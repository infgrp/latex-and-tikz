% 반드시 XeLatex으로 컴파일을 해야 합니다.
% 같이 제공하는 preamblex.tex은 trig.tex보다 상위 디렉토리에 두어야 컴파일이 됩니다. 두 파일을 같은 디렉토리에 두려면 몇 줄 아래에 있는 % AMS and mathtools
\usepackage{amsmath,amsthm,amssymb,marvosym,mathrsfs,amsfonts,amscd,mathtools}
% Hyperlinks and URLs
\usepackage{url}
\usepackage{hyperref}
\hypersetup{
	colorlinks,
	citecolor=BLACK,
	filecolor=BLACK,
	linkcolor=BLACK,
	urlcolor=BLACK
}

% Colors
\usepackage[usenames,dvipsnames]{xcolor}
\usepackage{tikz}
\usepackage{tkz-euclide}

% shadowing mdframed
\usepackage[framemethod=tikz]{mdframed}
\usetikzlibrary{shadows}
% Bold math
\usepackage{bm}

%Use Korean Letter when enumerate
\usepackage{dhucs-enumerate}

\usepackage{anyfontsize}

% Bra Ket (Dirac) Notation
\usepackage{braket}

% Slashed characters (e.g. in Dirac equation)
\usepackage{slashed}
\usepackage{pifont} % 원문자 사용시 필요한 패키지

% chapter decoration
\usepackage{type1cm}
\usepackage[explicit]{titlesec}

\titleformat{\chapter}[display]
{\normalfont\Large\rmfamily}
{\sffamily\flushright\fontsize{60}{0}\textbf{\textcolor{blue!40}{{\Huge\chaptername}~\thechapter\vskip0pt\rule{\textwidth}{2pt}}}}{0pt}
{\flushleft\fontsize{30}{0}{#1}\vskip60pt}
\titlespacing*{\chapter}
{0pt}{-40pt}{0pt}

%\usetikzlibrary{shadows}
\usetikzlibrary{shadows.blur}
\usetikzlibrary{shapes.symbols}

% Tcolorbox
\usepackage[most]{tcolorbox}

% Clean SI Units
\usepackage{siunitx}

% Enumerate thingies
\usepackage{enumitem}

% Cancel things out in equations
\usepackage[makeroom]{cancel}

\usepackage{multicol}

% Graphics and figures
\usepackage{graphicx}
\usepackage{wrapfig}
\usepackage{float}

\usepackage{cancel}

% Caption figures and tables
\usepackage{caption,subcaption}

% Generate symbols
\usepackage{textcomp} % Include this line to avoid output errors
\usepackage{gensymb}

% Make multiple rows in a table
\usepackage{multirow}

% Booktabs tables
\usepackage{booktabs}

%\usepackage[utopia,sfscaled]{mathdesign}
% Useful frames
\usepackage{mdframed}

% Comment-out large sections
\usepackage{comment}

% No auto-indent
\setlength{\parindent}{0pt}

% Asymptote - 3D vector graphics
\usepackage{asymptote}

% Tikz Package Stuff
\usepackage{pgf,tikz,pgfplots}
\usepackage{tikz-3dplot}
\usepackage{tabularx}
\usepackage{array}
\usepackage{colortbl}
\tcbuselibrary{skins}
\usepackage{tkz-euclide}

\newcolumntype{Y}{>{\raggedleft\arraybackslash}X}

\tcbset{tab1/.style={fonttitle=\bfseries\large,fontupper=\normalsize\sffamily,
		colback=yellow!10!white,colframe=red!75!black,colbacktitle=Salmon!40!white, halign=center,
		coltitle=black,center title,freelance,frame code={
			\foreach \n in {north east,north west,south east,south west}
			{\path [fill=red!75!black] (interior.\n) circle (3mm); };},}}

\tcbset{tab2/.style={enhanced,fonttitle=\bfseries,fontupper=\normalsize\sffamily, halign=center, box align=center,
		colback=yellow!10!white,colframe=red!50!black,colbacktitle=Salmon!40!white,
		coltitle=black,center title}}


% Use various tikz libraries
\usetikzlibrary{decorations.pathmorphing, decorations.markings, decorations.pathreplacing, patterns} % Decorate paths!
\usetikzlibrary{calc, patterns, shapes.geometric, positioning, through, intersections}
\usetikzlibrary{scopes}
\usetikzlibrary{angles, quotes}
\usetikzlibrary{svg.path}
\usetikzlibrary{arrows, arrows.meta}
\usetikzlibrary{fadings}
% pgfplots package settings
\pgfplotsset{compat=1.15}
% \pgfplotsset{width=10cm,compat=1.9} % Taken from latest overleaf.
% plot arc easily
\def\centerarc[#1](#2)(#3:#4:#5)% Syntax: [draw options] (center) (initial angle:final angle:radius)
{ \draw[#1] ($(#2)+({#5*cos(#3)},{#5*sin(#3)})$) arc (#3:#4:#5); }

% Awesome circled numbers
\newcommand*\circled[4]{\tikz[baseline=(char.base)]{\node[shape=circle, fill=#2, draw=#3, text=#4, inner sep=2pt] (char) {#1};}}

% Control size of text
\usepackage{relsize}

% Extend conditional commands
\usepackage{xifthen}
\usepackage{xcolor}
\definecolor{termcolor}{cmyk}{.21,.97,.0,.0}
\definecolor{darkred}{cmyk}{.27,1,1,.32}
\definecolor{darkblue}{cmyk}{1,.98,.10,.11}
\definecolor{darkgreen}{cmyk}{.29,0,87,0}
\definecolor{darkmycolor}{cmyk}{99,59,22,3}
\definecolor{for_eyes}{RGB}{253,247,228}
%change color of math equation
%\everymath{\color{darkred}}
% Scale math by size
\newcommand*{\Scale}[2][4]{\scalebox{#1}{\ensuremath{#2}}}

% Big integrals
\usepackage{bigints}

% Number equations within sections
\numberwithin{equation}{section}

% Generate blind text
\usepackage{blindtext}

% Useful symbols
\usepackage{marvosym}

\newcounter{problem}[section]
\newcounter{example}[section]

% cancel 색상 변경
\newcommand\Ccancel[2][black]{\renewcommand\CancelColor{\color{#1}}\cancel{#2}}

%%%% 원문자
\newcommand*\ocircled[1]{\tikz[baseline=(char.base)]{
		\node[shape=circle,draw,inner sep=2pt] (char) {#1};}}
	
%%%%% 보기 스타일 %%%%%
\usepackage{tabu}
\newcommand{\questwo}[2]{
	\vskip 6pt
	\noindent\begin{tabu}{X[0.2] X[6] X[0.2] X[6]}
		(1)&$#1$ &(2) &$#2$
	\end{tabu}
}
\newcommand{\questhree}[3]{
	\vskip 3pt
	\noindent\begin{tabu}{X[0.2] X[6] X[0.2] X[6] X[0.2] X[6]}
		(1)&$#1$ &(2) &$#2$ &(3) & $#3$
	\end{tabu}
	\vskip 5pt
}
\newcommand{\quesfour}[4]{
	\vskip 6pt
	\noindent\begin{tabu}{X[0.2] X[6] X[0.2] X[6]}
		(1)&$#1$ &(2) &$#2$\\
		(3)&$#3$ &(4) &$#4$
	\end{tabu}
}
\newcommand{\quesfive}[5]{
	\vskip 6pt
	\noindent\begin{tabu}{X[0.2] X[3] X[0.2] X[3] X[0.2] X[3]}
		(1)&$#1$ &(2) &$#2$ &(3) &$#3$\\
		(4)&$#4$ &(5) &$#5$
	\end{tabu}
}

\newcommand{\oquesfive}[5]{
	\vskip 6pt
	\noindent\begin{tabu}{X[0.2] X[3] X[0.2] X[3] X[0.2] X[3] X[0.2] X[3] X[0.2] X[3]}
		\ding{172}&$#1$ &\ding{173} &$#2$ &\ding{174} &$#3$&	\ding{175}&$#4$ &\ding{176} &$#5$
	\end{tabu}
}
%%%% sample(보기) %%%%
\newenvironment{sample}{\vskip 10pt\noindent\begin{tikzpicture}[yshift=1.5pt]%
		\draw[rounded corners=1ex,overlay,draw=blue] (0pt,-4pt) rectangle (19pt,9pt);
		\node[rectangle,overlay,xshift=9.8pt,yshift=2pt,color=blue] {{\footnotesize\sffamily 보기}};\end{tikzpicture}\phantom{\footnotesize\sffamily보기...}}{\vskip 10pt}
%%%% THEOREMS %%%%
\newenvironment{theorem}[1][\hspace{-0.36em}]
{
	\begin{mdframed}[backgroundcolor=purple!5, align=center, userdefinedwidth=40em, linecolor=purple!30, linewidth=2pt, roundcorner=7pt, innertopmargin=10pt, shadow=true, shadowcolor=black!20, roundcorner=7pt, innerbottommargin=10pt, frametitle = {#1}]
	}
	{
	\end{mdframed}
}

%%%% LEMMAS %%%%
\newenvironment{lemma}[1][\hspace{-0.36em}]
{
	\begin{mdframed}[backgroundcolor=black!8, align=center, userdefinedwidth=40em, topline=false, bottomline = false, leftline = false, rightline = false, frametitle = {#1 보조정리}]
	}
	{
	\end{mdframed}
}

%%%% COROLLARY %%%%
\newenvironment{corollary}[1][\hspace{-0.36em}]
{
	\begin{mdframed}[backgroundcolor=black!8, align=center, userdefinedwidth=40em, topline=false, bottomline = false, leftline = false, rightline = false, frametitle = {#1 따름정리}]
	}
	{
	\end{mdframed}
}

%%%% DEFINITIONS %%%%
\newenvironment{definition}[1][\hspace{-0.36em}]
{
	\begin{mdframed}[backgroundcolor=cyan!14, align=center, userdefinedwidth=40em, linecolor=cyan!60, linewidth=2pt,roundcorner=7pt, innertopmargin=10pt, shadow=true, shadowcolor=black!20, roundcorner=7pt, innerbottommargin=10pt, frametitle = {정의 : #1}]
	}
	{
	\end{mdframed}
}

%%%% PROPOSITION %%%%
\newenvironment{proposition}
{
	\begin{mdframed}[backgroundcolor=black!4, align=center, userdefinedwidth=40em, topline=false, bottomline = false, leftline = false, rightline = false, frametitle = {Proposition}]
	}
	{
	\end{mdframed}
}
%%%% PROBLEM %%%%
\newenvironment{problem}{\refstepcounter{problem}
	\begin{mdframed}[linecolor=blue!35, linewidth=2pt, roundcorner=7pt, innertopmargin=10pt, shadow=true, shadowcolor=black!20, roundcorner=7pt, innerbottommargin=10pt, backgroundcolor=blue!5]
		\noindent 
		\noindent\begin{tikzpicture}[overlay,xshift=6pt,yshift=5pt]
			\draw[fill=violet!15,draw=violet!15] (0,0) circle (8pt);
			\draw[fill=violet!15,draw=violet!15] (9pt,0) circle (8pt);
			\node[rectangle,overlay,xshift=4pt] {\color{black}\sffamily\bfseries 문제};
		\end{tikzpicture}{\phantom{.........}\fontspec[Scale=1.1]{TeX Gyre Adventor}\color{darkblue} \theproblem}\hspace{5pt}}{
\end{mdframed}}
%%%% SOLUTION OF PROBLEM %%%%
\newenvironment{psolution}{\begin{description}\item[{\begin{tikzpicture}%
				\draw[rounded corners=1ex,overlay] (-3pt,-3pt) rectangle (20pt,9pt);\end{tikzpicture}\footnotesize\sffamily풀이}\hspace{6pt}]}{\end{description}}
			
%%%% EXAMPLE %%%%		
\newenvironment{example}{\refstepcounter{example}
	\begin{mdframed}[roundcorner=7pt,linecolor=termcolor,linewidth=2pt,innertopmargin=10pt, shadow=true, shadowcolor=black!20, innerbottommargin=10pt, backgroundcolor=termcolor!2]
		\noindent 
		\noindent\begin{tikzpicture}[overlay,xshift=6pt,yshift=5pt]
			\draw[fill=darkred!60,draw=darkred!60] (0,0) circle (7pt);
			\draw[fill=darkred!60,draw=darkred!60] (9pt,0) circle (7pt);
			\node[rectangle,overlay,xshift=4pt] {\color{white}\sffamily\bfseries 예제};
		\end{tikzpicture}{\phantom{.........}\fontspec[Scale=1.1]{TeX Gyre Adventor}\color{darkred} \theexample}\hspace{5pt}}{
\end{mdframed}}		
%%%%%%% SOLUTION OF EXAMPLE %%%%%%%%%
\newenvironment{solution}{\begin{description}\item[{\begin{tikzpicture}%
				\draw[rounded corners=1ex,overlay] (-3pt,-3pt) rectangle (20pt,9pt);\end{tikzpicture}\footnotesize\sffamily풀이}\hspace{6pt}]}{\end{description}}

% 보기 박스 정의 시작
\tcbuselibrary{breakable, skins}
\tcbset{enhanced}
\newtcolorbox{ChoiceBox}[1]{
		enhanced,
		before skip=2ex, after skip=2ex,
		boxrule=0.5pt, colframe=black, colback=white, arc=0.5ex,
		boxsep=0.5ex, top=1.5ex, bottom=1.5ex, left=0.5em, right=o0.5em,
		colbacktitle=white, coltitle=black,
		attach boxed title to top center={xshift=0cm, yshift=-1.5mm},
		boxed title style={size=minimal, enhanced, boxrule=0.25pt, colframe=white},
		breakable=false, title ={< #1 >}
}
% 보기 박스 정의 끝.
	
% Change end-of-proof symbol
\renewcommand\qedsymbol{$\blacksquare$}
%overline
\newcommand{\ovr}[1]{\overline{\textrm{#1}}}
% trigonometric function
\newcommand{\cosrm}[1]{\cos \textrm{#1}}
\newcommand{\sinrm}[1]{\sin \textrm{#1}}
\newcommand{\tanrm}[1]{\tan \textrm{#1}}
\newcommand{\cotrm}[1]{\cot \textrm{#1}}
\newcommand{\cscrm}[1]{\csc \textrm{#1}}
\newcommand{\secrm}[1]{\sec \textrm{#1}}
%%%% BLACKBOARD BOLD %%%%
\newcommand{\bbN}{\mathbb{N}} % Natural numbers
\newcommand{\bbZ}{\mathbb{Z}} % Zahlen
\newcommand{\bbQ}{\mathbb{Q}} % Rational numbers
\newcommand{\bbR}{\mathbb{R}} % Real numbers
\newcommand{\bbC}{\mathbb{C}} % Complex numbers
\DeclareSymbolFont{bbold}{U}{bbold}{m}{n} % Identity matrix
\DeclareSymbolFontAlphabet{\mathbbold}{bbold} % Identity matrix
\newcommand{\identitymatrix}{\mathbbold{1}} % Identity matrix

%%%% CODE LISTING %%%%
\usepackage{listings}
\definecolor{greencomments}{HTML}{00BA00}
\definecolor{graynumbers}{HTML}{4F4F4F}
\definecolor{purplestrings}{HTML}{AD00AA}
\definecolor{backgroundcolor}{HTML}{E8E8E8}


%%%% UNIT BASIS VECTORS %%%%
\newcommand{\ihat}{\bm{\hat{\imath}}} % Cartesian i hat (x-direction)
\newcommand{\jhat}{\bm{\hat{\jmath}}} % Cartesian j hat (y-direction)
\newcommand{\khat}{\bm{\hat{k}}} % Cartesian k hat (z-direction)
\newcommand{\rhat}{\bm{\hat{r}}} % Spherical r hat
\newcommand{\phihat}{\bm{\hat{\phi}}} % Spherical phi hat
\newcommand{\thetahat}{\bm{\hat{\theta}}} % Spherical theta hat
\newcommand{\nhat}{\bm{\hat{n}}} % Unit normal vector
\newcommand{\rhohat}{\bm{\hat{\rho}}} % Cylindrical rho hat
\newcommand{\zhat}{\bm{\hat{z}}} % Cylindrical z hat


%%%% COLORS: DEFINITIONS AND COMMANDS %%%%
% Miscellaneous
\definecolor{DARKBLUE}{HTML}{040080}
\definecolor{DARKBROWN}{HTML}{8B4513}
\definecolor{LIGHTBROWN}{HTML}{CD853F}
\definecolor{PINK}{HTML}{D147BD}
\definecolor{LIGHTPINK}{HTML}{DC75CD}
\definecolor{GREENSCREEN}{HTML}{00FF00}
\definecolor{ORANGE}{HTML}{FF862F}
\newcommand{\DARKBLUE}{\color{DARKBLUE}}
\newcommand{\DARKBROWN}{\color{DARKBROWN}}
\newcommand{\LIGHTBROWN}{\color{LIGHTBROWN}}
\newcommand{\PINK}{\color{PINK}}
\newcommand{\LIGHTPINK}{\color{LIGHTPINK}}
\newcommand{\GREENSCREEN}{\color{GREENSCREEN}}
\newcommand{\ORANGE}{\color{ORANGE}}
% Blue
\definecolor{BLUEE}{HTML}{1C758A}
\definecolor{BLUED}{HTML}{29ABCA}
\definecolor{BLUEC}{HTML}{58C4DD}
\definecolor{BLUEB}{HTML}{9CDCEB}
\definecolor{BLUEA}{HTML}{C7E9F1}
\definecolor{BLUE}{HTML}{0000FF}
\newcommand{\BLUEE}{\color{BLUEE}}
\newcommand{\BLUED}{\color{BLUED}}
\newcommand{\BLUEC}{\color{BLUEC}}
\newcommand{\BLUEB}{\color{BLUEB}}
\newcommand{\BLUEA}{\color{BLUEA}}
\newcommand{\BLUE}{\color{BLUE}}
% Teal
\definecolor{TEALE}{HTML}{49A88F}
\definecolor{TEALD}{HTML}{55C1A7}
\definecolor{TEALC}{HTML}{5CD0B3}
\definecolor{TEALB}{HTML}{76DDC0}
\definecolor{TEALA}{HTML}{ACEAD7}
\definecolor{TEAL}{HTML}{00FFFF}
\newcommand{\TEALE}{\color{TEALE}}
\newcommand{\TEALD}{\color{TEALD}}
\newcommand{\TEALC}{\color{TEALC}}
\newcommand{\TEALB}{\color{TEALB}}
\newcommand{\TEALA}{\color{TEALA}}
\newcommand{\TEAL}{\color{TEAL}}
% Green
\definecolor{GREENE}{HTML}{699C52}
\definecolor{GREEND}{HTML}{77B05D}
\definecolor{GREENC}{HTML}{83C167}
\definecolor{GREENB}{HTML}{A6CF8C}
\definecolor{GREENA}{HTML}{C9E2AE}
\definecolor{GREEN}{HTML}{00FF00}
\newcommand{\GREENE}{\color{GREENE}}
\newcommand{\GREEND}{\color{GREEND}}
\newcommand{\GREENC}{\color{GREENC}}
\newcommand{\GREENB}{\color{GREENB}}
\newcommand{\GREENA}{\color{GREENA}}
\newcommand{\GREEN}{\color{GREEN}}
% Yellow
\definecolor{YELLOWE}{HTML}{E8C11C}
\definecolor{YELLOWD}{HTML}{F4D345}
\definecolor{YELLOWC}{HTML}{FFFF00}
\definecolor{YELLOWB}{HTML}{FFEA94}
\definecolor{YELLOWA}{HTML}{FFF1B6}
\definecolor{YELLOW}{HTML}{FFFF00}
\newcommand{\YELLOWE}{\color{YELLOWE}}
\newcommand{\YELLOWD}{\color{YELLOWD}}
\newcommand{\YELLOWC}{\color{YELLOWC}}
\newcommand{\YELLOWB}{\color{YELLOWB}}
\newcommand{\YELLOWA}{\color{YELLOWA}}
\newcommand{\YELLOW}{\color{YELLOW}}
% Gold
\definecolor{GOLDE}{HTML}{C78D46}
\definecolor{GOLDD}{HTML}{E1A158}
\definecolor{GOLDC}{HTML}{F0AC5F}
\definecolor{GOLDB}{HTML}{F9B775}
\definecolor{GOLDA}{HTML}{F7C797}
\newcommand{\GOLDE}{\color{GOLDE}}
\newcommand{\GOLDD}{\color{GOLDD}}
\newcommand{\GOLDC}{\color{GOLDC}}
\newcommand{\GOLDB}{\color{GOLDB}}
\newcommand{\GOLDA}{\color{GOLDA}}
% Red
\definecolor{REDE}{HTML}{CF5044}
\definecolor{REDD}{HTML}{E65A4C}
\definecolor{REDC}{HTML}{FC6255}
\definecolor{REDB}{HTML}{FF8080}
\definecolor{REDA}{HTML}{F7A1A3}
\definecolor{RED}{HTML}{FF0000}
\newcommand{\REDE}{\color{REDE}}
\newcommand{\REDD}{\color{REDD}}
\newcommand{\REDC}{\color{REDC}}
\newcommand{\REDB}{\color{REDB}}
\newcommand{\REDA}{\color{REDA}}
\newcommand{\RED}{\color{RED}}
% Maroon
\definecolor{MAROONE}{HTML}{94424F}
\definecolor{MAROOND}{HTML}{A24D61}
\definecolor{MAROONC}{HTML}{C55F73}
\definecolor{MAROONB}{HTML}{EC92AB}
\definecolor{MAROONA}{HTML}{ECABC1}
\newcommand{\MAROONE}{\color{MAROONE}}
\newcommand{\MAROOND}{\color{MAROOND}}
\newcommand{\MAROONC}{\color{MAROONC}}
\newcommand{\MAROONB}{\color{MAROONB}}
\newcommand{\MAROONA}{\color{MAROONA}}
% Purple
\definecolor{PURPLEE}{HTML}{644172}
\definecolor{PURPLED}{HTML}{715582}
\definecolor{PURPLEC}{HTML}{9A72AC}
\definecolor{PURPLEB}{HTML}{B189C6}
\definecolor{PURPLEA}{HTML}{CAA3E8}
\definecolor{PURPLE}{HTML}{FF00FF}
\newcommand{\PURPLEE}{\color{PURPLEE}}
\newcommand{\PURPLED}{\color{PURPLED}}
\newcommand{\PURPLEC}{\color{PURPLEC}}
\newcommand{\PURPLEB}{\color{PURPLEB}}
\newcommand{\PURPLEA}{\color{PURPLEA}}
\newcommand{\PURPLE}{\color{PURPLE}}
% White and Black
\definecolor{WHITE}{HTML}{FFFFFF}
\newcommand{\WHITE}{\color{WHITE}}
\definecolor{BLACK}{HTML}{000000}
\newcommand{\BLACK}{\color{BLACK}}
% Different Grays
\definecolor{LIGHTGRAY}{HTML}{BBBBBB}
\definecolor{GRAY}{HTML}{888888}
\definecolor{DARKGRAY}{HTML}{444444}
\definecolor{DARKERGRAY}{HTML}{222222}
\definecolor{GRAYBROWN}{HTML}{736357}
\newcommand{\LIGHTGRAY}{\color{LIGHTGRAY}}
\newcommand{\GRAY}{\color{GRAY}}
\newcommand{\DARKGRAY}{\color{DARKGRAY}}
\newcommand{\DARKERGRAY}{\color{DARKERGRAY}}
\newcommand{\GRAYBROWN}{\color{GRAYBROWN}}

를 % AMS and mathtools
\usepackage{amsmath,amsthm,amssymb,marvosym,mathrsfs,amsfonts,amscd,mathtools}
% Hyperlinks and URLs
\usepackage{url}
\usepackage{hyperref}
\hypersetup{
	colorlinks,
	citecolor=BLACK,
	filecolor=BLACK,
	linkcolor=BLACK,
	urlcolor=BLACK
}

% Colors
\usepackage[usenames,dvipsnames]{xcolor}
\usepackage{tikz}
\usepackage{tkz-euclide}

% shadowing mdframed
\usepackage[framemethod=tikz]{mdframed}
\usetikzlibrary{shadows}
% Bold math
\usepackage{bm}

%Use Korean Letter when enumerate
\usepackage{dhucs-enumerate}

\usepackage{anyfontsize}

% Bra Ket (Dirac) Notation
\usepackage{braket}

% Slashed characters (e.g. in Dirac equation)
\usepackage{slashed}
\usepackage{pifont} % 원문자 사용시 필요한 패키지

% chapter decoration
\usepackage{type1cm}
\usepackage[explicit]{titlesec}

\titleformat{\chapter}[display]
{\normalfont\Large\rmfamily}
{\sffamily\flushright\fontsize{60}{0}\textbf{\textcolor{blue!40}{{\Huge\chaptername}~\thechapter\vskip0pt\rule{\textwidth}{2pt}}}}{0pt}
{\flushleft\fontsize{30}{0}{#1}\vskip60pt}
\titlespacing*{\chapter}
{0pt}{-40pt}{0pt}

%\usetikzlibrary{shadows}
\usetikzlibrary{shadows.blur}
\usetikzlibrary{shapes.symbols}

% Tcolorbox
\usepackage[most]{tcolorbox}

% Clean SI Units
\usepackage{siunitx}

% Enumerate thingies
\usepackage{enumitem}

% Cancel things out in equations
\usepackage[makeroom]{cancel}

\usepackage{multicol}

% Graphics and figures
\usepackage{graphicx}
\usepackage{wrapfig}
\usepackage{float}

\usepackage{cancel}

% Caption figures and tables
\usepackage{caption,subcaption}

% Generate symbols
\usepackage{textcomp} % Include this line to avoid output errors
\usepackage{gensymb}

% Make multiple rows in a table
\usepackage{multirow}

% Booktabs tables
\usepackage{booktabs}

%\usepackage[utopia,sfscaled]{mathdesign}
% Useful frames
\usepackage{mdframed}

% Comment-out large sections
\usepackage{comment}

% No auto-indent
\setlength{\parindent}{0pt}

% Asymptote - 3D vector graphics
\usepackage{asymptote}

% Tikz Package Stuff
\usepackage{pgf,tikz,pgfplots}
\usepackage{tikz-3dplot}
\usepackage{tabularx}
\usepackage{array}
\usepackage{colortbl}
\tcbuselibrary{skins}
\usepackage{tkz-euclide}

\newcolumntype{Y}{>{\raggedleft\arraybackslash}X}

\tcbset{tab1/.style={fonttitle=\bfseries\large,fontupper=\normalsize\sffamily,
		colback=yellow!10!white,colframe=red!75!black,colbacktitle=Salmon!40!white, halign=center,
		coltitle=black,center title,freelance,frame code={
			\foreach \n in {north east,north west,south east,south west}
			{\path [fill=red!75!black] (interior.\n) circle (3mm); };},}}

\tcbset{tab2/.style={enhanced,fonttitle=\bfseries,fontupper=\normalsize\sffamily, halign=center, box align=center,
		colback=yellow!10!white,colframe=red!50!black,colbacktitle=Salmon!40!white,
		coltitle=black,center title}}


% Use various tikz libraries
\usetikzlibrary{decorations.pathmorphing, decorations.markings, decorations.pathreplacing, patterns} % Decorate paths!
\usetikzlibrary{calc, patterns, shapes.geometric, positioning, through, intersections}
\usetikzlibrary{scopes}
\usetikzlibrary{angles, quotes}
\usetikzlibrary{svg.path}
\usetikzlibrary{arrows, arrows.meta}
\usetikzlibrary{fadings}
% pgfplots package settings
\pgfplotsset{compat=1.15}
% \pgfplotsset{width=10cm,compat=1.9} % Taken from latest overleaf.
% plot arc easily
\def\centerarc[#1](#2)(#3:#4:#5)% Syntax: [draw options] (center) (initial angle:final angle:radius)
{ \draw[#1] ($(#2)+({#5*cos(#3)},{#5*sin(#3)})$) arc (#3:#4:#5); }

% Awesome circled numbers
\newcommand*\circled[4]{\tikz[baseline=(char.base)]{\node[shape=circle, fill=#2, draw=#3, text=#4, inner sep=2pt] (char) {#1};}}

% Control size of text
\usepackage{relsize}

% Extend conditional commands
\usepackage{xifthen}
\usepackage{xcolor}
\definecolor{termcolor}{cmyk}{.21,.97,.0,.0}
\definecolor{darkred}{cmyk}{.27,1,1,.32}
\definecolor{darkblue}{cmyk}{1,.98,.10,.11}
\definecolor{darkgreen}{cmyk}{.29,0,87,0}
\definecolor{darkmycolor}{cmyk}{99,59,22,3}
\definecolor{for_eyes}{RGB}{253,247,228}
%change color of math equation
%\everymath{\color{darkred}}
% Scale math by size
\newcommand*{\Scale}[2][4]{\scalebox{#1}{\ensuremath{#2}}}

% Big integrals
\usepackage{bigints}

% Number equations within sections
\numberwithin{equation}{section}

% Generate blind text
\usepackage{blindtext}

% Useful symbols
\usepackage{marvosym}

\newcounter{problem}[section]
\newcounter{example}[section]

% cancel 색상 변경
\newcommand\Ccancel[2][black]{\renewcommand\CancelColor{\color{#1}}\cancel{#2}}

%%%% 원문자
\newcommand*\ocircled[1]{\tikz[baseline=(char.base)]{
		\node[shape=circle,draw,inner sep=2pt] (char) {#1};}}
	
%%%%% 보기 스타일 %%%%%
\usepackage{tabu}
\newcommand{\questwo}[2]{
	\vskip 6pt
	\noindent\begin{tabu}{X[0.2] X[6] X[0.2] X[6]}
		(1)&$#1$ &(2) &$#2$
	\end{tabu}
}
\newcommand{\questhree}[3]{
	\vskip 3pt
	\noindent\begin{tabu}{X[0.2] X[6] X[0.2] X[6] X[0.2] X[6]}
		(1)&$#1$ &(2) &$#2$ &(3) & $#3$
	\end{tabu}
	\vskip 5pt
}
\newcommand{\quesfour}[4]{
	\vskip 6pt
	\noindent\begin{tabu}{X[0.2] X[6] X[0.2] X[6]}
		(1)&$#1$ &(2) &$#2$\\
		(3)&$#3$ &(4) &$#4$
	\end{tabu}
}
\newcommand{\quesfive}[5]{
	\vskip 6pt
	\noindent\begin{tabu}{X[0.2] X[3] X[0.2] X[3] X[0.2] X[3]}
		(1)&$#1$ &(2) &$#2$ &(3) &$#3$\\
		(4)&$#4$ &(5) &$#5$
	\end{tabu}
}

\newcommand{\oquesfive}[5]{
	\vskip 6pt
	\noindent\begin{tabu}{X[0.2] X[3] X[0.2] X[3] X[0.2] X[3] X[0.2] X[3] X[0.2] X[3]}
		\ding{172}&$#1$ &\ding{173} &$#2$ &\ding{174} &$#3$&	\ding{175}&$#4$ &\ding{176} &$#5$
	\end{tabu}
}
%%%% sample(보기) %%%%
\newenvironment{sample}{\vskip 10pt\noindent\begin{tikzpicture}[yshift=1.5pt]%
		\draw[rounded corners=1ex,overlay,draw=blue] (0pt,-4pt) rectangle (19pt,9pt);
		\node[rectangle,overlay,xshift=9.8pt,yshift=2pt,color=blue] {{\footnotesize\sffamily 보기}};\end{tikzpicture}\phantom{\footnotesize\sffamily보기...}}{\vskip 10pt}
%%%% THEOREMS %%%%
\newenvironment{theorem}[1][\hspace{-0.36em}]
{
	\begin{mdframed}[backgroundcolor=purple!5, align=center, userdefinedwidth=40em, linecolor=purple!30, linewidth=2pt, roundcorner=7pt, innertopmargin=10pt, shadow=true, shadowcolor=black!20, roundcorner=7pt, innerbottommargin=10pt, frametitle = {#1}]
	}
	{
	\end{mdframed}
}

%%%% LEMMAS %%%%
\newenvironment{lemma}[1][\hspace{-0.36em}]
{
	\begin{mdframed}[backgroundcolor=black!8, align=center, userdefinedwidth=40em, topline=false, bottomline = false, leftline = false, rightline = false, frametitle = {#1 보조정리}]
	}
	{
	\end{mdframed}
}

%%%% COROLLARY %%%%
\newenvironment{corollary}[1][\hspace{-0.36em}]
{
	\begin{mdframed}[backgroundcolor=black!8, align=center, userdefinedwidth=40em, topline=false, bottomline = false, leftline = false, rightline = false, frametitle = {#1 따름정리}]
	}
	{
	\end{mdframed}
}

%%%% DEFINITIONS %%%%
\newenvironment{definition}[1][\hspace{-0.36em}]
{
	\begin{mdframed}[backgroundcolor=cyan!14, align=center, userdefinedwidth=40em, linecolor=cyan!60, linewidth=2pt,roundcorner=7pt, innertopmargin=10pt, shadow=true, shadowcolor=black!20, roundcorner=7pt, innerbottommargin=10pt, frametitle = {정의 : #1}]
	}
	{
	\end{mdframed}
}

%%%% PROPOSITION %%%%
\newenvironment{proposition}
{
	\begin{mdframed}[backgroundcolor=black!4, align=center, userdefinedwidth=40em, topline=false, bottomline = false, leftline = false, rightline = false, frametitle = {Proposition}]
	}
	{
	\end{mdframed}
}
%%%% PROBLEM %%%%
\newenvironment{problem}{\refstepcounter{problem}
	\begin{mdframed}[linecolor=blue!35, linewidth=2pt, roundcorner=7pt, innertopmargin=10pt, shadow=true, shadowcolor=black!20, roundcorner=7pt, innerbottommargin=10pt, backgroundcolor=blue!5]
		\noindent 
		\noindent\begin{tikzpicture}[overlay,xshift=6pt,yshift=5pt]
			\draw[fill=violet!15,draw=violet!15] (0,0) circle (8pt);
			\draw[fill=violet!15,draw=violet!15] (9pt,0) circle (8pt);
			\node[rectangle,overlay,xshift=4pt] {\color{black}\sffamily\bfseries 문제};
		\end{tikzpicture}{\phantom{.........}\fontspec[Scale=1.1]{TeX Gyre Adventor}\color{darkblue} \theproblem}\hspace{5pt}}{
\end{mdframed}}
%%%% SOLUTION OF PROBLEM %%%%
\newenvironment{psolution}{\begin{description}\item[{\begin{tikzpicture}%
				\draw[rounded corners=1ex,overlay] (-3pt,-3pt) rectangle (20pt,9pt);\end{tikzpicture}\footnotesize\sffamily풀이}\hspace{6pt}]}{\end{description}}
			
%%%% EXAMPLE %%%%		
\newenvironment{example}{\refstepcounter{example}
	\begin{mdframed}[roundcorner=7pt,linecolor=termcolor,linewidth=2pt,innertopmargin=10pt, shadow=true, shadowcolor=black!20, innerbottommargin=10pt, backgroundcolor=termcolor!2]
		\noindent 
		\noindent\begin{tikzpicture}[overlay,xshift=6pt,yshift=5pt]
			\draw[fill=darkred!60,draw=darkred!60] (0,0) circle (7pt);
			\draw[fill=darkred!60,draw=darkred!60] (9pt,0) circle (7pt);
			\node[rectangle,overlay,xshift=4pt] {\color{white}\sffamily\bfseries 예제};
		\end{tikzpicture}{\phantom{.........}\fontspec[Scale=1.1]{TeX Gyre Adventor}\color{darkred} \theexample}\hspace{5pt}}{
\end{mdframed}}		
%%%%%%% SOLUTION OF EXAMPLE %%%%%%%%%
\newenvironment{solution}{\begin{description}\item[{\begin{tikzpicture}%
				\draw[rounded corners=1ex,overlay] (-3pt,-3pt) rectangle (20pt,9pt);\end{tikzpicture}\footnotesize\sffamily풀이}\hspace{6pt}]}{\end{description}}

% 보기 박스 정의 시작
\tcbuselibrary{breakable, skins}
\tcbset{enhanced}
\newtcolorbox{ChoiceBox}[1]{
		enhanced,
		before skip=2ex, after skip=2ex,
		boxrule=0.5pt, colframe=black, colback=white, arc=0.5ex,
		boxsep=0.5ex, top=1.5ex, bottom=1.5ex, left=0.5em, right=o0.5em,
		colbacktitle=white, coltitle=black,
		attach boxed title to top center={xshift=0cm, yshift=-1.5mm},
		boxed title style={size=minimal, enhanced, boxrule=0.25pt, colframe=white},
		breakable=false, title ={< #1 >}
}
% 보기 박스 정의 끝.
	
% Change end-of-proof symbol
\renewcommand\qedsymbol{$\blacksquare$}
%overline
\newcommand{\ovr}[1]{\overline{\textrm{#1}}}
% trigonometric function
\newcommand{\cosrm}[1]{\cos \textrm{#1}}
\newcommand{\sinrm}[1]{\sin \textrm{#1}}
\newcommand{\tanrm}[1]{\tan \textrm{#1}}
\newcommand{\cotrm}[1]{\cot \textrm{#1}}
\newcommand{\cscrm}[1]{\csc \textrm{#1}}
\newcommand{\secrm}[1]{\sec \textrm{#1}}
%%%% BLACKBOARD BOLD %%%%
\newcommand{\bbN}{\mathbb{N}} % Natural numbers
\newcommand{\bbZ}{\mathbb{Z}} % Zahlen
\newcommand{\bbQ}{\mathbb{Q}} % Rational numbers
\newcommand{\bbR}{\mathbb{R}} % Real numbers
\newcommand{\bbC}{\mathbb{C}} % Complex numbers
\DeclareSymbolFont{bbold}{U}{bbold}{m}{n} % Identity matrix
\DeclareSymbolFontAlphabet{\mathbbold}{bbold} % Identity matrix
\newcommand{\identitymatrix}{\mathbbold{1}} % Identity matrix

%%%% CODE LISTING %%%%
\usepackage{listings}
\definecolor{greencomments}{HTML}{00BA00}
\definecolor{graynumbers}{HTML}{4F4F4F}
\definecolor{purplestrings}{HTML}{AD00AA}
\definecolor{backgroundcolor}{HTML}{E8E8E8}


%%%% UNIT BASIS VECTORS %%%%
\newcommand{\ihat}{\bm{\hat{\imath}}} % Cartesian i hat (x-direction)
\newcommand{\jhat}{\bm{\hat{\jmath}}} % Cartesian j hat (y-direction)
\newcommand{\khat}{\bm{\hat{k}}} % Cartesian k hat (z-direction)
\newcommand{\rhat}{\bm{\hat{r}}} % Spherical r hat
\newcommand{\phihat}{\bm{\hat{\phi}}} % Spherical phi hat
\newcommand{\thetahat}{\bm{\hat{\theta}}} % Spherical theta hat
\newcommand{\nhat}{\bm{\hat{n}}} % Unit normal vector
\newcommand{\rhohat}{\bm{\hat{\rho}}} % Cylindrical rho hat
\newcommand{\zhat}{\bm{\hat{z}}} % Cylindrical z hat


%%%% COLORS: DEFINITIONS AND COMMANDS %%%%
% Miscellaneous
\definecolor{DARKBLUE}{HTML}{040080}
\definecolor{DARKBROWN}{HTML}{8B4513}
\definecolor{LIGHTBROWN}{HTML}{CD853F}
\definecolor{PINK}{HTML}{D147BD}
\definecolor{LIGHTPINK}{HTML}{DC75CD}
\definecolor{GREENSCREEN}{HTML}{00FF00}
\definecolor{ORANGE}{HTML}{FF862F}
\newcommand{\DARKBLUE}{\color{DARKBLUE}}
\newcommand{\DARKBROWN}{\color{DARKBROWN}}
\newcommand{\LIGHTBROWN}{\color{LIGHTBROWN}}
\newcommand{\PINK}{\color{PINK}}
\newcommand{\LIGHTPINK}{\color{LIGHTPINK}}
\newcommand{\GREENSCREEN}{\color{GREENSCREEN}}
\newcommand{\ORANGE}{\color{ORANGE}}
% Blue
\definecolor{BLUEE}{HTML}{1C758A}
\definecolor{BLUED}{HTML}{29ABCA}
\definecolor{BLUEC}{HTML}{58C4DD}
\definecolor{BLUEB}{HTML}{9CDCEB}
\definecolor{BLUEA}{HTML}{C7E9F1}
\definecolor{BLUE}{HTML}{0000FF}
\newcommand{\BLUEE}{\color{BLUEE}}
\newcommand{\BLUED}{\color{BLUED}}
\newcommand{\BLUEC}{\color{BLUEC}}
\newcommand{\BLUEB}{\color{BLUEB}}
\newcommand{\BLUEA}{\color{BLUEA}}
\newcommand{\BLUE}{\color{BLUE}}
% Teal
\definecolor{TEALE}{HTML}{49A88F}
\definecolor{TEALD}{HTML}{55C1A7}
\definecolor{TEALC}{HTML}{5CD0B3}
\definecolor{TEALB}{HTML}{76DDC0}
\definecolor{TEALA}{HTML}{ACEAD7}
\definecolor{TEAL}{HTML}{00FFFF}
\newcommand{\TEALE}{\color{TEALE}}
\newcommand{\TEALD}{\color{TEALD}}
\newcommand{\TEALC}{\color{TEALC}}
\newcommand{\TEALB}{\color{TEALB}}
\newcommand{\TEALA}{\color{TEALA}}
\newcommand{\TEAL}{\color{TEAL}}
% Green
\definecolor{GREENE}{HTML}{699C52}
\definecolor{GREEND}{HTML}{77B05D}
\definecolor{GREENC}{HTML}{83C167}
\definecolor{GREENB}{HTML}{A6CF8C}
\definecolor{GREENA}{HTML}{C9E2AE}
\definecolor{GREEN}{HTML}{00FF00}
\newcommand{\GREENE}{\color{GREENE}}
\newcommand{\GREEND}{\color{GREEND}}
\newcommand{\GREENC}{\color{GREENC}}
\newcommand{\GREENB}{\color{GREENB}}
\newcommand{\GREENA}{\color{GREENA}}
\newcommand{\GREEN}{\color{GREEN}}
% Yellow
\definecolor{YELLOWE}{HTML}{E8C11C}
\definecolor{YELLOWD}{HTML}{F4D345}
\definecolor{YELLOWC}{HTML}{FFFF00}
\definecolor{YELLOWB}{HTML}{FFEA94}
\definecolor{YELLOWA}{HTML}{FFF1B6}
\definecolor{YELLOW}{HTML}{FFFF00}
\newcommand{\YELLOWE}{\color{YELLOWE}}
\newcommand{\YELLOWD}{\color{YELLOWD}}
\newcommand{\YELLOWC}{\color{YELLOWC}}
\newcommand{\YELLOWB}{\color{YELLOWB}}
\newcommand{\YELLOWA}{\color{YELLOWA}}
\newcommand{\YELLOW}{\color{YELLOW}}
% Gold
\definecolor{GOLDE}{HTML}{C78D46}
\definecolor{GOLDD}{HTML}{E1A158}
\definecolor{GOLDC}{HTML}{F0AC5F}
\definecolor{GOLDB}{HTML}{F9B775}
\definecolor{GOLDA}{HTML}{F7C797}
\newcommand{\GOLDE}{\color{GOLDE}}
\newcommand{\GOLDD}{\color{GOLDD}}
\newcommand{\GOLDC}{\color{GOLDC}}
\newcommand{\GOLDB}{\color{GOLDB}}
\newcommand{\GOLDA}{\color{GOLDA}}
% Red
\definecolor{REDE}{HTML}{CF5044}
\definecolor{REDD}{HTML}{E65A4C}
\definecolor{REDC}{HTML}{FC6255}
\definecolor{REDB}{HTML}{FF8080}
\definecolor{REDA}{HTML}{F7A1A3}
\definecolor{RED}{HTML}{FF0000}
\newcommand{\REDE}{\color{REDE}}
\newcommand{\REDD}{\color{REDD}}
\newcommand{\REDC}{\color{REDC}}
\newcommand{\REDB}{\color{REDB}}
\newcommand{\REDA}{\color{REDA}}
\newcommand{\RED}{\color{RED}}
% Maroon
\definecolor{MAROONE}{HTML}{94424F}
\definecolor{MAROOND}{HTML}{A24D61}
\definecolor{MAROONC}{HTML}{C55F73}
\definecolor{MAROONB}{HTML}{EC92AB}
\definecolor{MAROONA}{HTML}{ECABC1}
\newcommand{\MAROONE}{\color{MAROONE}}
\newcommand{\MAROOND}{\color{MAROOND}}
\newcommand{\MAROONC}{\color{MAROONC}}
\newcommand{\MAROONB}{\color{MAROONB}}
\newcommand{\MAROONA}{\color{MAROONA}}
% Purple
\definecolor{PURPLEE}{HTML}{644172}
\definecolor{PURPLED}{HTML}{715582}
\definecolor{PURPLEC}{HTML}{9A72AC}
\definecolor{PURPLEB}{HTML}{B189C6}
\definecolor{PURPLEA}{HTML}{CAA3E8}
\definecolor{PURPLE}{HTML}{FF00FF}
\newcommand{\PURPLEE}{\color{PURPLEE}}
\newcommand{\PURPLED}{\color{PURPLED}}
\newcommand{\PURPLEC}{\color{PURPLEC}}
\newcommand{\PURPLEB}{\color{PURPLEB}}
\newcommand{\PURPLEA}{\color{PURPLEA}}
\newcommand{\PURPLE}{\color{PURPLE}}
% White and Black
\definecolor{WHITE}{HTML}{FFFFFF}
\newcommand{\WHITE}{\color{WHITE}}
\definecolor{BLACK}{HTML}{000000}
\newcommand{\BLACK}{\color{BLACK}}
% Different Grays
\definecolor{LIGHTGRAY}{HTML}{BBBBBB}
\definecolor{GRAY}{HTML}{888888}
\definecolor{DARKGRAY}{HTML}{444444}
\definecolor{DARKERGRAY}{HTML}{222222}
\definecolor{GRAYBROWN}{HTML}{736357}
\newcommand{\LIGHTGRAY}{\color{LIGHTGRAY}}
\newcommand{\GRAY}{\color{GRAY}}
\newcommand{\DARKGRAY}{\color{DARKGRAY}}
\newcommand{\DARKERGRAY}{\color{DARKERGRAY}}
\newcommand{\GRAYBROWN}{\color{GRAYBROWN}}

로 수정하시기 바랍니다.

\documentclass[11pt, a4paper]{book}
%\documentclass[11pt,a4paper]{article} 
\usepackage[T1]{fontenc}
\usepackage{kotex}
\usepackage[dvipsnames]{xcolor}
\usepackage{secdot}
% Set page geometry
\usepackage[margin=2.5cm,headsep=0.5cm]{geometry}
\usepackage{setspace}
\usepackage{versions}
\usepackage[most]{tcolorbox}
\usepackage{anyfontsize}
\PassOptionsToPackage{inline}{enumitem} %한글 enumerator 쓸때 필요
\usepackage{dhucs-enumitem} %한글 enumerator 쓸때 필요함.
\usepackage{sectsty}
%\includeversion{aftercal}
\excludeversion{aftercal}
\includeversion{nogeo}
%\excludeversion{nogeo}
\includeversion{psol}
%\excludeversion{psol}
% AMS and mathtools
\usepackage{amsmath,amsthm,amssymb,marvosym,mathrsfs,amsfonts,amscd,mathtools}
% Hyperlinks and URLs
\usepackage{url}
\usepackage{hyperref}
\hypersetup{
	colorlinks,
	citecolor=BLACK,
	filecolor=BLACK,
	linkcolor=BLACK,
	urlcolor=BLACK
}

% Colors
\usepackage[usenames,dvipsnames]{xcolor}
\usepackage{tikz}
\usepackage{tkz-euclide}

% shadowing mdframed
\usepackage[framemethod=tikz]{mdframed}
\usetikzlibrary{shadows}
% Bold math
\usepackage{bm}

%Use Korean Letter when enumerate
\usepackage{dhucs-enumerate}

\usepackage{anyfontsize}

% Bra Ket (Dirac) Notation
\usepackage{braket}

% Slashed characters (e.g. in Dirac equation)
\usepackage{slashed}
\usepackage{pifont} % 원문자 사용시 필요한 패키지

% chapter decoration
\usepackage{type1cm}
\usepackage[explicit]{titlesec}

\titleformat{\chapter}[display]
{\normalfont\Large\rmfamily}
{\sffamily\flushright\fontsize{60}{0}\textbf{\textcolor{blue!40}{{\Huge\chaptername}~\thechapter\vskip0pt\rule{\textwidth}{2pt}}}}{0pt}
{\flushleft\fontsize{30}{0}{#1}\vskip60pt}
\titlespacing*{\chapter}
{0pt}{-40pt}{0pt}

%\usetikzlibrary{shadows}
\usetikzlibrary{shadows.blur}
\usetikzlibrary{shapes.symbols}

% Tcolorbox
\usepackage[most]{tcolorbox}

% Clean SI Units
\usepackage{siunitx}

% Enumerate thingies
\usepackage{enumitem}

% Cancel things out in equations
\usepackage[makeroom]{cancel}

\usepackage{multicol}

% Graphics and figures
\usepackage{graphicx}
\usepackage{wrapfig}
\usepackage{float}

\usepackage{cancel}

% Caption figures and tables
\usepackage{caption,subcaption}

% Generate symbols
\usepackage{textcomp} % Include this line to avoid output errors
\usepackage{gensymb}

% Make multiple rows in a table
\usepackage{multirow}

% Booktabs tables
\usepackage{booktabs}

%\usepackage[utopia,sfscaled]{mathdesign}
% Useful frames
\usepackage{mdframed}

% Comment-out large sections
\usepackage{comment}

% No auto-indent
\setlength{\parindent}{0pt}

% Asymptote - 3D vector graphics
\usepackage{asymptote}

% Tikz Package Stuff
\usepackage{pgf,tikz,pgfplots}
\usepackage{tikz-3dplot}
\usepackage{tabularx}
\usepackage{array}
\usepackage{colortbl}
\tcbuselibrary{skins}
\usepackage{tkz-euclide}

\newcolumntype{Y}{>{\raggedleft\arraybackslash}X}

\tcbset{tab1/.style={fonttitle=\bfseries\large,fontupper=\normalsize\sffamily,
		colback=yellow!10!white,colframe=red!75!black,colbacktitle=Salmon!40!white, halign=center,
		coltitle=black,center title,freelance,frame code={
			\foreach \n in {north east,north west,south east,south west}
			{\path [fill=red!75!black] (interior.\n) circle (3mm); };},}}

\tcbset{tab2/.style={enhanced,fonttitle=\bfseries,fontupper=\normalsize\sffamily, halign=center, box align=center,
		colback=yellow!10!white,colframe=red!50!black,colbacktitle=Salmon!40!white,
		coltitle=black,center title}}


% Use various tikz libraries
\usetikzlibrary{decorations.pathmorphing, decorations.markings, decorations.pathreplacing, patterns} % Decorate paths!
\usetikzlibrary{calc, patterns, shapes.geometric, positioning, through, intersections}
\usetikzlibrary{scopes}
\usetikzlibrary{angles, quotes}
\usetikzlibrary{svg.path}
\usetikzlibrary{arrows, arrows.meta}
\usetikzlibrary{fadings}
% pgfplots package settings
\pgfplotsset{compat=1.15}
% \pgfplotsset{width=10cm,compat=1.9} % Taken from latest overleaf.
% plot arc easily
\def\centerarc[#1](#2)(#3:#4:#5)% Syntax: [draw options] (center) (initial angle:final angle:radius)
{ \draw[#1] ($(#2)+({#5*cos(#3)},{#5*sin(#3)})$) arc (#3:#4:#5); }

% Awesome circled numbers
\newcommand*\circled[4]{\tikz[baseline=(char.base)]{\node[shape=circle, fill=#2, draw=#3, text=#4, inner sep=2pt] (char) {#1};}}

% Control size of text
\usepackage{relsize}

% Extend conditional commands
\usepackage{xifthen}
\usepackage{xcolor}
\definecolor{termcolor}{cmyk}{.21,.97,.0,.0}
\definecolor{darkred}{cmyk}{.27,1,1,.32}
\definecolor{darkblue}{cmyk}{1,.98,.10,.11}
\definecolor{darkgreen}{cmyk}{.29,0,87,0}
\definecolor{darkmycolor}{cmyk}{99,59,22,3}
\definecolor{for_eyes}{RGB}{253,247,228}
%change color of math equation
%\everymath{\color{darkred}}
% Scale math by size
\newcommand*{\Scale}[2][4]{\scalebox{#1}{\ensuremath{#2}}}

% Big integrals
\usepackage{bigints}

% Number equations within sections
\numberwithin{equation}{section}

% Generate blind text
\usepackage{blindtext}

% Useful symbols
\usepackage{marvosym}

\newcounter{problem}[section]
\newcounter{example}[section]

% cancel 색상 변경
\newcommand\Ccancel[2][black]{\renewcommand\CancelColor{\color{#1}}\cancel{#2}}

%%%% 원문자
\newcommand*\ocircled[1]{\tikz[baseline=(char.base)]{
		\node[shape=circle,draw,inner sep=2pt] (char) {#1};}}
	
%%%%% 보기 스타일 %%%%%
\usepackage{tabu}
\newcommand{\questwo}[2]{
	\vskip 6pt
	\noindent\begin{tabu}{X[0.2] X[6] X[0.2] X[6]}
		(1)&$#1$ &(2) &$#2$
	\end{tabu}
}
\newcommand{\questhree}[3]{
	\vskip 3pt
	\noindent\begin{tabu}{X[0.2] X[6] X[0.2] X[6] X[0.2] X[6]}
		(1)&$#1$ &(2) &$#2$ &(3) & $#3$
	\end{tabu}
	\vskip 5pt
}
\newcommand{\quesfour}[4]{
	\vskip 6pt
	\noindent\begin{tabu}{X[0.2] X[6] X[0.2] X[6]}
		(1)&$#1$ &(2) &$#2$\\
		(3)&$#3$ &(4) &$#4$
	\end{tabu}
}
\newcommand{\quesfive}[5]{
	\vskip 6pt
	\noindent\begin{tabu}{X[0.2] X[3] X[0.2] X[3] X[0.2] X[3]}
		(1)&$#1$ &(2) &$#2$ &(3) &$#3$\\
		(4)&$#4$ &(5) &$#5$
	\end{tabu}
}

\newcommand{\oquesfive}[5]{
	\vskip 6pt
	\noindent\begin{tabu}{X[0.2] X[3] X[0.2] X[3] X[0.2] X[3] X[0.2] X[3] X[0.2] X[3]}
		\ding{172}&$#1$ &\ding{173} &$#2$ &\ding{174} &$#3$&	\ding{175}&$#4$ &\ding{176} &$#5$
	\end{tabu}
}
%%%% sample(보기) %%%%
\newenvironment{sample}{\vskip 10pt\noindent\begin{tikzpicture}[yshift=1.5pt]%
		\draw[rounded corners=1ex,overlay,draw=blue] (0pt,-4pt) rectangle (19pt,9pt);
		\node[rectangle,overlay,xshift=9.8pt,yshift=2pt,color=blue] {{\footnotesize\sffamily 보기}};\end{tikzpicture}\phantom{\footnotesize\sffamily보기...}}{\vskip 10pt}
%%%% THEOREMS %%%%
\newenvironment{theorem}[1][\hspace{-0.36em}]
{
	\begin{mdframed}[backgroundcolor=purple!5, align=center, userdefinedwidth=40em, linecolor=purple!30, linewidth=2pt, roundcorner=7pt, innertopmargin=10pt, shadow=true, shadowcolor=black!20, roundcorner=7pt, innerbottommargin=10pt, frametitle = {#1}]
	}
	{
	\end{mdframed}
}

%%%% LEMMAS %%%%
\newenvironment{lemma}[1][\hspace{-0.36em}]
{
	\begin{mdframed}[backgroundcolor=black!8, align=center, userdefinedwidth=40em, topline=false, bottomline = false, leftline = false, rightline = false, frametitle = {#1 보조정리}]
	}
	{
	\end{mdframed}
}

%%%% COROLLARY %%%%
\newenvironment{corollary}[1][\hspace{-0.36em}]
{
	\begin{mdframed}[backgroundcolor=black!8, align=center, userdefinedwidth=40em, topline=false, bottomline = false, leftline = false, rightline = false, frametitle = {#1 따름정리}]
	}
	{
	\end{mdframed}
}

%%%% DEFINITIONS %%%%
\newenvironment{definition}[1][\hspace{-0.36em}]
{
	\begin{mdframed}[backgroundcolor=cyan!14, align=center, userdefinedwidth=40em, linecolor=cyan!60, linewidth=2pt,roundcorner=7pt, innertopmargin=10pt, shadow=true, shadowcolor=black!20, roundcorner=7pt, innerbottommargin=10pt, frametitle = {정의 : #1}]
	}
	{
	\end{mdframed}
}

%%%% PROPOSITION %%%%
\newenvironment{proposition}
{
	\begin{mdframed}[backgroundcolor=black!4, align=center, userdefinedwidth=40em, topline=false, bottomline = false, leftline = false, rightline = false, frametitle = {Proposition}]
	}
	{
	\end{mdframed}
}
%%%% PROBLEM %%%%
\newenvironment{problem}{\refstepcounter{problem}
	\begin{mdframed}[linecolor=blue!35, linewidth=2pt, roundcorner=7pt, innertopmargin=10pt, shadow=true, shadowcolor=black!20, roundcorner=7pt, innerbottommargin=10pt, backgroundcolor=blue!5]
		\noindent 
		\noindent\begin{tikzpicture}[overlay,xshift=6pt,yshift=5pt]
			\draw[fill=violet!15,draw=violet!15] (0,0) circle (8pt);
			\draw[fill=violet!15,draw=violet!15] (9pt,0) circle (8pt);
			\node[rectangle,overlay,xshift=4pt] {\color{black}\sffamily\bfseries 문제};
		\end{tikzpicture}{\phantom{.........}\fontspec[Scale=1.1]{TeX Gyre Adventor}\color{darkblue} \theproblem}\hspace{5pt}}{
\end{mdframed}}
%%%% SOLUTION OF PROBLEM %%%%
\newenvironment{psolution}{\begin{description}\item[{\begin{tikzpicture}%
				\draw[rounded corners=1ex,overlay] (-3pt,-3pt) rectangle (20pt,9pt);\end{tikzpicture}\footnotesize\sffamily풀이}\hspace{6pt}]}{\end{description}}
			
%%%% EXAMPLE %%%%		
\newenvironment{example}{\refstepcounter{example}
	\begin{mdframed}[roundcorner=7pt,linecolor=termcolor,linewidth=2pt,innertopmargin=10pt, shadow=true, shadowcolor=black!20, innerbottommargin=10pt, backgroundcolor=termcolor!2]
		\noindent 
		\noindent\begin{tikzpicture}[overlay,xshift=6pt,yshift=5pt]
			\draw[fill=darkred!60,draw=darkred!60] (0,0) circle (7pt);
			\draw[fill=darkred!60,draw=darkred!60] (9pt,0) circle (7pt);
			\node[rectangle,overlay,xshift=4pt] {\color{white}\sffamily\bfseries 예제};
		\end{tikzpicture}{\phantom{.........}\fontspec[Scale=1.1]{TeX Gyre Adventor}\color{darkred} \theexample}\hspace{5pt}}{
\end{mdframed}}		
%%%%%%% SOLUTION OF EXAMPLE %%%%%%%%%
\newenvironment{solution}{\begin{description}\item[{\begin{tikzpicture}%
				\draw[rounded corners=1ex,overlay] (-3pt,-3pt) rectangle (20pt,9pt);\end{tikzpicture}\footnotesize\sffamily풀이}\hspace{6pt}]}{\end{description}}

% 보기 박스 정의 시작
\tcbuselibrary{breakable, skins}
\tcbset{enhanced}
\newtcolorbox{ChoiceBox}[1]{
		enhanced,
		before skip=2ex, after skip=2ex,
		boxrule=0.5pt, colframe=black, colback=white, arc=0.5ex,
		boxsep=0.5ex, top=1.5ex, bottom=1.5ex, left=0.5em, right=o0.5em,
		colbacktitle=white, coltitle=black,
		attach boxed title to top center={xshift=0cm, yshift=-1.5mm},
		boxed title style={size=minimal, enhanced, boxrule=0.25pt, colframe=white},
		breakable=false, title ={< #1 >}
}
% 보기 박스 정의 끝.
	
% Change end-of-proof symbol
\renewcommand\qedsymbol{$\blacksquare$}
%overline
\newcommand{\ovr}[1]{\overline{\textrm{#1}}}
% trigonometric function
\newcommand{\cosrm}[1]{\cos \textrm{#1}}
\newcommand{\sinrm}[1]{\sin \textrm{#1}}
\newcommand{\tanrm}[1]{\tan \textrm{#1}}
\newcommand{\cotrm}[1]{\cot \textrm{#1}}
\newcommand{\cscrm}[1]{\csc \textrm{#1}}
\newcommand{\secrm}[1]{\sec \textrm{#1}}
%%%% BLACKBOARD BOLD %%%%
\newcommand{\bbN}{\mathbb{N}} % Natural numbers
\newcommand{\bbZ}{\mathbb{Z}} % Zahlen
\newcommand{\bbQ}{\mathbb{Q}} % Rational numbers
\newcommand{\bbR}{\mathbb{R}} % Real numbers
\newcommand{\bbC}{\mathbb{C}} % Complex numbers
\DeclareSymbolFont{bbold}{U}{bbold}{m}{n} % Identity matrix
\DeclareSymbolFontAlphabet{\mathbbold}{bbold} % Identity matrix
\newcommand{\identitymatrix}{\mathbbold{1}} % Identity matrix

%%%% CODE LISTING %%%%
\usepackage{listings}
\definecolor{greencomments}{HTML}{00BA00}
\definecolor{graynumbers}{HTML}{4F4F4F}
\definecolor{purplestrings}{HTML}{AD00AA}
\definecolor{backgroundcolor}{HTML}{E8E8E8}


%%%% UNIT BASIS VECTORS %%%%
\newcommand{\ihat}{\bm{\hat{\imath}}} % Cartesian i hat (x-direction)
\newcommand{\jhat}{\bm{\hat{\jmath}}} % Cartesian j hat (y-direction)
\newcommand{\khat}{\bm{\hat{k}}} % Cartesian k hat (z-direction)
\newcommand{\rhat}{\bm{\hat{r}}} % Spherical r hat
\newcommand{\phihat}{\bm{\hat{\phi}}} % Spherical phi hat
\newcommand{\thetahat}{\bm{\hat{\theta}}} % Spherical theta hat
\newcommand{\nhat}{\bm{\hat{n}}} % Unit normal vector
\newcommand{\rhohat}{\bm{\hat{\rho}}} % Cylindrical rho hat
\newcommand{\zhat}{\bm{\hat{z}}} % Cylindrical z hat


%%%% COLORS: DEFINITIONS AND COMMANDS %%%%
% Miscellaneous
\definecolor{DARKBLUE}{HTML}{040080}
\definecolor{DARKBROWN}{HTML}{8B4513}
\definecolor{LIGHTBROWN}{HTML}{CD853F}
\definecolor{PINK}{HTML}{D147BD}
\definecolor{LIGHTPINK}{HTML}{DC75CD}
\definecolor{GREENSCREEN}{HTML}{00FF00}
\definecolor{ORANGE}{HTML}{FF862F}
\newcommand{\DARKBLUE}{\color{DARKBLUE}}
\newcommand{\DARKBROWN}{\color{DARKBROWN}}
\newcommand{\LIGHTBROWN}{\color{LIGHTBROWN}}
\newcommand{\PINK}{\color{PINK}}
\newcommand{\LIGHTPINK}{\color{LIGHTPINK}}
\newcommand{\GREENSCREEN}{\color{GREENSCREEN}}
\newcommand{\ORANGE}{\color{ORANGE}}
% Blue
\definecolor{BLUEE}{HTML}{1C758A}
\definecolor{BLUED}{HTML}{29ABCA}
\definecolor{BLUEC}{HTML}{58C4DD}
\definecolor{BLUEB}{HTML}{9CDCEB}
\definecolor{BLUEA}{HTML}{C7E9F1}
\definecolor{BLUE}{HTML}{0000FF}
\newcommand{\BLUEE}{\color{BLUEE}}
\newcommand{\BLUED}{\color{BLUED}}
\newcommand{\BLUEC}{\color{BLUEC}}
\newcommand{\BLUEB}{\color{BLUEB}}
\newcommand{\BLUEA}{\color{BLUEA}}
\newcommand{\BLUE}{\color{BLUE}}
% Teal
\definecolor{TEALE}{HTML}{49A88F}
\definecolor{TEALD}{HTML}{55C1A7}
\definecolor{TEALC}{HTML}{5CD0B3}
\definecolor{TEALB}{HTML}{76DDC0}
\definecolor{TEALA}{HTML}{ACEAD7}
\definecolor{TEAL}{HTML}{00FFFF}
\newcommand{\TEALE}{\color{TEALE}}
\newcommand{\TEALD}{\color{TEALD}}
\newcommand{\TEALC}{\color{TEALC}}
\newcommand{\TEALB}{\color{TEALB}}
\newcommand{\TEALA}{\color{TEALA}}
\newcommand{\TEAL}{\color{TEAL}}
% Green
\definecolor{GREENE}{HTML}{699C52}
\definecolor{GREEND}{HTML}{77B05D}
\definecolor{GREENC}{HTML}{83C167}
\definecolor{GREENB}{HTML}{A6CF8C}
\definecolor{GREENA}{HTML}{C9E2AE}
\definecolor{GREEN}{HTML}{00FF00}
\newcommand{\GREENE}{\color{GREENE}}
\newcommand{\GREEND}{\color{GREEND}}
\newcommand{\GREENC}{\color{GREENC}}
\newcommand{\GREENB}{\color{GREENB}}
\newcommand{\GREENA}{\color{GREENA}}
\newcommand{\GREEN}{\color{GREEN}}
% Yellow
\definecolor{YELLOWE}{HTML}{E8C11C}
\definecolor{YELLOWD}{HTML}{F4D345}
\definecolor{YELLOWC}{HTML}{FFFF00}
\definecolor{YELLOWB}{HTML}{FFEA94}
\definecolor{YELLOWA}{HTML}{FFF1B6}
\definecolor{YELLOW}{HTML}{FFFF00}
\newcommand{\YELLOWE}{\color{YELLOWE}}
\newcommand{\YELLOWD}{\color{YELLOWD}}
\newcommand{\YELLOWC}{\color{YELLOWC}}
\newcommand{\YELLOWB}{\color{YELLOWB}}
\newcommand{\YELLOWA}{\color{YELLOWA}}
\newcommand{\YELLOW}{\color{YELLOW}}
% Gold
\definecolor{GOLDE}{HTML}{C78D46}
\definecolor{GOLDD}{HTML}{E1A158}
\definecolor{GOLDC}{HTML}{F0AC5F}
\definecolor{GOLDB}{HTML}{F9B775}
\definecolor{GOLDA}{HTML}{F7C797}
\newcommand{\GOLDE}{\color{GOLDE}}
\newcommand{\GOLDD}{\color{GOLDD}}
\newcommand{\GOLDC}{\color{GOLDC}}
\newcommand{\GOLDB}{\color{GOLDB}}
\newcommand{\GOLDA}{\color{GOLDA}}
% Red
\definecolor{REDE}{HTML}{CF5044}
\definecolor{REDD}{HTML}{E65A4C}
\definecolor{REDC}{HTML}{FC6255}
\definecolor{REDB}{HTML}{FF8080}
\definecolor{REDA}{HTML}{F7A1A3}
\definecolor{RED}{HTML}{FF0000}
\newcommand{\REDE}{\color{REDE}}
\newcommand{\REDD}{\color{REDD}}
\newcommand{\REDC}{\color{REDC}}
\newcommand{\REDB}{\color{REDB}}
\newcommand{\REDA}{\color{REDA}}
\newcommand{\RED}{\color{RED}}
% Maroon
\definecolor{MAROONE}{HTML}{94424F}
\definecolor{MAROOND}{HTML}{A24D61}
\definecolor{MAROONC}{HTML}{C55F73}
\definecolor{MAROONB}{HTML}{EC92AB}
\definecolor{MAROONA}{HTML}{ECABC1}
\newcommand{\MAROONE}{\color{MAROONE}}
\newcommand{\MAROOND}{\color{MAROOND}}
\newcommand{\MAROONC}{\color{MAROONC}}
\newcommand{\MAROONB}{\color{MAROONB}}
\newcommand{\MAROONA}{\color{MAROONA}}
% Purple
\definecolor{PURPLEE}{HTML}{644172}
\definecolor{PURPLED}{HTML}{715582}
\definecolor{PURPLEC}{HTML}{9A72AC}
\definecolor{PURPLEB}{HTML}{B189C6}
\definecolor{PURPLEA}{HTML}{CAA3E8}
\definecolor{PURPLE}{HTML}{FF00FF}
\newcommand{\PURPLEE}{\color{PURPLEE}}
\newcommand{\PURPLED}{\color{PURPLED}}
\newcommand{\PURPLEC}{\color{PURPLEC}}
\newcommand{\PURPLEB}{\color{PURPLEB}}
\newcommand{\PURPLEA}{\color{PURPLEA}}
\newcommand{\PURPLE}{\color{PURPLE}}
% White and Black
\definecolor{WHITE}{HTML}{FFFFFF}
\newcommand{\WHITE}{\color{WHITE}}
\definecolor{BLACK}{HTML}{000000}
\newcommand{\BLACK}{\color{BLACK}}
% Different Grays
\definecolor{LIGHTGRAY}{HTML}{BBBBBB}
\definecolor{GRAY}{HTML}{888888}
\definecolor{DARKGRAY}{HTML}{444444}
\definecolor{DARKERGRAY}{HTML}{222222}
\definecolor{GRAYBROWN}{HTML}{736357}
\newcommand{\LIGHTGRAY}{\color{LIGHTGRAY}}
\newcommand{\GRAY}{\color{GRAY}}
\newcommand{\DARKGRAY}{\color{DARKGRAY}}
\newcommand{\DARKERGRAY}{\color{DARKERGRAY}}
\newcommand{\GRAYBROWN}{\color{GRAYBROWN}}

 % 작업하는 파일의 상위 디렉토리에 둬야 하는 파일.
\usetikzlibrary{intersections,decorations.text}
\definecolor{c1}{RGB}{62, 97, 127}
\definecolor{c2}{RGB}{104, 182, 182}
\definecolor{c3}{RGB}{107, 190, 190}
\definecolor{c4}{RGB}{100, 172, 174}
\title{함수의 극한 문제해결 전략}
\author{유익승}
\date{\today}
\newcommand{\plogo}{\fbox{\color{c4}$\mathcal{JBSH}$}}
\usepackage{fancyhdr,lastpage}
\pagestyle{fancy}
\fancyhf{}
\lhead{\color{c4}함수의 극한 문제해결 전략}
\rhead{\plogo}
\lfoot{\color{c1}\texttt{전북과학고등학교}}
\rfoot{\color{c1}\pageref{LastPage}페이지 중 \thepage 페이지}
\pagestyle{empty}
\begin{document}
	\pagecolor{for_eyes}
	\begin{tikzpicture}[overlay,remember picture]
		
		% Background color
		\fill[
		black!2]
		(current page.south west) rectangle (current page.north east);
		
		% Rectangles
		\shade[
		left color=Dandelion, 
		right color=Dandelion!40,
		transform canvas ={rotate around ={45:($(current page.north west)+(0,-6)$)}}] 
		($(current page.north west)+(0,-6)$) rectangle ++(9,1.5);
		
		\shade[
		left color=lightgray,
		right color=lightgray!50,
		rounded corners=0.75cm,
		transform canvas ={rotate around ={45:($(current page.north west)+(.5,-10)$)}}]
		($(current page.north west)+(0.5,-10)$) rectangle ++(15,1.5);
		
		\shade[
		left color=lightgray,
		rounded corners=0.3cm,
		transform canvas ={rotate around ={45:($(current page.north west)+(.5,-10)$)}}] ($(current page.north west)+(1.5,-9.55)$) rectangle ++(7,.6);
		
		\shade[
		left color=orange!80,
		right color=orange!60,
		rounded corners=0.4cm,
		transform canvas ={rotate around ={45:($(current page.north)+(-1.5,-3)$)}}]
		($(current page.north)+(-1.5,-3)$) rectangle ++(9,0.8);
		
		\shade[
		left color=red!80,
		right color=red!80,
		rounded corners=0.9cm,
		transform canvas ={rotate around ={45:($(current page.north)+(-3,-8)$)}}] ($(current page.north)+(-3,-8)$) rectangle ++(15,1.8);
		
		\shade[
		left color=orange,
		right color=Dandelion,
		rounded corners=0.9cm,
		transform canvas ={rotate around ={45:($(current page.north west)+(4,-15.5)$)}}]
		($(current page.north west)+(4,-15.5)$) rectangle ++(30,1.8);
		
		\shade[
		left color=RoyalBlue,
		right color=Emerald,
		rounded corners=0.75cm,
		transform canvas ={rotate around ={45:($(current page.north west)+(13,-10)$)}}]
		($(current page.north west)+(13,-10)$) rectangle ++(15,1.5);
		
		\shade[
		left color=lightgray,
		rounded corners=0.3cm,
		transform canvas ={rotate around ={45:($(current page.north west)+(18,-8)$)}}]
		($(current page.north west)+(18,-8)$) rectangle ++(15,0.6);
		
		\shade[
		left color=lightgray,
		rounded corners=0.4cm,
		transform canvas ={rotate around ={45:($(current page.north west)+(19,-5.65)$)}}]
		($(current page.north west)+(19,-5.65)$) rectangle ++(15,0.8);
		
		\shade[
		left color=OrangeRed,
		right color=red!80,
		rounded corners=0.6cm,
		transform canvas ={rotate around ={45:($(current page.north west)+(20,-9)$)}}] 
		($(current page.north west)+(20,-9)$) rectangle ++(14,1.2);
		
		% Year
		\draw[ultra thick,gray]
		($(current page.center)+(5,2)$) -- ++(0,-3cm) 
		node[
		midway,
		left=0.25cm,
		text width=5cm,
		align=right,
		black!75
		]
		{
			{\fontsize{25}{30} \selectfont \bf 심화수학I \\[10pt] 보충자료}
		} 
		node[
		midway,
		right=0.25cm,
		text width=6cm,
		align=left,
		orange]
		{
			{\fontsize{72}{86.4} \selectfont 2021}
		};
		
		% Title
		\node[align=center] at ($(current page.center)+(0,-5)$) 
		{
			{\fontsize{30}{35} \selectfont {{\bf 함수의 극한문제 해결 전략 몇 가지}}} \\[1cm]
			{\fontsize{16}{19.2} \selectfont \textcolor{orange}{ \bf 유익승}}\\[1cm]
			{\fontsize{20}{25} \selectfont \textcolor{blue}{ \bf 전북과학고등학교}}};
	\end{tikzpicture}
	
\setstretch{1.4}
\tikzset{every shadow/.style={opacity=1}}	
%	\maketitle
\newpage

	\chapter{\Huge 삼각함수/지수 로그함수의 극한}
	\pagestyle{fancy}
	 \pagenumbering{arabic}
	 \setcounter{page}{1} 
	 
삼각함수, 지수-로그함수의 극한을 한없이 가까워져 가는 값의 관점이 아니라 한없이 가까워져 가는 모양의 관점에서 보면 극한 문제를 쉽게 해결할 수 있을 때가 많다.  이러한 관점에서 삼각함수, 지수-로그함수의 극한 문제를 해결하고 이를 일반적인 함수에 적용하여 일반적인 함수의 극한 문제를 해결할 수 있는 전략으로 발전시켜 보자.

삼각함수와 지수, 로그 함수의 극한값의 계산은 극한의 의미를 정확하게 이해하면 간단히 해결할 수 있다. 두 함수 $f(x),\:g(x)$에 대하여 $\displaystyle\lim_{x\to a}\tfrac{f(x)}{g(x)}=k$가 의미하는 것은 $x$가 $a$에 한없이 가까워져 가면 $f(x)$는 $kg(x)$에 한없이 가까워져 간다는 것을 의미한다. 즉 $x\to a$일 때, $f(x)\to kg(x)$이다. 이것을 이용하면 삼각함수와 지수, 로그 함수의 극한의 계산을 매우 쉽게 할 수 있다.
\vskip 10pt
\section{삼각함수 극한의 계산}
극한 $\displaystyle\lim_{\theta\to 0}\tfrac{\sin\theta}{\theta}=1$, $\displaystyle\lim_{\theta\to0}\tfrac{\tan\theta}{\theta}=1$에서 각각
\[
\displaystyle\lim_{\theta\to 0}\frac{\sin\theta}{\theta}=\displaystyle\lim_{\theta\to 0}\frac{\sin\theta -\sin 0}{\theta -0}= 1,
\]
\[\displaystyle\lim_{\theta\to 0}\frac{\tan\theta}{\theta}=\displaystyle\lim_{\theta\to 0}\frac{\tan\theta -\tan 0}{\theta -0}= 1
\]
이므로 $y =\sin x$와 $y =\tan x$는 $x=0$에서 미분계수가 $1$임을 의미하고 따라서 $x\to 0$일 때, $\sin x\to x$, $\tan x\to x$임을 의미한다.

더 일반적으로 $x\to x_{0}$일 때, $f(x)\to 0$이면 $t =f(x)$라 하면 각각
\[\displaystyle\lim_{x\to x_{0}}\frac{\sin f(x)}{f(x)}=\displaystyle\lim_{t\to 0}\frac{\sin t}{t}=1,
\]
\[\displaystyle\lim_{x\to x_{0}}\frac{\tan f(x)}{f(x)}=\displaystyle\lim_{t\to 0}\frac{\tan t}{t}=1
\]
이므로 $x\to 0$일 때, $f(x)\to 0$이면 $\sin f(x)\to f(x)$, $\tan f(x)\to f(x)$임을 의미한다.

마찬가지로 
\[\displaystyle\lim_{\theta\to 0}\frac{1-\cos\theta}{\theta^{2}}=\frac{1}{2}
\]
이므로 $x\to 0$일 때, $1-\cos x\to\frac{1}{2}x^{2}$임을 의미하고 일반적으로 $x\to x_{0}$일 때, $f(x)\to 0$이면 $1-\cos f(x)\to \frac{1}{2}(f(x))^{2}$임을 의미한다.

이제 위의 사실을 이용하여 삼각함수의 극한 문제들을 풀어보자.
\vskip 10pt
\begin{example}
다음 극한값을 구하여라.
\begin{enumerate}
	\item $\displaystyle\lim_{x\to0}\frac{\sin 2x}{3x}$ 
	\item $\displaystyle\lim_{x\to0}\frac{\sin 3x}{\tan 2x}$ 
	\item $\displaystyle\lim_{x\to0}\frac{\sin(\sin 2x)}{\sin 3x}$
\end{enumerate}
\begin{solution}
	\begin{enumerate}
		\item $\displaystyle\lim_{x\to0}\frac{\sin 2x}{3x}=\displaystyle\lim_{x\to0}\frac{2x}{3x}=\frac{2}{3}$
		
		\item $\displaystyle\lim_{x\to0}\frac{\sin 3x}{\tan 2x}=\displaystyle\lim_{x\to0}\frac{3x}{2x}=\frac{3}{2}$
		
		\item $\displaystyle\lim_{x\to 0}\frac{\sin(\sin 2x)}{\sin 3x}=\displaystyle\lim_{x\to 0}\frac{\sin 2x}{3x}=\displaystyle\lim_{x\to 0}\frac{2x}{3x}=\frac{2}{3}$
	\end{enumerate}
\end{solution}
\end{example}

\vskip 10pt

\begin{example}
$\displaystyle\lim_{x\to0}\frac{\cos x -\cos 3x}{x\sin x}$의 극한값을 구하시오.
\begin{solution}
	\begin{align*}
		\displaystyle\lim_{x\to0}\frac{\cos x -\cos 3x}{x\sin x}
		&=\displaystyle\lim_{x\to0}\frac{(\cos x -1)+(1-\cos 3x)}{x^{2}}\\
		&=\displaystyle\lim_{x\to0}\frac{-\frac{1}{2}x^{2}+\frac{1}{2}(3x)^{2}}{x^{2}}= 4
	\end{align*}
\end{solution}
\end{example}
\vskip 10pt

\begin{problem} $\displaystyle\lim_{x\to 0}\frac{x^{3}}{\tan x -\sin  x}$의 값은?
\processifversion{psol}{
	\begin{psolution}
		\begin{align*}
			\displaystyle\lim_{x\to 0}\frac{x^{3}}{\tan x-\sin x}& =\displaystyle\lim_{x\to 0}\frac{x^{3}}{\tan x(1-\cos x)}\\
			& =\displaystyle\lim_{x\to0}\frac{x^{3}}{x\times\frac{1}{2}x^{2}}=2
		\end{align*}
\end{psolution}}
\end{problem}
\vskip 10pt
\begin{problem}
연속함수 $f(x)$에 대하여 $\displaystyle\lim_{x\to0}\tfrac{f(x)}{x-2\sin x}=4$일 때, $\displaystyle\lim_{x\to0}\tfrac{f(x)}{x+2\sin x}$의 값은?
\processifversion{psol}{
	\begin{psolution}
		$\displaystyle\lim_{x\to 0}\tfrac{f(x)}{x-2\sin x}=\displaystyle\lim_{x\to0}\tfrac{f(x)}{x-2x}=\displaystyle\lim_{x\to0}\tfrac{f(x)}{-x}=4$에서 $x\to0$이면 $f(x)\to -4x$임을 알 수 있다. 따라서 $\displaystyle\lim_{x\to0}\tfrac{f(x)}{x+2\sin x}=\displaystyle\lim_{x\to0}\tfrac{-4x}{3x}= -\frac{4}{3}$임을 알 수 있다.
\end{psolution}}
\end{problem}
\vskip 10pt
\begin{problem}
$\displaystyle\lim_{x\to 0}\frac{\cos 3x -\cos 2x}{x^{2}}$의 값을 구하시오.
\processifversion{psol}{
	\begin{psolution}
		\begin{align*}
			\displaystyle\lim_{x\to 0}\frac{\cos 3x -\cos 2x}{x^{2}}
			& =\displaystyle\lim_{x\to 0}\frac{(1-\cos 2x)-(1-\cos 3x)}{x^{2}}\\
			& =\displaystyle\lim_{x\to 0}\frac{\frac{1}{2}(2x)^{2}-\frac{1}{2}(3x)^{2}}{x^{2}}= -\frac{5}{2}
		\end{align*}
		이다.
\end{psolution}}
\end{problem}
\vskip 10pt
\begin{problem}\textbf{(2009 6월평가원)} $f(x)$가 $\displaystyle\lim_{x\to 0}\tfrac{f(x)}{1-\cos\left(x^{2}\right)}=2$를 만족시킬 때,
$\displaystyle\lim_{x\to 0}\tfrac{f(x)}{x^{p}}=q$이다. $p+q$의 값은?(단, $p>0$, $q>0$이다.)
\processifversion{psol}{
	\begin{psolution}
		$x\to 0$일 때, $1-\cos(x^{2})\to\frac{1}{2}x^{4}$이고 $\displaystyle\lim_{x\to 0}\tfrac{f(x)}{1-\cos\left(x^{2}\right)}=2$이므로 $x\to 0$일 때, $f(x)\to x^{4}$이다. 따라서 $p=4$일 때만 $\displaystyle\lim_{x\to 0}\tfrac{f(x)}{x^{p}}$는 $0$이 아닌 값 $1$에 수렴한다. 그러므로 $p+q=5$이다.
\end{psolution}}
\end{problem}
\vskip 10pt
\begin{problem}\textbf{(2011 6월평가원)} $\displaystyle\lim_{x\to 0}\tfrac{e^{2x^{2}}-1}{\tan x\sin x}$의 값은?
\processifversion{psol}{
	\begin{psolution}
		$x\to 0$일 때, $e^{2x^{2}-1}\to 2x^{2}$이고 $\tan x\sin x\to x^{2}$이므로 구하는 극한값은 $2$이다.
\end{psolution}}
\end{problem}
\vskip 10pt

\begin{problem}
다음 극한값을 구하시오.
\begin{enumerate}
	\item $\displaystyle\lim_{x\to 0}\frac{\tan(\sin 4x)}{\sin(\tan 2x)}$       
	\item $\displaystyle\lim_{x\to 0}\frac{\sin 9x}{\sin x +\sin 3x +\sin 5x}$
\end{enumerate}
\processifversion{psol}{
	\begin{psolution}
		\begin{enumerate}
			\item $x\to 0$이면 $\sin 4x\to 4x$, $\tan 2x\to 2x$이므로 
			\[
			\displaystyle\lim_{x\to 0}\frac{\tan(\sin 4x)}{\sin(\tan 2x)}=\displaystyle\lim_{x\to}\frac{\tan 4x}{\sin 2x}=2
			\]
			이다.
			\item $x\to 0$일 때, $\sin ax\to ax$이므로 
			\[
			\displaystyle\lim_{x\to 0}\frac{\sin 9x}{\sin x +\sin 3x +\sin 5x}=\displaystyle\lim_{x\to 0}\frac{9x}{x+3x+5x}=1
			\]
			이다.
		\end{enumerate}
\end{psolution}}
\end{problem}
\vskip 10pt
앞에서 논의한 것을 활용하면 위의 예제와 문제들에서와 같이 삼각함수의 극한 문제는 간단한 다항식의 극한 문제가 된다.

\vskip 10pt
\section{지수-로그함수 극한의 계산}
지수함수와 로그함수의 극한 $\displaystyle\lim_{x\to 0}\frac{e^{x}-1}{x}=1$, $\displaystyle\lim_{x\to 0}\frac{\ln(1+x)}{x}=1$
에서 각각
\[
\displaystyle\lim_{tx\to 0}\frac{e^{x}-1}{x}=\displaystyle\lim_{x\to 0}\frac{e^{x}-e^{0}}{x-0}=1,
\] 
\[
\displaystyle\lim_{x\to 0}\frac{\ln(1+x)}{x}=\displaystyle\lim_{x\to 0}\frac{\ln(1+x)-\ln 1}{x-0}=1
\]
이므로 $y = e^{x}$와 $y =\ln(1+x)$는 $x=0$에서 미분계수가 $1$임을 의미하고 따라서 $x\to0$일 때, $e^{x}-1\to x$, $e^{x}\to 1+x$, $\ln(1+x)\to x$임을 의미한다.


더 일반적으로 $x\to x_{0}$일 때, $f(x)\to0$이면 $t =f(x)$라 하면 각각
\[
\displaystyle\lim_{x\to x_{0}}\frac{e^{f(x)}-1}{f(x)}=\displaystyle\lim_{t\to 0}\frac{e^{t}-1}{t}=1,
\] 
\[
\displaystyle\lim_{x\to x_{0}}\frac{\ln(1+f(x))}{f(x)}=\displaystyle\lim_{t\to 0}\frac{\ln(1+t)}{t}=1
\]
이므로 $x\to x_{0}$일 때, $f(x)\to 0$이면 
$e^{f(x)}-1\to f(x)$, $e^{f(x)}\to 1 +f(x)$, $\ln(1+f(x))\to f(x)$
임을 의미한다.

이제 위의 사실을 이용하여 지수함수와 로그함수의 극한 문제들을 몇 개 풀어보자.
\vskip 10pt
\begin{example}
다음 극한값을 구하여라.
\begin{enumerate}
	\item $\displaystyle\lim_{x\to 0}\frac{e^{2x}-1}{\sin x}$       
	\item $\displaystyle\lim_{x\to 0}\frac{\ln(1+2x)}{3x}$	
\end{enumerate}
\begin{solution}
	\begin{enumerate}
		\item $\displaystyle\lim_{x\to 0}\frac{e^{2x}-1}{\sin x}=\displaystyle\lim_{x\to 0}\frac{2x}{x}=2$
		\item $\displaystyle\lim_{x\to 0}\frac{\ln(1+2x)}{3x}=\displaystyle\lim_{x\to 0}\frac{2x}{3x}=\frac{2}{3}$
	\end{enumerate}
\end{solution}
\end{example}
\vskip 10pt

\begin{example}
$\displaystyle\lim_{x\to 0}\frac{1-\cos^{2}x}{x\ln(1+x)}$의 값은?
\begin{solution}
	\[\displaystyle\lim_{x\to 0}\frac{1-\cos^{2}x}{x\ln(1+x)}=\displaystyle\lim_{x\to 0}\frac{\sin^{2}x}{x^{2}}=1
	\]
\end{solution}
\end{example}
\vskip 10pt
\begin{problem}
\textbf{(2010학년도 6월 평가원)} $\displaystyle\lim_{x\to 0}\tfrac{e^{1-\sin x}- e^{1-\tan x}}{\tan x -\sin x}$의 값은?
\processifversion{psol}{
	\begin{psolution}
		\[
		\displaystyle\lim_{x\to 0}\frac{e^{1-\sin x}- e^{1-\tan x}}{\tan x -\sin x}=\displaystyle\lim_{x\to 0}e^{1-\tan x}\times\frac{e^{\tan x -\sin x}-1}{\tan x -\sin x}= e
		\]
		이다. 왜냐하면 $x\to 0$일 때, $\tan x -\sin x\to 0$이므로 $e^{\tan x -\sin x}-1\to \tan x -\sin x$이므로
		$\displaystyle\lim_{x\to 0}\tfrac{e^{\tan x-\sin x}-1}{\tan x-\sin x}=1$이고 따라서 
		\[
		\displaystyle\lim_{x\to 0}\frac{e^{1-\sin x}-e^{1-\tan x}}{\tan x-\sin x}=e
		\]
		이다.
\end{psolution}}
\end{problem}
\vskip 10pt
\begin{problem}
연속함수 $f(x)$가 $\displaystyle\lim_{x\to 0}\frac{e^{2x}-1}{f(x)}=10$을 만족할 때, $\displaystyle\lim_{x\to 0}\frac{f(x)}{x}$의 값은?
\processifversion{psol}{
	\begin{psolution}
		$x\to 0$일 때, $e^{2x}-1\to 2x$이므로 $\displaystyle\lim_{x\to 0}\frac{e^{2x}-1}{f(x)}=10$에서 $x\to 0$일 때, $f(x)\to\frac{1}{5}x$임을 알수 있다. 따라서 $\displaystyle\lim_{x\to 0}\frac{f(x)}{x}=\frac{1}{5}$이다.
\end{psolution}}
\end{problem}
\vskip 10pt

\begin{problem}
삼차함수 $f(x)$가 $f(-x)=-f(x)$를 만족시킬 때, $\displaystyle\lim_{x\to 0}\frac{f(\sin x)}{\sin f(x)}$의 값은?
\processifversion{psol}{
	\begin{psolution} 삼차함수가 $f(-x)= -f(x)$이면 기함수이고 원점을 지난다. 따라서 $x\to 0$일 때, $f(x)\to 0$이므로 $\sin x\to x$, $\sin f(x)\to f(x)$이므로 구하는 극한값은 $1$이다.
\end{psolution}}
\end{problem}
\vskip 10pt

이제 앞에서 언급한 극한에 대한 의미를 일반화 하면 $\displaystyle\lim_{x\to a}\tfrac{f(x)}{g(x)}=\alpha$라는 사실은
$x\to a$일 때, $f(x)\to\alpha g(x)$임을 의미함을 알 수 있다. 이제 이러한 관점에서 다음의 문제를 풀어보자.
\vskip 10pt
\begin{problem}
$\displaystyle\lim_{x\to\infty}\frac{\ln(2x+3)}{\ln(3x+5)}$의 값을 구하면?
\processifversion{psol}{
	\begin{psolution}
		$\displaystyle\lim_{x\to\infty}\tfrac{2x+3}{2x}=1$이므로 $x\to\infty$일 때, $2x+3\to2x$이고 마찬가지 방법으로 $3x+5\to 3x$이므로  
		\[
		\displaystyle\lim_{x\to\infty}\frac{\ln(2x+3)}{\ln(3x+5)}=\displaystyle\lim_{x\to\infty}\frac{\ln 2x}{\ln 3x}=\displaystyle\lim_{x\to\infty}\frac{\ln x+\ln 2}{\ln x+\ln 3}=1
		\]
		이다.
\end{psolution}}
\end{problem}
\vskip 10pt
\begin{problem}
\textbf{(2010학년도 평가원 6월)} 함수 $f(x)$에 대하여 옳은 것만을 <보기>에서 있는 대로 고르시오.
\begin{tcolorbox}[enhanced,attach boxed title to top center={yshift=-3mm,yshifttext=-1mm},
	colback=blue!10!white,colframe=blue!75!black,colbacktitle=black!40!white,
	title=< 보기 >, fonttitle=\bfseries,
	boxed title style={size=small,colframe=red!50!white} ]
	\begin{enumerate}[label={\jaso*.}]
		\item $f(x)=x^{2}$이면 $\displaystyle\lim_{x\to 0}\frac{e^{f(x)}-1}{x}=0$이다.
		
		\item $\displaystyle\lim_{x\to 0}\frac{e^{x}-1}{f(x)}=1$이면 $\displaystyle\lim_{x\to 0}\frac{3^{x}-1}{f(x)}=\ln 3$이다.
		
		\item $\displaystyle\lim_{x\to 0}f(x)=0$이면 $\displaystyle\lim_{x\to 0}\frac{e^{f(x)}-1}{x}$이 존재한다.
	\end{enumerate}
\end{tcolorbox}
\processifversion{psol}{
	\begin{psolution}
		ㄱ에서 $x\to 0$일 때, $e^{f(x)}-1\to x^{2}$이므로 성립한다. ㄴ에서 $x\to 0$일 때, $f(x)\to x$이므로 성립한다. ㄷ에서 $x\to 0$일 때, $f(x)\to 0$이므로 $e^{f(x)}-1\to f(x)$이다. 그런데 $f(x)= | x |$이면 극한값이 존재하지 않으므로 성립하지 않는다.
\end{psolution}}
\end{problem}
\vskip 10pt
\begin{problem}
$\displaystyle\lim_{n\to\infty}4^{n}\left(1 -\cos\frac{\theta}{2^{n}}\right)$의 값을 구하시오.
\processifversion{psol}{
	\begin{psolution} $\frac{1}{2^{n}}=t$로 치환하면 $t\to   0^+$이고 
		\[
		\displaystyle\lim_{n\to\infty}4^{n}\left(1 -\cos\frac{\theta}{2^{n}}\right)=\displaystyle\lim_{t\to  0^+}\frac{1-\cos t\theta}{t^{2}}=\frac{1}{2}\theta^{2}
		\]
		이다.
\end{psolution}}
\end{problem}
\vskip 10pt
\begin{problem}
함수 $f(x)$가 $\displaystyle\lim_{x\to 0}\frac{\ln(1+2x)}{f(x)}=2$를 만족시킬 때, 항상 옳은 것만을 <보기>에서 있는 대로 고르시오.
\begin{tcolorbox}[enhanced,attach boxed title to top center={yshift=-3mm,yshifttext=-1mm},
	colback=blue!10!white,colframe=blue!75!black,colbacktitle=black!40!white,
	title=< 보기 >, fonttitle=\bfseries,
	boxed title style={size=small,colframe=red!50!white}]
	\begin{enumerate}	
		\item $\displaystyle\lim_{x\to 0}\frac{xf(x)}{\tan 3x}=0$
		
		\item $\displaystyle\lim_{x\to 0}\frac{\ln(1+2x)-x}{f(x)}=1$
		
		\item $\displaystyle\lim_{x\to 0}\frac{x\ln(1+2x)}{\{f(x)\}^{2}}=2$
	\end{enumerate}
\end{tcolorbox}
\processifversion{psol}{
	\begin{psolution} $\displaystyle\lim_{x\to 0}\frac{\ln(1+2x)}{f(x)}=2$이므로 $x\to 0$일 때, $f(x)\to x$이다. 따라서 ㄱ, ㄴ, ㄷ이 모두 성립함을 쉽게 알 수 있다. 
\end{psolution}}
\end{problem}
\vskip 10pt
\begin{problem}
두 다항함수 $f(x)$, $g(x)$에 대하여 $\displaystyle\lim_{x\to 0}\frac{f(x)}{x}=\displaystyle\lim_{x\to 0}\frac{g(x)}{x^{2}}=1$일 때, 
\[
\displaystyle\lim_{x\to 0}\frac{4xf(x)+g(x)}{xf(x)-2g(x)}
\]
의 값을 구하시오.
\processifversion{psol}{
	\begin{psolution}
		$x\to 0$일 때, $f(x)\to x$, $g(x)\to x^{2}$이므로
		\[
		\displaystyle\lim_{x\to 0}\frac{4xf(x)+g(x)}{xf(x)-2g(x)}=\displaystyle\lim_{x\to 0}\frac{4x^{2}+x^{2}}{x^{2}-2x^{2}}= -5
		\]
		이다.
\end{psolution}}
\end{problem}
\vskip 10pt

지금까지 논의한 지수, 로그함수의 극한의 계산 방법은 극한의 계산에서 $x\to a$일 때, $f(x)$가 어떤 값에 한없이 가까워져가느냐에 관점을 두는 것이 아니라 $f(x)$가 어떤 직선 또는 곡선에 가까워져 가느냐에 관점을 두는 것이다. 즉, 접근해가는 값이 아니라 접근해 가는 {\color{red}‘모양’}에 관심을 두는 것이다. 이러한 관점에서 접근하면 어떤 함수의 극한의 문제들을 매우 쉽게 해결할 수 도 있고 함수의 극한값 계산 문제를 제작하는 데 있어서도 매우 유용하다. 이제 지금까지 논의한 것을 좀 더 일반화하는 방법을 생각하자.

	\chapter{\Huge 일반적인 함수의 극한 계산 전략}
		이 장에서는 먼저 미분을 사용하지 않는 방법인 극한 변수의 치환에 의한 방법과 미분을 활용하는 방법 3가지를 소개한다. 먼저 극한 변수의 치환에 의한 방법을 알아보자.
	
\section{극한 변수 치환법}
	 $x\to a$일 때 $y=f(x)$의 극한값을 구하는데 $x= 2s$로 치환하는 것은 양변을 시간 $t$에 대하여 미분하면 $\frac{dx}{dt}=2\frac{ds}{dt}$가 된다. 따라서 $x= 2s$로 치환하는 것은 $x$가 $a$에 접근해가는 속도를 바꾸는 행위로 볼 수 있다. 따라서 이러한 치환을 ‘극한변수 속도 조절하기’라고도 할 수 있을 것이다. 이러한 치환은 때때로 로피탈의 정리를 써야만 하는 상황에서 매우 재미있는 문제 풀이를 우리에게 제공한다. 특히 이 전략은 $x\to 0$ 또는 $x\to\infty$일 때 적용되는 사례가 많다. 이 방법의 지도에서 유의해야할 사항은 이미 주어진 극한이 수렴함을 전재하는 것이므로 정당화 수준에서 엄밀한 극한의 계산보다는 극한값을 구할 수 있다는 정당화 수준에서 가르쳐야 할 것이다. 몇 가지 예를 살펴본다.
	
	\vspace{1em}
	\begin{example}
	다음 극한값을 구하시오.
\begin{equation*}
	\displaystyle\lim_{x\to\infty}\frac{\ln x}{x}
\end{equation*}
\begin{solution}
	$x = 2t$로 치환하자. $x\to\infty$이면 $t\to\infty$이다. 구하는 극한값을 $L$이라 하면
\begin{align*}
L =\displaystyle\lim_{x\to\infty}\frac{\ln x}{x}
& =\displaystyle\lim_{t\to\infty}\frac{\ln 2t}{2t}\\
& =\displaystyle\lim_{t\to\infty}\frac{\ln t +\ln 2}{2t}\\
& =\displaystyle\lim_{t\to\infty}\left(\frac{1}{2}\frac{\ln t}{t}+\frac{\ln 2}{2t}\right)=\frac{1}{2}L +0
\end{align*}
즉 $L =\frac{1}{2}L$이므로 $L=0$임을 알 수 있다.
\end{solution}
\end{example}

\vspace{1em}
\begin{example}
	다음의 극한값을 구하시오.
	\begin{equation*}
		\displaystyle\lim_{x\to 0}\frac{x-\sin x}{x^{3}}
	\end{equation*}

\begin{solution}
	 $x = 2t$로 치환하자. $x\to 0$이면 $t\to 0$이다. 구하는 극한값을 $L$이라 하면
	\begin{align*}
		L & =\displaystyle\lim_{x\to 0}\frac{x-\sin x}{x^{3}}\\
		& =\displaystyle\lim_{t\to 0}\frac{2t -\sin 2t}{8t^{3}}\\
		& =\displaystyle\lim_{t\to 0}\frac{2t - 2\sin t\cos t}{8t^{3}}\\
		& =\displaystyle\lim_{t\to 0}\frac{t-\sin t\cos t}{4t^{3}}\\
		& =\displaystyle\lim_{t\to 0}\frac{t-\sin t +\sin t(1-\cos t)}{4t^{3}}\\
		& =\displaystyle\lim_{t\to 0}\left(\frac{1}{4}\cdot \frac{t-\sin t}{t^{3}}+\frac{1}{4}\left(\frac{\sin t}{t} \cdot \frac{1-\cos t}{t^{2}}\right)\right)\\
		& =\frac{1}{4}L +\frac{1}{8}
	\end{align*}
	이므로 $L =\frac{1}{6}$이다. 한편 $x = 3t$로 치환하고 3배각 공식을 이용할 수도 있다. 이 경우에는 $\displaystyle \displaystyle\lim_{\theta\to 0} \frac{\sin{\theta}}{\theta}=1$이 활용된다.
		\begin{align*}
		L & =\displaystyle\lim_{x\to 0}\frac{x-\sin x}{x^{3}}\\
		& =\displaystyle\lim_{t\to 0}\frac{3t -\sin 3t}{27t^{3}}\\
		& =\displaystyle\lim_{t\to 0}\frac{3t - 3\sin t +4\sin^{3}t}{27t^{3}}\\
		& =\displaystyle\lim_{t\to 0}\left(\frac{1}{9}\frac{t-\sin t}{t^{3}}+\frac{4}{27}\left(\frac{\sin t}{t}\right)^{3}\right)\\
		& =\frac{1}{9}L +\frac{4}{27}
	\end{align*}
	이므로 마찬가지로 $L =\frac{1}{6}$임을 알 수 있다.
\end{solution}
\end{example}

	이 방법을 적용하면 다음의 두 문제도 쉽게 해결 할 수 있다. 
	\vspace{1em}
	
\begin{problem}
다음이 성립함을 보여라.
\begin{equation*}
\displaystyle\lim_{x\to 0}\frac{\tan x - x}{x^{3}}=\frac{1}{3}
\end{equation*}
\end{problem}
\vspace{1em}
\begin{problem}
	다음이 성립함을 보여라.
	\begin{equation*}
	\displaystyle\lim_{x\to 0}\frac{\sinh x - x}{x^{3}}=\frac{1}{6}
	\end{equation*}
\end{problem}
\vspace{1em}
	이러한 문제들을 로피탈 정리를 사용하지 못하게 하고 문제를 해결하라고 학생들에게 제시하면 대부분 포기하게 될 것이다.  예를 하나 더 제시한다.

\begin{example}
	다음 극한값을 구하시오.
	\begin{equation*}
		\displaystyle\lim_{x\to 0}\frac{\cos x - 1 +\frac{1}{2}x^{2}}{x^{4}}
	\end{equation*}
	\begin{solution}
	$x = 2t$로 치환하자. $x\to 0$이면 $t\to 0$이다. 구하는 극한값을 $L$이라 하면
		\begin{align*}
		L & =\displaystyle\lim_{x\to 0}\frac{\cos x - 1 +\frac{1}{2}x^{2}}{x^{4}}\\
		& =\displaystyle\lim_{t\to 0}\frac{\cos 2t - 1 + 2t^{2}}{16 t^{4}}\\
		& =\displaystyle\lim_{t\to 0}\frac{2\cos^{2}t-2+2t^{2}}{16t^{4}}\\
		& =\displaystyle\lim_{t\to 0}\frac{\cos^{2}t -1 + t^{2}}{8t^{4}}\\
		& =\displaystyle\lim_{t\to 0}\frac{(\cos t-1)(\cos t +1) +t^{2}}{8t^{4}}\\
		& =\displaystyle\lim_{t\to 0}\frac{(\cos t -1 +\frac{1}{2}t^{2})(1+\cos t)+ t^{2}-\frac{1}{2}t^{2}-\frac{1}{2}t^{2}\cos t}{8t^{4}}\\
		& =\displaystyle\lim_{t\to 0}\left\{\frac{\cos t -1 +\frac{1}{2}t^{2}}{8t^{4}}\cdot (1+\cos t)+\frac{1}{2}\cdot\frac{1-\cos t}{8 t^{2}}\right\}\\
		& =\frac{1}{4}L +\frac{1}{32}
	\end{align*}
	이므로 $L =\frac{1}{24}$이다.
	\end{solution}
\end{example}

	지금까지 삼각함수가 포함된 극한을 계산하였다. 지수함수에서도 가능함을 살펴보자. 마지막으로 예 하나를 더 살펴보자.
	\begin{example}
		다음 극한값을 구하시오.
		\begin{equation*}
				\displaystyle\lim_{x\to 0}\frac{e^{2x}-2x-1}{x^{2}}
		\end{equation*}
		\begin{solution}
		 $x = 2t$로 치환하자. $x\to 0$이면 $t\to 0$이다. 구하는 극한값을 $L$이라 하면
		\begin{align*}
			L& =\displaystyle\lim_{x\to 0}\frac{e^{2x}-2x-1}{x^{2}}\\
			& =\displaystyle\lim_{t\to 0}\frac{e^{4t}-4t-1}{4t^{2}}\\
			& =\displaystyle\lim_{t\to 0}\frac{(e^{2t}-2t-1)(e^{2t}+1)+ 2t(e^{2t}-1)}{4t^{2}}\\
			& =\displaystyle\lim_{t\to 0}\left\{\frac{e^{2t}-2t-1}{4t^{2}}\left(e^{2t}+1\right)+\frac{1}{2}\frac{e^{2t}-1}{t}\right\}\\
			& =\frac{1}{2}L + 1
		\end{align*}
		이다. 따라서 $L = 2$이다. 
		\end{solution}
	\end{example}
	비슷한 방법으로 다음을 증명할 수 있다.
\vspace{1em}
\begin{problem}
	 다음이 성립함을 보여라.
	\begin{equation*}
	\displaystyle\lim_{x\to 0}\frac{e^{x}-1-x}{\cos x -1}=-1
	\end{equation*}
\end{problem}
\vspace{1em}
	다음의 예제는 앞의 예제들의 결과로부터 쉽게 얻을 수도 있고 지금까지의 방법으로 계산할 수 도 있다.
\vspace{1em}
\begin{example}
	 다음 극한값을 구하시오.
	\begin{equation*}
	\displaystyle\lim_{x\to 0}\frac{x-\sin x}{x-\tan x}= -\frac{1}{2}
	\end{equation*}
\begin{solution}
	 위의 결과들을 안다고 가정하면 다음과 같이 문제를 해결 할 수 있다.
\begin{equation*}
	\displaystyle\lim_{x\to 0}\frac{x-\sin x}{x-\tan x}=\displaystyle\lim_{x\to 0}\frac{\frac{x-\sin x}{x^{3}}}{\frac{x-\tan x}{x^{3}}}=\frac{\frac{1}{6}}{-\frac{1}{3}}= -\frac{1}{2}
\end{equation*}	

\end{solution}
\begin{solution}
 $x = 2t$로 치환하자. $x\to 0$이면 $t\to 0$이다. 구하는 극한값을 $L$이라 하면
	\begin{align*}
		L=\displaystyle\lim_{x\to 0}\frac{x-\sin x}{x-\tan x}
		& =\displaystyle\lim_{t\to 0}\frac{2t -\sin 2t}{2t -\tan 2t}\\
		& =\displaystyle\lim_{t\to 0}\frac{2t - 2\sin t\cos t}{2t -\frac{2\tan t}{1-\tan^{2}t}}\\
		& =\displaystyle\lim_{t\to 0}\frac{t(1-\tan^{2}t)-\sin t\cos t(1-\tan^{2}t)}{t(1-\tan^{2}t)-\tan t}\\
		& =\displaystyle\lim_{t\to 0}\frac{t-\sin t +\sin t(1-\cos t)- t\tan^{2}t +\cos t\sin t\tan^{2}t}{t -\tan t - t\tan^{2}t}\\
		& =\displaystyle\lim_{t\to 0}\frac{\frac{t-\sin t}{t -\tan t}+\frac{\sin t(1-\cos t)}{t-\tan t}-\frac{t\tan^{2}t}{t -\tan t}+\frac{\cos t\sin t\tan^{2}t}{t-\tan t}}{1 -\frac{t\tan^{2}t}{t -\tan t}}\\
		& =\frac{L +\frac{3}{2}+3 - 3}{1+3}
	\end{align*} 
	이다. 따라서 $L =-\frac{1}{2}$이다.
\end{solution}
\end{example}
\vspace{1em}
	
	한편 지금까지 언급한 전략을 극한 변수의 치환이라는 측면 보다 좀 더 일반화된 측면에서 보면 이 전략은 극한값의 계산 과정에서 구해야 할 극한값의 상수 배가 나타나게 하여 구해야 할 극한값에 대한 일차 방정식을 얻는 방법이라 할 수 있다. 
	
\vspace{1em}
\begin{example}\label{exam:rev_trans}
	다음 극한값을 구하시오.
	\begin{equation*}
	\displaystyle\lim_{x\to\infty}\left(x - x^{2}\ln\left(\frac{1+x}{x}\right)\right)
	\end{equation*}
\begin{solution}
	$x =\frac{1}{t}$로 치환하면 $t\to 0^{+}$이다. 따라서
	\begin{align*}
		\lim_{x\to\infty}\left(x - x^{2}\ln\left(\frac{1+x}{x}\right)\right)
		& =lim_{t\to 0^{+}}\frac{1-\frac{1}{t}\ln(1+t)}{t}\\
		& =\lim_{t\to 0^{+}}\frac{\frac{1}{t^{2}}\ln(1+t)-\frac{1}{t(1+t)}}{1}(\text{ by L'Hospital's Theorem})\\
		& =\lim_{t\to 0^{+}}\frac{\frac{1}{t}\ln(1+t)-\frac{1}{1+t}}{t}\\
		& =\lim_{t\to 0^{+}}\frac{\frac{1}{t}\ln(1+t)-1 + 1-\frac{1}{1+t}}{t}\\
		& =\lim_{t\to 0^{+}}\left(\frac{\frac{1}{t}\ln(1+t)-1}{t}+\frac{1}{1+t}\right)
	\end{align*}
	에서 구하는 극한값을 $L$이라 하면 $L = -L + 1$을 얻고 따라서 $L =\frac{1}{2}$이다.
\end{solution}
\end{example}
\vspace{1em}
	\begin{problem}
 다음 극한값을 구하시오.
\begin{equation*}
	\displaystyle\lim_{x\rightarrow 0}\frac{\pi\sin x -\sin\pi x}{x(\cos x -\cos\pi x)}
\end{equation*}	
\processifversion{psol}{
\begin{psolution}
	$\displaystyle L=\displaystyle\lim_{x\rightarrow 0}\frac{\pi\sin x -\sin\pi x}{x(\cos x -\cos\pi x)}$라 하고 $x = 2t$로 치환하면 $x \to 0$일 때 $t \to 0$이다. 이제
	\begin{align*}
\displaystyle\lim_{x\rightarrow 0}\frac{\pi\sin x -\sin\pi x}{x(\cos x -\cos\pi x)}
		& =\displaystyle\lim_{t\rightarrow 0}\frac{\pi\sin 2t -\sin 2\pi t}{2t(\cos 2t -\cos 2\pi t)}\\
		& =\frac{1}{2}\displaystyle\lim_{t\rightarrow 0}\frac{2\pi\sin t\cos t - 2\sin\pi t\cos\pi t}{t\left(2\cos^{2}t - 1 -\left(2\cos^{2}\pi t-1\right)\right)}\\
		& =\frac{1}{2}\displaystyle\lim_{t\rightarrow 0}\frac{2\pi\sin t\cos t - 2\sin\pi t\cos\pi t}{t\left(2\cos^{2}t - 2\cos^{2}\pi t\right)}\\
		& =\frac{1}{2}\displaystyle\lim_{t\rightarrow 0}\frac{\pi\sin t\cos t -\sin\pi t\cos\pi t}{t(\cos t -\cos\pi t)}\cdot\frac{1}{(\cos t +\cos\pi t)}\\
		& =\frac{1}{4}\displaystyle\lim_{t\rightarrow 0}\frac{\pi\sin t\cos t -\pi\sin t\cos\pi t +\pi\sin t\cos\pi t-\sin\pi t\cos\pi t}{t(\cos t -\cos\pi t)}\\
		& =\frac{1}{4}\displaystyle\lim_{t\rightarrow 0}\frac{\pi\sin t(\cos t -\cos\pi t)+\cos\pi t(\pi\sin t -\sin\pi t)}{t(\cos t -\cos\pi t)}\\
		& =\frac{1}{4}\left(\pi\displaystyle\lim_{t\rightarrow 0}\frac{\sin t}{t}+\displaystyle\lim_{t\rightarrow 0}\frac{\pi\sin t -\sin\pi t}{t(\cos t -\cos\pi t)}\right)\\
		& =\frac{\pi}{4}+\frac{1}{4}L
	\end{align*}
	이므로 $L =\frac{\pi}{3}$이다.
\end{psolution}}
\end{problem}
\vspace{1em}
	다음에 제시하는 문제는 급수전개로 해결하는 것은 가능하다. 급수 전개는 아주 나중에 학습하는 내용이므로 극한의 성질에 위배되지 않으면서 정교하게 문제를 풀어보자.
	\vspace{1em}
	
	\begin{example}
		 다음 극한값을 구하시오.
		 \begin{equation*}
		 	\displaystyle\lim_{n\to\infty}e^{2n}\left(1-\frac{2}{n}\right)^{n^{2}}
		 \end{equation*}
	 \begin{solution}
	먼저
	
	\begin{align*}
		\displaystyle\lim_{n\to\infty}e^{2n}\left(1-\frac{2}{n}\right)^{n^{2}}
		& =\displaystyle\lim_{n\to\infty}\left(e^{\frac{2}{n}}\left(1-\frac{2}{n}\right)\right)^{n^{2}}\\
		& =\displaystyle\lim_{n\to\infty}\left(1+\left(e^{\frac{2}{n}}\left(1-\frac{2}{n}\right)-1\right)\right)^{\frac{1}{\left(e^{\frac{2}{n}}\left(1-\frac{2}{n}\right)-1\right)}\cdot n^{2}\left(e^{\frac{2}{n}\left(1-\frac{2}{n}\right)-1}\right)}
	\end{align*}
	이다. 그런데
	\begin{equation*}
		\displaystyle\lim_{n\to\infty}\left(1+\left(e^{\frac{2}{n}}\left(1-\frac{2}{n}\right)-1\right)\right)^{\frac{1}{\left(e^{\frac{2}{n}}\left(1-\frac{2}{n}\right)-1\right)}}= e
	\end{equation*}
	이므로 $\displaystyle\lim_{n\to\infty}n^{2}\left(e^{\frac{2}{n}}\left(1-\frac{2}{n}\right)-1\right)$의 극한값만 구하면 된다. $t =\frac{1}{n}$으로 치환하면 $n\to\infty$이면 $t\to 0^{+}$이므로 
	\begin{align*}
		\displaystyle\lim_{n\to\infty}n^{2}\left(e^{\frac{2}{n}}\left(1-\frac{2}{n}\right)-1\right)
		& =\displaystyle\lim_{t\to 0^{+}}\frac{e^{2t}(1-2t)-1}{t^{2}}\\
		& =\displaystyle\lim_{t\to 0^{+}}\frac{e^{2t}-2te^{2t}-1}{t^{2}}\\
		& =\displaystyle\lim_{t\to 0^{+}}\frac{(e^{2t}-2t-1)- 2t(e^{2t}-1)}{t^{2}}\\
		& =\displaystyle\lim_{t\to 0^{+}}\left\{\frac{e^{2t}-2t-1}{t^{2}}-\frac{2\left(e^{2t}-1\right)}{t}\right\}\\
		& =\displaystyle\lim_{t\to 0^{+}}\frac{e^{2t}-2t-1}{t^{2}}- 2\displaystyle\lim_{t\to 0^{+}}\frac{e^{2t}-1}{t}\\
		& = 2 -4 =-2
	\end{align*}
	이다. 따라서 구하는 극한값은 $e^{-2}$이다.  	
	 \end{solution}
 \begin{solution}\textbf{(오류에 의한 풀이 1)}
 		\begin{align*}
 		\displaystyle\lim_{n\to\infty}e^{2n}\left(1-\frac{2}{n}\right)^{n^{2}}
 		& =\displaystyle\lim_{n\to\infty}\left(\left(1+\frac{2}{n}\right)^{\frac{n}{2}}\right)^{2n}\left(1-\frac{2}{n}\right)^{n^{2}}\\
 		& =\displaystyle\lim_{n\to\infty}\left(1+\frac{2}{n}\right)^{n^{2}}\left(1-\frac{2}{n}\right)^{n^{2}}\\
 		& =\displaystyle\lim_{n\to\infty}\left(1-\frac{4}{n^{2}}\right)^{n^{2}}\\
 		& =\displaystyle\lim_{n\to\infty}\left(1-\frac{4}{n^{2}}\right)^{\left(-\frac{n^{2}}{4}\right)(-4)}=e^{-4}
 	\end{align*}
 \end{solution}
\begin{solution}\textbf{(오류에 의한 풀이 2)}
	\begin{align*}
		\displaystyle\lim_{n\to\infty}e^{2n}\left(1-\frac{2}{n}\right)^{n^{2}}
		& =\displaystyle\lim_{n\to\infty}e^{2n}\left(1-\frac{2}{n}\right)^{n}\cdot\left(1-\frac{2}{n}\right)^{n}\cdots\left(1-\frac{2}{n}\right)^{n}\\
		& =\displaystyle\lim_{n\to\infty}e^{2n}\left\{\left(1-\frac{2}{n}\right)^{\left(-\frac{n}{2}\right)(-2)}\left(1-\frac{2}{n}\right)^{\left(-\frac{n}{2}\right)(-2)}\cdots\left(1-\frac{2}{n}\right)^{\left(-\frac{n}{2}\right)(-2)}\right\}\\
		& =\displaystyle\lim_{n\to\infty}e^{2n}\left(e^{-2}e^{-2}\cdots e^{-2}\right)\\
		& =\displaystyle\lim_{n\to\infty}e^{2n}\cdot e^{-2n}=1
	\end{align*}
\end{solution}
\end{example} 
	\begin{problem}
위의 두 오류에 의한 풀이의 오류를 설명하시오.		
	\end{problem}
\vspace{1em}
다음의 예제는 예제 \ref{exam:rev_trans}과 마찬가지로 $x =\frac{1}{t}$로 치환하는 것이다.  
\begin{example}
	 다음 극한값을 구하시오.
	\begin{equation*}
		\displaystyle\lim_{x\to \infty}x\left(\arctan x -\frac{\pi}{2}e^{\frac{1}{x}}\right)
	\end{equation*}
\begin{solution}
	$\displaystyle\lim_{x \to \infty}x\left(\arctan x -\frac{\pi}{2}e^{\frac{1}{x}}\right)$에서 $x =\frac{1}{t}$로 치환하자. 그러면 $x \to \infty$일 때, $t \to 0^{+}$이다. 따라서 $\arctan(t)+\arctan\left(\frac{1}{t}\right)=\frac{\pi}{2}$이므로 다음이 성립한다.
	\begin{align*}
		\displaystyle\lim_{x\rightarrow\infty}x\left(\arctan x -\frac{\pi}{2}e^{\frac{1}{x}}\right)
		& =\displaystyle\lim_{t\rightarrow 0^{+}}\frac{\arctan\left(\frac{1}{t}\right)-\frac{\pi}{2}e^{t}}{t}\\
		& =\displaystyle\lim_{t\rightarrow 0^{+}}\frac{\frac{\pi}{2}-\arctan(t)-\frac{\pi}{2}e^{t}}{t}\\
		& = -\displaystyle\lim_{t\rightarrow 0^{+}}\left(\frac{\arctan(t)}{t}+\frac{\pi}{2}\cdot\frac{e^{t}-1}{t}\right)\\
		& =-\left(1+\frac{\pi}{2}\right)
	\end{align*}
\end{solution}
\end{example}
\begin{problem}
다음 극한값을 구하시오.
	\begin{equation*}
		\displaystyle\lim_{x\rightarrow 0}\frac{1-\frac{1}{2}x^{2}-\cos\left(\frac{x}{1-x^{2}}\right)}{x^{4}}
	\end{equation*}
\end{problem}
\vspace{1em}
\begin{problem}
	다음 극한값을 구하시오.
	\begin{equation*}
	\displaystyle\lim_{x\rightarrow 0}\left(\frac{1}{\sin^{2}x}+\frac{1}{\tan^{2}x}-\frac{2}{x^{2}}\right)
	\end{equation*}
\end{problem}
\vspace{1em}
\begin{problem}
	다음 극한값을 구하시오.
	\begin{equation*}
		\lim\limits_{x \to 0}\frac{\frac{\sin x}{x}- \cos 2x}{x^2}
	\end{equation*}
\end{problem}
\vspace{1em}
\section{선형근사에 의한 극한의 계산}
선형근사에 의한 극한 문제의 해결은 계산해야 할 함수의 극한값이 매우 복잡한 형태일 때, $x=a$에서 접선의 방정식을 복잡한 함수를 대신하여 극한값의 계산에 이용하는 것이다. 몇 개의 예제를 통해서 살펴보자.

\vskip 10pt
\begin{example} $\displaystyle\lim_{x\to\infty}\left(\sqrt{x^{2}+2x}-x\right)+\displaystyle\lim_{x\to -\infty}\frac{1}{\sqrt{x^{2}-2x}+x}$의 값은?
\begin{solution}
	먼저 $\displaystyle\lim_{x\to\infty}\left(\sqrt{x^{2}+2x}-x\right)$를 계산하면
	\begin{align*}
		\displaystyle\lim_{x\to\infty}\left(\sqrt{x^{2}+2x}-x\right)
		& =\displaystyle\lim_{x\to\infty}\frac{x^{2}+2x -x^{2}}{\sqrt{x^{2}+2x}+x}\\
		& =\displaystyle\lim_{x\to\infty}\frac{2x}{\sqrt{x^{2}+2x}+x}=1
	\end{align*}
	이다. 즉 $\displaystyle\lim_{x\to\infty}\left(\sqrt{x^{2}+2x}-x\right)=1$이다. 이 극한값의 계산 결과를 분석해 보자. 이 극한값의 계산 결과는
	\[
	x\to\infty \Rightarrow \sqrt{x^{2}+2x}\to x+1
	\] 즉, 
	$y =\sqrt{x^{2}+2x}$의 점근선이 $y = x+1$임을 의미한다. 따라서 $x$가 한없이 커질 때, $\sqrt{x^{2}+2x}$는 $x+1$에 한없이 가까워져 간다. 이것은 $x\to\infty$일 때, $\sqrt{x^{2}+2x}$ 대신 $x+1$을 대신 써서 계산해도 됨을 의미한다. 이제 $\displaystyle\lim_{x\to -\infty}\frac{1}{\sqrt{x^{2}-2x}+x}$을 계산하기 위해 $t= -x$로 치환하면 
	\[
	\displaystyle\lim_{x\to -\infty}\frac{1}{\sqrt{x^{2}-2x}+x}=\displaystyle\lim_{t\to\infty}\frac{1}{\sqrt{t^{2}+2t}-t}
	\]
	이다. 그런데 $t\to\infty$일 때, $\sqrt{t^{2}+2t}\to t+1$이므로 
	\[
	\displaystyle\lim_{t\to\infty}\frac{1}{\sqrt{t^{2}+2t}-t}=\displaystyle\lim_{t\to\infty}\frac{1}{(t+1)-t}= 1
	\]
	임을 쉽게 알 수 있다. 
\end{solution}
\end{example}

\vskip 10pt
이것을 다음과 같이 일반화 할 수 있다. 즉 $a>0$일 때,
\begin{align*}
\displaystyle\lim_{x\to\infty}\left(\sqrt{ax^{2}+bx+c}-\sqrt{a}x\right)
& =\displaystyle\lim_{x\to\infty}\frac{bx+c}{\left(\sqrt{ax^{2}+bx+c}+\sqrt{a}x\right)}\\
& =\displaystyle\lim_{x\to\infty}\frac{b+\frac{c}{x}}{\sqrt{a +\frac{b}{x}+\frac{c}{x^{2}}}+\sqrt{a}}\\
&=\frac{b}{2\sqrt{a}}
\end{align*}
이므로 $x\to\infty$일 때, $\sqrt{ax^{2}+bx+c}\to\sqrt{a}x +\frac{b}{2\sqrt{a}}$임을 알 수 있고 이 사실을 극한값 계산에 활용할 수 있다. 이제 이 사실을 이용하여 극한값을 간단히 계산할 수 있음을 살펴보자.

\begin{example}
$\displaystyle\lim_{x\to\infty}\left(2x -\sqrt{4x^{2}-x}\right)$의 값은?
\begin{solution}
	$x\to\infty$일 때 $\sqrt{4x^{2}-x}\to 2x -\frac{1}{4}$이므로 구하는 극한값은 $\frac{1}{4}$이다.
\end{solution}
\end{example}
\vskip 10pt

\begin{problem}
다음 극한값을 구하여라.
\begin{enumerate}
	\item $\displaystyle\lim_{x\to\infty}\left(\sqrt{4x^{2}-3x}- 2x\right)$
	\item $\displaystyle\lim_{x\to\infty}\left(\sqrt{x^{2}+4x}-\sqrt{x^{2}}\right)$
\end{enumerate}
\processifversion{psol}{
	\begin{psolution}
		\begin{enumerate}
			\item 공식을 적용하면 $\displaystyle\lim_{x\to\infty}\left(\sqrt{4x^{2}-3x}- 2x\right)= -\frac{3}{2}$임을 알 수 있다.
			\item 마찬가지로 $\displaystyle\lim_{x\to\infty}\left(\sqrt{x^{2}+4x}-\sqrt{x^{2}}\right)=2$임을 알 수 있다.
		\end{enumerate}	
\end{psolution}}
\end{problem}
\vskip 10pt

\begin{problem}
\textbf{(평가원 기출)} 곡선 $y =\sqrt{x}$위의 점 $(t,\:\sqrt{t})$에서 점 $(1,\: 0)$까지의 거리를 $d_{1}$, 점 $(2,\: 0)$가지의 거리를 $d_{2}$라 할 때, $\displaystyle\lim_{t\to\infty}\left(d_{1}-d_{2}\right)$의 값을 구하시오.
\processifversion{psol}{
	\begin{psolution}
		\[
		d_1 =\sqrt{(t-1)^2 + \sqrt{t}^2} =\sqrt{t^2 - t +1} \rightarrow t-\frac{1}{2}
		\]이고
		\[	
		d_2 = \sqrt{(t-2)^2 + \sqrt{t}^2} =\sqrt{t^2 - 3t +4} \rightarrow t-\frac{3}{2}
		\]이므로 
		$\displaystyle\lim_{t\to\infty}\left(d_{1}-d_{2}\right)=1$이다.	
		
		한편 그림
		\begin{figure}[H]
			\begin{center}
				\begin{tikzpicture}[domain=0:3.5, samples=200, >=latex]
					% \draw[very thin,color=gray] (-2.6,-1.6) grid (2.4,1.6);
					\filldraw[blue] (1,0) circle (1.5pt) node [below] {$\textrm{A}$};
					\filldraw[blue] (2,0) circle (1.5pt) node [below] {$\textrm{B}$};
					\filldraw[blue] (3,1.7321) circle (1.5pt) node [above] {$\textrm{P}$};
					\draw[->] [line width=0.3mm] (-0.5,0) -- (3.5,0) node[right] {$x$};
					\draw[->] [line width=0.3mm] (0,-0.5) -- (0,1.9) node[above] {$y$};
					\draw[domain=0:3.5, color=blue,line width=0.3mm] plot (\x, {sqrt(\x)}) node [above] {$\sqrt{x}$};
					\node[below left] at (0, 0) {\small \text{O}} ;
					\draw [line width=0.3mm, color=blue] (1,0) -- (3, 1.7321);
					\draw [line width=0.3mm, color=blue] (2,0) -- (3, 1.732);
				\end{tikzpicture}
			\end{center}
		\end{figure}
		로부터 점 $\text{P}$가 원점으로부터 한없이 멀어지면 $d_1 -d_2 = \overline{\textrm{AB}}$임을 알 수 있다.	
\end{psolution}}
\end{problem}
\vskip 10pt


한편 $x\to\infty$인 경우가 아니더라도 함수의 극한의 의미를 해석하여 문제를 해결할 수 있다. 다음의 예제는 분자를 유리화하여 상당히 긴 대수적 연산을 통해 $a,\: b$를 구할 수 있다. 그러나 $x=0$에서 $\sqrt{x^{2}+x+1}$의 선형 근사를 통해 문제를 간단히 해결할 수 있다.
\vskip 10pt

\begin{example}
$\displaystyle\lim_{x\to 0}\frac{\sqrt{x^{2}+x+1}-(ax+b)}{x}=\frac{5}{2}$를 만족시키는 두 실수 $a,\: b$의 합 $a+b$의 값을 구하시오.
\begin{solution}
	먼저 $f(x)=\sqrt{x^{2}+x+1}$의 $x =0$에서의 미분계수를 구하자. 
	\[
	f^{\prime}(0)=\displaystyle\lim_{x\to 0}\frac{f(x)-f(0)}{x-0}=\displaystyle\lim_{x\to 0}\frac{\sqrt{x^{2}+x+1}-1}{x}=\frac{1}{2}
	\]
	이다. 이 미분계수의 결과를 해석하면 $x\to 0$일 때, 
	\[
	\sqrt{x^{2}+x+1}\to\frac{1}{2}x +1
	\]
	임을 의미한다. 
	\begin{align*}
		\displaystyle\lim_{x\to 0}\frac{\sqrt{x^{2}+x+1}-(ax+b)}{x}
		& =\displaystyle\lim_{x\to 0}\frac{\frac{1}{2}x+1-(ax+b)}{x}\\
		& =\displaystyle\lim_{x\to 0}\frac{\left(\frac{1}{2}-a\right)x+(1-b)}{x}\\
		&=\frac{5}{2}
	\end{align*}
	이다. 따라서 $\frac{1}{2}-a =\frac{5}{2}$, $1-b =0$이므로 $a = -2$, $b =1$을 쉽게 얻을 수 있다.
\end{solution}
\end{example}
\vskip 10pt

이 예제를 일반화 하자. $x\to a$일 때 어떤 함수의 극한을 계산하는데 대수적으로 다루기 어려운 함수 $f(x)$가 포함되어 있는 경우에
\[
f'(a)=\displaystyle\lim_{x\to a}\frac{f(x)-f(a)}{x-a}
\]
이므로 $x\to a$일 때 $f(x)\to(x-a)f'(a)+f(a)$라 할 수 있다. 예제 3의 경우는 $f(x)=\sqrt{x^{2}+x+1}$인 경우이다. 이제 이러한 방법을 이용하여 다음 문제를 풀어보자.
\vskip 10pt
\begin{example}
\textbf{(평가원 기출)} $\displaystyle\lim_{x\to 0}\frac{20x}{\sqrt{4+x}-\sqrt{4-x}}$의 값을 구하시오.
\begin{solution}
	$f(x)=\sqrt{4+x}$라 하면 $f'(0)=\frac{1}{4}$이므로 $\displaystyle\lim_{x\to 0}\frac{\sqrt{4+x}-2}{x}=\frac{1}{4}$이다. 즉 $x\to 0$일 때, $\sqrt{4+x}\to\frac{1}{4}x +2$이고 $\sqrt{4-x}\to -\frac{1}{4}x+2$이다. 따라서
	\begin{align*}
		\displaystyle\lim_{x\to 0}\frac{20x}{\sqrt{4+x}-\sqrt{4-x}}&=\displaystyle\lim_{x\to 0}\frac{20x}{\left(\frac{1}{4}x+2\right)-\left(-\frac{1}{4}x+2\right)}\\
		&=40
	\end{align*}
	이다.
\end{solution}
\end{example}
\vskip 10pt
\begin{problem} $\displaystyle\lim_{x\to 0}\frac{\sqrt{1+2x}-\sqrt{1-2x}}{x}$의 값을 구하시오.
\processifversion{psol}{
	\begin{psolution}
		함수 $y =\frac{1}{\sqrt{1+2x}}$의 $x=0$에서의 접선의 방정식은 $y=x+1$이고  $y =\frac{1}{\sqrt{1-2x}}$의 $x=0$에서의 접선의 방정식은 $y=-x+1$이이므로 구하는 극한값은 $2$임을 알 수 있다.
\end{psolution}}
\end{problem} 

\vspace{1em}
\section{평균변화율을 이용한 극한의 계산}
 $\frac{0}{0}$꼴의 극한값의 계산을 미분 특히 평균값 정리를 배운 3학년 학생의 경우도 여전히 미분의 지식을 사용하지 않고 대수적 연수만으로 계산하는 경우가 있다. 그러나 고 3 수험생의 경우 실전에서는 최대한 문제를 빨리 풀 수 있도록 이미 배운 모든 지식을 총 동원하여 문제를 해결 하는 것이 좋다. $\frac{0}{0}$꼴의 극한값을 계산하는 편리한 방법으로 로피탈의 정리가 있으나 로피탈의 정리를 쓰면 문제가 더 복잡해 지는 경우가 있다. 평균변화율의 개념을 활용한 함수의 극한의 계산은 처음 문제보다 더 복잡해지는 경우가 없으면서도 계산을 매우 빠르게 수행할 수 있다는 장점이 있다. 

평균변화율의 개념을 활용한 함수의 극한값의 계산에서 중요한 것은 주어진 극한의 계산에서 주어진 식이 어떤 함수의 특정한 구간에서의 평균변화율이라는 것을 분석해 내는 것이 중요하다. 
\vspace{1em}
\begin{example}
	다항함수 $f(x)$에 대하여 $\displaystyle\lim_{x\to 1}\frac{f(x)-f(1)}{x-1}=10$일 때,
	\begin{equation*}
		\displaystyle\lim_{h\to 0}\frac{f(1+h)-f(1-2h)}{h}
	\end{equation*}
의 값은?
\begin{solution}\textbf{(일반적인 풀이)}
 $\displaystyle\lim_{x\to 1}\frac{f(x)-f(1)}{x-1}=f'(1)$이므로 $f'(1)=10$이고,
	\begin{align*}
		\displaystyle\lim_{h\to 0}\frac{f(1+h)-f(1-2h)}{h}
		& =\displaystyle\lim_{h\to 0}\frac{f(1+h)-f(1)+ f(1)-f(1-2h)}{h}\\
		& =\displaystyle\lim_{h\to 0}\frac{f(1+h)-f(1)}{h}+\displaystyle\lim_{h\to 0}\frac{f(1-2h)-f(1)}{-2h}\times 2 \\
		& = f^{\prime}(1)+ 2 f^{\prime}(1)		\\
		& =3f^{\prime}(1)=30
	\end{align*}
\end{solution}
\begin{solution}\textbf{(평균변화율의 개념을 활용한 풀이)}
	
$\displaystyle\lim_{h\to 0}\frac{f(1+h)-f(1-2h)}{h}$을
\begin{equation*}
	 3\lim_{h\to 0}\frac{f(1+h)-f(1-2h)}{3h}=3\lim_{h\to 0}\frac{f(1+h)-f(1-2h)}{(1+h)-(1-2h)}
\end{equation*}
로 변형하면 $\frac{f(1+h)-f(1-2h)}{(1+h)-(1-2h)}$은 함수 $y=f(x)$의 구간 $[1-2h,\:1+h]$에서의 평균변화율이고 이 구간은 $h\to 0$일 때, 한 점 $x=1$에 수렴해 간다. 따라서 
	\begin{equation*}
	3\lim_{h\to 0}\frac{f(1+h)-f(1-2h)}{3h}=3\lim_{h\to 0}\frac{f(1+h)-f(1-2h)}{(1+h)-(1-2h)}=3f^{\prime}(1)
	\end{equation*}
	이다. 
\end{solution}
\end{example}

평균변화율의 개념을 활용한 풀이는 주어진 $\frac{0}{0}$꼴의 극한값의 계산식이 $\frac{f(A)-f(B)}{X}$일 때, $X = k(A-B)$의 꼴로 고쳐주기만 하면 간단히 계산된다는 장점이 있다.
\vspace{1em}
\begin{example}
	$f^{\prime}(3)=2$일 때, $\displaystyle\lim_{h\to 0}\frac{f(3 +2h)-f(3-3h)}{2h}$의 값을 구하시오.
	\begin{solution}
		\begin{align*}
		 \lim_{h\to 0}\frac{f(3 +2h)-f(3-3h)}{2h} &=\displaystyle\lim_{h\to 0}\frac{f(3+2h)-f(3-3h)}{5h}\cdot\frac{5}{2}\\
		 &=f^{\prime}(3)\times\frac{5}{2}=5
		\end{align*}
	\end{solution}
\end{example}
\vspace{1em}

\begin{problem}
	$\displaystyle\lim_{x\to 0}\frac{e^{1-\sin x}-e^{1-\tan x}}{\tan x -\sin x}$의 값을 구하시오.
	
\processifversion{psol}{
\begin{psolution}
	$\frac{e^{1-\sin x}-e^{1-\tan x}}{\tan x -\sin x}=\frac{e^{1-\sin x}-e^{1-\tan x}}{(1-\sin x)-(1-\tan x)}$이므로 $\frac{e^{1-\sin x}-e^{1-\tan x}}{\tan x -\sin x}$는 구간 $[1- \tan x,\:1-\sin x]$에서 함수 $y=e^{x}$의 평균변화율이다. 그런데 $x\to 1$이면 구간 $[1- \tan x,\:1-\sin x]$은 한 점 $x=1$에 수렴해 가고 따라서 평균변화율은 $x=1$에서의 순간변화율 $e$에 수렴함을 알 수 있다.
\end{psolution}}
\end{problem}
\vspace{1em}
\begin{problem}
	$f(x)=\begin{cases}
		\frac{x^{2}+x-12}{x-3}&(x\ne 3)\\
		a &	(x=3)
	\end{cases}$
가 모든 실수 $x$에서 연속일 때, $a$의 값은?
\processifversion{psol}{
\begin{psolution}
	$x=3$에서 연속이면 모든 실수 $x$에서 연속이고 $x=3$에서 연속이려면	
	\begin{equation*}
	a =\displaystyle\lim_{x\to 3}\frac{x^{2}+x-12}{x-3}
	\end{equation*}
	이어야 한다. $\frac{x^{2}+x-12}{x-3}$은 구간 $[3,\:x]$에서 함수 $f(x)=x^{2}+x$의 평균변화율이고 $x\to 3$이면 이 평균변화율은 $x=3$에서의 순간변화율에 수렴해가므로 구하는 $a$의 값은 $7$이다.
\end{psolution}}
\end{problem}
\vspace{1em}
\begin{problem}
	 다음 극한값을 구하시오.
	\begin{enumerate}
		\item $\displaystyle\lim_{x\to +0}\frac{e^{x}-e^{\sin x}}{x-\sin x}$
		\item $\displaystyle\lim_{x\to 0}\frac{\sin x -\sin(\sin x)}{x -\sin x}$
	\end{enumerate}
\processifversion{psol}{
\begin{psolution}
	\begin{enumerate}
		\item $\frac{e^{x}-e^{\sin x}}{x-\sin x}$는 구간 $(\sin x,\:x)$에서 함수 $y=e^{x}$의 평균변화율이다. $x\to 0+$이면 구간 
		$(\sin x,\:x)$은 한 점 $x=0$에 수렴해 가므로 $\displaystyle\lim_{x\to 0+}\tfrac{e^{x}-e^{\sin x}}{x-\sin x}$은 $y=e^{x}$의 $x=0$에서의 순간변화율 즉, 미분계수이다. 따라서 구하는 값은 $1$이다.
		\item $\frac{\sin x -\sin(\sin x)}{x -\sin x}$은 구간 $(\sin x,\:x)$에서 함수 $y=\sin x$의 평균변화율이다. $x\to 0$이면 구간 $(\sin x,\:x)$은 한 점 $x=0$에 수렴해 가므로 $\displaystyle\lim_{x\to 0}\tfrac{\sin x -\sin(\sin x)}{x -\sin x}$은 $y=\sin x$의 $x=0$에서의 순간변화율 즉, 미분계수이다. 따라서 구하는 값은 $1$이다.
	\end{enumerate}
\end{psolution}}
\end{problem}
\vspace{1em}
\section{함수의 극한 문제 해결}
\begin{problem}
	$\lim\limits_{n\to\infty}\left(\sqrt{2n^{2}+n}-a\sqrt{2n^{2}-n}\right)$의 극한값이 존재할 때의 $a$값과 그 때의 극한값을 $b$라 할 때, $a^{2}+b^{2}$의 값은?
\processifversion{psol}{
\begin{psolution}
	\begin{align*}
		\lim_{n\to\infty}\left(\sqrt{2n^{2}+n}-a\sqrt{2n^{2}-n}\right)
		& =\lim_{n\to\infty}\frac{2n^{2}+n-a^{2}(2n^{2}-n)}{\sqrt{2n^{2}+n}+a\sqrt{2n^{2}-n}}\\
		& =\lim_{n\to\infty}\frac{2n^{2}(1-a^{2})+n(1+a^{2})}{n\left(\sqrt{2+\frac{1}{n}}+a\sqrt{2-\frac{1}{n}}\right)}\\
		& =\lim_{n\to\infty}\frac{2n(1-a^{2})+(1+a^{2})}{\sqrt{2+\frac{1}{n}}+a\sqrt{2-\frac{1}{n}}}
	\end{align*}
	이므로 $a=1$일 때, 구하는 극한값은 $\frac{\sqrt{2}}{2}$이다. 따라서 $a^{2}+b^{2}=\frac{3}{2}$이다.
\end{psolution}}
\end{problem}
\vspace{1em}

\begin{problem}
	$\lim\limits_{x\to 1}\frac{\sqrt[3]{x^{2}+7}-\sqrt{x+3}}{x^{2}-3x+2}$의 값을 구하면?
\processifversion{psol}{
	\begin{psolution}
		\begin{align*}
			\lim_{x\to 1}\frac{\sqrt[3]{x^{2}+7}-\sqrt{x+3}}{x^{2}-3x+2}
			& =\lim_{x\to 1}\frac{\sqrt[3]{x^{2}+7}-2}{x^{2}-3x+2}+\lim_{x\to 1}\frac{2-\sqrt{x+3}}{x^{2}-3x+2}\\
			& =\lim_{x\to 1}\frac{x+1}{(x-2)\left(\sqrt[3]{(x^{2}+7)^{2}}+2\sqrt[3]{x^{2}+7}+4\right)}\\ &\phantom{+}+\lim_{x\to 1}\frac{1}{(2-x)(2+\sqrt{x+3)}}\\
			& = -\frac{2}{12}+\frac{1}{4}=\frac{1}{12}
		\end{align*}
\end{psolution}}
\end{problem}
\vspace{1em}

\begin{problem}
	$\lim\limits_{x\to 0}\frac{\sqrt{1+\sin^{2}x}-\cos x}{1-\sqrt{1+\tan^{2}x}}$의 값은?
\processifversion{psol}{
	\begin{psolution}
	\begin{align*}
		\lim_{x\to 0}\frac{\sqrt{1+\sin^{2}x}-\cos x}{1-\sqrt{1+\tan^{2}x}}
		& =\lim_{x\to 0}\frac{(1+\sin^{2}x-\cos^{2}x)(1+\sqrt{1+\tan^{2}x})}{(1-1-\tan^{2}x)(\sqrt{1+\sin^{2}x}+\cos x)}\\
		& =\lim_{x\to 0}\frac{2\sin^{2}x(1+\sqrt{1+\tan^{2}x})}{-\tan^{2}x\left(\sqrt{1+\sin^{2}x}+\cos x\right)}\\
		& =\lim_{x\to 0}\frac{-2\left(1+\sqrt{1+\tan^{2}x}\right)}{\sqrt{1+\sin^{2}x}+\cos x}\\
		& = -2
	\end{align*}	
\end{psolution}}
\end{problem}
\vspace{1em}

\begin{problem}
	$\lim\limits_{n\to\infty}\left(\frac{2n^{2}-3}{2n^{2}-n+1}\right)^{\frac{n^{2}-1}{n}}$의 값은?
\processifversion{psol}{
	\begin{psolution}
		\begin{align*}
			\lim_{n\to\infty}\left(\frac{2n^{2}-3}{2n^{2}-n+1}\right)^{\frac{n^{2}-1}{n}}
			& =\lim_{n\to\infty}\left(1+\frac{n-4}{2n^{2}-n+1}\right)^{\frac{n^{2}-1}{n}}\\
			& =\lim_{n\to\infty}\left[\left(1+\frac{n-4}{2n^{2}-n+1}\right)^{\frac{2n^{2}-n+1}{n-4}}\right]^{\frac{(n-4)(n^{2}-1)}{2n^{3}-n^{2}+n}}\\
			& = e^{\lim_{n\to\infty}\frac{n^{3}-4n^{2}-n+4}{2n^{3}-2n^{2}+n}}= e^{\frac{1}{2}}
		\end{align*}
\end{psolution}}
\end{problem}
\vspace{1em}

\begin{problem}
	$\lim\limits_{x\to 0}\left(e^{x}+\sin x\right)^{\frac{1}{x}}$의 값은?
\processifversion{psol}{
	\begin{psolution}
		\begin{align*}
			\lim_{x\to 0}\left(e^{x}+\sin x\right)^{\frac{1}{x}}
			& =\lim_{x\to 0}\left[e^{x}\left(1+\frac{\sin x}{e^{x}}\right)\right]^{\frac{1}{x}}\\
			& =\lim_{x\to 0}\left(e^{x}\right)^{\frac{1}{x}}\cdot\lim_{x\to 0}\left[\left(1+\frac{\sin x}{e^{x}}\right)^{\frac{e^{x}}{\sin x}}\right]^{\frac{\sin x}{xe^{x}}}\\
			& = e\cdot e^{\lim_{x\to 0}\frac{\sin x}{x}\cdot\frac{1}{e^{x}}}=e^{2}
		\end{align*}
\end{psolution}}
\end{problem}
\vspace{1em}


\begin{problem}
	$a,\: b$가 양의 실수일 때, $\lim\limits_{n\to\infty}\left(\frac{a-1+\sqrt[n]{b}}{a}\right)^{n}$의 값은?
\processifversion{psol}{
	\begin{psolution}
		\begin{align*}
			\lim_{n\to\infty}\left(\frac{a-1+\sqrt[n]{b}}{a}\right)^{n}
			& =\lim_{n\to\infty}\left[\left(1+\frac{\sqrt[n]{b}-1}{a}\right)^{\frac{a}{\sqrt[n]{b}-1}}\right]^{\frac{n(\sqrt[n]{b}-1)}{a}}\\
			& = e^{\frac{1}{a}\lim_{n\to\infty}\frac{b^{\frac{1}{n}}-1}{\frac{1}{n}}}\\
			& =e^{\frac{\ln b}{a}}=b^{\frac{1}{a}}
		\end{align*}
\end{psolution}}
\end{problem}
\vspace{1em}


\begin{problem}
	\[
	a_{n}=\begin{cases}
		1&(n\le k,\: k\text{는 자연수})\\
		\frac{(n+1)^{k}-n^{k}}{_{n}C_{k-1}}&(n>k)
	\end{cases}
	\]
	일 때, 다음 물음에 답하시오.
	\begin{enumerate}
		\item $\lim\limits_{n\to\infty}a_{n}$을 구하시오.
		
		\item $b_{n}=1+\sum\limits_{k=1}^{n}k \cdot \lim\limits_{n\to\infty}a_{n}$이라 할 때, \[\lim_{n\to\infty}\left(\frac{b_{n}^{2}}{b_{n-1} b_{n+1}}\right)^{n}\]의 값을 구하시오.
	\end{enumerate}
\processifversion{psol}{
	\begin{psolution}
		\begin{enumerate}
			\item \begin{align*}
		\lim_{n\to\infty}a_{n}
		& =\lim_{n\to\infty}\frac{(n+1)^{k}-n^{k}}{_{n}C_{k}}\\
		& =\lim_{n\to\infty}\frac{(k-1)!kn^{k-1}+\cdots +(k-1)!}{(n-k+2)(n-k+3)\cdots n}\\
		& =\frac{k!\cdot n^{k-1}+\cdots}{n^{k-1}+\cdots}=k!
	\end{align*}
			 \item $b_{n}=1 +\sum_{k=1}^{n}k\cdot k! =1+\sum_{k=1}^{n}((k+1)!-k!)=(n+1)!$이므로
			 \begin{equation*}
			 	\displaystyle \lim_{n\to\infty}\left(\frac{b_{n}^{2}}{b_{n-1}b_{n+1}}\right)^{n}=\lim_{n\to\infty}\left(1-\frac{1}{n}\right)^{n}=e^{-1}
			 \end{equation*}
			이다.	
\end{enumerate}
\end{psolution}}
\end{problem}
\vspace{1em}


\begin{problem}
	$a>0$일 때, $\lim\limits_{x\to 0}\frac{(a+x)^{x}-1}{x}$의 값을 구하시오.
\processifversion{psol}{
	\begin{psolution}
		\begin{align*}
			\lim_{x\to 0}\frac{(a+x)^{x}-1}{x}
			& =\lim_{x\to 0}\frac{e^{x\ln(a+x)}-1}{x}\\
			& =\lim_{x\to 0}\frac{e^{x\ln(a+x)}-1}{x\ln(a+x)}\cdot\ln(a+x)\\
			& =\ln a
		\end{align*}
\end{psolution}}
\end{problem}
\vspace{1em}


\begin{problem}
	$a_{1}=\frac{3}{2}$이고 $a_{n+1}=\frac{a_{n}^{2}-a_{n}+1}{a_{n}}$일 때 다음 물음에 답하시오.
	\begin{itemize}
		\item수열 $\left\{a_{n}\right\}$이 수렴함을 보이시오.
		\item $\lim\limits_{n\to\infty}a_{n}$의 값을 구하시오.
	\end{itemize}
\processifversion{psol}{
	\begin{psolution}
\begin{itemize}
	\item 산술-기하 부등식에 의해
	\begin{equation*}
	a_{n+1}=\frac{a_{n}^{2}-a_{n}+1}{a_{n}}=a_{n}+\frac{1}{a_{n}}-1\ge 1
	\end{equation*}
	이므로 주어진 수열은 아래로 유계이고 	
	$a_{n+1}-a_{n}=\frac{1}{a_{n}}-1\le 0$이므로 감소수열이다. 따라서 주어진 수열은 수렴한다.
	\item 주어진 수열이 수렴하므로 $\lim_{n\to\infty}a_{n}=l$이라 하면
		$l =\frac{l^{2}-l+1}{l}$에서 $l=1$이다.
\end{itemize}		
\end{psolution}}
\end{problem}
\vspace{1em}


\begin{problem}
	$a>0$, $b \in \left(a,\: 2a\right)$이다. $x_{0}=b$이고 $x_{n+1}=a+\sqrt{x_{n}(2a-x_{n})}$을 만족시키는 수열 $x_{n}$의 극한값이 존재할 때, $\frac{b}{a}$의 값을 구하시오.
\processifversion{psol}{
	\begin{psolution}
       $x_{1}= a+\sqrt{2ab-b^{2}}$이고 
		\begin{align*}
			x_{2}
			& = a+\sqrt{(a+\sqrt{2ab-b^{2}})(a-\sqrt{2ab-b^{2}})}\\
			& = a+\sqrt{a^{2}-2ab+b^{2}}\\
			& = a+ | a-b | =b
		\end{align*}
		이므로 주어진 수열은 주기수열이고 따라서 $\lim_{k\to\infty}x_{2k}=b$, $\lim_{k\to\infty}x_{2k+1}= a+\sqrt{2ab-b^{2}}$이다. 이 수열이 수렴하기 위한 필요충분조건은 $b=a+\sqrt{2ab-b^{2}}$이고 따라서 
		\begin{equation*}
			\frac{b}{a}=1+\frac{\sqrt{2}}{2}
		\end{equation*}
	이다. 
\end{psolution}}
\end{problem}
\vspace{1em}


\begin{problem}
	$\lim\limits_{n\to\infty}\sum\limits_{k=1}^{n}\frac{k}{4k^{4}+1}$의 값을 구하시오.
\processifversion{psol}{
	\begin{psolution}
		\begin{align*}
			\lim_{n\to\infty}\sum_{k=1}^{n}\frac{k}{4k^{4}+1}
			& =\lim_{n\to\infty}\left(\frac{1}{4}\sum_{k=1}^{n}\frac{1}{2k^{2}-2k+1}-\frac{1}{4}\sum_{k=1}^{n}\frac{1}{2k^{2}+2k+1}\right)\\
			& =\frac{1}{4}\lim_{n\to\infty}\left(1-\frac{1}{2n^{2}+2n+1}\right)=\frac{1}{4}
		\end{align*}
\end{psolution}}
\end{problem}
\vspace{1em}


\begin{problem}
	$\lim\limits_{x\to 0}\frac{\ln(\cos 4x)}{\ln(\cos 2x)}$의 값은? 더 일반적으로 $\lim\limits_{x\to 0}\frac{\ln(\cos ax)}{\ln(\cos bx)}$의 값을 구하시오.
\processifversion{psol}{
	\begin{psolution}
		 $\lim_{x\to 0}\frac{\ln(\cos ax)}{\ln(\cos bx)}$의 극한값을 구하면 원래의 문제는 해결되므로 이 극한값만 계산하자.
		\begin{align*}
			\lim_{x\to 0}\frac{\ln(\cos ax)}{\ln(\cos bx)}
			& =\lim_{x\to 0}\frac{(\cos ax-1)\ln(1+\cos ax-1)^{\frac{1}{\cos ax-1}}}{(\cos bx-1)\ln(1+\cos bx-1)^{\frac{1}{\cos bx-1}}}\\
			& =\lim_{x\to 0}\frac{-2\sin^{2}\frac{ax}{2}}{-2\sin^{2}\frac{bx}{2}}=\frac{a^{2}}{b^{2}}
		\end{align*}
\end{psolution}}
\end{problem}
\vspace{1em}


\begin{problem}
	$a,\: b$가 양의 실수일 때, $\lim\limits_{n\to\infty}\left(\frac{\sqrt[n]{a}+\sqrt[n]{b}}{2}\right)^{n}$의 값을 구하시오.
\processifversion{psol}{
	\begin{psolution}
		$\lim_{n\to\infty}n(\sqrt[n]{a}-1)=\lim_{n\to\infty}\frac{a^{\frac{1}{n}}-1}{\frac{1}{n}}=\ln a$이므로
		\begin{align*}
			\lim_{n\to\infty}\left(\frac{\sqrt[n]{a}+\sqrt[n]{b}}{2}\right)^{n}
			& =\lim_{n\to\infty}\left(1+\frac{\sqrt[n]{a}-1+\sqrt[n]{b}-1}{2}\right)^{n}\\
			& =\lim_{n\to\infty}\left[\left(1+\frac{\sqrt[n]{a}-1+\sqrt[n]{b}-1}{2}\right)^{\frac{2}{\sqrt[n]{a}-1+\sqrt[n]{b}-1}}\right]^{\frac{n(\sqrt[n]{a}-1)+n(\sqrt[n]{b}-1)}{2}}\\
			& =e^{\lim_{n\to\infty}\frac{n(\sqrt[n]{a}-1)+n(\sqrt[n]{b}-1)}{2}}\\
			& = e^{\frac{\ln a +\ln b}{2}}=e^{\ln\sqrt{ab}}=\sqrt{ab}
		\end{align*}
\end{psolution}}
\end{problem}
\vspace{1em}


\begin{problem}
	$\lim\limits_{x\to 1^{-0}}\frac{1-\cos\left(4\cos^{-1}x\right)}{1-x^{2}}$의 값은?$\left(\text{단, } x \in \left(0,\:\frac{\pi}{2}\right)\text{이다.}\right)$
\processifversion{psol}{
	\begin{psolution}
		\begin{align*}
			\lim_{x\to 1-0}\frac{1-\cos\left(4\cos^{-1}x\right)}{1-x^{2}}
			& =\lim_{x\to 1-0}\frac{2\sin^{2}\left(2\cos^{-1}x\right)}{1-x^{2}}\\
			& =\lim_{x\to 1-0}\frac{2\sin^{2}\left(2\cos^{-1}x\right)}{(2\cos^{-1}x)^{2}}\cdot\lim_{x\to 1-0}\frac{4(\cos^{-1}x)^{2}}{1-x^{2}}\\
			& = 2\cdot\lim_{x\to 1-0}\frac{4(\cos^{-1}x)^{2}}{1-x^{2}}\\
			& =8\lim_{y\to 0+}\frac{y^{2}}{\sin^{2}y}=8
		\end{align*}
\end{psolution}}
\end{problem}
\vspace{1em}


\begin{problem}
	$\lim\limits_{x\to 0}\frac{\tan{x} -\tan^{-1}x}{x^{2}}$의 값을 구하시오.
\processifversion{psol}{
	\begin{psolution}
		먼저 $x \in \left(0,\:\frac{\pi}{2}\right)$일 때,
\begin{equation*}
0 <\frac{\tan x-x}{x^{2}}<\frac{\tan x-\sin x}{x^{2}}=\frac{\tan x(1-\cos x)}{x^{2}}=\frac{2\tan x\sin^{2}\frac{x}{2}}{x^{2}}
\end{equation*}		
		이고 
\begin{equation*}
	\lim_{x\to 0}\frac{2\tan x\sin^{2}\frac{x}{2}}{x^{2}}=\lim_{x\to 0}\frac{\tan x}{2}\cdot\lim_{x\to 0}\left(\frac{\sin\frac{x}{2}}{\frac{x}{2}}\right)^{2}=0
\end{equation*}
	이므로 $\lim_{x\to 0}\frac{\tan x -x}{x^{2}}=0$이다. 이제,
	\begin{align*}
			\lim_{x\to 0}\frac{\tan x -\tan^{-1}x}{x^{2}}
			& =\lim_{x\to 0}\frac{tanx-x}{x^{2}}+\lim_{x\to 0}\frac{x-\tan^{-1}x}{x^{2}}\\
			& =\lim_{x\to 0}\frac{x-\tan^{-1}x}{x^{2}}\\
			& =\lim_{y\to 0}\frac{\tan y-y}{\tan^{2}y}\\
			& =\lim_{y\to 0}\frac{\tan y-y}{y^{2}}\cdot\frac{y^{2}}{\tan^{2}y}=0
		\end{align*}
\end{psolution}}
\end{problem}
\vspace{1em}


\begin{problem}
	$\lim\limits_{n\to\infty}\cos(n\pi\sqrt[2n]{e})$의 값을 구하시오.
\processifversion{psol}{
	\begin{psolution}
		\begin{align*}
			\left|\lim_{n\to\infty}\cos(n\pi\sqrt[2n]{e})\right|
			& =\lim_{n\to\infty}\left|(-1)^{n}\cos(n\pi\sqrt[2n]{e}-n\pi)\right| \\
			& =\lim_{n\to\infty}\left|\cos\left(\frac{\pi}{2}\cdot\frac{e^{\frac{1}{2n}}-1}{\frac{1}{2n}}\right)\right| \\
			& =\left|\cos\frac{\pi}{2}\cdot\lim_{n\to\infty}\frac{e^{\frac{1}{2n}}-1}{\frac{1}{2n}}\right|\\
			& =\left|\cos\frac{\pi}{2}\right| =0
		\end{align*}
		이므로 $\displaystyle \lim_{n\to\infty}\cos(n\pi\sqrt[2n]{e})=0$이다.	
\end{psolution}}
\end{problem}
\vspace{1em}


\begin{problem}
	$\lim\limits_{n\to\infty}\left(\frac{n+1}{n}\right)^{\tan\frac{(n-1)\pi}{2n}}$의 값을 구하시오.
\processifversion{psol}{
	\begin{psolution}
		\begin{align*}
			\lim_{n\to\infty}\left(\frac{n+1}{n}\right)^{\tan\frac{(n-1)\pi}{2n}}
			& =\displaystyle\lim_{n\to\infty}\left[\left(1+\frac{1}{n}\right)^{n}\right]^{\frac{1}{n}\tan\frac{(n-1)\pi}{2n}}\\
			& = e^{\displaystyle\lim_{n\to\infty}\frac{\tan\frac{(n-1)\pi}{2n}}{n}}\\
			& = e^{\displaystyle\lim_{n\to\infty}\frac{\tan\left(\frac{\pi}{2}-\frac{\pi}{2n}\right)}{n}}\\
			& = e^{\displaystyle\lim_{n\to\infty}\frac{\cot\frac{\pi}{2n}}{n}}\\
			& = e^{\frac{2}{\pi}\displaystyle\lim_{n\to\infty}\frac{\frac{\pi}{2n}}{\tan\frac{\pi}{2n}}}=e^{\frac{2}{\pi}}
		\end{align*}
\end{psolution}}
\end{problem}
\vspace{1em}


\begin{problem}
	$\lim\limits_{n\to\infty}n\ln\tan\left(\frac{\pi}{4}+\frac{\pi}{n}\right)$의 값을 구하시오.
\processifversion{psol}{
	\begin{psolution}
		\begin{align*}
			\lim_{n\to\infty}n\ln\tan\left(\frac{\pi}{4}+\frac{\pi}{n}\right)
			& =\lim_{n\to\infty}\tan\left(\frac{\pi}{4}+\frac{\pi}{n}\right)^{n}\\
			& =\ln\left[\left(1+\tan\left(\frac{\pi}{4}+\frac{\pi}{n}\right)-1\right)^{\frac{1}{\tan\left(\frac{\pi}{4}+\frac{\pi}{n}\right)-1}}\right]^{n\left(\tan\left(\frac{\pi}{4}+\frac{\pi}{n}\right)-1\right)}\\
			& =\ln e^{\displaystyle \lim_{n\to\infty}n\left(\tan\left(\frac{\pi}{4}+\frac{\pi}{n}\right)-1\right)}\\
			& =\lim_{n\to\infty}n\left(\tan\left(\frac{\pi}{4}+\frac{\pi}{n}\right)-1\right)\\
			& =\lim_{n\to\infty}n\left(\frac{1+\tan\frac{\pi}{n}}{1-\tan\frac{\pi}{n}}-1\right)\\
			& =\lim_{n\to\infty}\frac{2n\tan\frac{\pi}{n}}{1-\tan\frac{\pi}{n}}\\
			& = 2\lim_{n\to\infty}n\tan\frac{\pi}{n}\\
			& =2\pi\lim_{n\to\infty}\frac{\tan\frac{\pi}{n}}{\frac{\pi}{n}}=2\pi 
	\end{align*}
\end{psolution}}
\end{problem}
\vspace{1em}


\begin{problem}
	$\tan\alpha +\cot\alpha =n(n\ge 2)$의 근을 $\alpha_{n}$이라 할 때,
	\begin{equation*}
		\lim\limits_{n\to\infty}(\sin\alpha_{n}+\cos\alpha_{n})^{n}
	\end{equation*}
	의 값을 구하시오.$\left(\text{단, }0<\alpha_{n}<\frac{\pi}{4}\text{이다.}\right)$
\processifversion{psol}{
	\begin{psolution}
		\begin{align*}
			\lim_{n\to\infty}(\sin\alpha_{n}+\cos\alpha_{n})^{n}
			& =\lim_{n\to\infty}\left[\left(\sin\alpha_{n}+\cos\alpha_{n}\right)^{2}\right]^{\frac{n}{2}}\\
			& =\lim_{n\to\infty}\left(1+ 2\cos\alpha_{n}\sin\alpha_{n}\right)^{\frac{n}{2}}\\
			& =\lim_{n\to\infty}\left(1+\frac{2}{\frac{\sin^{2}\alpha_{n}+\cos^{2}\alpha_{n}}{\cos\alpha_{n}\sin\alpha_{n}}}\right)^{\frac{n}{2}}\\
			& =\lim_{n\to\infty}\left(1+\frac{2}{\tan\alpha_{n}+\cot\alpha_{n}}\right)^{\frac{n}{2}}\\
			& =\lim_{n\to\infty}\left(1+\frac{2}{n}\right)^{\frac{n}{2}}=e
		\end{align*}
\end{psolution}}
\end{problem}
\vspace{1em}


\begin{problem}
	$\lim\limits_{x\to 0}\frac{2^{\tan^{-1}x}-2^{\sin^{-1}x}}{2^{\tan x}- 2^{\sin x}}$의 값을 구하시오.
\processifversion{psol}{
	\begin{psolution}
		\begin{align*}
	\lim_{x\to 0}\frac{2^{\tan^{-1}x}-2^{\sin^{-1}x}}{2^{\tan x}- 2^{\sin x}}&=\lim_{x\to 0}\frac{2^{\sin^{-1}x}(2^{\tan^{-1}x-\sin^{-1}x}-1)}{2^{\sin x}(2^{\tan x-\sin x}-1)}\\
	 &=\lim_{x\to 0}\frac{2^{\tan^{-1}x -\sin^{-1}x}-1}{2^{\tan x-\sin x}-1}\\
	 &=\lim_{x\to 0}\frac{2^{\tan^{-1}x-\sin^{-1}x}-1}{\tan^{-1}x-\sin^{-1}x}\cdot\lim_{x\to 0}\frac{\tan x-\sin x}{2^{\tan x-\sin x}-1}\\
	 &\phantom{lim}\cdot\lim_{x\to 0}\frac{\tan^{-1}x-\sin^{-1}x}{\tan x-\sin x}\\
	 &=\ln 2\cdot\frac{1}{\ln 2}\cdot\lim_{x\to 0}\frac{\tan^{-1}x-\sin^{-1}x}{\tan x-\sin x}\\
	 &=\lim_{x\to 0}\frac{\tan^{-1}x-\sin^{-1}x}{x^{3}}\cdot\lim_{x\to 0}\frac{x^{3}}{\tan x-\sin x}\\
     &=\lim_{x\to 0}\frac{\tan^{-1}x-\sin^{-1}x}{\tan\left(\tan^{-1}x-\sin^{-1}x\right)}\times\lim_{x\to 0}\frac{\tan\left(\tan^{-1}x-\sin^{-1}x\right)}{x^{3}}\\
			&\times\lim_{x\to 0}\frac{x^{3}}{\tan x(1-\cos x)}\\
     &=\lim_{x\to 0}\frac{\frac{x-\frac{x}{\sqrt{1-x^{2}}}}{1+\frac{x^{2}}{\sqrt{1-x^{2}}}}}{x^{3}}\cdot\lim_{x\to 0}\frac{x^{3}}{2 \tan x\sin^{2}\frac{x}{2}}\\
	 &= 2\lim_{x\to 0}\frac{\sqrt{1-x^{2}}-1}{x^{2}\left(\sqrt{1-x^{2}}+x^{2}\right)}\\
	 &= 2\lim_{x\to 0}\frac{-x^{2}}{x^{2}\left(\sqrt{1-x^{2}}+x^{2}\right)\left(\sqrt{1-x^{2}}+1\right)}=-1
	\end{align*}
\end{psolution}}
\end{problem}
\vspace{1em}


\begin{problem}
	$\lim\limits_{x\to 0}\frac{\tan^{-1}x-\sin^{-1}x}{x^{3}}$의 값을 구하시오.
\processifversion{psol}{
	\begin{psolution}
		\begin{align*}
	\lim_{x\rightarrow 0}\frac{\tan^{-1}x-\sin^{-1}x}{x^{3}}
			& =\lim_{x\rightarrow 0}\frac{\tan^{-1}x-\sin^{-1}x}{\tan\left(\tan^{-1}x-\sin^{-1}x\right)}\cdot\frac{\tan(\tan^{-1}x-\sin^{-1}x)}{x^{3}}\\
			& =\lim_{x\rightarrow 0}\frac{\tan(\tan^{-1}x-\sin^{-1}x)}{x^{3}}\\
			& =\lim_{x\rightarrow 0}\frac{1}{x^{3}}\cdot\frac{x-\frac{x}{\sqrt{1-x^{2}}}}{1+\frac{x^{2}}{\sqrt{1-x^{2}}}}\\
			& =\lim_{x\rightarrow 0}\frac{1}{x^{2}}\cdot\frac{\sqrt{1-x^{2}}-1}{\sqrt{1-x^{2}}+x^{2}}\\
			& =\lim_{x\rightarrow 0}\frac{-1}{\left(\sqrt{1-x^{2}}+x^{2}\right)\left(\sqrt{1-x^{2}}+1\right)}= -\frac{1}{2}
		\end{align*}
\end{psolution}}
\end{problem}
\vspace{1em}


\begin{problem}
	$\lim\limits_{n\to\infty}\sqrt[n]{\frac{3^{3n}(n!)^{3}}{(3n)!}}$의 값을 구하시오.
\processifversion{psol}{
	\begin{psolution}
	$a_{n}=\frac{3^{3n}(n!)^{3}}{(3n)!}$라 하면
	\begin{align*}
		\lim_{n\to\infty}\sqrt[n]{a_{n}}
		& =\lim_{n\to\infty}\frac{a_{n+1}}{a_{n}}\\
		& =\lim_{n\to\infty}\frac{3^{3n+3}[(n+1)!]^{3}}{(3n+3)!}\cdot\frac{(3n)!}{3^{3n}(n!)^{3}}\\
		& =\lim_{n\to\infty}\frac{27(n+1)^{3}}{(3n+1)(3n+2)(3n+3)}=1
	\end{align*}	
\end{psolution}}
\end{problem}
\vspace{1em}


\end{document}