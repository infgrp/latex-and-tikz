% 반드시 XeLatex으로 컴파일을 해야 합니다.
% 같이 제공하는 preamblex.tex은 trig.tex보다 상위 디렉토리에 두어야 컴파일이 됩니다. 두 파일을 같은 디렉토리에 두려면 몇 줄 아래에 있는 % AMS and mathtools
\usepackage{amsmath,amsthm,amssymb,marvosym,mathrsfs,amsfonts,amscd,mathtools}
% Hyperlinks and URLs
\usepackage{url}
\usepackage{hyperref}
\hypersetup{
	colorlinks,
	citecolor=BLACK,
	filecolor=BLACK,
	linkcolor=BLACK,
	urlcolor=BLACK
}

% Colors
\usepackage[usenames,dvipsnames]{xcolor}
\usepackage{tikz}
\usepackage{tkz-euclide}

% shadowing mdframed
\usepackage[framemethod=tikz]{mdframed}
\usetikzlibrary{shadows}
% Bold math
\usepackage{bm}

%Use Korean Letter when enumerate
\usepackage{dhucs-enumerate}

\usepackage{anyfontsize}

% Bra Ket (Dirac) Notation
\usepackage{braket}

% Slashed characters (e.g. in Dirac equation)
\usepackage{slashed}
\usepackage{pifont} % 원문자 사용시 필요한 패키지

% chapter decoration
\usepackage{type1cm}
\usepackage[explicit]{titlesec}

\titleformat{\chapter}[display]
{\normalfont\Large\rmfamily}
{\sffamily\flushright\fontsize{60}{0}\textbf{\textcolor{blue!40}{{\Huge\chaptername}~\thechapter\vskip0pt\rule{\textwidth}{2pt}}}}{0pt}
{\flushleft\fontsize{30}{0}{#1}\vskip60pt}
\titlespacing*{\chapter}
{0pt}{-40pt}{0pt}

%\usetikzlibrary{shadows}
\usetikzlibrary{shadows.blur}
\usetikzlibrary{shapes.symbols}

% Tcolorbox
\usepackage[most]{tcolorbox}

% Clean SI Units
\usepackage{siunitx}

% Enumerate thingies
\usepackage{enumitem}

% Cancel things out in equations
\usepackage[makeroom]{cancel}

\usepackage{multicol}

% Graphics and figures
\usepackage{graphicx}
\usepackage{wrapfig}
\usepackage{float}

\usepackage{cancel}

% Caption figures and tables
\usepackage{caption,subcaption}

% Generate symbols
\usepackage{textcomp} % Include this line to avoid output errors
\usepackage{gensymb}

% Make multiple rows in a table
\usepackage{multirow}

% Booktabs tables
\usepackage{booktabs}

%\usepackage[utopia,sfscaled]{mathdesign}
% Useful frames
\usepackage{mdframed}

% Comment-out large sections
\usepackage{comment}

% No auto-indent
\setlength{\parindent}{0pt}

% Asymptote - 3D vector graphics
\usepackage{asymptote}

% Tikz Package Stuff
\usepackage{pgf,tikz,pgfplots}
\usepackage{tikz-3dplot}
\usepackage{tabularx}
\usepackage{array}
\usepackage{colortbl}
\tcbuselibrary{skins}
\usepackage{tkz-euclide}

\newcolumntype{Y}{>{\raggedleft\arraybackslash}X}

\tcbset{tab1/.style={fonttitle=\bfseries\large,fontupper=\normalsize\sffamily,
		colback=yellow!10!white,colframe=red!75!black,colbacktitle=Salmon!40!white, halign=center,
		coltitle=black,center title,freelance,frame code={
			\foreach \n in {north east,north west,south east,south west}
			{\path [fill=red!75!black] (interior.\n) circle (3mm); };},}}

\tcbset{tab2/.style={enhanced,fonttitle=\bfseries,fontupper=\normalsize\sffamily, halign=center, box align=center,
		colback=yellow!10!white,colframe=red!50!black,colbacktitle=Salmon!40!white,
		coltitle=black,center title}}


% Use various tikz libraries
\usetikzlibrary{decorations.pathmorphing, decorations.markings, decorations.pathreplacing, patterns} % Decorate paths!
\usetikzlibrary{calc, patterns, shapes.geometric, positioning, through, intersections}
\usetikzlibrary{scopes}
\usetikzlibrary{angles, quotes}
\usetikzlibrary{svg.path}
\usetikzlibrary{arrows, arrows.meta}
\usetikzlibrary{fadings}
% pgfplots package settings
\pgfplotsset{compat=1.15}
% \pgfplotsset{width=10cm,compat=1.9} % Taken from latest overleaf.
% plot arc easily
\def\centerarc[#1](#2)(#3:#4:#5)% Syntax: [draw options] (center) (initial angle:final angle:radius)
{ \draw[#1] ($(#2)+({#5*cos(#3)},{#5*sin(#3)})$) arc (#3:#4:#5); }

% Awesome circled numbers
\newcommand*\circled[4]{\tikz[baseline=(char.base)]{\node[shape=circle, fill=#2, draw=#3, text=#4, inner sep=2pt] (char) {#1};}}

% Control size of text
\usepackage{relsize}

% Extend conditional commands
\usepackage{xifthen}
\usepackage{xcolor}
\definecolor{termcolor}{cmyk}{.21,.97,.0,.0}
\definecolor{darkred}{cmyk}{.27,1,1,.32}
\definecolor{darkblue}{cmyk}{1,.98,.10,.11}
\definecolor{darkgreen}{cmyk}{.29,0,87,0}
\definecolor{darkmycolor}{cmyk}{99,59,22,3}
\definecolor{for_eyes}{RGB}{253,247,228}
%change color of math equation
%\everymath{\color{darkred}}
% Scale math by size
\newcommand*{\Scale}[2][4]{\scalebox{#1}{\ensuremath{#2}}}

% Big integrals
\usepackage{bigints}

% Number equations within sections
\numberwithin{equation}{section}

% Generate blind text
\usepackage{blindtext}

% Useful symbols
\usepackage{marvosym}

\newcounter{problem}[section]
\newcounter{example}[section]

% cancel 색상 변경
\newcommand\Ccancel[2][black]{\renewcommand\CancelColor{\color{#1}}\cancel{#2}}

%%%% 원문자
\newcommand*\ocircled[1]{\tikz[baseline=(char.base)]{
		\node[shape=circle,draw,inner sep=2pt] (char) {#1};}}
	
%%%%% 보기 스타일 %%%%%
\usepackage{tabu}
\newcommand{\questwo}[2]{
	\vskip 6pt
	\noindent\begin{tabu}{X[0.2] X[6] X[0.2] X[6]}
		(1)&$#1$ &(2) &$#2$
	\end{tabu}
}
\newcommand{\questhree}[3]{
	\vskip 3pt
	\noindent\begin{tabu}{X[0.2] X[6] X[0.2] X[6] X[0.2] X[6]}
		(1)&$#1$ &(2) &$#2$ &(3) & $#3$
	\end{tabu}
	\vskip 5pt
}
\newcommand{\quesfour}[4]{
	\vskip 6pt
	\noindent\begin{tabu}{X[0.2] X[6] X[0.2] X[6]}
		(1)&$#1$ &(2) &$#2$\\
		(3)&$#3$ &(4) &$#4$
	\end{tabu}
}
\newcommand{\quesfive}[5]{
	\vskip 6pt
	\noindent\begin{tabu}{X[0.2] X[3] X[0.2] X[3] X[0.2] X[3]}
		(1)&$#1$ &(2) &$#2$ &(3) &$#3$\\
		(4)&$#4$ &(5) &$#5$
	\end{tabu}
}

\newcommand{\oquesfive}[5]{
	\vskip 6pt
	\noindent\begin{tabu}{X[0.2] X[3] X[0.2] X[3] X[0.2] X[3] X[0.2] X[3] X[0.2] X[3]}
		\ding{172}&$#1$ &\ding{173} &$#2$ &\ding{174} &$#3$&	\ding{175}&$#4$ &\ding{176} &$#5$
	\end{tabu}
}
%%%% sample(보기) %%%%
\newenvironment{sample}{\vskip 10pt\noindent\begin{tikzpicture}[yshift=1.5pt]%
		\draw[rounded corners=1ex,overlay,draw=blue] (0pt,-4pt) rectangle (19pt,9pt);
		\node[rectangle,overlay,xshift=9.8pt,yshift=2pt,color=blue] {{\footnotesize\sffamily 보기}};\end{tikzpicture}\phantom{\footnotesize\sffamily보기...}}{\vskip 10pt}
%%%% THEOREMS %%%%
\newenvironment{theorem}[1][\hspace{-0.36em}]
{
	\begin{mdframed}[backgroundcolor=purple!5, align=center, userdefinedwidth=40em, linecolor=purple!30, linewidth=2pt, roundcorner=7pt, innertopmargin=10pt, shadow=true, shadowcolor=black!20, roundcorner=7pt, innerbottommargin=10pt, frametitle = {#1}]
	}
	{
	\end{mdframed}
}

%%%% LEMMAS %%%%
\newenvironment{lemma}[1][\hspace{-0.36em}]
{
	\begin{mdframed}[backgroundcolor=black!8, align=center, userdefinedwidth=40em, topline=false, bottomline = false, leftline = false, rightline = false, frametitle = {#1 보조정리}]
	}
	{
	\end{mdframed}
}

%%%% COROLLARY %%%%
\newenvironment{corollary}[1][\hspace{-0.36em}]
{
	\begin{mdframed}[backgroundcolor=black!8, align=center, userdefinedwidth=40em, topline=false, bottomline = false, leftline = false, rightline = false, frametitle = {#1 따름정리}]
	}
	{
	\end{mdframed}
}

%%%% DEFINITIONS %%%%
\newenvironment{definition}[1][\hspace{-0.36em}]
{
	\begin{mdframed}[backgroundcolor=cyan!14, align=center, userdefinedwidth=40em, linecolor=cyan!60, linewidth=2pt,roundcorner=7pt, innertopmargin=10pt, shadow=true, shadowcolor=black!20, roundcorner=7pt, innerbottommargin=10pt, frametitle = {정의 : #1}]
	}
	{
	\end{mdframed}
}

%%%% PROPOSITION %%%%
\newenvironment{proposition}
{
	\begin{mdframed}[backgroundcolor=black!4, align=center, userdefinedwidth=40em, topline=false, bottomline = false, leftline = false, rightline = false, frametitle = {Proposition}]
	}
	{
	\end{mdframed}
}
%%%% PROBLEM %%%%
\newenvironment{problem}{\refstepcounter{problem}
	\begin{mdframed}[linecolor=blue!35, linewidth=2pt, roundcorner=7pt, innertopmargin=10pt, shadow=true, shadowcolor=black!20, roundcorner=7pt, innerbottommargin=10pt, backgroundcolor=blue!5]
		\noindent 
		\noindent\begin{tikzpicture}[overlay,xshift=6pt,yshift=5pt]
			\draw[fill=violet!15,draw=violet!15] (0,0) circle (8pt);
			\draw[fill=violet!15,draw=violet!15] (9pt,0) circle (8pt);
			\node[rectangle,overlay,xshift=4pt] {\color{black}\sffamily\bfseries 문제};
		\end{tikzpicture}{\phantom{.........}\fontspec[Scale=1.1]{TeX Gyre Adventor}\color{darkblue} \theproblem}\hspace{5pt}}{
\end{mdframed}}
%%%% SOLUTION OF PROBLEM %%%%
\newenvironment{psolution}{\begin{description}\item[{\begin{tikzpicture}%
				\draw[rounded corners=1ex,overlay] (-3pt,-3pt) rectangle (20pt,9pt);\end{tikzpicture}\footnotesize\sffamily풀이}\hspace{6pt}]}{\end{description}}
			
%%%% EXAMPLE %%%%		
\newenvironment{example}{\refstepcounter{example}
	\begin{mdframed}[roundcorner=7pt,linecolor=termcolor,linewidth=2pt,innertopmargin=10pt, shadow=true, shadowcolor=black!20, innerbottommargin=10pt, backgroundcolor=termcolor!2]
		\noindent 
		\noindent\begin{tikzpicture}[overlay,xshift=6pt,yshift=5pt]
			\draw[fill=darkred!60,draw=darkred!60] (0,0) circle (7pt);
			\draw[fill=darkred!60,draw=darkred!60] (9pt,0) circle (7pt);
			\node[rectangle,overlay,xshift=4pt] {\color{white}\sffamily\bfseries 예제};
		\end{tikzpicture}{\phantom{.........}\fontspec[Scale=1.1]{TeX Gyre Adventor}\color{darkred} \theexample}\hspace{5pt}}{
\end{mdframed}}		
%%%%%%% SOLUTION OF EXAMPLE %%%%%%%%%
\newenvironment{solution}{\begin{description}\item[{\begin{tikzpicture}%
				\draw[rounded corners=1ex,overlay] (-3pt,-3pt) rectangle (20pt,9pt);\end{tikzpicture}\footnotesize\sffamily풀이}\hspace{6pt}]}{\end{description}}

% 보기 박스 정의 시작
\tcbuselibrary{breakable, skins}
\tcbset{enhanced}
\newtcolorbox{ChoiceBox}[1]{
		enhanced,
		before skip=2ex, after skip=2ex,
		boxrule=0.5pt, colframe=black, colback=white, arc=0.5ex,
		boxsep=0.5ex, top=1.5ex, bottom=1.5ex, left=0.5em, right=o0.5em,
		colbacktitle=white, coltitle=black,
		attach boxed title to top center={xshift=0cm, yshift=-1.5mm},
		boxed title style={size=minimal, enhanced, boxrule=0.25pt, colframe=white},
		breakable=false, title ={< #1 >}
}
% 보기 박스 정의 끝.
	
% Change end-of-proof symbol
\renewcommand\qedsymbol{$\blacksquare$}
%overline
\newcommand{\ovr}[1]{\overline{\textrm{#1}}}
% trigonometric function
\newcommand{\cosrm}[1]{\cos \textrm{#1}}
\newcommand{\sinrm}[1]{\sin \textrm{#1}}
\newcommand{\tanrm}[1]{\tan \textrm{#1}}
\newcommand{\cotrm}[1]{\cot \textrm{#1}}
\newcommand{\cscrm}[1]{\csc \textrm{#1}}
\newcommand{\secrm}[1]{\sec \textrm{#1}}
%%%% BLACKBOARD BOLD %%%%
\newcommand{\bbN}{\mathbb{N}} % Natural numbers
\newcommand{\bbZ}{\mathbb{Z}} % Zahlen
\newcommand{\bbQ}{\mathbb{Q}} % Rational numbers
\newcommand{\bbR}{\mathbb{R}} % Real numbers
\newcommand{\bbC}{\mathbb{C}} % Complex numbers
\DeclareSymbolFont{bbold}{U}{bbold}{m}{n} % Identity matrix
\DeclareSymbolFontAlphabet{\mathbbold}{bbold} % Identity matrix
\newcommand{\identitymatrix}{\mathbbold{1}} % Identity matrix

%%%% CODE LISTING %%%%
\usepackage{listings}
\definecolor{greencomments}{HTML}{00BA00}
\definecolor{graynumbers}{HTML}{4F4F4F}
\definecolor{purplestrings}{HTML}{AD00AA}
\definecolor{backgroundcolor}{HTML}{E8E8E8}


%%%% UNIT BASIS VECTORS %%%%
\newcommand{\ihat}{\bm{\hat{\imath}}} % Cartesian i hat (x-direction)
\newcommand{\jhat}{\bm{\hat{\jmath}}} % Cartesian j hat (y-direction)
\newcommand{\khat}{\bm{\hat{k}}} % Cartesian k hat (z-direction)
\newcommand{\rhat}{\bm{\hat{r}}} % Spherical r hat
\newcommand{\phihat}{\bm{\hat{\phi}}} % Spherical phi hat
\newcommand{\thetahat}{\bm{\hat{\theta}}} % Spherical theta hat
\newcommand{\nhat}{\bm{\hat{n}}} % Unit normal vector
\newcommand{\rhohat}{\bm{\hat{\rho}}} % Cylindrical rho hat
\newcommand{\zhat}{\bm{\hat{z}}} % Cylindrical z hat


%%%% COLORS: DEFINITIONS AND COMMANDS %%%%
% Miscellaneous
\definecolor{DARKBLUE}{HTML}{040080}
\definecolor{DARKBROWN}{HTML}{8B4513}
\definecolor{LIGHTBROWN}{HTML}{CD853F}
\definecolor{PINK}{HTML}{D147BD}
\definecolor{LIGHTPINK}{HTML}{DC75CD}
\definecolor{GREENSCREEN}{HTML}{00FF00}
\definecolor{ORANGE}{HTML}{FF862F}
\newcommand{\DARKBLUE}{\color{DARKBLUE}}
\newcommand{\DARKBROWN}{\color{DARKBROWN}}
\newcommand{\LIGHTBROWN}{\color{LIGHTBROWN}}
\newcommand{\PINK}{\color{PINK}}
\newcommand{\LIGHTPINK}{\color{LIGHTPINK}}
\newcommand{\GREENSCREEN}{\color{GREENSCREEN}}
\newcommand{\ORANGE}{\color{ORANGE}}
% Blue
\definecolor{BLUEE}{HTML}{1C758A}
\definecolor{BLUED}{HTML}{29ABCA}
\definecolor{BLUEC}{HTML}{58C4DD}
\definecolor{BLUEB}{HTML}{9CDCEB}
\definecolor{BLUEA}{HTML}{C7E9F1}
\definecolor{BLUE}{HTML}{0000FF}
\newcommand{\BLUEE}{\color{BLUEE}}
\newcommand{\BLUED}{\color{BLUED}}
\newcommand{\BLUEC}{\color{BLUEC}}
\newcommand{\BLUEB}{\color{BLUEB}}
\newcommand{\BLUEA}{\color{BLUEA}}
\newcommand{\BLUE}{\color{BLUE}}
% Teal
\definecolor{TEALE}{HTML}{49A88F}
\definecolor{TEALD}{HTML}{55C1A7}
\definecolor{TEALC}{HTML}{5CD0B3}
\definecolor{TEALB}{HTML}{76DDC0}
\definecolor{TEALA}{HTML}{ACEAD7}
\definecolor{TEAL}{HTML}{00FFFF}
\newcommand{\TEALE}{\color{TEALE}}
\newcommand{\TEALD}{\color{TEALD}}
\newcommand{\TEALC}{\color{TEALC}}
\newcommand{\TEALB}{\color{TEALB}}
\newcommand{\TEALA}{\color{TEALA}}
\newcommand{\TEAL}{\color{TEAL}}
% Green
\definecolor{GREENE}{HTML}{699C52}
\definecolor{GREEND}{HTML}{77B05D}
\definecolor{GREENC}{HTML}{83C167}
\definecolor{GREENB}{HTML}{A6CF8C}
\definecolor{GREENA}{HTML}{C9E2AE}
\definecolor{GREEN}{HTML}{00FF00}
\newcommand{\GREENE}{\color{GREENE}}
\newcommand{\GREEND}{\color{GREEND}}
\newcommand{\GREENC}{\color{GREENC}}
\newcommand{\GREENB}{\color{GREENB}}
\newcommand{\GREENA}{\color{GREENA}}
\newcommand{\GREEN}{\color{GREEN}}
% Yellow
\definecolor{YELLOWE}{HTML}{E8C11C}
\definecolor{YELLOWD}{HTML}{F4D345}
\definecolor{YELLOWC}{HTML}{FFFF00}
\definecolor{YELLOWB}{HTML}{FFEA94}
\definecolor{YELLOWA}{HTML}{FFF1B6}
\definecolor{YELLOW}{HTML}{FFFF00}
\newcommand{\YELLOWE}{\color{YELLOWE}}
\newcommand{\YELLOWD}{\color{YELLOWD}}
\newcommand{\YELLOWC}{\color{YELLOWC}}
\newcommand{\YELLOWB}{\color{YELLOWB}}
\newcommand{\YELLOWA}{\color{YELLOWA}}
\newcommand{\YELLOW}{\color{YELLOW}}
% Gold
\definecolor{GOLDE}{HTML}{C78D46}
\definecolor{GOLDD}{HTML}{E1A158}
\definecolor{GOLDC}{HTML}{F0AC5F}
\definecolor{GOLDB}{HTML}{F9B775}
\definecolor{GOLDA}{HTML}{F7C797}
\newcommand{\GOLDE}{\color{GOLDE}}
\newcommand{\GOLDD}{\color{GOLDD}}
\newcommand{\GOLDC}{\color{GOLDC}}
\newcommand{\GOLDB}{\color{GOLDB}}
\newcommand{\GOLDA}{\color{GOLDA}}
% Red
\definecolor{REDE}{HTML}{CF5044}
\definecolor{REDD}{HTML}{E65A4C}
\definecolor{REDC}{HTML}{FC6255}
\definecolor{REDB}{HTML}{FF8080}
\definecolor{REDA}{HTML}{F7A1A3}
\definecolor{RED}{HTML}{FF0000}
\newcommand{\REDE}{\color{REDE}}
\newcommand{\REDD}{\color{REDD}}
\newcommand{\REDC}{\color{REDC}}
\newcommand{\REDB}{\color{REDB}}
\newcommand{\REDA}{\color{REDA}}
\newcommand{\RED}{\color{RED}}
% Maroon
\definecolor{MAROONE}{HTML}{94424F}
\definecolor{MAROOND}{HTML}{A24D61}
\definecolor{MAROONC}{HTML}{C55F73}
\definecolor{MAROONB}{HTML}{EC92AB}
\definecolor{MAROONA}{HTML}{ECABC1}
\newcommand{\MAROONE}{\color{MAROONE}}
\newcommand{\MAROOND}{\color{MAROOND}}
\newcommand{\MAROONC}{\color{MAROONC}}
\newcommand{\MAROONB}{\color{MAROONB}}
\newcommand{\MAROONA}{\color{MAROONA}}
% Purple
\definecolor{PURPLEE}{HTML}{644172}
\definecolor{PURPLED}{HTML}{715582}
\definecolor{PURPLEC}{HTML}{9A72AC}
\definecolor{PURPLEB}{HTML}{B189C6}
\definecolor{PURPLEA}{HTML}{CAA3E8}
\definecolor{PURPLE}{HTML}{FF00FF}
\newcommand{\PURPLEE}{\color{PURPLEE}}
\newcommand{\PURPLED}{\color{PURPLED}}
\newcommand{\PURPLEC}{\color{PURPLEC}}
\newcommand{\PURPLEB}{\color{PURPLEB}}
\newcommand{\PURPLEA}{\color{PURPLEA}}
\newcommand{\PURPLE}{\color{PURPLE}}
% White and Black
\definecolor{WHITE}{HTML}{FFFFFF}
\newcommand{\WHITE}{\color{WHITE}}
\definecolor{BLACK}{HTML}{000000}
\newcommand{\BLACK}{\color{BLACK}}
% Different Grays
\definecolor{LIGHTGRAY}{HTML}{BBBBBB}
\definecolor{GRAY}{HTML}{888888}
\definecolor{DARKGRAY}{HTML}{444444}
\definecolor{DARKERGRAY}{HTML}{222222}
\definecolor{GRAYBROWN}{HTML}{736357}
\newcommand{\LIGHTGRAY}{\color{LIGHTGRAY}}
\newcommand{\GRAY}{\color{GRAY}}
\newcommand{\DARKGRAY}{\color{DARKGRAY}}
\newcommand{\DARKERGRAY}{\color{DARKERGRAY}}
\newcommand{\GRAYBROWN}{\color{GRAYBROWN}}

를 % AMS and mathtools
\usepackage{amsmath,amsthm,amssymb,marvosym,mathrsfs,amsfonts,amscd,mathtools}
% Hyperlinks and URLs
\usepackage{url}
\usepackage{hyperref}
\hypersetup{
	colorlinks,
	citecolor=BLACK,
	filecolor=BLACK,
	linkcolor=BLACK,
	urlcolor=BLACK
}

% Colors
\usepackage[usenames,dvipsnames]{xcolor}
\usepackage{tikz}
\usepackage{tkz-euclide}

% shadowing mdframed
\usepackage[framemethod=tikz]{mdframed}
\usetikzlibrary{shadows}
% Bold math
\usepackage{bm}

%Use Korean Letter when enumerate
\usepackage{dhucs-enumerate}

\usepackage{anyfontsize}

% Bra Ket (Dirac) Notation
\usepackage{braket}

% Slashed characters (e.g. in Dirac equation)
\usepackage{slashed}
\usepackage{pifont} % 원문자 사용시 필요한 패키지

% chapter decoration
\usepackage{type1cm}
\usepackage[explicit]{titlesec}

\titleformat{\chapter}[display]
{\normalfont\Large\rmfamily}
{\sffamily\flushright\fontsize{60}{0}\textbf{\textcolor{blue!40}{{\Huge\chaptername}~\thechapter\vskip0pt\rule{\textwidth}{2pt}}}}{0pt}
{\flushleft\fontsize{30}{0}{#1}\vskip60pt}
\titlespacing*{\chapter}
{0pt}{-40pt}{0pt}

%\usetikzlibrary{shadows}
\usetikzlibrary{shadows.blur}
\usetikzlibrary{shapes.symbols}

% Tcolorbox
\usepackage[most]{tcolorbox}

% Clean SI Units
\usepackage{siunitx}

% Enumerate thingies
\usepackage{enumitem}

% Cancel things out in equations
\usepackage[makeroom]{cancel}

\usepackage{multicol}

% Graphics and figures
\usepackage{graphicx}
\usepackage{wrapfig}
\usepackage{float}

\usepackage{cancel}

% Caption figures and tables
\usepackage{caption,subcaption}

% Generate symbols
\usepackage{textcomp} % Include this line to avoid output errors
\usepackage{gensymb}

% Make multiple rows in a table
\usepackage{multirow}

% Booktabs tables
\usepackage{booktabs}

%\usepackage[utopia,sfscaled]{mathdesign}
% Useful frames
\usepackage{mdframed}

% Comment-out large sections
\usepackage{comment}

% No auto-indent
\setlength{\parindent}{0pt}

% Asymptote - 3D vector graphics
\usepackage{asymptote}

% Tikz Package Stuff
\usepackage{pgf,tikz,pgfplots}
\usepackage{tikz-3dplot}
\usepackage{tabularx}
\usepackage{array}
\usepackage{colortbl}
\tcbuselibrary{skins}
\usepackage{tkz-euclide}

\newcolumntype{Y}{>{\raggedleft\arraybackslash}X}

\tcbset{tab1/.style={fonttitle=\bfseries\large,fontupper=\normalsize\sffamily,
		colback=yellow!10!white,colframe=red!75!black,colbacktitle=Salmon!40!white, halign=center,
		coltitle=black,center title,freelance,frame code={
			\foreach \n in {north east,north west,south east,south west}
			{\path [fill=red!75!black] (interior.\n) circle (3mm); };},}}

\tcbset{tab2/.style={enhanced,fonttitle=\bfseries,fontupper=\normalsize\sffamily, halign=center, box align=center,
		colback=yellow!10!white,colframe=red!50!black,colbacktitle=Salmon!40!white,
		coltitle=black,center title}}


% Use various tikz libraries
\usetikzlibrary{decorations.pathmorphing, decorations.markings, decorations.pathreplacing, patterns} % Decorate paths!
\usetikzlibrary{calc, patterns, shapes.geometric, positioning, through, intersections}
\usetikzlibrary{scopes}
\usetikzlibrary{angles, quotes}
\usetikzlibrary{svg.path}
\usetikzlibrary{arrows, arrows.meta}
\usetikzlibrary{fadings}
% pgfplots package settings
\pgfplotsset{compat=1.15}
% \pgfplotsset{width=10cm,compat=1.9} % Taken from latest overleaf.
% plot arc easily
\def\centerarc[#1](#2)(#3:#4:#5)% Syntax: [draw options] (center) (initial angle:final angle:radius)
{ \draw[#1] ($(#2)+({#5*cos(#3)},{#5*sin(#3)})$) arc (#3:#4:#5); }

% Awesome circled numbers
\newcommand*\circled[4]{\tikz[baseline=(char.base)]{\node[shape=circle, fill=#2, draw=#3, text=#4, inner sep=2pt] (char) {#1};}}

% Control size of text
\usepackage{relsize}

% Extend conditional commands
\usepackage{xifthen}
\usepackage{xcolor}
\definecolor{termcolor}{cmyk}{.21,.97,.0,.0}
\definecolor{darkred}{cmyk}{.27,1,1,.32}
\definecolor{darkblue}{cmyk}{1,.98,.10,.11}
\definecolor{darkgreen}{cmyk}{.29,0,87,0}
\definecolor{darkmycolor}{cmyk}{99,59,22,3}
\definecolor{for_eyes}{RGB}{253,247,228}
%change color of math equation
%\everymath{\color{darkred}}
% Scale math by size
\newcommand*{\Scale}[2][4]{\scalebox{#1}{\ensuremath{#2}}}

% Big integrals
\usepackage{bigints}

% Number equations within sections
\numberwithin{equation}{section}

% Generate blind text
\usepackage{blindtext}

% Useful symbols
\usepackage{marvosym}

\newcounter{problem}[section]
\newcounter{example}[section]

% cancel 색상 변경
\newcommand\Ccancel[2][black]{\renewcommand\CancelColor{\color{#1}}\cancel{#2}}

%%%% 원문자
\newcommand*\ocircled[1]{\tikz[baseline=(char.base)]{
		\node[shape=circle,draw,inner sep=2pt] (char) {#1};}}
	
%%%%% 보기 스타일 %%%%%
\usepackage{tabu}
\newcommand{\questwo}[2]{
	\vskip 6pt
	\noindent\begin{tabu}{X[0.2] X[6] X[0.2] X[6]}
		(1)&$#1$ &(2) &$#2$
	\end{tabu}
}
\newcommand{\questhree}[3]{
	\vskip 3pt
	\noindent\begin{tabu}{X[0.2] X[6] X[0.2] X[6] X[0.2] X[6]}
		(1)&$#1$ &(2) &$#2$ &(3) & $#3$
	\end{tabu}
	\vskip 5pt
}
\newcommand{\quesfour}[4]{
	\vskip 6pt
	\noindent\begin{tabu}{X[0.2] X[6] X[0.2] X[6]}
		(1)&$#1$ &(2) &$#2$\\
		(3)&$#3$ &(4) &$#4$
	\end{tabu}
}
\newcommand{\quesfive}[5]{
	\vskip 6pt
	\noindent\begin{tabu}{X[0.2] X[3] X[0.2] X[3] X[0.2] X[3]}
		(1)&$#1$ &(2) &$#2$ &(3) &$#3$\\
		(4)&$#4$ &(5) &$#5$
	\end{tabu}
}

\newcommand{\oquesfive}[5]{
	\vskip 6pt
	\noindent\begin{tabu}{X[0.2] X[3] X[0.2] X[3] X[0.2] X[3] X[0.2] X[3] X[0.2] X[3]}
		\ding{172}&$#1$ &\ding{173} &$#2$ &\ding{174} &$#3$&	\ding{175}&$#4$ &\ding{176} &$#5$
	\end{tabu}
}
%%%% sample(보기) %%%%
\newenvironment{sample}{\vskip 10pt\noindent\begin{tikzpicture}[yshift=1.5pt]%
		\draw[rounded corners=1ex,overlay,draw=blue] (0pt,-4pt) rectangle (19pt,9pt);
		\node[rectangle,overlay,xshift=9.8pt,yshift=2pt,color=blue] {{\footnotesize\sffamily 보기}};\end{tikzpicture}\phantom{\footnotesize\sffamily보기...}}{\vskip 10pt}
%%%% THEOREMS %%%%
\newenvironment{theorem}[1][\hspace{-0.36em}]
{
	\begin{mdframed}[backgroundcolor=purple!5, align=center, userdefinedwidth=40em, linecolor=purple!30, linewidth=2pt, roundcorner=7pt, innertopmargin=10pt, shadow=true, shadowcolor=black!20, roundcorner=7pt, innerbottommargin=10pt, frametitle = {#1}]
	}
	{
	\end{mdframed}
}

%%%% LEMMAS %%%%
\newenvironment{lemma}[1][\hspace{-0.36em}]
{
	\begin{mdframed}[backgroundcolor=black!8, align=center, userdefinedwidth=40em, topline=false, bottomline = false, leftline = false, rightline = false, frametitle = {#1 보조정리}]
	}
	{
	\end{mdframed}
}

%%%% COROLLARY %%%%
\newenvironment{corollary}[1][\hspace{-0.36em}]
{
	\begin{mdframed}[backgroundcolor=black!8, align=center, userdefinedwidth=40em, topline=false, bottomline = false, leftline = false, rightline = false, frametitle = {#1 따름정리}]
	}
	{
	\end{mdframed}
}

%%%% DEFINITIONS %%%%
\newenvironment{definition}[1][\hspace{-0.36em}]
{
	\begin{mdframed}[backgroundcolor=cyan!14, align=center, userdefinedwidth=40em, linecolor=cyan!60, linewidth=2pt,roundcorner=7pt, innertopmargin=10pt, shadow=true, shadowcolor=black!20, roundcorner=7pt, innerbottommargin=10pt, frametitle = {정의 : #1}]
	}
	{
	\end{mdframed}
}

%%%% PROPOSITION %%%%
\newenvironment{proposition}
{
	\begin{mdframed}[backgroundcolor=black!4, align=center, userdefinedwidth=40em, topline=false, bottomline = false, leftline = false, rightline = false, frametitle = {Proposition}]
	}
	{
	\end{mdframed}
}
%%%% PROBLEM %%%%
\newenvironment{problem}{\refstepcounter{problem}
	\begin{mdframed}[linecolor=blue!35, linewidth=2pt, roundcorner=7pt, innertopmargin=10pt, shadow=true, shadowcolor=black!20, roundcorner=7pt, innerbottommargin=10pt, backgroundcolor=blue!5]
		\noindent 
		\noindent\begin{tikzpicture}[overlay,xshift=6pt,yshift=5pt]
			\draw[fill=violet!15,draw=violet!15] (0,0) circle (8pt);
			\draw[fill=violet!15,draw=violet!15] (9pt,0) circle (8pt);
			\node[rectangle,overlay,xshift=4pt] {\color{black}\sffamily\bfseries 문제};
		\end{tikzpicture}{\phantom{.........}\fontspec[Scale=1.1]{TeX Gyre Adventor}\color{darkblue} \theproblem}\hspace{5pt}}{
\end{mdframed}}
%%%% SOLUTION OF PROBLEM %%%%
\newenvironment{psolution}{\begin{description}\item[{\begin{tikzpicture}%
				\draw[rounded corners=1ex,overlay] (-3pt,-3pt) rectangle (20pt,9pt);\end{tikzpicture}\footnotesize\sffamily풀이}\hspace{6pt}]}{\end{description}}
			
%%%% EXAMPLE %%%%		
\newenvironment{example}{\refstepcounter{example}
	\begin{mdframed}[roundcorner=7pt,linecolor=termcolor,linewidth=2pt,innertopmargin=10pt, shadow=true, shadowcolor=black!20, innerbottommargin=10pt, backgroundcolor=termcolor!2]
		\noindent 
		\noindent\begin{tikzpicture}[overlay,xshift=6pt,yshift=5pt]
			\draw[fill=darkred!60,draw=darkred!60] (0,0) circle (7pt);
			\draw[fill=darkred!60,draw=darkred!60] (9pt,0) circle (7pt);
			\node[rectangle,overlay,xshift=4pt] {\color{white}\sffamily\bfseries 예제};
		\end{tikzpicture}{\phantom{.........}\fontspec[Scale=1.1]{TeX Gyre Adventor}\color{darkred} \theexample}\hspace{5pt}}{
\end{mdframed}}		
%%%%%%% SOLUTION OF EXAMPLE %%%%%%%%%
\newenvironment{solution}{\begin{description}\item[{\begin{tikzpicture}%
				\draw[rounded corners=1ex,overlay] (-3pt,-3pt) rectangle (20pt,9pt);\end{tikzpicture}\footnotesize\sffamily풀이}\hspace{6pt}]}{\end{description}}

% 보기 박스 정의 시작
\tcbuselibrary{breakable, skins}
\tcbset{enhanced}
\newtcolorbox{ChoiceBox}[1]{
		enhanced,
		before skip=2ex, after skip=2ex,
		boxrule=0.5pt, colframe=black, colback=white, arc=0.5ex,
		boxsep=0.5ex, top=1.5ex, bottom=1.5ex, left=0.5em, right=o0.5em,
		colbacktitle=white, coltitle=black,
		attach boxed title to top center={xshift=0cm, yshift=-1.5mm},
		boxed title style={size=minimal, enhanced, boxrule=0.25pt, colframe=white},
		breakable=false, title ={< #1 >}
}
% 보기 박스 정의 끝.
	
% Change end-of-proof symbol
\renewcommand\qedsymbol{$\blacksquare$}
%overline
\newcommand{\ovr}[1]{\overline{\textrm{#1}}}
% trigonometric function
\newcommand{\cosrm}[1]{\cos \textrm{#1}}
\newcommand{\sinrm}[1]{\sin \textrm{#1}}
\newcommand{\tanrm}[1]{\tan \textrm{#1}}
\newcommand{\cotrm}[1]{\cot \textrm{#1}}
\newcommand{\cscrm}[1]{\csc \textrm{#1}}
\newcommand{\secrm}[1]{\sec \textrm{#1}}
%%%% BLACKBOARD BOLD %%%%
\newcommand{\bbN}{\mathbb{N}} % Natural numbers
\newcommand{\bbZ}{\mathbb{Z}} % Zahlen
\newcommand{\bbQ}{\mathbb{Q}} % Rational numbers
\newcommand{\bbR}{\mathbb{R}} % Real numbers
\newcommand{\bbC}{\mathbb{C}} % Complex numbers
\DeclareSymbolFont{bbold}{U}{bbold}{m}{n} % Identity matrix
\DeclareSymbolFontAlphabet{\mathbbold}{bbold} % Identity matrix
\newcommand{\identitymatrix}{\mathbbold{1}} % Identity matrix

%%%% CODE LISTING %%%%
\usepackage{listings}
\definecolor{greencomments}{HTML}{00BA00}
\definecolor{graynumbers}{HTML}{4F4F4F}
\definecolor{purplestrings}{HTML}{AD00AA}
\definecolor{backgroundcolor}{HTML}{E8E8E8}


%%%% UNIT BASIS VECTORS %%%%
\newcommand{\ihat}{\bm{\hat{\imath}}} % Cartesian i hat (x-direction)
\newcommand{\jhat}{\bm{\hat{\jmath}}} % Cartesian j hat (y-direction)
\newcommand{\khat}{\bm{\hat{k}}} % Cartesian k hat (z-direction)
\newcommand{\rhat}{\bm{\hat{r}}} % Spherical r hat
\newcommand{\phihat}{\bm{\hat{\phi}}} % Spherical phi hat
\newcommand{\thetahat}{\bm{\hat{\theta}}} % Spherical theta hat
\newcommand{\nhat}{\bm{\hat{n}}} % Unit normal vector
\newcommand{\rhohat}{\bm{\hat{\rho}}} % Cylindrical rho hat
\newcommand{\zhat}{\bm{\hat{z}}} % Cylindrical z hat


%%%% COLORS: DEFINITIONS AND COMMANDS %%%%
% Miscellaneous
\definecolor{DARKBLUE}{HTML}{040080}
\definecolor{DARKBROWN}{HTML}{8B4513}
\definecolor{LIGHTBROWN}{HTML}{CD853F}
\definecolor{PINK}{HTML}{D147BD}
\definecolor{LIGHTPINK}{HTML}{DC75CD}
\definecolor{GREENSCREEN}{HTML}{00FF00}
\definecolor{ORANGE}{HTML}{FF862F}
\newcommand{\DARKBLUE}{\color{DARKBLUE}}
\newcommand{\DARKBROWN}{\color{DARKBROWN}}
\newcommand{\LIGHTBROWN}{\color{LIGHTBROWN}}
\newcommand{\PINK}{\color{PINK}}
\newcommand{\LIGHTPINK}{\color{LIGHTPINK}}
\newcommand{\GREENSCREEN}{\color{GREENSCREEN}}
\newcommand{\ORANGE}{\color{ORANGE}}
% Blue
\definecolor{BLUEE}{HTML}{1C758A}
\definecolor{BLUED}{HTML}{29ABCA}
\definecolor{BLUEC}{HTML}{58C4DD}
\definecolor{BLUEB}{HTML}{9CDCEB}
\definecolor{BLUEA}{HTML}{C7E9F1}
\definecolor{BLUE}{HTML}{0000FF}
\newcommand{\BLUEE}{\color{BLUEE}}
\newcommand{\BLUED}{\color{BLUED}}
\newcommand{\BLUEC}{\color{BLUEC}}
\newcommand{\BLUEB}{\color{BLUEB}}
\newcommand{\BLUEA}{\color{BLUEA}}
\newcommand{\BLUE}{\color{BLUE}}
% Teal
\definecolor{TEALE}{HTML}{49A88F}
\definecolor{TEALD}{HTML}{55C1A7}
\definecolor{TEALC}{HTML}{5CD0B3}
\definecolor{TEALB}{HTML}{76DDC0}
\definecolor{TEALA}{HTML}{ACEAD7}
\definecolor{TEAL}{HTML}{00FFFF}
\newcommand{\TEALE}{\color{TEALE}}
\newcommand{\TEALD}{\color{TEALD}}
\newcommand{\TEALC}{\color{TEALC}}
\newcommand{\TEALB}{\color{TEALB}}
\newcommand{\TEALA}{\color{TEALA}}
\newcommand{\TEAL}{\color{TEAL}}
% Green
\definecolor{GREENE}{HTML}{699C52}
\definecolor{GREEND}{HTML}{77B05D}
\definecolor{GREENC}{HTML}{83C167}
\definecolor{GREENB}{HTML}{A6CF8C}
\definecolor{GREENA}{HTML}{C9E2AE}
\definecolor{GREEN}{HTML}{00FF00}
\newcommand{\GREENE}{\color{GREENE}}
\newcommand{\GREEND}{\color{GREEND}}
\newcommand{\GREENC}{\color{GREENC}}
\newcommand{\GREENB}{\color{GREENB}}
\newcommand{\GREENA}{\color{GREENA}}
\newcommand{\GREEN}{\color{GREEN}}
% Yellow
\definecolor{YELLOWE}{HTML}{E8C11C}
\definecolor{YELLOWD}{HTML}{F4D345}
\definecolor{YELLOWC}{HTML}{FFFF00}
\definecolor{YELLOWB}{HTML}{FFEA94}
\definecolor{YELLOWA}{HTML}{FFF1B6}
\definecolor{YELLOW}{HTML}{FFFF00}
\newcommand{\YELLOWE}{\color{YELLOWE}}
\newcommand{\YELLOWD}{\color{YELLOWD}}
\newcommand{\YELLOWC}{\color{YELLOWC}}
\newcommand{\YELLOWB}{\color{YELLOWB}}
\newcommand{\YELLOWA}{\color{YELLOWA}}
\newcommand{\YELLOW}{\color{YELLOW}}
% Gold
\definecolor{GOLDE}{HTML}{C78D46}
\definecolor{GOLDD}{HTML}{E1A158}
\definecolor{GOLDC}{HTML}{F0AC5F}
\definecolor{GOLDB}{HTML}{F9B775}
\definecolor{GOLDA}{HTML}{F7C797}
\newcommand{\GOLDE}{\color{GOLDE}}
\newcommand{\GOLDD}{\color{GOLDD}}
\newcommand{\GOLDC}{\color{GOLDC}}
\newcommand{\GOLDB}{\color{GOLDB}}
\newcommand{\GOLDA}{\color{GOLDA}}
% Red
\definecolor{REDE}{HTML}{CF5044}
\definecolor{REDD}{HTML}{E65A4C}
\definecolor{REDC}{HTML}{FC6255}
\definecolor{REDB}{HTML}{FF8080}
\definecolor{REDA}{HTML}{F7A1A3}
\definecolor{RED}{HTML}{FF0000}
\newcommand{\REDE}{\color{REDE}}
\newcommand{\REDD}{\color{REDD}}
\newcommand{\REDC}{\color{REDC}}
\newcommand{\REDB}{\color{REDB}}
\newcommand{\REDA}{\color{REDA}}
\newcommand{\RED}{\color{RED}}
% Maroon
\definecolor{MAROONE}{HTML}{94424F}
\definecolor{MAROOND}{HTML}{A24D61}
\definecolor{MAROONC}{HTML}{C55F73}
\definecolor{MAROONB}{HTML}{EC92AB}
\definecolor{MAROONA}{HTML}{ECABC1}
\newcommand{\MAROONE}{\color{MAROONE}}
\newcommand{\MAROOND}{\color{MAROOND}}
\newcommand{\MAROONC}{\color{MAROONC}}
\newcommand{\MAROONB}{\color{MAROONB}}
\newcommand{\MAROONA}{\color{MAROONA}}
% Purple
\definecolor{PURPLEE}{HTML}{644172}
\definecolor{PURPLED}{HTML}{715582}
\definecolor{PURPLEC}{HTML}{9A72AC}
\definecolor{PURPLEB}{HTML}{B189C6}
\definecolor{PURPLEA}{HTML}{CAA3E8}
\definecolor{PURPLE}{HTML}{FF00FF}
\newcommand{\PURPLEE}{\color{PURPLEE}}
\newcommand{\PURPLED}{\color{PURPLED}}
\newcommand{\PURPLEC}{\color{PURPLEC}}
\newcommand{\PURPLEB}{\color{PURPLEB}}
\newcommand{\PURPLEA}{\color{PURPLEA}}
\newcommand{\PURPLE}{\color{PURPLE}}
% White and Black
\definecolor{WHITE}{HTML}{FFFFFF}
\newcommand{\WHITE}{\color{WHITE}}
\definecolor{BLACK}{HTML}{000000}
\newcommand{\BLACK}{\color{BLACK}}
% Different Grays
\definecolor{LIGHTGRAY}{HTML}{BBBBBB}
\definecolor{GRAY}{HTML}{888888}
\definecolor{DARKGRAY}{HTML}{444444}
\definecolor{DARKERGRAY}{HTML}{222222}
\definecolor{GRAYBROWN}{HTML}{736357}
\newcommand{\LIGHTGRAY}{\color{LIGHTGRAY}}
\newcommand{\GRAY}{\color{GRAY}}
\newcommand{\DARKGRAY}{\color{DARKGRAY}}
\newcommand{\DARKERGRAY}{\color{DARKERGRAY}}
\newcommand{\GRAYBROWN}{\color{GRAYBROWN}}

로 수정하시기 바랍니다.

\documentclass[11pt, a4paper]{book}
%\documentclass[11pt,a4paper]{article} 
\usepackage[T1]{fontenc}
\usepackage{kotex}

\usepackage{secdot}
% Set page geometry
\usepackage[margin=2.5cm,headsep=0.5cm]{geometry}
\usepackage{setspace}
\usepackage{versions}

\usepackage{sectsty}
%\includeversion{aftercal}
\excludeversion{aftercal}
\includeversion{nogeo}
%\excludeversion{nogeo}
\includeversion{psol}
%\excludeversion{psol}
% AMS and mathtools
\usepackage{amsmath,amsthm,amssymb,marvosym,mathrsfs,amsfonts,amscd,mathtools}
% Hyperlinks and URLs
\usepackage{url}
\usepackage{hyperref}
\hypersetup{
	colorlinks,
	citecolor=BLACK,
	filecolor=BLACK,
	linkcolor=BLACK,
	urlcolor=BLACK
}

% Colors
\usepackage[usenames,dvipsnames]{xcolor}
\usepackage{tikz}
\usepackage{tkz-euclide}

% shadowing mdframed
\usepackage[framemethod=tikz]{mdframed}
\usetikzlibrary{shadows}
% Bold math
\usepackage{bm}

%Use Korean Letter when enumerate
\usepackage{dhucs-enumerate}

\usepackage{anyfontsize}

% Bra Ket (Dirac) Notation
\usepackage{braket}

% Slashed characters (e.g. in Dirac equation)
\usepackage{slashed}
\usepackage{pifont} % 원문자 사용시 필요한 패키지

% chapter decoration
\usepackage{type1cm}
\usepackage[explicit]{titlesec}

\titleformat{\chapter}[display]
{\normalfont\Large\rmfamily}
{\sffamily\flushright\fontsize{60}{0}\textbf{\textcolor{blue!40}{{\Huge\chaptername}~\thechapter\vskip0pt\rule{\textwidth}{2pt}}}}{0pt}
{\flushleft\fontsize{30}{0}{#1}\vskip60pt}
\titlespacing*{\chapter}
{0pt}{-40pt}{0pt}

%\usetikzlibrary{shadows}
\usetikzlibrary{shadows.blur}
\usetikzlibrary{shapes.symbols}

% Tcolorbox
\usepackage[most]{tcolorbox}

% Clean SI Units
\usepackage{siunitx}

% Enumerate thingies
\usepackage{enumitem}

% Cancel things out in equations
\usepackage[makeroom]{cancel}

\usepackage{multicol}

% Graphics and figures
\usepackage{graphicx}
\usepackage{wrapfig}
\usepackage{float}

\usepackage{cancel}

% Caption figures and tables
\usepackage{caption,subcaption}

% Generate symbols
\usepackage{textcomp} % Include this line to avoid output errors
\usepackage{gensymb}

% Make multiple rows in a table
\usepackage{multirow}

% Booktabs tables
\usepackage{booktabs}

%\usepackage[utopia,sfscaled]{mathdesign}
% Useful frames
\usepackage{mdframed}

% Comment-out large sections
\usepackage{comment}

% No auto-indent
\setlength{\parindent}{0pt}

% Asymptote - 3D vector graphics
\usepackage{asymptote}

% Tikz Package Stuff
\usepackage{pgf,tikz,pgfplots}
\usepackage{tikz-3dplot}
\usepackage{tabularx}
\usepackage{array}
\usepackage{colortbl}
\tcbuselibrary{skins}
\usepackage{tkz-euclide}

\newcolumntype{Y}{>{\raggedleft\arraybackslash}X}

\tcbset{tab1/.style={fonttitle=\bfseries\large,fontupper=\normalsize\sffamily,
		colback=yellow!10!white,colframe=red!75!black,colbacktitle=Salmon!40!white, halign=center,
		coltitle=black,center title,freelance,frame code={
			\foreach \n in {north east,north west,south east,south west}
			{\path [fill=red!75!black] (interior.\n) circle (3mm); };},}}

\tcbset{tab2/.style={enhanced,fonttitle=\bfseries,fontupper=\normalsize\sffamily, halign=center, box align=center,
		colback=yellow!10!white,colframe=red!50!black,colbacktitle=Salmon!40!white,
		coltitle=black,center title}}


% Use various tikz libraries
\usetikzlibrary{decorations.pathmorphing, decorations.markings, decorations.pathreplacing, patterns} % Decorate paths!
\usetikzlibrary{calc, patterns, shapes.geometric, positioning, through, intersections}
\usetikzlibrary{scopes}
\usetikzlibrary{angles, quotes}
\usetikzlibrary{svg.path}
\usetikzlibrary{arrows, arrows.meta}
\usetikzlibrary{fadings}
% pgfplots package settings
\pgfplotsset{compat=1.15}
% \pgfplotsset{width=10cm,compat=1.9} % Taken from latest overleaf.
% plot arc easily
\def\centerarc[#1](#2)(#3:#4:#5)% Syntax: [draw options] (center) (initial angle:final angle:radius)
{ \draw[#1] ($(#2)+({#5*cos(#3)},{#5*sin(#3)})$) arc (#3:#4:#5); }

% Awesome circled numbers
\newcommand*\circled[4]{\tikz[baseline=(char.base)]{\node[shape=circle, fill=#2, draw=#3, text=#4, inner sep=2pt] (char) {#1};}}

% Control size of text
\usepackage{relsize}

% Extend conditional commands
\usepackage{xifthen}
\usepackage{xcolor}
\definecolor{termcolor}{cmyk}{.21,.97,.0,.0}
\definecolor{darkred}{cmyk}{.27,1,1,.32}
\definecolor{darkblue}{cmyk}{1,.98,.10,.11}
\definecolor{darkgreen}{cmyk}{.29,0,87,0}
\definecolor{darkmycolor}{cmyk}{99,59,22,3}
\definecolor{for_eyes}{RGB}{253,247,228}
%change color of math equation
%\everymath{\color{darkred}}
% Scale math by size
\newcommand*{\Scale}[2][4]{\scalebox{#1}{\ensuremath{#2}}}

% Big integrals
\usepackage{bigints}

% Number equations within sections
\numberwithin{equation}{section}

% Generate blind text
\usepackage{blindtext}

% Useful symbols
\usepackage{marvosym}

\newcounter{problem}[section]
\newcounter{example}[section]

% cancel 색상 변경
\newcommand\Ccancel[2][black]{\renewcommand\CancelColor{\color{#1}}\cancel{#2}}

%%%% 원문자
\newcommand*\ocircled[1]{\tikz[baseline=(char.base)]{
		\node[shape=circle,draw,inner sep=2pt] (char) {#1};}}
	
%%%%% 보기 스타일 %%%%%
\usepackage{tabu}
\newcommand{\questwo}[2]{
	\vskip 6pt
	\noindent\begin{tabu}{X[0.2] X[6] X[0.2] X[6]}
		(1)&$#1$ &(2) &$#2$
	\end{tabu}
}
\newcommand{\questhree}[3]{
	\vskip 3pt
	\noindent\begin{tabu}{X[0.2] X[6] X[0.2] X[6] X[0.2] X[6]}
		(1)&$#1$ &(2) &$#2$ &(3) & $#3$
	\end{tabu}
	\vskip 5pt
}
\newcommand{\quesfour}[4]{
	\vskip 6pt
	\noindent\begin{tabu}{X[0.2] X[6] X[0.2] X[6]}
		(1)&$#1$ &(2) &$#2$\\
		(3)&$#3$ &(4) &$#4$
	\end{tabu}
}
\newcommand{\quesfive}[5]{
	\vskip 6pt
	\noindent\begin{tabu}{X[0.2] X[3] X[0.2] X[3] X[0.2] X[3]}
		(1)&$#1$ &(2) &$#2$ &(3) &$#3$\\
		(4)&$#4$ &(5) &$#5$
	\end{tabu}
}

\newcommand{\oquesfive}[5]{
	\vskip 6pt
	\noindent\begin{tabu}{X[0.2] X[3] X[0.2] X[3] X[0.2] X[3] X[0.2] X[3] X[0.2] X[3]}
		\ding{172}&$#1$ &\ding{173} &$#2$ &\ding{174} &$#3$&	\ding{175}&$#4$ &\ding{176} &$#5$
	\end{tabu}
}
%%%% sample(보기) %%%%
\newenvironment{sample}{\vskip 10pt\noindent\begin{tikzpicture}[yshift=1.5pt]%
		\draw[rounded corners=1ex,overlay,draw=blue] (0pt,-4pt) rectangle (19pt,9pt);
		\node[rectangle,overlay,xshift=9.8pt,yshift=2pt,color=blue] {{\footnotesize\sffamily 보기}};\end{tikzpicture}\phantom{\footnotesize\sffamily보기...}}{\vskip 10pt}
%%%% THEOREMS %%%%
\newenvironment{theorem}[1][\hspace{-0.36em}]
{
	\begin{mdframed}[backgroundcolor=purple!5, align=center, userdefinedwidth=40em, linecolor=purple!30, linewidth=2pt, roundcorner=7pt, innertopmargin=10pt, shadow=true, shadowcolor=black!20, roundcorner=7pt, innerbottommargin=10pt, frametitle = {#1}]
	}
	{
	\end{mdframed}
}

%%%% LEMMAS %%%%
\newenvironment{lemma}[1][\hspace{-0.36em}]
{
	\begin{mdframed}[backgroundcolor=black!8, align=center, userdefinedwidth=40em, topline=false, bottomline = false, leftline = false, rightline = false, frametitle = {#1 보조정리}]
	}
	{
	\end{mdframed}
}

%%%% COROLLARY %%%%
\newenvironment{corollary}[1][\hspace{-0.36em}]
{
	\begin{mdframed}[backgroundcolor=black!8, align=center, userdefinedwidth=40em, topline=false, bottomline = false, leftline = false, rightline = false, frametitle = {#1 따름정리}]
	}
	{
	\end{mdframed}
}

%%%% DEFINITIONS %%%%
\newenvironment{definition}[1][\hspace{-0.36em}]
{
	\begin{mdframed}[backgroundcolor=cyan!14, align=center, userdefinedwidth=40em, linecolor=cyan!60, linewidth=2pt,roundcorner=7pt, innertopmargin=10pt, shadow=true, shadowcolor=black!20, roundcorner=7pt, innerbottommargin=10pt, frametitle = {정의 : #1}]
	}
	{
	\end{mdframed}
}

%%%% PROPOSITION %%%%
\newenvironment{proposition}
{
	\begin{mdframed}[backgroundcolor=black!4, align=center, userdefinedwidth=40em, topline=false, bottomline = false, leftline = false, rightline = false, frametitle = {Proposition}]
	}
	{
	\end{mdframed}
}
%%%% PROBLEM %%%%
\newenvironment{problem}{\refstepcounter{problem}
	\begin{mdframed}[linecolor=blue!35, linewidth=2pt, roundcorner=7pt, innertopmargin=10pt, shadow=true, shadowcolor=black!20, roundcorner=7pt, innerbottommargin=10pt, backgroundcolor=blue!5]
		\noindent 
		\noindent\begin{tikzpicture}[overlay,xshift=6pt,yshift=5pt]
			\draw[fill=violet!15,draw=violet!15] (0,0) circle (8pt);
			\draw[fill=violet!15,draw=violet!15] (9pt,0) circle (8pt);
			\node[rectangle,overlay,xshift=4pt] {\color{black}\sffamily\bfseries 문제};
		\end{tikzpicture}{\phantom{.........}\fontspec[Scale=1.1]{TeX Gyre Adventor}\color{darkblue} \theproblem}\hspace{5pt}}{
\end{mdframed}}
%%%% SOLUTION OF PROBLEM %%%%
\newenvironment{psolution}{\begin{description}\item[{\begin{tikzpicture}%
				\draw[rounded corners=1ex,overlay] (-3pt,-3pt) rectangle (20pt,9pt);\end{tikzpicture}\footnotesize\sffamily풀이}\hspace{6pt}]}{\end{description}}
			
%%%% EXAMPLE %%%%		
\newenvironment{example}{\refstepcounter{example}
	\begin{mdframed}[roundcorner=7pt,linecolor=termcolor,linewidth=2pt,innertopmargin=10pt, shadow=true, shadowcolor=black!20, innerbottommargin=10pt, backgroundcolor=termcolor!2]
		\noindent 
		\noindent\begin{tikzpicture}[overlay,xshift=6pt,yshift=5pt]
			\draw[fill=darkred!60,draw=darkred!60] (0,0) circle (7pt);
			\draw[fill=darkred!60,draw=darkred!60] (9pt,0) circle (7pt);
			\node[rectangle,overlay,xshift=4pt] {\color{white}\sffamily\bfseries 예제};
		\end{tikzpicture}{\phantom{.........}\fontspec[Scale=1.1]{TeX Gyre Adventor}\color{darkred} \theexample}\hspace{5pt}}{
\end{mdframed}}		
%%%%%%% SOLUTION OF EXAMPLE %%%%%%%%%
\newenvironment{solution}{\begin{description}\item[{\begin{tikzpicture}%
				\draw[rounded corners=1ex,overlay] (-3pt,-3pt) rectangle (20pt,9pt);\end{tikzpicture}\footnotesize\sffamily풀이}\hspace{6pt}]}{\end{description}}

% 보기 박스 정의 시작
\tcbuselibrary{breakable, skins}
\tcbset{enhanced}
\newtcolorbox{ChoiceBox}[1]{
		enhanced,
		before skip=2ex, after skip=2ex,
		boxrule=0.5pt, colframe=black, colback=white, arc=0.5ex,
		boxsep=0.5ex, top=1.5ex, bottom=1.5ex, left=0.5em, right=o0.5em,
		colbacktitle=white, coltitle=black,
		attach boxed title to top center={xshift=0cm, yshift=-1.5mm},
		boxed title style={size=minimal, enhanced, boxrule=0.25pt, colframe=white},
		breakable=false, title ={< #1 >}
}
% 보기 박스 정의 끝.
	
% Change end-of-proof symbol
\renewcommand\qedsymbol{$\blacksquare$}
%overline
\newcommand{\ovr}[1]{\overline{\textrm{#1}}}
% trigonometric function
\newcommand{\cosrm}[1]{\cos \textrm{#1}}
\newcommand{\sinrm}[1]{\sin \textrm{#1}}
\newcommand{\tanrm}[1]{\tan \textrm{#1}}
\newcommand{\cotrm}[1]{\cot \textrm{#1}}
\newcommand{\cscrm}[1]{\csc \textrm{#1}}
\newcommand{\secrm}[1]{\sec \textrm{#1}}
%%%% BLACKBOARD BOLD %%%%
\newcommand{\bbN}{\mathbb{N}} % Natural numbers
\newcommand{\bbZ}{\mathbb{Z}} % Zahlen
\newcommand{\bbQ}{\mathbb{Q}} % Rational numbers
\newcommand{\bbR}{\mathbb{R}} % Real numbers
\newcommand{\bbC}{\mathbb{C}} % Complex numbers
\DeclareSymbolFont{bbold}{U}{bbold}{m}{n} % Identity matrix
\DeclareSymbolFontAlphabet{\mathbbold}{bbold} % Identity matrix
\newcommand{\identitymatrix}{\mathbbold{1}} % Identity matrix

%%%% CODE LISTING %%%%
\usepackage{listings}
\definecolor{greencomments}{HTML}{00BA00}
\definecolor{graynumbers}{HTML}{4F4F4F}
\definecolor{purplestrings}{HTML}{AD00AA}
\definecolor{backgroundcolor}{HTML}{E8E8E8}


%%%% UNIT BASIS VECTORS %%%%
\newcommand{\ihat}{\bm{\hat{\imath}}} % Cartesian i hat (x-direction)
\newcommand{\jhat}{\bm{\hat{\jmath}}} % Cartesian j hat (y-direction)
\newcommand{\khat}{\bm{\hat{k}}} % Cartesian k hat (z-direction)
\newcommand{\rhat}{\bm{\hat{r}}} % Spherical r hat
\newcommand{\phihat}{\bm{\hat{\phi}}} % Spherical phi hat
\newcommand{\thetahat}{\bm{\hat{\theta}}} % Spherical theta hat
\newcommand{\nhat}{\bm{\hat{n}}} % Unit normal vector
\newcommand{\rhohat}{\bm{\hat{\rho}}} % Cylindrical rho hat
\newcommand{\zhat}{\bm{\hat{z}}} % Cylindrical z hat


%%%% COLORS: DEFINITIONS AND COMMANDS %%%%
% Miscellaneous
\definecolor{DARKBLUE}{HTML}{040080}
\definecolor{DARKBROWN}{HTML}{8B4513}
\definecolor{LIGHTBROWN}{HTML}{CD853F}
\definecolor{PINK}{HTML}{D147BD}
\definecolor{LIGHTPINK}{HTML}{DC75CD}
\definecolor{GREENSCREEN}{HTML}{00FF00}
\definecolor{ORANGE}{HTML}{FF862F}
\newcommand{\DARKBLUE}{\color{DARKBLUE}}
\newcommand{\DARKBROWN}{\color{DARKBROWN}}
\newcommand{\LIGHTBROWN}{\color{LIGHTBROWN}}
\newcommand{\PINK}{\color{PINK}}
\newcommand{\LIGHTPINK}{\color{LIGHTPINK}}
\newcommand{\GREENSCREEN}{\color{GREENSCREEN}}
\newcommand{\ORANGE}{\color{ORANGE}}
% Blue
\definecolor{BLUEE}{HTML}{1C758A}
\definecolor{BLUED}{HTML}{29ABCA}
\definecolor{BLUEC}{HTML}{58C4DD}
\definecolor{BLUEB}{HTML}{9CDCEB}
\definecolor{BLUEA}{HTML}{C7E9F1}
\definecolor{BLUE}{HTML}{0000FF}
\newcommand{\BLUEE}{\color{BLUEE}}
\newcommand{\BLUED}{\color{BLUED}}
\newcommand{\BLUEC}{\color{BLUEC}}
\newcommand{\BLUEB}{\color{BLUEB}}
\newcommand{\BLUEA}{\color{BLUEA}}
\newcommand{\BLUE}{\color{BLUE}}
% Teal
\definecolor{TEALE}{HTML}{49A88F}
\definecolor{TEALD}{HTML}{55C1A7}
\definecolor{TEALC}{HTML}{5CD0B3}
\definecolor{TEALB}{HTML}{76DDC0}
\definecolor{TEALA}{HTML}{ACEAD7}
\definecolor{TEAL}{HTML}{00FFFF}
\newcommand{\TEALE}{\color{TEALE}}
\newcommand{\TEALD}{\color{TEALD}}
\newcommand{\TEALC}{\color{TEALC}}
\newcommand{\TEALB}{\color{TEALB}}
\newcommand{\TEALA}{\color{TEALA}}
\newcommand{\TEAL}{\color{TEAL}}
% Green
\definecolor{GREENE}{HTML}{699C52}
\definecolor{GREEND}{HTML}{77B05D}
\definecolor{GREENC}{HTML}{83C167}
\definecolor{GREENB}{HTML}{A6CF8C}
\definecolor{GREENA}{HTML}{C9E2AE}
\definecolor{GREEN}{HTML}{00FF00}
\newcommand{\GREENE}{\color{GREENE}}
\newcommand{\GREEND}{\color{GREEND}}
\newcommand{\GREENC}{\color{GREENC}}
\newcommand{\GREENB}{\color{GREENB}}
\newcommand{\GREENA}{\color{GREENA}}
\newcommand{\GREEN}{\color{GREEN}}
% Yellow
\definecolor{YELLOWE}{HTML}{E8C11C}
\definecolor{YELLOWD}{HTML}{F4D345}
\definecolor{YELLOWC}{HTML}{FFFF00}
\definecolor{YELLOWB}{HTML}{FFEA94}
\definecolor{YELLOWA}{HTML}{FFF1B6}
\definecolor{YELLOW}{HTML}{FFFF00}
\newcommand{\YELLOWE}{\color{YELLOWE}}
\newcommand{\YELLOWD}{\color{YELLOWD}}
\newcommand{\YELLOWC}{\color{YELLOWC}}
\newcommand{\YELLOWB}{\color{YELLOWB}}
\newcommand{\YELLOWA}{\color{YELLOWA}}
\newcommand{\YELLOW}{\color{YELLOW}}
% Gold
\definecolor{GOLDE}{HTML}{C78D46}
\definecolor{GOLDD}{HTML}{E1A158}
\definecolor{GOLDC}{HTML}{F0AC5F}
\definecolor{GOLDB}{HTML}{F9B775}
\definecolor{GOLDA}{HTML}{F7C797}
\newcommand{\GOLDE}{\color{GOLDE}}
\newcommand{\GOLDD}{\color{GOLDD}}
\newcommand{\GOLDC}{\color{GOLDC}}
\newcommand{\GOLDB}{\color{GOLDB}}
\newcommand{\GOLDA}{\color{GOLDA}}
% Red
\definecolor{REDE}{HTML}{CF5044}
\definecolor{REDD}{HTML}{E65A4C}
\definecolor{REDC}{HTML}{FC6255}
\definecolor{REDB}{HTML}{FF8080}
\definecolor{REDA}{HTML}{F7A1A3}
\definecolor{RED}{HTML}{FF0000}
\newcommand{\REDE}{\color{REDE}}
\newcommand{\REDD}{\color{REDD}}
\newcommand{\REDC}{\color{REDC}}
\newcommand{\REDB}{\color{REDB}}
\newcommand{\REDA}{\color{REDA}}
\newcommand{\RED}{\color{RED}}
% Maroon
\definecolor{MAROONE}{HTML}{94424F}
\definecolor{MAROOND}{HTML}{A24D61}
\definecolor{MAROONC}{HTML}{C55F73}
\definecolor{MAROONB}{HTML}{EC92AB}
\definecolor{MAROONA}{HTML}{ECABC1}
\newcommand{\MAROONE}{\color{MAROONE}}
\newcommand{\MAROOND}{\color{MAROOND}}
\newcommand{\MAROONC}{\color{MAROONC}}
\newcommand{\MAROONB}{\color{MAROONB}}
\newcommand{\MAROONA}{\color{MAROONA}}
% Purple
\definecolor{PURPLEE}{HTML}{644172}
\definecolor{PURPLED}{HTML}{715582}
\definecolor{PURPLEC}{HTML}{9A72AC}
\definecolor{PURPLEB}{HTML}{B189C6}
\definecolor{PURPLEA}{HTML}{CAA3E8}
\definecolor{PURPLE}{HTML}{FF00FF}
\newcommand{\PURPLEE}{\color{PURPLEE}}
\newcommand{\PURPLED}{\color{PURPLED}}
\newcommand{\PURPLEC}{\color{PURPLEC}}
\newcommand{\PURPLEB}{\color{PURPLEB}}
\newcommand{\PURPLEA}{\color{PURPLEA}}
\newcommand{\PURPLE}{\color{PURPLE}}
% White and Black
\definecolor{WHITE}{HTML}{FFFFFF}
\newcommand{\WHITE}{\color{WHITE}}
\definecolor{BLACK}{HTML}{000000}
\newcommand{\BLACK}{\color{BLACK}}
% Different Grays
\definecolor{LIGHTGRAY}{HTML}{BBBBBB}
\definecolor{GRAY}{HTML}{888888}
\definecolor{DARKGRAY}{HTML}{444444}
\definecolor{DARKERGRAY}{HTML}{222222}
\definecolor{GRAYBROWN}{HTML}{736357}
\newcommand{\LIGHTGRAY}{\color{LIGHTGRAY}}
\newcommand{\GRAY}{\color{GRAY}}
\newcommand{\DARKGRAY}{\color{DARKGRAY}}
\newcommand{\DARKERGRAY}{\color{DARKERGRAY}}
\newcommand{\GRAYBROWN}{\color{GRAYBROWN}}

 % 작업하는 파일의 상위 디렉토리에 둬야 하는 파일.
\usetikzlibrary{intersections,decorations.text}
\definecolor{c1}{RGB}{62, 97, 127}
\definecolor{c2}{RGB}{104, 182, 182}
\definecolor{c3}{RGB}{107, 190, 190}
\definecolor{c4}{RGB}{100, 172, 174}
\title{Ces\`{a}ro-Stoltz 정리와 급수의 수렴 판정법}
\author{유익승}
\date{\today}
\newcommand{\plogo}{\fbox{\color{c4}$\mathcal{JBSH}$}}
\usepackage{fancyhdr,lastpage}
\pagestyle{fancy}
\fancyhf{}
\lhead{\color{c4}Ces\`{a}ro-Stoltz and Convergence test of Series}
\rhead{\plogo}
\lfoot{\color{c1}\texttt{전북과학고등학교}}
\rfoot{\color{c1}\pageref{LastPage}페이지 중 \thepage 페이지}
\pagestyle{empty}
\begin{document}
	\begin{tikzpicture}[overlay,remember picture]
		
		% Background color
		\fill[
		black!2]
		(current page.south west) rectangle (current page.north east);
		
		% Rectangles
		\shade[
		left color=Dandelion, 
		right color=Dandelion!40,
		transform canvas ={rotate around ={45:($(current page.north west)+(0,-6)$)}}] 
		($(current page.north west)+(0,-6)$) rectangle ++(9,1.5);
		
		\shade[
		left color=lightgray,
		right color=lightgray!50,
		rounded corners=0.75cm,
		transform canvas ={rotate around ={45:($(current page.north west)+(.5,-10)$)}}]
		($(current page.north west)+(0.5,-10)$) rectangle ++(15,1.5);
		
		\shade[
		left color=lightgray,
		rounded corners=0.3cm,
		transform canvas ={rotate around ={45:($(current page.north west)+(.5,-10)$)}}] ($(current page.north west)+(1.5,-9.55)$) rectangle ++(7,.6);
		
		\shade[
		left color=orange!80,
		right color=orange!60,
		rounded corners=0.4cm,
		transform canvas ={rotate around ={45:($(current page.north)+(-1.5,-3)$)}}]
		($(current page.north)+(-1.5,-3)$) rectangle ++(9,0.8);
		
		\shade[
		left color=red!80,
		right color=red!80,
		rounded corners=0.9cm,
		transform canvas ={rotate around ={45:($(current page.north)+(-3,-8)$)}}] ($(current page.north)+(-3,-8)$) rectangle ++(15,1.8);
		
		\shade[
		left color=orange,
		right color=Dandelion,
		rounded corners=0.9cm,
		transform canvas ={rotate around ={45:($(current page.north west)+(4,-15.5)$)}}]
		($(current page.north west)+(4,-15.5)$) rectangle ++(30,1.8);
		
		\shade[
		left color=RoyalBlue,
		right color=Emerald,
		rounded corners=0.75cm,
		transform canvas ={rotate around ={45:($(current page.north west)+(13,-10)$)}}]
		($(current page.north west)+(13,-10)$) rectangle ++(15,1.5);
		
		\shade[
		left color=lightgray,
		rounded corners=0.3cm,
		transform canvas ={rotate around ={45:($(current page.north west)+(18,-8)$)}}]
		($(current page.north west)+(18,-8)$) rectangle ++(15,0.6);
		
		\shade[
		left color=lightgray,
		rounded corners=0.4cm,
		transform canvas ={rotate around ={45:($(current page.north west)+(19,-5.65)$)}}]
		($(current page.north west)+(19,-5.65)$) rectangle ++(15,0.8);
		
		\shade[
		left color=OrangeRed,
		right color=red!80,
		rounded corners=0.6cm,
		transform canvas ={rotate around ={45:($(current page.north west)+(20,-9)$)}}] 
		($(current page.north west)+(20,-9)$) rectangle ++(14,1.2);
		
		% Year
		\draw[ultra thick,gray]
		($(current page.center)+(5,2)$) -- ++(0,-3cm) 
		node[
		midway,
		left=0.25cm,
		text width=5cm,
		align=right,
		black!75
		]
		{
			{\fontsize{25}{30} \selectfont \bf 심화수학I \\[10pt] 보충자료}
		} 
		node[
		midway,
		right=0.25cm,
		text width=6cm,
		align=left,
		orange]
		{
			{\fontsize{72}{86.4} \selectfont 2021}
		};
		
		% Title
		\node[align=center] at ($(current page.center)+(0,-5)$) 
		{
			{\fontsize{30}{35} \selectfont {{\bf 함수의 극한문제 해결 전략 몇 가지}}} \\[1cm]
			{\fontsize{16}{19.2} \selectfont \textcolor{orange}{ \bf 유익승}}\\[1cm]
			{\fontsize{20}{25} \selectfont \textcolor{blue}{ \bf 전북과학고등학교}}};
	\end{tikzpicture}
	
\setstretch{1.4}
\tikzset{every shadow/.style={opacity=1}}	
%	\maketitle
\newpage

	\chapter{\Huge Ces\`{a}ro-Stolz정리}
	\pagestyle{fancy}
	 \pagenumbering{arabic}
	 \setcounter{page}{1} 
Ces\`{a}ro-Stolz정리는 함수의 극한에서 로피탈 정리와 같은 의미를 갖는, 수열의 극한에서의 정리이다. 함수의 극한에서 로피탈 정리의 활용이 매우 강력한 문제해결의 도구가 되는 것처럼 수열의 극한과 급수의 계산에서 매우 유용하게 사용할 수 있는 정리이다. 
\vspace{1em}
\begin{theorem}[(Ces\`{a}ro-Stolz 1)] 수열 $\left\{x_{n}\right\}$, $\left\{y_{n}\right\}$에 대하여, $\left\{y_{n}\right\}$은 증가수열이며 $\displaystyle \lim_{n\to\infty}y_{n}=\infty$일 때,
\begin{equation*}
	\lim_{n\to\infty}\frac{x_{n+1}-x_{n}}{y_{n+1}-y_{n}}=L
\end{equation*}
이면 $\displaystyle \lim_{n\to\infty}\frac{x_{n}}{y_{n}}=L$ 이다.
\end{theorem}
\begin{proof}
$n\to\infty$일 때, $\frac{x_{n+1}-x_{n}}{y_{n+1}-y_{n}}\to L$이므로 임의의 양의 실수 $\epsilon$에 대하여 어떤 자연수 $n_{0}$이 존재하여 $n\ge n_{0}$인 모든 자연수 $n$에 대하여
\begin{equation*}
L-\frac{\epsilon}{2}<\frac{x_{n+1}-x_{n}}{y_{n+1}-y_{n}}< L +\frac{\epsilon}{2}
\end{equation*}
이 성립한다. $y_{n+1}-y_{n}>0$이므로
\begin{equation*}
\left(L-\frac{\epsilon}{2}\right)\left(y_{n+1}-y_{n}\right)< x_{n+1}-x_{n}<\left(L+\frac{\epsilon}{2}\right)\left(y_{n+1}-y_{n}\right)
\end{equation*}
이고
\begin{align*}
	\left(L-\frac{\epsilon}{2}\right)\left(y_{n_{0}+1}-y_{n_{0}}\right) < x_{n_{0}+1}&-x_{n_{0}}<\left(L+\frac{\epsilon}{2}\right)\left(y_{n_{0}+1}-y_{n_{0}}\right)\\
	\left(L-\frac{\epsilon}{2}\right)\left(y_{n_{0}+2}-y_{n_{0}+1}\right)<x_{n_{0}+2}&-x_{n_{0}+1}<\left(L+\frac{\epsilon}{2}\right)\left(y_{n_{0}+2}-y_{n_{0}+1}\right)\\
	&\vdots \\
	\left(L-\frac{\epsilon}{2}\right)\left(y_{m}-y_{m-1}\right)< x_{m}&-x_{m-1}<\left(L+\frac{\epsilon}{2}\right)\left(y_{m}-y_{m-1}\right)
\end{align*}
이다. 이제 같은 변끼리 더하면, 망원합(telescoping sum)에 의해,
\begin{equation*}
	\left(L-\frac{\epsilon}{2}\right)\left(y_{m}-y_{n_{0}}\right)< x_{m}-x_{n_{0}}<\left(L+\frac{\epsilon}{2}\right)\left(y_{m}-y_{n_{0}}\right)
\end{equation*}
을 얻는다. 양변을 $y_{m}$으로 나누고 변형하면
\begin{equation*}
L -\frac{\epsilon}{2}+\left(-L\frac{y_{n_{0}}}{y_{m}}+\frac{\epsilon}{2}\cdot\frac{y_{n_{0}}}{y_{m}}+\frac{x_{n_{0}}}{y_{m}}\right)<\frac{x_{m}}{y_{m}}< L +\frac{\epsilon}{2}+\left(-L\frac{y_{n_{0}}}{y_{m}}-\frac{\epsilon}{2}\cdot\frac{y_{n_{0}}}{y_{m}}+\frac{x_{n_{0}}}{y_{m}}\right)
\end{equation*}
이다. $y_{n}\to\infty$이므로 $n_{1}> n_{0}$인 자연수 $n_{1}$이 존재하여 $m\ge n_{1}$을 만족하는 모든 자연수 $m$에 대하여 
\begin{equation*}
\left | -L\frac{y_{n_{0}}}{y_{m}}+\frac{\epsilon}{2}\cdot\frac{y_{n_{0}}}{y_{m}}+\frac{x_{n_{0}}}{y_{m}}\right | <\frac{\epsilon}{2}
\end{equation*}을 만족 시키도록  할 수 있다.

즉, $m\ge n_{1}$인 모든 자연수 $m$에 대하여
\begin{equation*}
	L-\epsilon <\frac{x_{m}}{y_{m}}< L +\epsilon
\end{equation*}
이 성립한다. $\epsilon$은 임의의 양의 실수이므로 $\frac{x_{n}}{y_{n}}$은 $L$에 수렴한다.
\end{proof}

\vspace{1em}
  비슷한 방법으로 다음의 정리를 증명할 수 있다.
\begin{theorem}[(Ces\`{a}ro-Stoltz 2)]
	수열 $\left\{x_{n}\right\}$, $\left\{y_{n}\right\}$에 대하여, $\left\{y_{n}\right\}$은 감소수열이며
	\begin{equation*}
	 \lim_{n\to\infty}x_{n}=\lim_{n\to\infty}y_{n}= 0(y_{n}\ne 0)
	\end{equation*}
이고,
	\begin{equation*}
			\lim_{n\to\infty}\frac{x_{n+1}-x_{n}}{y_{n+1}-y_{n}}=L
	\end{equation*}
	이면 $\displaystyle \lim_{n\to\infty}\frac{x_{n}}{y_{n}}=L$ 이다.
\end{theorem}
\vspace{1em}
\begin{example}
 자연수 $n$에 대하여 $\displaystyle \lim_{n\to\infty}\frac{1+3+ 5+\cdots +(2n-1)}{2+4+6+\,\cdots \, + 2n}$의 값은?
\begin{solution}
Ces\`{a}ro-Stolz정리를 적용하기 위해 $x_{n}=1+3+5+\, \cdots \, +(2n-1)$
$y_{n}=2+4+6+\, \cdots \, + 2n$이라 하면 $x_{n+1}-x_{n}=2n+1$, $y_{n+1}-y_{n}=2n+2$이므로 
\begin{equation*}
\lim_{n\to\infty}\frac{x_{n+1}-x_{n}}{y_{n+1}-y_{n}}=\lim_{n\to\infty}\frac{2n+1}{2n+2}=1
\end{equation*}
이다. 따라서  
\[
\lim_{n\to\infty}\frac{1+3+ 5+\,\cdots \, +(2n-1)}{2+4+6+\, \cdots \, + 2n}=1
\]
이다.
\end{solution}
\end{example}
\vspace{1em}
\begin{example}
	자연수 $n$에 대하여 $\displaystyle \lim_{n\to\infty}\frac{n^{3}}{1^{2}+2^{2}+\, \cdots\,  + n^{2}}$의 값은?
\begin{solution}
	 $x_{n}=n^{3}$, $y_{n}=1^{2}+2^{2}+\,\cdots \, + n^{2}$라 하면, 
	\begin{align*}
	\lim_{n\to\infty}\frac{x_{n+1}-x_{n}}{y_{n+1}-y_{n}}&=\lim_{n\to\infty}\frac{(n+1)^{3}-n^{3}}{(n+1)^{2}}\\
	&=\lim_{n\to\infty}\frac{3n^{2}+3n+1}{n^{2}+2n+1}=3
	\end{align*}
	이므로 $\displaystyle \lim_{n\to\infty}\frac{n^{3}}{1^{2}+2^{2}+\, \cdots \,+ n^{2}}=3$이다.
\end{solution}
\end{example}
\vspace{1em}
\begin{example}[(2006 대수능)]
수열 $\left\{a_{n}\right\}$이 모든 자연수 $n$에 대하여 $n<a_{n}<n+1$을 만족시킬 때, $\displaystyle \lim_{n\to\infty}\frac{n^{2}}{a_{1}+a_{2}+\, \cdots \, +a_{n}}$의 값은?
\begin{solution}
$x_{n}=n^{2}$, $y_{n}=a_{1}+a_{2}+ \,\cdots \, + a_{n}$라 하면,
\begin{equation*}
	\frac{x_{n+1}-x_{n}}{y_{n+1}-y_{n}}=\frac{2n+1}{a_{n+1}}
\end{equation*}	
	이고
\begin{equation*}
	\frac{2n+1}{n+2}<\frac{2n+1}{a_{n}}<\frac{2n+1}{n+1}
\end{equation*}
이므로 조임정리(Squeeze Theorem)에 의해 $\displaystyle \lim_{n\to\infty}\frac{2n+1}{a_{n+1}}=2$
이고 따라서
\begin{equation*}
	\lim_{n\to\infty}\frac{n^{2}}{a_{1}+a_{2}+\,\cdots \, +a_{n}}=2
\end{equation*}
이다.
\end{solution}
\end{example}
\vspace{1em}
\begin{problem}
 실수열 $\left\{a_{n}\right\}$에 대하여 $s_{n}=\frac{a_{1}+a_{2}+\,\cdots\, + a_{n}}{n}$라 할 때, $\displaystyle\lim_{n\to\infty}a_{n}=L$이면 $\displaystyle\lim_{n\to\infty}s_{n}=L$임을 보여라.
\processifversion{psol}{%
	\begin{psolution}
		 $x_{n}=a_{1}+a_{2}+\cdots + a_{n}$, $y_{n}=n$이라 하면 $\frac{x_{n+1}-x_{n}}{y_{n+1}-y_{n}}= a_{n+1}$이고 
		
		$\frac{x_{n}}{y_{n}}=\frac{a_{1}+a_{2}+\, \cdots \,+ a_{n}}{n}$이므로 문제가 증명되었다.
	\end{psolution}
}
\end{problem}
\vspace{1em}
\begin{problem}
 $\displaystyle \lim_{n\to\infty}\frac{1 + 2\sqrt{2}+3\sqrt{3}+\,\cdots \, + n\sqrt{n}}{n^{2}\sqrt{n}}$의 값을 구하시오.
\processifversion{psol}{%
	\begin{psolution}
$x_{n}=1 + 2\sqrt{2}+ 3\sqrt{3}+\,\cdots \, + n\sqrt{n}$, $y_{n}= n^{2}\sqrt{n}$이라 하면
	\begin{align*}
		\lim_{n\to\infty}\frac{x_{n+1}-x_{n}}{y_{n+1}-y_{n}}
		& =\lim_{n\to\infty}\frac{(n+1)\sqrt{n+1}}{(n+1)^{2}\sqrt{n+1}-n^{2}\sqrt{n}}\\
		& =\lim_{n\to\infty}\frac{\sqrt{(n+1)^{3}}}{\sqrt{(n+1)^{5}}-\sqrt{n^{5}}}\\
		& =\lim_{n\to\infty}\frac{\sqrt{(n+1)^{3}}\left(\sqrt{(n+1)^{5}}+\sqrt{n^{5}}\right)}{(n+1)^{5}-n^{5}}\\
		& =\lim_{n\to\infty}\frac{n^{4}+4n^{3}+6n^{2}+4n+1+\sqrt{n^{8}+3n^{7}+3n^{6}+n^{5}}}{5n^{4}+10n^{3}+10n^{2}+5n+1}\\
		& =\frac{2}{5}
	\end{align*}
	이다. 따라서 $\displaystyle \lim_{n\to\infty}\frac{x_{n}}{y_{n}}=\frac{2}{5}$이다.
\end{psolution}
}
\end{problem}
\vspace{1em}
\begin{problem}
	 $\displaystyle \lim_{n\to\infty}\sum_{k=1}^{n}\frac{\sqrt{_{n+k}C_{2}}}{n^{2}}$의 값을 구하시오.
	\processifversion{psol}{%
\begin{psolution}
	\begin{equation*}
		\lim_{n\to\infty}\sum_{k=1}^{n}\frac{\sqrt{_{n+k}C_{2}}}{n^{2}}=\displaystyle \lim_{n\to\infty}\frac{\sum_{k=1}^{n}\sqrt{_{n+k}C_{2}}}{n^{2}}
	\end{equation*}
이고 $\displaystyle x_{n}=\sum_{k=1}^{n}\sqrt{_{n+k}C_{k}}$, $y_{n}=n^{2}$이라 하면,
	\begin{align*}
		\lim_{n\to\infty}\frac{x_{n+1}-x_{n}}{y_{n+1}-y_{n}}=\lim_{n\to\infty}\frac{\sqrt{_{2n+1}C_{2}}}{(n+1)^{2}-n^{2}}
		& =\lim_{n\to\infty}\frac{\sqrt{\frac{2n(2n+1)}{2}}}{2n+1}\\
		& =\lim_{n\to\infty}\sqrt{\frac{n}{2n+1}}=\frac{\sqrt{2}}{2}
	\end{align*}
	이므로 $\displaystyle \lim_{n\to\infty}\frac{x_{n}}{y_{n}}=\frac{\sqrt{2}}{2}$이다.
\end{psolution}	
}
\end{problem}
\vspace{1em}
\begin{problem}
수열 $\left\{a_{n}\right\}$의 일반항 $a_{n}$이 $a_{n}=\frac{1+7+13 +\,\cdots \,+(6n-5)}{1+4+7+ \,\cdots \,+(3n-2)}$일 때, $\displaystyle \lim_{n\to\infty}a_{n}$의 값을 구하시오.
\processifversion{psol}{%
	\begin{psolution}
	$x_{n}= 1 + 7 + 13 +\,\cdots \,+(6n-5)$, $y_{n}=1 + 4+ 7 +\,\cdots \, +(3n-2)$라 하면
$a_{n}=\frac{x_{n}}{y_{n}}$이고 
\begin{equation*}
\lim_{n\to\infty}\frac{x_{n+1}-x_{n}}{y_{n+1}-y_{n}}=\frac{6}{3}=2
\end{equation*}
이므로 $\displaystyle \lim_{n\to\infty}a_{n}= 2$이다.
\end{psolution}
}
\end{problem}
\vspace{1em}
\begin{problem}[(2012학년도 6월 모의수능)] 자연수 $n$에 대하여 직선 $x=n$이 두 곡선 $y=2^{x}$, $y=3^{x}$과 만나는 점을 각각 $\textrm{P}_{n}$, $\textrm{Q}_{n}$이라 하자. 삼각형 $\textrm{P}_{n}\textrm{Q}_{n}\textrm{P}_{n-1}$의 넓이를 $S_{n}$이라 하고, $\displaystyle T_{n}=\sum_{k=1}^{n}S_{k}$라 할 때, $\displaystyle \lim_{n\to\infty}\frac{T_{n}}{3^{n}}$의 값은?(단, 점 $\textrm{P}_{0}$의 좌표는 $(0,\: 1)$이다.)
\processifversion{psol}{%
	\begin{psolution}다음 그림에서
		\begin{figure}[H]
			\begin{center}
				\begin{tikzpicture}[scale=1]
					
					\draw[-latex, ultra thick] (-2, 0) -- (3.1, 0) node[below] {\Large $x$} ;
					\draw[-latex, ultra thick] (0, -1) -- (0, 7) node[left] {\Large $y$} ;
					\node[below left] at (0,0) {\large\textrm{O}};
					\node[below left] at (1.1,0) {$n-1$};
					\draw[scale=1, domain=-2:1.7, smooth, variable=\x, blue, ultra thick] plot (\x, {3^(\x)}) node[above, blue]{\large $y=3^{x}$};
					
					\draw[scale =1, domain=-1.8:2.6, smooth, variable=\x, red, ultra thick] plot(\x, {2^\x}) node[above, red]{\large $y=2^{x}$};
					\draw[dotted, ultra thick] (1,-0.5) -- (1,6.7);
					\draw[dotted, ultra thick] (1.6,-0.5) -- (1.6, 6.5);
					
					\node at (1.6,3.0314) [right]{$\textrm{P}_n$}; 
					\node at (1.6,5.8) [right]{$\textrm{Q}_n$}; 
					\node[below right] at (1.6, 0) {$n$};
					
					\draw[fill=red, draw=black] (1, 2) node[below right]{$\textrm{P}_{n-1}$} circle (0.07cm);
					\draw[fill=red, draw=black] (1.6,3.0314) circle (0.07cm);
					\draw[fill=blue, draw=black] (1.6,5.8) circle (0.07cm);
					\draw[ultra thick, line join=round] (1,2) -- (1.6, 3.0314) -- (1.6, 5.8) --cycle;
				\end{tikzpicture}
				\end{center}
			\end{figure}
		삼각형 $\textrm{P}_{n}\textrm{Q}_{n}\textrm{P}_{n-1}$의 넓이가 $S_{n}$이므로 $S_{n}=\frac{1}{2}(3^{n}-2^{n})$이고
	\begin{equation*}
		 \lim_{n\to\infty}\frac{T_{n}}{3^{n}}=\lim_{n\to\infty}\frac{T_{n}-T_{n-1}}{3^{n}-3^{n-1}}=\lim_{n\to\infty}\frac{S_{n}}{2\cdot 3^{n-1}}
	\end{equation*}
이다. 따라서
\begin{equation*}
	\lim_{n\to\infty}\frac{T_{n}}{3^{n}}=\lim_{n\to\infty}\frac{\frac{1}{2}\left(3^{n}-2^{n}\right)}{2\cdot 3^{n-1}}=\frac{3}{4}
\end{equation*}
이다.
\end{psolution}
}
\end{problem}
\vspace{1em}
\begin{theorem}
양의 실수열 $\left\{a_{n}\right\}$에 대하여 $\displaystyle \lim_{n\to\infty}\frac{a_{n+1}}{a_{n}}=l$이면 $\displaystyle \lim_{n\to\infty}\sqrt[n]{a_{n}}=l$이다.
\end{theorem}
\begin{proof} $\ln\sqrt[n]{a_{n}}=\ln a_{n}^{\frac{1}{n}}=\frac{1}{n}\ln a_{n}=\frac{\ln a_{n}}{n}$이다. 이제 $x_{n}=\ln a_{n}$, $y_{n}=n$이라 하고 $x_{n},\: y_{n}$에 Ces\`{a}ro-Stoltz 정리1을 적용하면
\begin{align*}
	\lim_{n\to\infty}\frac{\ln a_{n}}{n}&=\lim_{n\to\infty}\frac{\ln a_{n+1}-\ln a_{n}}{(n+1)-n}\\
	&=\lim_{n\to\infty}\left(\ln a_{n+1}-\ln a_{n}\right)\\
	&=\ln\left(\lim_{n\to\infty}\frac{a_{n+1}}{a_{n}}\right)
\end{align*}
이 성립하고 $\displaystyle \lim_{n\to\infty}\frac{a_{n+1}}{a_{n}}$이 존재하므로 Ces\`{a}ro-Stoltz 정리1에 의해
\begin{equation*}
	\ln\lim_{n\to\infty}\sqrt[n]{a_{n}}=\lim_{n\to\infty}\frac{\ln a_{n}}{n}=\ln\lim_{n\to\infty}\frac{a_{n+1}}{a_{n}}
\end{equation*}
이므로 주어진 정리가 증명되었다.
\end{proof}
\vspace{1em}
\begin{example}
 $\displaystyle \lim_{n\to\infty}\sqrt[n]{\frac{3^{3n}(n!)^{3}}{(3n)!}}$의 값을 구하여라.
\begin{solution}
$a_{n}=\frac{3^{3n}(n!)^{3}}{(3n)!}$이라 하면
\begin{align*}
	\lim_{n\to\infty}\sqrt[n]{a_{n}}
	& =\lim_{n\to\infty}\frac{a_{n+1}}{a_{n}}\\
	& =\lim_{n\to\infty}\frac{3^{3n+3}[(n+1)!]^{3}}{(3n+3)!}\cdot\frac{(3n)!}{3^{3n}(n!)^{3}}\\
	& =\lim_{n\to\infty}\frac{27(n+1)^{3}}{(3n+1)(3n+2)(3n+3))}\\
	& = 1
\end{align*}
이다.
\end{solution}
\end{example}
\vspace{1em}
\begin{problem}
다음 수열의 극한값을 각각 계산하여라.

\phantom{문제 6.  }$a_{n}=\sqrt[n]{\frac{(n+1)!}{(2n+1)!\cdot 3^{n}}}$,  $\quad$ $b_{n}=\sqrt[n]{\frac{((n+1)!)^{2}}{(2n+1)!\cdot 3^{n}}}$
\end{problem}

\chapter{\Huge 무한급수의 수렴성 판정법}
  무한급수 $\displaystyle \sum_{n=1}^{\infty}a_{n}$이 주어졌을 때 우리는 이 급수에 대하여 다음의 질문을 할 수가 있을 것이다.
  \begin{itemize}
  	\item 주어진 무한급수는 수렴하는가?
  	\item 주어진 무한급수가 수렴한다면 어떤 값에 수렴하는가?
  \end{itemize}

그런데 일반적으로 급수의 합을 정확하게 구할 수 있는 경우는 매우 드물다. 예를 들어 무한등비급수나 주어진 급수가 망원합의 형태로 고칠 수 있는 경우 또는 어떤 함수의 정적분 형태로 고칠 수 있는 경우가 거의 전부라고 할 수 있다. 급수의 합을 정확하게 구할 수 없는 경우는 수치적으로 근삿값을 구해야 하는 데 이러한 경우 급수가 수렴하는지의 여부가 중요하다. 따라서 이 장에서는 급수의 수렴 판단법에 대하여 알아 본다.

\section{비교판정법}
 비교 판정법의 기본적인 아이디어는 이미 알려진 급수의 수렴, 발산에 관한 정보를 이용하여 주어진 급수의 수렴, 발산을 파악하는 것이다. 예를 들어 
 \begin{equation*}
 	\sum_{n=1}^{\infty} \frac{1}{2^{n}+1}
 \end{equation*}
의 수렴은 무한등비급수 $\sum_{n=1}^{\infty} \frac{1}{2^n}$가 수렴한다는 사실과 모든 자연수 $n$에 대하여
\begin{equation*}
	\frac{1}{2^{n}+1} < \frac{1}{2^{n}}
\end{equation*}
으로부터 주어진 급수가 수렴함을 파악할 수 있다. 이제 이와 관련된 내용을 좀 더 엄밀하게 증명해 보자.

 임의의 자연수 $n$에 대하여 $a_{n}\geq 0$이면 이 수열의 부분합으로 이루어지는 수열 $S_{n} = \displaystyle \sum_{k=1}^{n}a_{k}$는 증가수열이다. 따라서 모든 자연수 $n$에 대하여 $0 \leq a_{n} \leq b_{n}$이고 $\displaystyle \sum_{n=1}^{\infty}b_{n}$이 $T$로 수렴하면
 \begin{equation*}
 	S_{n} = \sum_{k=1}^{n} a_{k} \le \sum_{k=1}^{n}b_{k} \le T
 \end{equation*}
이므로 $\{S_{n}\}$은 위로 유계인 증가수열이고 따라서 단조수렴 정리에 의해 $\displaystyle \sum_{n=1}^{\infty}a_{n}$은 수렴한다. 한편 이것의 대우 또한 참이므로 무한급수 $\displaystyle \sum_{n=1}^{\infty}a_{n}$가 발산하면 $\displaystyle \sum_{n=1}^{\infty}b_{n}$도 발산한다. 이것을 {\color{red}비교 판정법}이라고 한다.

 임의의 무한급수의 첫째 항부터 임의의 유한 항까지의 합은 무한급수의 수렴 발산에 영향을 주지 않으므로 위의 내용을 일반화하여 정리하면 다음과 같다.
 \vspace{1em}
 
 \begin{theorem}[비교판정법]
  어떤 자연수 $n_{0}$에 대하여 $n\ge n_{0}$일 때 $0 \le a_{n} \le b_{n}$이라고 하자. 그러면 다음이 성립한다.
 	\begin{itemize}
 		\item $\displaystyle \sum_{n=1}^{\infty}b_{n}$이 수렴하면 $\displaystyle \sum_{n=1}^{\infty}a_{n}$도 수렴한다.
 		\item $\displaystyle \sum_{n=1}^{\infty}a_{n}$이 발산하면 $\displaystyle \sum_{n=1}^{\infty}b_{n}$도 발산한다.
 	\end{itemize}
 	\end{theorem}
 \vspace{1em}
 이제 간단한 예제들을 살펴보자.
 
 \begin{example}
 	다음 급수는 수렴하는지 발산하는지 판단하시오.
 	
 \qquad \quad (1) $\displaystyle \sum_{n=1}^{\infty}\frac{1}{n^{2}}$ \qquad \qquad (2) $\displaystyle \sum_{n=1}^{\infty}\frac{1}{n} \sin\left(\frac{1}{n}\right)$ \qquad \qquad (3) $\displaystyle \sum_{n=1}^{\infty}\sin\left(\frac{1}{n}\right)$
 	\begin{solution}
 		\begin{enumerate}[label=(\arabic*)]
 			\item $0 \le \frac{1}{n^{2}} \le \frac{2}{n(n+1)}$이고
 		\begin{equation*}
 			\sum_{n=1}^{\infty} \frac{2}{n(n+1)} = 2\sum_{n=1}^{\infty}\frac{1}{n(n+1)} = 2
 		\end{equation*}
 	이다. 따라서 비교판정법에 의하여 주어진 급수는 수렴한다.
 		\item $0 \le x \le 1$이면  $x \ge \sin(x)$이므로 
 		\begin{equation*}
 			0 \le \frac{1}{n} \sin\left(\frac{1}{n}\right)
 		\end{equation*}
 	이다. (1)에서  $\displaystyle \sum_{n=1}^{\infty}\frac{1}{n^{2}}$이 수렴하므로 비교판정법에 의하여 주어진 급수는 수렴한다.
 		\item 원점과 점 $\left(\frac{\pi}{2}, \, 1\right)$을 연결한 직선 $y=\frac{2}{\pi}x$는 다음 그림과 같이 구간 $\left[0, \, \frac{\pi}{2}\right]$에서 $y=\sin(x)$보다 아래에 있다.
 		\begin{figure}[H]
 			\begin{center}
 				\begin{tikzpicture}[scale=2.5]
 					\draw[-latex, ultra thick] (-0.6, 0) -- (2.2, 0) node[below] {\large $x$} ;
 					\draw[-latex, ultra thick] (0, -0.7) -- (0, 1.8) node[left] {\large $y$} ;
 					\node[below right] at (0,0) {\large \textrm{O}};
 					
 					\draw[scale=1, domain=-0.4:1.8, smooth, variable=\x, blue, ultra thick] plot (\x, {sin(\x r)}) node[below right, blue]{\large $y=\sin(x)$};
 					\draw [very thick, dotted, red, domain=-0.5:1.7] plot (\x, {\x}) node [left]{\large $y=x$};
 					\draw [very thick, dotted, black, domain=-0.5:2.1] plot (\x, {2*\x /pi}) node[above]{\large $y=\frac{2x}{\pi}$};
 					\draw[dotted, very thick] ({pi/2}, 0) node[below]{\large $\frac{\pi}{2}$}  |- (0,1) node[left]{\large $1$} ;
 					
 				\end{tikzpicture}
 			\end{center}
 		\end{figure}
 		즉 $0 \le x \le \frac{\pi}{2}$이면 
 		\begin{equation*}
 			\frac{2}{\pi}x \le \sin(x)
 		\end{equation*}
 	이다. 그런데 $0 \le \frac{1}{n} \le 1$이므로 
 	\begin{equation*}
 		\sum_{n=1}^{\infty} \sin \left(\frac{1}{n}\right) \ge \frac{2}{\pi} \sum_{n=1}^{\infty}\frac{1}{n} =\infty
 	\end{equation*}
 이다. 따라서 비교판정법에 의하여 주어진 급수는 발산한다.
 		\end{enumerate}
 	\end{solution}
 \end{example} 
\vspace{1em}
\begin{example}
	무한급수 $\displaystyle \sum_{n=1}^{\infty}\frac{5}{2n^{2}+4n+2}$가 수렴하는지 발산하는지를 판단하시오.
	\begin{solution}
		임의의 자연수 $n$에 대하여 다음이 성립한다.
		\begin{equation*}
			\frac{5}{2n^{2}+4n+2} < \frac{5}{2n^2}
		\end{equation*}
		이다. 그런데 
		\begin{equation*}
			\sum_{n=1}^{\infty} \frac{5}{2n^{2}} =\frac{5}{2} \sum_{n=1}^{\infty} \frac{1}{n^{2}}
		\end{equation*}
		이다. 그런데 무한급수 $\sum_{n=1}^{\infty}\frac{1}{n^{2}}$이 수렴하므로 비교판정법에 의해 주어진 급수는 수렴한다.
	\end{solution}
\end{example}
\vspace{1em}
\begin{problem}
	다음 급수가 수렴하는지 발산하는지를 판정하시오.
	\begin{equation*}
	 (1) \sum_{n=1}^{\infty} \frac{2n}{1+n^{2}} \qquad \qquad \qquad \qquad \qquad (2) \sum_{n=1}^{\infty} \frac{1}{n !} \qquad \qquad \qquad \qquad \qquad \qquad \qquad 
	\end{equation*}
\end{problem}
\vspace{1em}
\begin{problem}
	무한급수 $\displaystyle \sum_{k=1}^{\infty}\frac{\ln k}{k}$가 수렴하는지 발산하는지를 판단하시오.
\end{problem}
\vspace{1em}

  비교판정법을 적용하다 보면 우리는 어떤 급수가 수렴함을 보일 때는 이 급수의 수열보다 큰 수열로 만들어진 수렴하는 급수를 찾아야 하고, 반대로 어떤 급수가 발산함을 보일 때는 이 급수의 수열보다 작은 수열로 만들어진 발산하는 급수를 찾아야 함을 알 수 있다. 그런데 직관적으로 수렴 또는 발산이 직관적으로 명확한데 반해 비교판정법을 적용하기는 어려운 경우가 있다. 예를 들어 무한급수
  \begin{equation*}
  	\sum_{n=1}^{\infty} \frac{1}{2^{n}-1}
  \end{equation*}
은 직관적으로 수렴하는 것으로 판단되지만 부등식
\begin{equation*}
	\frac{1}{2^{n}-1} > \frac{1}{2^{n}}
\end{equation*}
은 비교판정법에의해 주어진 급수가 수렴함을 판단하는데 도움이 되지 않는다. 이러한 경우에 유용한 수단이 존재한다. 다음의 부등식은 이 경우에 유용하다. 

 \begin{theorem}[극한 비교판정법]
 	무한급수 $\displaystyle \sum_{n=1}^{\infty}a_{n}$, $\displaystyle \sum_{n=1}^{\infty}b_{n}$은 모든 항이 양수이다. 어떤 양의 실수 $c$에 대하여
 	\begin{equation*}
 		\lim\limits_{n \to \infty} \frac{a_{n}}{b_{n}} = c
 	\end{equation*}
 이면 두 급수는 모두 수렴하거나 모두 발산한다.
 \end{theorem}
\begin{proof}
	두 양수 $m$과 $M$가 $m < c<M$을 만족시킨다고 하자. $n$을 크게 하여  $\frac{a_{n}}{b_n}$와 $c$이 원하는 만큼 가깝게 할 수 있으므로 큰 자연수 $N$이 존재하여 $n>N$인 모든 자연수 $n$에 대하여
	\begin{equation*}
		m < \frac{a_{n}}{b_{n}} <M
	\end{equation*} 
이 성립한다. $b_{n}>0$이므로 $n>N$인 모든 자연수 $n$에 대하여
	\begin{equation*}
	mb_{n} < a_{n} < M b_{n}
\end{equation*} 
이다.  무한급수 $\displaystyle \sum_{n=N+1}^{\infty}b_{n}$이 수렴하면  $\displaystyle M \sum_{n=N+1}^{\infty}b_{n}$도 수렴하므로 무한급수 $\displaystyle \sum_{n=N+1}^{\infty}a_{n}$도 수렴한다. 한편  무한급수 $\displaystyle \sum_{n=N+1}^{\infty}b_{n}$이 발산하면  $\displaystyle m\sum_{n=N+1}^{\infty}b_{n}$도 발산하고 따라서 비교판정법에 의해 $\displaystyle \sum_{n=N+1}^{\infty}a_{n}$도 발산한다.
\end{proof}
\vspace{1em}
\begin{example}
	무한급수 $\displaystyle \sum_{n=1}^{\infty}\frac{1}{2^{n}-1}$이 수렴하는지 발산하는지를 판정하시오.
	\begin{solution}
		극한 비교판정법을 적용하기 위하여 $a_{n}=\frac{1}{2^{n}-1}$, $b_{n}=\frac{1}{2^{n}}$이라 하자. 그런데
		\begin{align*}
			\lim_{n \to \infty}\frac{a_{n}}{b_{n}} &= \lim_{n \to \infty}  \frac{1/(2^n -1)}{1/2^n} \\
			&= \lim_{n \to \infty} \frac{2^n}{2^n -1} \\
			& = \lim_{n \to \infty} \frac{1}{1-1/2^n} = 1>0
		\end{align*}
	\end{solution}
이고 무한급수 $\sum_{n=1}^{\infty}1/2^n$이 수렴하는 무한등비급수이므로 주어진 급수는 수렴한다.
\end{example}
\vspace{1em}
\begin{problem}
	다음 무한급수가 수렴하는지 발산하는지를 판정하시오.
	\begin{equation*}
		\sum_{n=1}^{\infty} \frac{2n^2 +3n}{\sqrt{5+n^{5}}}
	\end{equation*}
\end{problem}

\section{교대급수 판정법}

무한급수에서 이웃하는 두 항의 부호가 항상 반대인 급수를 {\color{red}교대급수}라고 한다. 교대급수의 일반항은 양의 수열 $\{a_{n}\}$에 대하여 다음과 같이 쓸 수 있다.
\begin{equation*}
	(-1)^{n-1}a_{n} \text{ 또는 } (-1)^{n}a_{n}
\end{equation*}
이제 다음과 같이 홀수 항까지의 부분합 $T_{n}$과 짝수항까지의 부분합 $U_{n}$을 정의하자.
\begin{align*}
	T_{n} & = a_{1}-a_{2} + a_3 + \, \cdots \, -a_{2n-2} + a_{2n-1} \\
	&= a_1 -(a_2 -a_3 ) - \, \cdots, \,-(a_{2n-2}-a_{2n-1})\\
	U_{n} & =a_{1}-a_{2} + a_3 -a_4 + \, \cdots \, +a_{2n-1} - a_{2n}\\
	&= (a_1 -a_2 ) + (a_3 -a_4 ) + \, \cdots \, + (a_{2n-1}-a_{2n} )
\end{align*}
이제 $T_{n}- U_{n} =a_{2n}\ge 0$이므로 $U_{n} \le T_{n}$이고 모든 자연수 $n$에 대하여 $a_{n} \ge a_{n+1}$이면 $\{T_n\}$, $\{U_{n}\}$은 각각 감소수열, 증가수열이다. 따라서 다음의 부등식을 얻는다.
\begin{equation*}
	U_1 \le U_2 \le \, \cdots \, \le U_n \le T_n \le \, \cdots \, \le T_2 \le T_1
\end{equation*}
이 부등식으로 부터 우리는 $\{T_n\}$은 아래로 유계, $\{U_n\}$은 위로 유계이다. 
따라서 단조수렴 정리에 의하여 $\{T_n\}$과 $\{U_n\}$은  수렴한다. 한편
\begin{equation*}
	\lim\limits_{n \to \infty}(T_n - U_n ) = \lim\limits_{n \to \infty}a_{2n}
\end{equation*}
이므로 $\lim\limits_{n \to \infty}a_n =0$이면 $\lim\limits_{n\to\infty}T_n = \lim\limits_{n\to\infty}U_n $이다. 이것을 정리하면 다음과 같다.
\vspace{1em}
\begin{theorem}[교대급수 판정법]
	모든 자연수 $n$에 대하여 $a_{n} \ge 0$일 때
	\begin{enumerate}
		\item 모든 자연수 $n$에 대하여 $a_{n+1} \le a_n$이고
		\item $\lim\limits_{n\to\infty}a_{n}=0$
	\end{enumerate}
이면 주어진 교대급수 $\displaystyle \sum_{n=1}^{\infty}(-1)^{n-1}a_{n}$은 수렴한다.
\end{theorem}

예를 들어 {\color{red}조화교대급수}
\begin{equation*}
	1-\frac{1}{2} + \frac{1}{3} - \frac{1}{4} + \, \cdots \, + (-1)^{n-1}\frac{1}{n} + \, \cdots
\end{equation*}
는 수렴함을 알 수 있다. 실제로 이 무한급수는 $\ln 2$에 수렴함이 알려져 있다.
\vspace{1em}
\begin{example}
	다음 교대급수의 수렴, 발산을 판정하시오.
	\begin{equation*}
		\sum_{n=1}^{\infty} \frac{(-1)^{n-1}3n}{4n-1}
	\end{equation*}
\begin{solution}
	\begin{equation*}
		\lim\limits_{n\to \infty}a_{n} =\lim\limits_{n\to \infty}\frac{3n}{4n-1} =\frac{3}{4}
	\end{equation*}
이므로 주어진 급수는 수렴하지 않는다.
\end{solution}
\end{example}
\vspace{1em}
\begin{problem}
	다음 급수가 수렴하는지 여부를 판정하시오.
	\begin{enumerate}[label=(\arabic*)]
		\item $\displaystyle \sum_{n=1}^{\infty}(-1)^{n}\left(\sqrt{n^{2}+n}-n\right)$
		\item $\displaystyle \sum_{n=1}^{\infty}(-1)^{n}\frac{n^2}{n^2 +n+1}$
	\end{enumerate}
\end{problem}

교대급수는 $n$항까지의 합이 교대급수가 수렴하는 값의 좋은 근사값이 된다. 즉 다음이 성림한다.
\vspace{1em}
\begin{theorem}[교대급수의 오차의 한계]
	모든 자연수 $n$에 대하여 $a_{n} \ge 0$이고 $\displaystyle \sum_{n=1}^{\infty}(-1)^{n-1}a_n =s$일 때
	\begin{enumerate}[label=\arabic*)]
		\item 모든 자연수 $n$에 대하여 $a_{n+1} \le a_n$이고
		\item $\lim\limits_{n\to\infty}a_{n}=0$
	\end{enumerate}
	이면 
	\begin{equation*}
		\vert R_{n} \vert= \vert s - s_n \vert \le a_{n+1}
	\end{equation*}
이다.
\end{theorem}
\begin{proof}
	$s_n = \displaystyle \sum_{k=1}^{n}(-1)^{k-1}a_{k}$라 하면 
	교대급수 수렴정리에서 우리는 항상 $s_n$과 $s_{n+1}$ 사이에 항상 $s$가 존재함을 알 수 있다. 따라서 다음이 성립한다.
	\begin{equation*}
			\vert R_{n} \vert = \vert s - s_n \vert \le \vert s_{n+1} -s_{n}\vert = a_{n+1}
	\end{equation*}
\end{proof}

\section{비율 판정법(Ratio Test)}
먼저 임의의 무한급수 $\displaystyle \sum_{n=1}^{\infty}a_{n}$에 대하여 우리는 다음의 무한급수를 생각할 수 있다.
\begin{equation*}
	\sum_{n=1}^{\infty}\vert a_n \vert = \vert a_1 \vert + \vert a_2 \vert + \vert a_3 \vert + \, \cdots 
\end{equation*}
이 급수는 처음 주어진 급수의 모든 항에 절댓값을 취하여 얻어지는 급수이다.

\begin{definition}[절대수렴]
	무한급수 $\displaystyle \sum_{n=1}^{\infty}\vert a_n \vert$이 수렴하면 무한급수 $\displaystyle \sum_{n=1}^{\infty}a_{n}$은 {\color{red}절대수렴}한다고 한다.
\end{definition}
\vspace{1em}
\begin{example}
	무한급수 $\displaystyle \sum_{n=1}^{\infty}\frac{(-1)^{n-1}}{n^{2}}$은 절대수렴하고 $\displaystyle \sum_{n=1}^{\infty}\frac{(-1)^{n-1}}{n}$은 절대수렴하지 않음을 보이시오.
	\begin{solution}
		무한급수
		\begin{equation*}
			\sum_{n=1}^{\infty}\left| \frac{(-1)^{n-1}}{n^{2}}\right|  =\sum_{n=1}^{\infty} \frac{1}{1^2} +\frac{1}{2^2} +\frac{1}{3^{2}} +\frac{1}{4^2} + \, \cdots
		\end{equation*}
	이므로 절대수렴한다. 한편 무한급수 $\displaystyle \sum_{n=1}^{\infty}\frac{(-1)^{n-1}}{n}$은 조화급수가 발산하므로 절대수렴하지 않는다.
	\end{solution}
\end{example}
\vspace{1em}
\begin{definition}[조건부 수렴]
	무한급수 $\displaystyle \sum_{n=1}^{\infty}a_{n}$이 수렴하나 절대수렴하지는 않을 때 주어진 급수는 {\color{red}조건부 수렴}한다고 한다.
\end{definition}
\vspace{1em}
\begin{theorem}
	어떤 무한급수 $\displaystyle \sum_{n=1}^{\infty}a_{n}$가 절대수렴하면 이 급수는 수렴한다.
\end{theorem}
\begin{proof}
	부등식
	\begin{equation*}
		0 \le a_n + \vert a_n \vert \le 2\vert a_n \vert
	\end{equation*}
이 성립한다. 따라서 무한급수 $\displaystyle \sum_{n=1}^{\infty}a_{n}$가 절대수렴하면 $\displaystyle \sum_{n=1}^{\infty}\vert a_n \vert$이 수렴하고 따라서 $2 \displaystyle \sum_{n=1}^{\infty}\vert a_n \vert$이 수렴한다. 따라서 비교판정법에 의해 무한급수 $\displaystyle \sum_{n=1}^{\infty}\left(a_n + \vert a_n \vert \right)$는 수렴한다. 그런데
\begin{equation*}
	\sum_{n=1}^{\infty}a_{n} =\sum_{n=1}^{\infty}\left(a_n + \vert a_n \vert \right) - \sum_{n=1}^{\infty}\vert a_n \vert
\end{equation*}
이므로 무한급수 $\displaystyle \sum_{n=1}^{\infty}a_{n}$는 수렴한다.
\end{proof}
\vspace{1em}

\begin{problem}
	무한급수
	\begin{equation*}
		\sum_{n=1}^{\infty}\frac{\cos(n)}{n^{2}} = \frac{\cos(1)}{1^2} +\frac{\cos(2)}{2^2} + \frac{\cos(3)}{2^{2}} + \, \cdots
	\end{equation*}
가 수렴하는지 발산하는지를 판정하시오.
\end{problem}
\vspace{1em}

\begin{theorem}[비율판정법]
	무한급수 $\displaystyle \sum_{n=1}^{\infty}a_{n}$에 대하여 $\displaystyle \lim\limits_{n\to \infty}\left|\frac{a_{n+1}}{a_{n}}\right| =L$이라고 하자.
\begin{enumerate}[label=(\arabic*)]
	\item $L<1$이면 무한급수 $\displaystyle \sum_{n=1}^{\infty}a_{n}$는  절대수렴한다. 따라서 수렴한다.
	\item $L>1$이면 무한급수 $\displaystyle \sum_{n=1}^{\infty}a_{n}$는 발산한다.
	\item $L=1$이면 비율판정법은 주어진 급수의 수렴, 발산을 판정하지못한다. 즉 급수의 수렴, 발산에 대한 어떠한 정보도 주지 않는다.
\end{enumerate}
\end{theorem}
\begin{proof}
	\begin{enumerate}[label=(\arabic*)]
		\item 증명의 기본적인 아이디어는 수렴하는 무한등비급수와 이를 이용한 비교판정법을 적용하는 것이다. $L<1$이므로 $L<r<1$을 만족시키는 실수 $r$을 선택할 수 있다. 그런데 
		\begin{equation*}
			\displaystyle \lim\limits_{n\to \infty}\left|\frac{a_{n+1}}{a_{n}}\right| =L \qquad \text{이고} \qquad L<r
		\end{equation*}
	이므로 충분히 큰 $N$에 대하여 다음이 성립한다.
	\begin{equation*}
		n \ge N \text{인 모든 }n \text{에 대하여 } \left|\frac{a_{n+1}}{a_{n}}\right| <r
	\end{equation*}
즉,
\begin{equation}\label{eqn:geo}
	n \ge N \text{인 모든 }n \text{에 대하여 } \left|a_{n+1}\right| < \left|a_{n}\right| r
\end{equation}
이다. 식 (\ref{eqn:geo})에 $N, \,N+1, \, N+2,\, \ldots$를 순차적으로 대입하면
\begin{align*}
	\left|a_{N+1}\right| &< \left|a_{N}\right| r\\
		\left|a_{N+2}\right| &< \left|a_{N+1}\right| r < \left|a_{N}\right| r^{2}\\
			\left|a_{N+3}\right| &< \left|a_{N+2}\right| r < \left|a_{N}\right| r^{3}
\end{align*}
을 얻고 일반적으로 
\begin{equation}\label{eqn:geoineq}
	k \ge 1\text{인 모든 } k\text{에 대하여} \left|a_{N+k}\right| < \left|a_{N}\right| r^{k}
\end{equation}
이다. 이제 무한등비급수
\begin{equation*}
	\sum_{n=1}^{\infty}\vert a_N \vert r^{k} = \vert a_N \vert r + \vert a_N \vert r^2 + \vert a_N \vert r^3 + \, \cdots 
\end{equation*}
는 $0<r<1$이므로 수렴한다. 따라서 부등식 (\ref{eqn:geoineq})과 비교판정법에 의해 무한급수
\begin{equation*}
	\sum_{n=N+1}^{\infty}\vert a_n \vert =\sum_{k=1}^{\infty}\vert a_{N+k} \vert = \vert a_{N+1} \vert + \vert a_{N+2} \vert + \vert a_{N+3} \vert +\, \cdots 
\end{equation*}
은 수렴한다.
	\item $\vert a_{n+1}/a_{n} \vert \to L>1$ 또는 $\vert a_{n+1}/a_{n} \vert \to \infty$이면 $\vert a_{n+1}/a_{n} \vert$는 적당한 자연수 $N$이 존재하여 $n \ge N$인 모든 자연수 $n$에 대하여 $\vert a_{n+1}/a_{n} \vert >1$이다. 즉 $n \ge N$이면 $\vert a_{n+1}\vert > \vert a_{n} \vert$
이므로 $\lim\limits_{n \to \infty}a_{n} \neq 0$이므로 주어진 급수는 발산한다.
	\end{enumerate}
\end{proof}
\vspace{1em}
\begin{example}
	다음 급수가 수렴하는지 발산하는지 판정하시오.
	\begin{equation*}
		\qquad (1) \sum_{n=1}^{\infty} \frac{2^{n}}{n^{2}} \hspace{2.5cm} (2) \sum_{n=1}^{\infty} \frac{2^n}{n!} \hspace{2.5cm} (3) \sum_{n=1}^{\infty}\frac{n!}{n^{n}} \hspace{3.5cm}
	\end{equation*}
	\begin{solution}
		\begin{enumerate}[label=(\arabic*)]
			\item $\left|\frac{a_{n+1}}{a_{n}}\right|=\frac{2^{n+1}/(n+1)^{2}}{2^{n}/n^{2}}=\frac{2n^2}{(n+1)^2}$이므로 $\lim\limits_{n \to \infty}\left|\frac{a_{n+1}}{a_{n}}\right|=2>1$이다. 따라서 주어진 급수는 비율판정법에 의해 발산한다.
			\item $\left|\frac{a_{n+1}}{a_{n}}\right|=\frac{2^{n+1}/(n+1)!}{2^{n}/n!}=\frac{2}{(n+1)}$이므로 $\lim\limits_{n \to \infty}\left|\frac{a_{n+1}}{a_{n}}\right|=0<1$이다. 따라서 주어진 급수는 비율판정법에 의해 수렴한다.
			\item $\left|\frac{a_{n+1}}{a_{n}}\right|=\frac{(n+1)!/(n+1)^{n+1}}{n!/n^{n}}=\frac{n^{n}}{(n+1)^{n+1}} =\frac{1}{\left(1+\frac{1}{n}\right)^n}$이고 $\lim\limits_{n \to \infty}\left(1+\frac{1}{n}\right)^{n}=e$이므로 $\lim\limits_{n \to \infty}\left|\frac{a_{n+1}}{a_{n}}\right|=\frac{1}{e}<1$이다. 따라서 주어진 급수는 비율판정법에 의해 수렴한다.
		\end{enumerate}
	\end{solution}
\end{example}
\vspace{1em}
\begin{problem}
	다음 점화식으로 정의되는 수열 $\{a_n\}$에 대하여 급수  $\displaystyle \sum_{n=1}^{\infty}a_{n}$은 수렴하는지 발산하는지 판정하시오.
	\begin{enumerate}[label=(\arabic*)]
		\item $a_1 =3$, $a_{n+1} =\frac{4n-3}{3n+2} a_{n}$
		\item $a_1 = 5$, $a_{n+1} =\frac{\ln (n+1)}{\sqrt{n}}a_n$
	\end{enumerate}
\end{problem}

\section{근 판정법(Root test)}
 근 판정법은 제곱근이 포함된 무한급수의 수렴, 발산의 판정에 편리한 판정법이다. 이제 다음을 증명하자.
 \begin{theorem}[근 판정법]
 	 무한급수 $\displaystyle \sum_{n=1}^{\infty}a_{n}$에 대하여 $\displaystyle \lim\limits_{n\to \infty}\sqrt[n]{\left|a_{n}\right|} =L$이라고 하자.
 	\begin{enumerate}[label=(\arabic*)]
 		\item $L<1$이면 무한급수 $\displaystyle \sum_{n=1}^{\infty}a_{n}$는 절대수렴한다. 따라서 수렴한다.
 		\item $L>1$이면 무한급수 $\displaystyle \sum_{n=1}^{\infty}a_{n}$는 발산한다.
 		\item $L=1$이면 근 판정법은 주어진 급수의 수렴, 발산을 판정하지못한다. 즉 급수의 수렴, 발산에 대한 어떠한 정보도 주지 않는다.
 	\end{enumerate}
 \end{theorem}
 \begin{proof}
 	\begin{enumerate}[label=(\arabic*)]
 		\item 증명의 기본적인 아이디어는 수렴하는 무한등비급수와 이를 이용한 비교판정법을 적용하는 것이다. $L<1$이므로 $L<r<1$을 만족시키는 실수 $r$을 선택할 수 있다. 그런데 
 		\begin{equation*}
 			\displaystyle \lim\limits_{n\to \infty}\sqrt[n]{\vert a_{n}\vert} =L \qquad \text{이고} \qquad L<r
 		\end{equation*}
 		이므로 충분히 큰 $N$에 대하여 다음이 성립한다.
 		\begin{equation*}
 			n \ge N \text{인 모든 }n \text{에 대하여 } \sqrt[n]{\vert a_{n}\vert} <r
 		\end{equation*}
 		즉,
 		\begin{equation*}
 			n \ge N \text{인 모든 }n \text{에 대하여 } \left|a_{n}\right| < r^{n}
 		\end{equation*}
 		이다. 따라서 비교판정법에 의하여 주어진 급수는 수렴한다.
 		\item (1)의 증명과 비율판정법의 증명을 참고하면 쉽게 증명할 수 있다.
 	\end{enumerate}
 \end{proof}
\vspace{1em}
\begin{example}
	다음 무한급수의 수렴, 발산을 판단하시오.
	\begin{equation*}
		\sum_{n=1}^{\infty}\left(\frac{2n+3}{3n+2}\right)^{n}
	\end{equation*}
\begin{solution}
	\begin{align*}
		a_{n} &=\left(\frac{2n+3}{3n+2}\right)^n \\
		\sqrt[n]{\vert a_{n}\vert} &= \frac{2n+3}{3n+2} = \frac{2+\frac{3}{n}}{3+\frac{2}{n}}  \to \frac{2}{3} <1
	\end{align*}
이므로 근 판정법에 의해 주어진 급수는 수렴한다.
\end{solution}
\end{example}
\vspace{1em}

 어떤 급수가 절대수렴하는가 조건부 수렴하는가는 이 급수의 항들의 순서를 바꾸어 더해도 되는가의 여부와 관계가 있다. 잘 알다시피 유한한 실수들의 합은 순서를 어떻게 바꾸어 더해도 그 값이 일정하다. 그러나 무한급수는 더하는 순서를 바꾸면 그 값이 변한다. 예를 들어 교대 조화급수
 \begin{equation*}
 	\sum_{k=1}^{\infty} \frac{(-1)^{k+1}}{k}=\frac{1}{1}-\frac{1}{2}+\frac{1}{3}-\frac{1}{4}+\frac{1}{5}-\frac{1}{6}+\cdots+\frac{(-1)^{n+1}}{n}+\cdots 
 \end{equation*}
는 $\ln 2$에 수렴함이 알려져 있다. 
\begin{equation*}
	\ln 2=\frac{1}{1}-\frac{1}{2}+\frac{1}{3}-\frac{1}{4}+\frac{1}{5}-\frac{1}{6}+\cdots+\frac{(-1)^{n+1}}{n}+\cdots 
\end{equation*}
그런데 
\begin{align*}
	&1-\frac{1}{2}-\frac{1}{4}+\frac{1}{3}-\frac{1}{6}-\frac{1}{8}+\frac{1}{5} -\frac{1}{10} -\frac{1}{12} + \cdots \\
	=& \left(1-\frac{1}{2}\right) -\frac{1}{4} +\left(\frac{1}{3}-\frac{1}{6}\right)-\frac{1}{8} +\left(\frac{1}{5} -\frac{1}{10}\right) -\frac{1}{12} + \cdots \\
	=& \frac{1}{2} -\frac{1}{4} +\frac{1}{6} -\frac{1}{8} + \frac{1}{10} -\frac{1}{12} +\frac{1}{14} - \frac{1}{16} + \cdots \\
	=& \frac{1}{2} \left( \frac{1}{1}-\frac{1}{2}+\frac{1}{3}-\frac{1}{4}+\frac{1}{5}-\frac{1}{6}+\cdots \right)  =\frac{1}{2} \ln 2
\end{align*}
이다. 즉,
\begin{equation*}
	1-\frac{1}{2}-\frac{1}{4}+\frac{1}{3}-\frac{1}{6}-\frac{1}{8}+\frac{1}{5} -\frac{1}{10} -\frac{1}{12} + {분모}\cdots =\ln 2
\end{equation*}
이므로 무한급수는 더하는 순서를 바꾸면 다른 값이 된다는 것을 알 수 있다. 일반적으로 조건부 수렴하는 무한급수의 항들의 더하는 순서를 적당히 바꾸면 임의의 실수에 수렴하게 할 수 있다는 것이 Riemann에 의해 증명되었다. 이의 증명은 그렇게 어렵지 않으니 독자들에게 연습문제로 남기겠다.
\end{document}