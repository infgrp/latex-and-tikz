\documentclass[a4paper]{article}
\usepackage{secdot}
\usepackage[T1]{fontenc}
\usepackage{amsmath}
\usepackage{setspace}
\usepackage{kotex}
\usepackage{versions}
\includeversion{psol}
%\excludeversion{psol}
\usepackage[margin=3.5cm,headsep=0.5cm]{geometry}
% AMS and mathtools
\usepackage{amsmath,amsthm,amssymb,marvosym,mathrsfs,amsfonts,amscd,mathtools}
% Hyperlinks and URLs
\usepackage{url}
\usepackage{hyperref}
\hypersetup{
	colorlinks,
	citecolor=BLACK,
	filecolor=BLACK,
	linkcolor=BLACK,
	urlcolor=BLACK
}

% Colors
\usepackage[usenames,dvipsnames]{xcolor}
\usepackage{tikz}
\usepackage{tkz-euclide}

% shadowing mdframed
\usepackage[framemethod=tikz]{mdframed}
\usetikzlibrary{shadows}
% Bold math
\usepackage{bm}

%Use Korean Letter when enumerate
\usepackage{dhucs-enumerate}

\usepackage{anyfontsize}

% Bra Ket (Dirac) Notation
\usepackage{braket}

% Slashed characters (e.g. in Dirac equation)
\usepackage{slashed}
\usepackage{pifont} % 원문자 사용시 필요한 패키지

% chapter decoration
\usepackage{type1cm}
\usepackage[explicit]{titlesec}

\titleformat{\chapter}[display]
{\normalfont\Large\rmfamily}
{\sffamily\flushright\fontsize{60}{0}\textbf{\textcolor{blue!40}{{\Huge\chaptername}~\thechapter\vskip0pt\rule{\textwidth}{2pt}}}}{0pt}
{\flushleft\fontsize{30}{0}{#1}\vskip60pt}
\titlespacing*{\chapter}
{0pt}{-40pt}{0pt}

%\usetikzlibrary{shadows}
\usetikzlibrary{shadows.blur}
\usetikzlibrary{shapes.symbols}

% Tcolorbox
\usepackage[most]{tcolorbox}

% Clean SI Units
\usepackage{siunitx}

% Enumerate thingies
\usepackage{enumitem}

% Cancel things out in equations
\usepackage[makeroom]{cancel}

\usepackage{multicol}

% Graphics and figures
\usepackage{graphicx}
\usepackage{wrapfig}
\usepackage{float}

\usepackage{cancel}

% Caption figures and tables
\usepackage{caption,subcaption}

% Generate symbols
\usepackage{textcomp} % Include this line to avoid output errors
\usepackage{gensymb}

% Make multiple rows in a table
\usepackage{multirow}

% Booktabs tables
\usepackage{booktabs}

%\usepackage[utopia,sfscaled]{mathdesign}
% Useful frames
\usepackage{mdframed}

% Comment-out large sections
\usepackage{comment}

% No auto-indent
\setlength{\parindent}{0pt}

% Asymptote - 3D vector graphics
\usepackage{asymptote}

% Tikz Package Stuff
\usepackage{pgf,tikz,pgfplots}
\usepackage{tikz-3dplot}
\usepackage{tabularx}
\usepackage{array}
\usepackage{colortbl}
\tcbuselibrary{skins}
\usepackage{tkz-euclide}

\newcolumntype{Y}{>{\raggedleft\arraybackslash}X}

\tcbset{tab1/.style={fonttitle=\bfseries\large,fontupper=\normalsize\sffamily,
		colback=yellow!10!white,colframe=red!75!black,colbacktitle=Salmon!40!white, halign=center,
		coltitle=black,center title,freelance,frame code={
			\foreach \n in {north east,north west,south east,south west}
			{\path [fill=red!75!black] (interior.\n) circle (3mm); };},}}

\tcbset{tab2/.style={enhanced,fonttitle=\bfseries,fontupper=\normalsize\sffamily, halign=center, box align=center,
		colback=yellow!10!white,colframe=red!50!black,colbacktitle=Salmon!40!white,
		coltitle=black,center title}}


% Use various tikz libraries
\usetikzlibrary{decorations.pathmorphing, decorations.markings, decorations.pathreplacing, patterns} % Decorate paths!
\usetikzlibrary{calc, patterns, shapes.geometric, positioning, through, intersections}
\usetikzlibrary{scopes}
\usetikzlibrary{angles, quotes}
\usetikzlibrary{svg.path}
\usetikzlibrary{arrows, arrows.meta}
\usetikzlibrary{fadings}
% pgfplots package settings
\pgfplotsset{compat=1.15}
% \pgfplotsset{width=10cm,compat=1.9} % Taken from latest overleaf.
% plot arc easily
\def\centerarc[#1](#2)(#3:#4:#5)% Syntax: [draw options] (center) (initial angle:final angle:radius)
{ \draw[#1] ($(#2)+({#5*cos(#3)},{#5*sin(#3)})$) arc (#3:#4:#5); }

% Awesome circled numbers
\newcommand*\circled[4]{\tikz[baseline=(char.base)]{\node[shape=circle, fill=#2, draw=#3, text=#4, inner sep=2pt] (char) {#1};}}

% Control size of text
\usepackage{relsize}

% Extend conditional commands
\usepackage{xifthen}
\usepackage{xcolor}
\definecolor{termcolor}{cmyk}{.21,.97,.0,.0}
\definecolor{darkred}{cmyk}{.27,1,1,.32}
\definecolor{darkblue}{cmyk}{1,.98,.10,.11}
\definecolor{darkgreen}{cmyk}{.29,0,87,0}
\definecolor{darkmycolor}{cmyk}{99,59,22,3}
\definecolor{for_eyes}{RGB}{253,247,228}
%change color of math equation
%\everymath{\color{darkred}}
% Scale math by size
\newcommand*{\Scale}[2][4]{\scalebox{#1}{\ensuremath{#2}}}

% Big integrals
\usepackage{bigints}

% Number equations within sections
\numberwithin{equation}{section}

% Generate blind text
\usepackage{blindtext}

% Useful symbols
\usepackage{marvosym}

\newcounter{problem}[section]
\newcounter{example}[section]

% cancel 색상 변경
\newcommand\Ccancel[2][black]{\renewcommand\CancelColor{\color{#1}}\cancel{#2}}

%%%% 원문자
\newcommand*\ocircled[1]{\tikz[baseline=(char.base)]{
		\node[shape=circle,draw,inner sep=2pt] (char) {#1};}}
	
%%%%% 보기 스타일 %%%%%
\usepackage{tabu}
\newcommand{\questwo}[2]{
	\vskip 6pt
	\noindent\begin{tabu}{X[0.2] X[6] X[0.2] X[6]}
		(1)&$#1$ &(2) &$#2$
	\end{tabu}
}
\newcommand{\questhree}[3]{
	\vskip 3pt
	\noindent\begin{tabu}{X[0.2] X[6] X[0.2] X[6] X[0.2] X[6]}
		(1)&$#1$ &(2) &$#2$ &(3) & $#3$
	\end{tabu}
	\vskip 5pt
}
\newcommand{\quesfour}[4]{
	\vskip 6pt
	\noindent\begin{tabu}{X[0.2] X[6] X[0.2] X[6]}
		(1)&$#1$ &(2) &$#2$\\
		(3)&$#3$ &(4) &$#4$
	\end{tabu}
}
\newcommand{\quesfive}[5]{
	\vskip 6pt
	\noindent\begin{tabu}{X[0.2] X[3] X[0.2] X[3] X[0.2] X[3]}
		(1)&$#1$ &(2) &$#2$ &(3) &$#3$\\
		(4)&$#4$ &(5) &$#5$
	\end{tabu}
}

\newcommand{\oquesfive}[5]{
	\vskip 6pt
	\noindent\begin{tabu}{X[0.2] X[3] X[0.2] X[3] X[0.2] X[3] X[0.2] X[3] X[0.2] X[3]}
		\ding{172}&$#1$ &\ding{173} &$#2$ &\ding{174} &$#3$&	\ding{175}&$#4$ &\ding{176} &$#5$
	\end{tabu}
}
%%%% sample(보기) %%%%
\newenvironment{sample}{\vskip 10pt\noindent\begin{tikzpicture}[yshift=1.5pt]%
		\draw[rounded corners=1ex,overlay,draw=blue] (0pt,-4pt) rectangle (19pt,9pt);
		\node[rectangle,overlay,xshift=9.8pt,yshift=2pt,color=blue] {{\footnotesize\sffamily 보기}};\end{tikzpicture}\phantom{\footnotesize\sffamily보기...}}{\vskip 10pt}
%%%% THEOREMS %%%%
\newenvironment{theorem}[1][\hspace{-0.36em}]
{
	\begin{mdframed}[backgroundcolor=purple!5, align=center, userdefinedwidth=40em, linecolor=purple!30, linewidth=2pt, roundcorner=7pt, innertopmargin=10pt, shadow=true, shadowcolor=black!20, roundcorner=7pt, innerbottommargin=10pt, frametitle = {#1}]
	}
	{
	\end{mdframed}
}

%%%% LEMMAS %%%%
\newenvironment{lemma}[1][\hspace{-0.36em}]
{
	\begin{mdframed}[backgroundcolor=black!8, align=center, userdefinedwidth=40em, topline=false, bottomline = false, leftline = false, rightline = false, frametitle = {#1 보조정리}]
	}
	{
	\end{mdframed}
}

%%%% COROLLARY %%%%
\newenvironment{corollary}[1][\hspace{-0.36em}]
{
	\begin{mdframed}[backgroundcolor=black!8, align=center, userdefinedwidth=40em, topline=false, bottomline = false, leftline = false, rightline = false, frametitle = {#1 따름정리}]
	}
	{
	\end{mdframed}
}

%%%% DEFINITIONS %%%%
\newenvironment{definition}[1][\hspace{-0.36em}]
{
	\begin{mdframed}[backgroundcolor=cyan!14, align=center, userdefinedwidth=40em, linecolor=cyan!60, linewidth=2pt,roundcorner=7pt, innertopmargin=10pt, shadow=true, shadowcolor=black!20, roundcorner=7pt, innerbottommargin=10pt, frametitle = {정의 : #1}]
	}
	{
	\end{mdframed}
}

%%%% PROPOSITION %%%%
\newenvironment{proposition}
{
	\begin{mdframed}[backgroundcolor=black!4, align=center, userdefinedwidth=40em, topline=false, bottomline = false, leftline = false, rightline = false, frametitle = {Proposition}]
	}
	{
	\end{mdframed}
}
%%%% PROBLEM %%%%
\newenvironment{problem}{\refstepcounter{problem}
	\begin{mdframed}[linecolor=blue!35, linewidth=2pt, roundcorner=7pt, innertopmargin=10pt, shadow=true, shadowcolor=black!20, roundcorner=7pt, innerbottommargin=10pt, backgroundcolor=blue!5]
		\noindent 
		\noindent\begin{tikzpicture}[overlay,xshift=6pt,yshift=5pt]
			\draw[fill=violet!15,draw=violet!15] (0,0) circle (8pt);
			\draw[fill=violet!15,draw=violet!15] (9pt,0) circle (8pt);
			\node[rectangle,overlay,xshift=4pt] {\color{black}\sffamily\bfseries 문제};
		\end{tikzpicture}{\phantom{.........}\fontspec[Scale=1.1]{TeX Gyre Adventor}\color{darkblue} \theproblem}\hspace{5pt}}{
\end{mdframed}}
%%%% SOLUTION OF PROBLEM %%%%
\newenvironment{psolution}{\begin{description}\item[{\begin{tikzpicture}%
				\draw[rounded corners=1ex,overlay] (-3pt,-3pt) rectangle (20pt,9pt);\end{tikzpicture}\footnotesize\sffamily풀이}\hspace{6pt}]}{\end{description}}
			
%%%% EXAMPLE %%%%		
\newenvironment{example}{\refstepcounter{example}
	\begin{mdframed}[roundcorner=7pt,linecolor=termcolor,linewidth=2pt,innertopmargin=10pt, shadow=true, shadowcolor=black!20, innerbottommargin=10pt, backgroundcolor=termcolor!2]
		\noindent 
		\noindent\begin{tikzpicture}[overlay,xshift=6pt,yshift=5pt]
			\draw[fill=darkred!60,draw=darkred!60] (0,0) circle (7pt);
			\draw[fill=darkred!60,draw=darkred!60] (9pt,0) circle (7pt);
			\node[rectangle,overlay,xshift=4pt] {\color{white}\sffamily\bfseries 예제};
		\end{tikzpicture}{\phantom{.........}\fontspec[Scale=1.1]{TeX Gyre Adventor}\color{darkred} \theexample}\hspace{5pt}}{
\end{mdframed}}		
%%%%%%% SOLUTION OF EXAMPLE %%%%%%%%%
\newenvironment{solution}{\begin{description}\item[{\begin{tikzpicture}%
				\draw[rounded corners=1ex,overlay] (-3pt,-3pt) rectangle (20pt,9pt);\end{tikzpicture}\footnotesize\sffamily풀이}\hspace{6pt}]}{\end{description}}

% 보기 박스 정의 시작
\tcbuselibrary{breakable, skins}
\tcbset{enhanced}
\newtcolorbox{ChoiceBox}[1]{
		enhanced,
		before skip=2ex, after skip=2ex,
		boxrule=0.5pt, colframe=black, colback=white, arc=0.5ex,
		boxsep=0.5ex, top=1.5ex, bottom=1.5ex, left=0.5em, right=o0.5em,
		colbacktitle=white, coltitle=black,
		attach boxed title to top center={xshift=0cm, yshift=-1.5mm},
		boxed title style={size=minimal, enhanced, boxrule=0.25pt, colframe=white},
		breakable=false, title ={< #1 >}
}
% 보기 박스 정의 끝.
	
% Change end-of-proof symbol
\renewcommand\qedsymbol{$\blacksquare$}
%overline
\newcommand{\ovr}[1]{\overline{\textrm{#1}}}
% trigonometric function
\newcommand{\cosrm}[1]{\cos \textrm{#1}}
\newcommand{\sinrm}[1]{\sin \textrm{#1}}
\newcommand{\tanrm}[1]{\tan \textrm{#1}}
\newcommand{\cotrm}[1]{\cot \textrm{#1}}
\newcommand{\cscrm}[1]{\csc \textrm{#1}}
\newcommand{\secrm}[1]{\sec \textrm{#1}}
%%%% BLACKBOARD BOLD %%%%
\newcommand{\bbN}{\mathbb{N}} % Natural numbers
\newcommand{\bbZ}{\mathbb{Z}} % Zahlen
\newcommand{\bbQ}{\mathbb{Q}} % Rational numbers
\newcommand{\bbR}{\mathbb{R}} % Real numbers
\newcommand{\bbC}{\mathbb{C}} % Complex numbers
\DeclareSymbolFont{bbold}{U}{bbold}{m}{n} % Identity matrix
\DeclareSymbolFontAlphabet{\mathbbold}{bbold} % Identity matrix
\newcommand{\identitymatrix}{\mathbbold{1}} % Identity matrix

%%%% CODE LISTING %%%%
\usepackage{listings}
\definecolor{greencomments}{HTML}{00BA00}
\definecolor{graynumbers}{HTML}{4F4F4F}
\definecolor{purplestrings}{HTML}{AD00AA}
\definecolor{backgroundcolor}{HTML}{E8E8E8}


%%%% UNIT BASIS VECTORS %%%%
\newcommand{\ihat}{\bm{\hat{\imath}}} % Cartesian i hat (x-direction)
\newcommand{\jhat}{\bm{\hat{\jmath}}} % Cartesian j hat (y-direction)
\newcommand{\khat}{\bm{\hat{k}}} % Cartesian k hat (z-direction)
\newcommand{\rhat}{\bm{\hat{r}}} % Spherical r hat
\newcommand{\phihat}{\bm{\hat{\phi}}} % Spherical phi hat
\newcommand{\thetahat}{\bm{\hat{\theta}}} % Spherical theta hat
\newcommand{\nhat}{\bm{\hat{n}}} % Unit normal vector
\newcommand{\rhohat}{\bm{\hat{\rho}}} % Cylindrical rho hat
\newcommand{\zhat}{\bm{\hat{z}}} % Cylindrical z hat


%%%% COLORS: DEFINITIONS AND COMMANDS %%%%
% Miscellaneous
\definecolor{DARKBLUE}{HTML}{040080}
\definecolor{DARKBROWN}{HTML}{8B4513}
\definecolor{LIGHTBROWN}{HTML}{CD853F}
\definecolor{PINK}{HTML}{D147BD}
\definecolor{LIGHTPINK}{HTML}{DC75CD}
\definecolor{GREENSCREEN}{HTML}{00FF00}
\definecolor{ORANGE}{HTML}{FF862F}
\newcommand{\DARKBLUE}{\color{DARKBLUE}}
\newcommand{\DARKBROWN}{\color{DARKBROWN}}
\newcommand{\LIGHTBROWN}{\color{LIGHTBROWN}}
\newcommand{\PINK}{\color{PINK}}
\newcommand{\LIGHTPINK}{\color{LIGHTPINK}}
\newcommand{\GREENSCREEN}{\color{GREENSCREEN}}
\newcommand{\ORANGE}{\color{ORANGE}}
% Blue
\definecolor{BLUEE}{HTML}{1C758A}
\definecolor{BLUED}{HTML}{29ABCA}
\definecolor{BLUEC}{HTML}{58C4DD}
\definecolor{BLUEB}{HTML}{9CDCEB}
\definecolor{BLUEA}{HTML}{C7E9F1}
\definecolor{BLUE}{HTML}{0000FF}
\newcommand{\BLUEE}{\color{BLUEE}}
\newcommand{\BLUED}{\color{BLUED}}
\newcommand{\BLUEC}{\color{BLUEC}}
\newcommand{\BLUEB}{\color{BLUEB}}
\newcommand{\BLUEA}{\color{BLUEA}}
\newcommand{\BLUE}{\color{BLUE}}
% Teal
\definecolor{TEALE}{HTML}{49A88F}
\definecolor{TEALD}{HTML}{55C1A7}
\definecolor{TEALC}{HTML}{5CD0B3}
\definecolor{TEALB}{HTML}{76DDC0}
\definecolor{TEALA}{HTML}{ACEAD7}
\definecolor{TEAL}{HTML}{00FFFF}
\newcommand{\TEALE}{\color{TEALE}}
\newcommand{\TEALD}{\color{TEALD}}
\newcommand{\TEALC}{\color{TEALC}}
\newcommand{\TEALB}{\color{TEALB}}
\newcommand{\TEALA}{\color{TEALA}}
\newcommand{\TEAL}{\color{TEAL}}
% Green
\definecolor{GREENE}{HTML}{699C52}
\definecolor{GREEND}{HTML}{77B05D}
\definecolor{GREENC}{HTML}{83C167}
\definecolor{GREENB}{HTML}{A6CF8C}
\definecolor{GREENA}{HTML}{C9E2AE}
\definecolor{GREEN}{HTML}{00FF00}
\newcommand{\GREENE}{\color{GREENE}}
\newcommand{\GREEND}{\color{GREEND}}
\newcommand{\GREENC}{\color{GREENC}}
\newcommand{\GREENB}{\color{GREENB}}
\newcommand{\GREENA}{\color{GREENA}}
\newcommand{\GREEN}{\color{GREEN}}
% Yellow
\definecolor{YELLOWE}{HTML}{E8C11C}
\definecolor{YELLOWD}{HTML}{F4D345}
\definecolor{YELLOWC}{HTML}{FFFF00}
\definecolor{YELLOWB}{HTML}{FFEA94}
\definecolor{YELLOWA}{HTML}{FFF1B6}
\definecolor{YELLOW}{HTML}{FFFF00}
\newcommand{\YELLOWE}{\color{YELLOWE}}
\newcommand{\YELLOWD}{\color{YELLOWD}}
\newcommand{\YELLOWC}{\color{YELLOWC}}
\newcommand{\YELLOWB}{\color{YELLOWB}}
\newcommand{\YELLOWA}{\color{YELLOWA}}
\newcommand{\YELLOW}{\color{YELLOW}}
% Gold
\definecolor{GOLDE}{HTML}{C78D46}
\definecolor{GOLDD}{HTML}{E1A158}
\definecolor{GOLDC}{HTML}{F0AC5F}
\definecolor{GOLDB}{HTML}{F9B775}
\definecolor{GOLDA}{HTML}{F7C797}
\newcommand{\GOLDE}{\color{GOLDE}}
\newcommand{\GOLDD}{\color{GOLDD}}
\newcommand{\GOLDC}{\color{GOLDC}}
\newcommand{\GOLDB}{\color{GOLDB}}
\newcommand{\GOLDA}{\color{GOLDA}}
% Red
\definecolor{REDE}{HTML}{CF5044}
\definecolor{REDD}{HTML}{E65A4C}
\definecolor{REDC}{HTML}{FC6255}
\definecolor{REDB}{HTML}{FF8080}
\definecolor{REDA}{HTML}{F7A1A3}
\definecolor{RED}{HTML}{FF0000}
\newcommand{\REDE}{\color{REDE}}
\newcommand{\REDD}{\color{REDD}}
\newcommand{\REDC}{\color{REDC}}
\newcommand{\REDB}{\color{REDB}}
\newcommand{\REDA}{\color{REDA}}
\newcommand{\RED}{\color{RED}}
% Maroon
\definecolor{MAROONE}{HTML}{94424F}
\definecolor{MAROOND}{HTML}{A24D61}
\definecolor{MAROONC}{HTML}{C55F73}
\definecolor{MAROONB}{HTML}{EC92AB}
\definecolor{MAROONA}{HTML}{ECABC1}
\newcommand{\MAROONE}{\color{MAROONE}}
\newcommand{\MAROOND}{\color{MAROOND}}
\newcommand{\MAROONC}{\color{MAROONC}}
\newcommand{\MAROONB}{\color{MAROONB}}
\newcommand{\MAROONA}{\color{MAROONA}}
% Purple
\definecolor{PURPLEE}{HTML}{644172}
\definecolor{PURPLED}{HTML}{715582}
\definecolor{PURPLEC}{HTML}{9A72AC}
\definecolor{PURPLEB}{HTML}{B189C6}
\definecolor{PURPLEA}{HTML}{CAA3E8}
\definecolor{PURPLE}{HTML}{FF00FF}
\newcommand{\PURPLEE}{\color{PURPLEE}}
\newcommand{\PURPLED}{\color{PURPLED}}
\newcommand{\PURPLEC}{\color{PURPLEC}}
\newcommand{\PURPLEB}{\color{PURPLEB}}
\newcommand{\PURPLEA}{\color{PURPLEA}}
\newcommand{\PURPLE}{\color{PURPLE}}
% White and Black
\definecolor{WHITE}{HTML}{FFFFFF}
\newcommand{\WHITE}{\color{WHITE}}
\definecolor{BLACK}{HTML}{000000}
\newcommand{\BLACK}{\color{BLACK}}
% Different Grays
\definecolor{LIGHTGRAY}{HTML}{BBBBBB}
\definecolor{GRAY}{HTML}{888888}
\definecolor{DARKGRAY}{HTML}{444444}
\definecolor{DARKERGRAY}{HTML}{222222}
\definecolor{GRAYBROWN}{HTML}{736357}
\newcommand{\LIGHTGRAY}{\color{LIGHTGRAY}}
\newcommand{\GRAY}{\color{GRAY}}
\newcommand{\DARKGRAY}{\color{DARKGRAY}}
\newcommand{\DARKERGRAY}{\color{DARKERGRAY}}
\newcommand{\GRAYBROWN}{\color{GRAYBROWN}}


\newcommand{\Z}{\mathbb{Z}}
\newcommand{\vrm}[1]{\overrightarrow{\textrm{#1}}}

\title{미적분에서의 존재성 증명의 한 전략}
\author{infgrp@hanmail.net}
\date{\today}

\begin{document}
\maketitle

\setstretch{1.5}


\section{논의의 기초}

    미적분학에서 존재성 증명에 활용되는 가장 기본적인 정리는 최대-최소 정리와 사잇값의 정리이다. 왜냐하면 이들 정리는 다른 정리에 의존하지 않고 단지 폐구간과 이 구간에서 연속인 함수라는 두 수학적 개념만을 활용하여 증명되는 수학적 사실이기 때문이다. 사실 이 두 정리의 증명은 고등학교 수준을 훨씬 뛰어 넘기 때문에 고등학교에서는 증명하지 않고 그 사실만을, 이들을 기반으로 하는 정리의 증명에 활용한다. 
    
    이들 정리 다음으로 기본적인 정리는 Rolle의 정리이다. Rolle의 정리는 최대-최소 정리를 활용하여 증명된다. 따라서 Rolle의 정리는 최대-최소 정리 다음 수준의 기본적인 정리라 할 수 있다. 이제 Rolle의 정리와 그 증명을 살펴보자. 
\vspace{1em}
\begin{theorem}[Rolle의 정리]
함수 $f:[a,\: b]\to\mathbb{R}$가 구간 $[a,\: b]$에서 연속이고 개구간 $(a,\: b)$에서 미분가능하고
 $f(a)=f(b)$이면 $f^{\prime}(c)=0$을 만족시키는 실수 $c \in(a,\: b)$가 존재한다.
\begin{proof}
 주어진 함수가 폐구간 $[a,\: b]$에서 연속이므로 최대-최소정리에 의하여 이 함수는 이 구간에서 최댓값과 최솟값을 갖는다. 만약 이 구간의 양 끝점의 한 점에서 최댓값을 갖고 나머지 한 점에서 최솟값을 가지면 이 함수는 상수함수이다. 이 경우 미분계수가 정의되는 구간 $(a, \;b)$위의 모든 점에서 미분계수가 $0$이다. 이제 구간 내부의 한 점 $c$에서 최댓값을 갖는 경우를 생각하자. (최솟값의 경우는 함수 $-f$를 생각하면 되므로 이 경우는 증명을 생략하자.) 

이제 $f(c)$가 최댓값이므로  $h>0$이면 
\[
\frac{f(c+h)-f(c)}{h} \le 0
\]
이다. 따라서 점 $c$에서의 우극한 $f(c^{+})$는 다음과 같다.
\[
f(c^{+}) = \lim\limits_{h \to 0^{+}} \frac{f(c+h)-f(c)}{h} \le 0
\]
마찬가지 방법으로 $f(c)$가 최댓값 이므로  $h<0$이면 
\[
\frac{f(c+h)-f(c)}{h} \ge 0
\]
이다. 따라서 점 $c$에서의 좌극한 $f(c^{-})$는 다음과 같다.
\[
f(c^{-}) = \lim\limits_{h \to 0^{-}} \frac{f(c+h)-f(c)}{h} \ge 0
\]
그런데 점 $c$에서 미분계수가 존재하므로 $f^{+}(c)= f^{-}(c)$이므로 $f^{\prime}(c)=0$이다.
\end{proof}
\end{theorem}
\vspace{1em}

이제 일반적으로 평균값의 정리(Mean Vaule Theorem)로 알려진 Lagrange의 평균값 정리에 대하여 알아보자. 정리의 증명은 주어진 함수와 주어진 구간을 지나는 직선을 이용하고 Rolle의 정리를 적용하여 쉽게 증명할 수 있다.

\begin{theorem}[Lagrange의 평균값 정리]
 함수 $f:[a,\: b]\to\mathrm{R}$가 구간 $[a,\: b]$에서 연속이고 개구간 $(a,\: b)$에서 미분가능하고 $f(a)=f(b)$이면 $f^{\prime}(c)=\frac{f(b)-f(a)}{b-a}$를 만족시키는 실수 $c \in(a,\: b)$가 존재한다.
 \begin{proof}
 	먼저 두 점 $(a, \;f(a))$와 $(b, \;f(g))$를 지나는 직선의 방정식을 $g(x)$라 하면
 	\[
 	g(x) = \frac{f(b)-f(a)}{b-a} \left(x-a\right) + f(a)
 	\]
 	이다. 이제 $h(x) = f(x) - g(x)$라 정의 하면 함수 $h(x)$는 폐구간 $[a,\: b]$에서 연속이고 개구간 $(a,\; b)$에서 미분가능하다. 또 $h(a)=h(b)=0$이므로 함수 $h(x)$에 Rolle의 정리를 적용할 수 있으므로 Rolle의 정리에 의해 $c \in \left(a, \; b\right)$가 존재하여 $h^{\prime}(c)=0$을 만족시킨다. 그런데 
 	\[
 	h^{\prime}(x) = f^{\prime}(x) - \frac{f(b)-f(a)}{b-a}
 	\]
 	이므로 $h^{\prime}(c)=0$이면 $f^{\prime}(c)=\frac{f(b)-f(a)}{b-a}$이므로 정리가 증명되었다.
 \end{proof}
\end{theorem}

Lagrange의 평균값 정리는 뒤에서 존재성 증명의 한 전략을 탐구하고 이 전략을 적용하여 위의 방법과는 조금 다르게 증명할 것이다. 이제 Cauchy의 평균값 정리에 대하여 알아보자.

\vspace{1em}
\begin{theorem}[Cauchy의 평균값 정리]
	두 함수 $f(x), \;g(x)$가 폐구간 $[a, \; b]$에서 연속이고 개구간 $(a, \; b)$에서 미분가능하면 다음을 만족시키는 실수 $c(a<c<b)$가 존재한다.
	\[
	(g(b)-g(a))f^{\prime}(c) = (f(b)-f(a))g^{\prime}(c)
	\]
	\begin{proof}
		$g(a)=g(b)$인 경우는 롤의 정리와 같아지므로 매우 쉽다.  이제 $g(a) \ne g(b)$라 가정하자. 이제 다음과 같이 $h(x)$를 정의하자.
		\[
		h(x) = f(x) - \frac{f(b)-f(a)}{g(b)-g(a)} \cdot g(x)
		\]
		간단한 계산에 의해 $h(a) =h(b)$이고 가정에 의해 $h(x)$는 폐구간 $[a,\;b]$에서 연속이고 개구간 $(a, \; b)$에서 미분가능하므로 롤의 정리에 의해 다음을 만족시키는 $c(a<c<b)$가 존재한다.
		\[
		h^{\prime}(c) =0 = f^{\prime}(c) - \frac{f(b)-f(a)}{g(b)-g(a)} \cdot g^{\prime}(c)
		\]
		즉,
		\[
		(g(b)-g(a))f^{\prime}(c) = (f(b)-f(a))g^{\prime}(c)
		\]
		이 성립한다.
	\end{proof}
\end{theorem}
  Cauchy의 평균값 정리의 증명도 뒤에서 존재성 증명의 전략을 탐구하고 이 전략을 적용하여 다시 증명해 볼 것이다. 이제 Lagrange의 평균값 정리를 활용할 수 있는 다양한 상황에 대하여 탐구해 보자.
  
\section{Lagrange의 평균값 정리와 Cauchy의 평균값 정리의 활용 범위와 예제}

Lagrange의 평균값 정리는 미적분학에서 매우 다양하게 활용된다. 그 구체적인 사례를 열거하면 다음과 같다.
\begin{enumerate}[label=\arabic*)]
	\item 등식의 증명
	\item 부등식의 증명
	\item 함수와 도함수의 성질 탐구
	\item 평균값 정리 문제의 결론 유도를 위한 Lagrange의 평균값 정리의 적용
	\item 방정식의 근의 존재성과 유일성의 판정
	\item 함수의 극한값의 계산에의 응용
\end{enumerate}
이제 이들 사례에 해당하는 예제들을 탐구해 보자.


\begin{problem}
	평균값 정리를 이용하여 다음 등식을 증명하시오.
	\[
	\sin^{-1} x + \cos^{-1} x = \frac{\pi}{2} \quad (-1 \le x \le 1)
	\]
	\processifversion{psol}{%
	\begin{psolution}
			$F(x)=\sin^{-1}x + \cos^{-1}x -\frac{\pi}{2}$라 하자. $F(0)=0$임은 분명하고 $F(x)$는 폐구간 $\left[-1, \; 1\right]$에서 연속이고 개구간 $\left(-1, \;1\right)$에서 미분가능하다. 따라서 평균값 정리에 의해 $\zeta \in \left(x, \;1\right)\left(0<x<1\right)$가 존재하여 
		\[
		\frac{F(x)-F(0)}{x-0} = F ^{\prime}(\zeta)
		\]
		를 만족시킨다. 그런데 $F^{\prime}(x) =\frac{1}{\sqrt{1-x^2}} - \frac{1}{\sqrt{1-x^2}}=0$이므로 $F(\zeta)=0$이고 $F(x)-F(0)=0$, 즉 $F(x)=F(0)$이다. 따라서 
		\[
		F(x) = \sin^{-1}x + \cos^{-1} x -\frac{\pi}{2} =0
		\]
		비슷한 방법으로 $-1<x<0$인 경우도 증명할 수 있고 따라서 문제가 증명되었다.
	\end{psolution}
}
\end{problem}
	\vspace{1em}
	
이제  부등식의 증명에 활용되는 예제를 하나 해결해 보자. 미분 부등식과 관련이 있는 부등식은 평균값의 정리가 거의 사용된다고 보면 된다. 미분 부등식은 미분 방정식의 형태에서 등식이 부등식의 형태로 바뀐 것을 의미한다. 이러한 부등식의 증명에서 평균값의 정리를 사용할 수 있도록 부등식 전체를 대수적으로 재구성하는 것은 매우 중요하다.
	
	\begin{problem}
		$x>0$일 때, 다음 부등식을 증명하시오.
		\[
		\frac{x}{1+x}< \ln(1+x)< x
		\]
		\processifversion{psol}{
\begin{psolution}
	$f(t) =\ln t$라 하면 $f^{\prime}(t) =\frac{1}{t}$이고 $f(1) =\ln 1 =0$이다. 한편 $f(t)$는 폐구간 $\left[1, \;1+x\right](x>0)$에서 평균값의 정리가 성립하므로 $\zeta \in \left(1, \; 1+x\right)$가 존재하여 
	\[
f(1+x) - f(1) = f^{\prime}(\zeta) \left[(1+x)- 1\right]
	\]
	이다. 즉 $\ln (1+x)-\ln 1 = \frac{1}{\zeta}x (1<\zeta<1+x)$. 그런데 $1<\zeta<1+x$이므로
	\begin{align*}
		\frac{1}{1+x} < \frac{1}{\zeta} < 1 & \Leftrightarrow \frac{x}{1+x} < \frac{x}{\zeta}<x \\
		&\Leftrightarrow \frac{x}{1+x} < \ln(1+x) < x
	\end{align*}
이다.
\end{psolution}		
}\end{problem}

\vspace{1em}
이제 어떤 함수와 그 도함수의 관계에 관련된 성질을 탐구하는 데 평균값 정리가 어떻게 활용될 수 있는 가를 탐구해 보자. 다음 문제에 주목해 보자.

\begin{problem}
	함수 $f(x)$가 구간 $\left[0, \;1\right]$에서 연속이며 $\left(0, \;1\right)$에서 미분 가능하고  $f^{\prime}(x)$가 유계이면 이 구간에서 $f(x)$도 유계임을 보이시오.
	\processifversion{psol}{%
\begin{psolution}
	$x_0, \;x \in (0, \;1)$이고 $x_0 <x$라 가정하자. 그러면 $f(x)$는 구간 $\left[x_0,\;x\right]$에서 평균값 정리를 만족시킨다. 즉, $\zeta \in \left(x_0,\;x\right)$가 존재하여
	\[
	f(x) -f(x_0) = f^{\prime}(\zeta)(x-x_0)
	\]
	이고 $x_0,\;x \in (0, \;1)$이므로 $\vert x-x_0 \vert<1$이고 
	\[
	f(x) = f(x_0) + f^{\prime}(\zeta)(x-x_0)
	\]
	에서
	\begin{align*}
			\vert f(x) \vert  &=\vert f(x_0) + f^{\prime}(\zeta)(x-x_0) \vert \\
			& < \vert f(x_0)\vert  + \vert f^{\prime}(\zeta)\vert \vert x-x_0 \vert \\
			& < \vert f(x_0) \vert + \vert f^{\prime}(\zeta) \vert			
	\end{align*}
이다. 따라서 $f^{\prime}(x)$가 유계이면 이 구간에서 $f(x)$도 유계이다.
\end{psolution}	
}
\end{problem}

\vspace{1em}

이제 특정한 함수에 대한 평균값 정리 문제의 결론 유도를 위한 Lagrange의 평균값 정리의 응용에 대하여 알아보자. 이 경우는 두 가지 유형이 있을 수 있는데 먼저 단순 평균값 정리 문제이다. 이것은 특정한 함수에 평균값 정리를 적용하여 그에 따른 결론을 얻는 것이고 복합 평균값 정리 문제는 먼저 Lagrange의 평균값 정리를 적용하고 Cauchy의 평균값 정리를 적용하여 $\zeta, \eta$에 관한 식을 얻는 것이다. 

먼저 단순 평균값 정리 문제를 살펴보자.

\vspace{1em}
\begin{problem}
	함수 $f(x)$가 구간 $\left[a, \;b\right]$에서 연속이고 $\left(a, \;b\right)$에서 미분 가능하면 다음을 만족시키는 $\zeta \in \left(a, \; b\right)$가 존재함을 보여라.
	\[
	\frac{bf(b)-af(a)}{b-a} = f(\zeta) + \zeta f^{\prime}(\zeta)
	\]
	\processifversion{psol}{%
\begin{psolution}
	함수 $f(x)$가 구간 $\left[a, \;b\right]$에서 연속이고 $\left(a, \;b\right)$에서 미분 가능하므로 함수 $F(x) = x f(x)$도 구간 $\left[a, \;b\right]$에서 연속이고 $\left(a, \;b\right)$에서 미분 가능하다. 따라서 Lagrange의 평균값 정리에 의해 $\zeta \in \left(a, \;b\right)$가 존재하여
	\[
	\frac{F(b)-F(a)}{b-a} = F^{\prime}(\zeta)
	\]
	를 만족시킨다. 그런데 $F^{\prime}(x) = f(x) + x f^{\prime}(x)$이므로
	\[
	\frac{bf(b)-af(a)}{b-a} = f(\zeta) + \zeta f^{\prime}(\zeta)
	\]
	이다.
\end{psolution}	
}
\end{problem}
\vspace{1em}
\begin{problem}
	함수 $f(x)$가 구간 $\left[a, \;b\right]$에서 연속이고 $\left(a, \;b\right)$에서 미분 가능하면 $\zeta, \;\eta \in \left(a, \;b\right)$가 존재하여 다음을 만족시킨다.
	\[
	f^{\prime}(\zeta) = \frac{f^{\prime}(\eta)}{\eta} \cdot \frac{a+b}{2}
	\]
	
	\processifversion{psol}{%
\begin{psolution}
	Lagrange의 평균값 정리에 의해 다음을 만족시키는 $\zeta \in \left(a, \;b\right)$가 존재한다.
	\[
	f^{\prime}(\zeta) =\frac{f(b)-f(a)}{b-a}
	\]
	이제 $g(x)=x^2$이라 하자. 두 함수 $f, \;g$는 모두 폐구간 $\left[a, \;b\right]$에서 연속이고 $\left(a, \;b\right)$에서 미분가능하므로 Cauchy의 평균값 정리에 의해 
	\[
	\frac{f(b)-f(a)}{g(b)-g(a)} = \frac{f^{\prime}(\eta)}{g^{\prime}(\eta)} = \frac{f^{\prime}(\eta)}{2\eta}
	\]
	를 만족시키는 $\eta \in \left(a, \;b\right)$가 존재한다. 따라서
	\[
	\frac{f(b)-f(a)}{b^2-a^2} = \frac{f^{\prime}(\eta)}{2 \eta} \Leftrightarrow \frac{f(b)-f(a)}{b-a} = \frac{f^{\prime}(\eta)}{2 \eta} \cdot (a+b)
	\]
	이므로 문제가 증명되었다.
\end{psolution}	
}
\end{problem}
\vspace{1em}
한편 Lagrange의 평균값 정리는 방정식의 해의 존재성과 유일성의 판단에 활용될 수 있다. 다음 문제는 이러한 사례의 전형적인 예이다.
\begin{problem}
	함수 $f$의 정의역은 $\left[0, \infty\right)$이고 이 구간에서 연속이며 구간 $\left(0, \; \infty\right)$에서 미분가능하다. 
	\[
	f^{\prime}(x) \ge k >0, \quad f(0) <0
	\]
	이면 $f(x)$는 구간 $\left(0, \; \infty\right)$에서 유일한 해를 갖는다.
	\processifversion{psol}{%
\begin{psolution}
	$x>0$일 때, $f^{\prime}(x)>0$이므로 $f(x)$는 구간 $\left[0, \; \infty \right)$에서 단조증가한다. 따라서 $f(x)$는 구간 $\left[0, \; \infty \right)$에서 많아야 하나의 실근을 갖는다. 구간 $\left[0, \; x\right]$에서 $f(x)$는 연속이고 $\left(0, \; x\right)$에서 미분가능하므로 
	\[
	f(x) - f(0) = f^{\prime}(\zeta)x  \ge kx
	\]
	이고
	\[
	f(x) \ge f(0) + kx >0
	\]
	이다. 따라서 $x>-\frac{f(0)}{k}$이면 $f(x)>0$이다. 즉, $b > -\frac{f(0)}{k}$이도록 $b$를 선택하면 $f(b)>0$이고 구간 $\left[0, \; b\right]$에서 $f(x)$는 연속이므로 사잇값 정리에 의해 $x_0 \in \left(0,\;b\right)$가 존재하여 $f(x_0)=0$이다. $f(x)$는 구간 $\left(0, \; b\right)$에서 적어도 하나의 실근을 갖고 구간 $\left(0,\; \infty\right)$에서 많아야 하나의 해를 가지므로 이 구간에서 유일한 해를 갖는다.
\end{psolution}	
}
\end{problem}

마지막으로 평균값 정리는 극한값의 계산에 응용될 수 있다.  고등학교 수준에서 어려운 꽤 많은 극한 문제들이 평균값 정리를 활용하면 쉽게 계산되므로 이 경우에 해당하는 문제들을 찾아서 해결해 보는 것도 좋을 것이다.

\begin{problem}
	극한값 
	\[
	\lim\limits_{x \to \infty} x^2 \left[\arctan(x+1)- \arctan x\right]
	\]
	의 값을 구하시오.
	\processifversion{psol}{%
\begin{psolution}
	$f(x)= \arctan x$라고 하면, $f^{\prime}(x) =\frac{1}{1+x^2}$이고 $f(x)$는 구간 $\left[x, \; x+1\right]$에서 Lagrange의 평균값 정리를 만족시킨다. 따라서
	\begin{align*}
		f(x+1)-f(x) &=\arctan(x+1) - \arctan x\\
		&=f^{\prime}(\zeta)\left[(x+1)-x\right] =\frac{1}{1+\zeta^{2}}
	\end{align*}
이고
\begin{align*}
	\lim\limits_{x \to \infty} x^2 \left[\arctan(x+1)-\arctan x\right] &= \lim\limits_{x \to \infty} x^2 \cdot \frac{1}{1+\zeta^2} \quad (x<\zeta < x+1)\\
	&= \lim\limits_{x \to \infty} \frac{x^2}{1+\zeta^2} = 1
\end{align*}
\end{psolution}	
}
\end{problem}
\vspace{1em}

지금까지 우리는 Lagrange의 평균값 정리를 활용하여 해결할 수 있는 문제의 범위를 살펴보았다. 이제 우리는 평균값 정리를 활용하여 어떤 존재성 증명할 할 때 적용할 수 있는 문제해결 전략 하나를 탐구하자.


\section{존재성 증명 문제해결의 한 전략 탐구}

 Lagrange의 평균값 정리, Cauchy의 평균값 정리를 비롯하여 많은 미적분에서의 존재성 문제를 해결하는 데 적용이 가능한 문제해결 전략을 탐구해 보자. 이러한 전략을 구사할 수 있는 유형의 문제는 다음과 같은 형태를 갖는다.
 \[
 G(f^{\prime}(c), c) = H(f(a), f(b), a, b)
 \]
 또는
 \[
 G(f(c), c) = H(F(c), a, b)
 \]
이 두 식의 유형에서 $c$의 존재성을 묻는다고 가정하자.  위의 식에서 $F(x)$는 $f(x)$의 한 부정적분이고 $H(X, \;Y, \;Z,\; \cdots )$, $G(X,\;Y,\;Z, \; \cdots)$는 각각 $H, \;G$가 $X,\; Y,\; Z,\; \cdots$에 대한 함수임을 의미한다.


 이제  이러한 식이 주어졌을 때, 존재성 증명의 전략은 다음과 같다.
 \begin{enumerate}[label = \arabic*)]
 	\item  $c$를 변수 $x$라 두고 양 변을 적절히 변형하고  적분한다. 이 때 적분상수는 $0$으로 한다.
 	\item 1)에서 얻어진 미지수를 포함하는 양 변을 한 변에서 다른 한 변을 빼거나 나눈다.
 	\item 2)에서 얻어지는 $h(x)=0$ 또는 $J(x)=1$에 대하여 $y=h(x)$에 평균값 정리를 적용한다.
 	\item 3)에서 얻어지는 결과를 대수적으로 적절히 조작하면 원하는 존재성 정리가 완성된다.
 \end{enumerate}

\begin{example}
	$f:[0,\: 1]\to\mathbb{R}$이 $[0,\: 1]$에서 연속이고 $\int_{0}^{1}f(x)dx=0$을 만족시킨다고 할 때, $f(c)=\int_{0}^{c}f(x)dx$인 $c\in(0,\: 1)$이 존재함을 보여라. 
	\begin{solution}
		\textbf{(존재성 증명의 전략)} $f(c)=\int_{0}^{c}f(x)dx$로부터 $f(x)=\int_{0}^{x}f(t)dt$를 생각하자.  $F(x)=\int_{0}^{x}f(t)dt$라 하면 $F(x)=f(x)$이다. 즉 $F^{\prime}(x)= F(x)$로부터 $\frac{F^{\prime}(x)}{F(x)}=1$를 얻을 수 있다. 이제 양변을 적분하면
		\[
		\ln \vert F(x)\vert =x+C
		\]
		이고 $C=0$으로 놓으면 $e^{-x}\vert F(x)\vert =1$이다. 이제 
		\[
		g(x)=e^{-x}\int_{0}^{x}f(t)dt
		\]
		를 생각한다.
		
		$g$는 폐구간 $[0,\: 1]$에서 연속이고 개구간 $(0,\: 1)$ 에서 미분가능하다.
		
		$g(0)=g(1)=0$이므로  Rolle의 정리에 의해 $g^{\prime}(c)=0$이 되는 $c\in(0,\: 1)$이 존재한다. 
		\[
		g^{\prime}(x)= - e^{-x}\int_{0}^{x}f(t)dt +e^{-x}f(x)
		\]
		이므로 
		\[
		0 = g^{\prime}(c)= -e^{-c}\int_{0}^{c}f(t)dt + e^{-c}f(c)
		\]
		이고 문제가 증명되었다.
	\end{solution}
\end{example}
\vspace{1em}

\begin{problem} $f:[0,\: 1]\to\mathbb{R}$은 연속이고 $\int_{0}^{1}f(x)dx=0$을 만족한다. 이때, 
	\[
	(1-c)f(c)=c\int_{0}^{c}f(x)dx
	\]
	를 만족시키는 $c\in(0,\: 1)$이 존재함을 보여라.
	\processifversion{psol}{%
\begin{psolution}
	$(1-c)f(c)=c\int_{0}^{c}f(x)dx$로부터 $(1-x)f(x)=x\int_{0}^{x}f(t)dt$를 고려해 보자.
	
	 $F(x)=\int_{0}^{x}f(t)dt$ 라 하면 식 
	\[
	(1-x)f(x)=x\int_{0}^{x}f(t)dt
	\]
	은 $(1-x)F^{\prime}(x)=x F(x)$로 나타낼 수 있다. 즉
	\[
	\frac{x}{1-x}=\frac{F^{\prime}(x)}{F(x)}
	\]
	이고 양변을 적분하면
	\[
	\ln \vert F(x)\vert =-x-\ln \vert -x+1 \vert
	\]
	이고 이로부터 $\vert F(x)\vert =\frac{1}{e^{x}\vert -x+1 \vert}$, 즉 $e^{x}(1-x)\vert F(x)\vert =1$이다. 따라서 
	\[
	g(x)=e^{x}(1-x)\int_{0}^{x}f(t)dt
	\]
	라 하면 $g(0)=g(1)$이므로 롤의 정리에 의해 $g^{\prime}(c)=0$인 $c$가 존재하고 이는 $(1-c)f(c)=c\int_{0}^{c}f(x)dx$와 동치이다.
\end{psolution}	
}
\end{problem}

\vspace{1em}
\begin{problem}
	$f$가 $[a,\: b]$에서 연속이고 $(a,\: b)$에서 미분가능하다. 또한 구간 $(a,\: b)$에서 실근을 갖지 않는다. 이때, $\frac{f^{\prime}(c)}{f(c)}=\frac{1}{a-c}+\frac{1}{b-c}$를 만족하는 $c$가 존재함을 보여라. 
	
	\processifversion{psol}{%
	\begin{psolution}
		식 $\frac{f^{\prime}(c)}{f(c)}=\frac{1}{a-c}+\frac{1}{b-c}$로부터 
		\[
		\frac{f^{\prime}(x)}{f(x)}=\frac{1}{a-x}+\frac{1}{b-x}
		\]
		를 생각하자. 양변을 적분하면
		\[
		\ln \vert f(x)\vert =-\ln \vert(a-x)(b-x)\vert +c
		\]
		을 얻고 이 식에서 c를 0으로 놓으면 $\vert f(x)\vert \vert(a-x)(b-x)\vert =1$
		
		이다. 따라서 
		\[
		g(x)=f(x)(a-x)(b-x)
		\]
		라 하면 $g(a)=g(b)=0$이므로 롤의 정리에 의해 $g^{\prime}(c)=0$인 $c\in(a,\: b)$가 존재한다. 한편 $g^{\prime}(c)=0$은
		\[
		f^{\prime}(c)(a-c)(b-c)=f(c)(b-c)+f(c)(a-c)
		\]
		와 동치이다.
	\end{psolution}
}
\end{problem}
\vspace{1em}
\begin{problem}
	함수 $f :[0,\: 1]\to\mathrm{R}$이 미분가능하고 $f(0)=0$이다. 모든 $x \in(0,\: 1)$에 대하여 $f(x)>0$이 성립할 때
	\[
	\frac{2f^{\prime}(c)}{f(c)}=\frac{f^{\prime}(1-c)}{f(1-c)}
	\]
	를 만족시키는 $c \in(0,\: 1)$이 존재함을 보여라. 또 $d \in(0,\: 1)$에 대하여 
	\[
	\frac{3f^{\prime}(d)}{f(d)}=\frac{f^{\prime}(1-d)}{f(1-d)}
	\]
	를 만족시키는가?
	\processifversion{psol}{%
\begin{psolution}
	지금까지의 존재성 증명의 전략을 생각하면 $g(x)= f(x)^{2}f(1-x)$를 생각하면 된다. $f$가 미분가능하므로 $g$도 미분가능하고 $g(0)=0$, $g(1)=0$이므로 롤의 정리에 의해 $g^{\prime}(c)=0$을 만족시키는 실수 $c \in(0,\: 1)$가 존재한다. 즉 
	\[
	\frac{2f^{\prime}(c)}{f(c)}=\frac{f^{\prime}(1-c)}{f(1-c)}
	\]
	이 성립한다.
	
	두 번째 질문은 $h(x)=f(x)^{3}f(1-x)$를 생각하면 된다.
\end{psolution}	
}
\end{problem}
\vspace{1em}
\begin{problem}
	$f:[0,\: 1]\to\mathbb{R}$이 연속이고 $\int_{0}^{1}f(x)dx=0$이다. 다음을 만족시키는 $c\in(0,\: 1)$이 존재함을 보여라. 
	\begin{itemize}
		\item $f$가 $(0,\: 1)$에서 미분가능할 때, $f(c)=f^{\prime}(c)\int_{0}^{c}f(x)dx$
		
		\item $\frac{f(c)}{c}=\int_{0}^{c}f(x)dx$
		
		\item $cf(c)=\int_{c}^{1}f(x)dx$
	\end{itemize}
\end{problem}

\section{존재성 증명 전략을 활용한 평균값 정리들의 증명}

이제 위의 존재성 증명 전략을 적용하여 Lagrange의 평균값 정리와 Cauchy의 평균값 정리를 증명해 보자. 먼저 Lagrange의 평균값 정리의 증명에 존재성 증명 전략을 적용해 보자. 

$f^{\prime}(c)=\frac{f(b)-f(a)}{b-a}$에 존재성 증명 전략을 적용하자. 먼저 $f^{\prime}(c)(b-a) = (f(b)-f(a))$의 양 변을 적분하고 적분상수 $c$를 $0$으로 두면 
\[
f(x)(b-a) = (f(b)-f(a)) x
\]
이다. 이제 $h(x) = f(x)(b-a)-(f(b)-f(a))x$라 하자. 그러면 $h(x)$는 $f(x)$와 마찬가지로 $\left[a, \;b\right]$에서 연속이고 개구간 $\left(a, \; b\right)$에서 미분 가능하다. 또한 $h(a)=h(b)=bf(a)-af(b)$이므로 Rolle의 정리에 의해 $h^{\prime}(c)=0$인 $c(c \in \left(a, \;b\right))$가 존재한다. 그런데
\[
h^{\prime}(x) = f^{\prime}(x)(b-a) - (f(b)-f(a))
\]
이므로 $h^{\prime}(c)=0$은
\[
 f^{\prime}(x)(b-a) - (f(b)-f(a)) =0
\]
과 동치이고 따라서 Lagrange의 평균값 정리가 증명되었다. 전통적인 Lagrange의 평균값 정리에서 사용함수와는 조금 다른 함수로 증명되는 것에 주목할 필요가 있다.

이제 Cauchy의 평균값 정리의 증명에 존재성 증명의 전략을 적용해 보자. Cauchy의 평균값 정리는 
	두 함수 $f(x), \;g(x)$가 폐구간 $[a, \; b]$에서 연속이고 개구간 $(a, \; b)$에서 미분가능할 때
\[
(g(b)-g(a))f^{\prime}(c) = (f(b)-f(a))g^{\prime}(c)
\]
을 만족시키는  $c(a<c<b)$가 존재함을 보이는 것이므로
\[
(g(b)-g(a))f^{\prime}(x) = (f(b)-f(a))g^{\prime}(x)
\]
의 양변을 적분하고 적분상수를 $0$으로 한 다음 우변을 좌변으로 이항한 것을 $h(x)$라 하면
\[
h(x) = (g(b)-g(a))f(x) - (f(b)-f(a))g(x)
\]
이다. $h(x)$는 폐구간 $[a, \; b]$에서 연속이고 개구간 $(a, \; b)$에서 미분가능하고
\[
h(a) =h(b) = f(a)g(b) - f(b)g(a)
\]
이므로 Rolle의 정리에 의해 $h^{\prime}(c)=0$을 만족시키는 $c\left(c \in \left(a, \; b\right)\right)$가 존재한다. 따라서 Cauchy의 평균값 정리가 증명되었다.

\section{마치며}

지금까지 우리는 미적분학에서 존재성 증명과 관련된 정리들 사이의 위계성, 즉 어떤 정리에 의해 새로운 정리가 증명될 때 새로운 정리를 증명할 때 사용되는 정리가 더 근본적인 정리라는 것에 대하여 논하였고, Lagrange의 평균값 정리의 사용 범위에 대하여 구체적인 사례들과 함께 알아 보았다. 마지막으로 존재성 증명 전략 하나를 탐구하고 이 전략을 적용하여 여러 존재성 문제들을 해결해 보았고 마지막으로 존재성 증명 전략을 이용하여 Lagrange의 평균값 정리와 Cauchy의 평균값 정리를 증명할 수 있음을 보였다. 

존재성 증명 전략과 관련하여 재미있는 사실 중의 하나는 Rolle의 정리에 존재성 증명 전략을 적용하면 함수 $f(x)$ 자신이 나온다는 것이다. 즉, 존재성 증명 전략이 Rolle의 정리 증명에는 실패한다는 사실이다. 


\end{document}