\documentclass[a4paper]{article}
\usepackage{secdot}
% AMS and mathtools
\usepackage{amsmath,amsthm,amssymb,marvosym,mathrsfs,amsfonts,amscd,mathtools}
% Hyperlinks and URLs
\usepackage{url}
\usepackage{hyperref}
\hypersetup{
	colorlinks,
	citecolor=BLACK,
	filecolor=BLACK,
	linkcolor=BLACK,
	urlcolor=BLACK
}

% Colors
\usepackage[usenames,dvipsnames]{xcolor}
\usepackage{tikz}
\usepackage{tkz-euclide}

% shadowing mdframed
\usepackage[framemethod=tikz]{mdframed}
\usetikzlibrary{shadows}
% Bold math
\usepackage{bm}

%Use Korean Letter when enumerate
\usepackage{dhucs-enumerate}

\usepackage{anyfontsize}

% Bra Ket (Dirac) Notation
\usepackage{braket}

% Slashed characters (e.g. in Dirac equation)
\usepackage{slashed}
\usepackage{pifont} % 원문자 사용시 필요한 패키지

% chapter decoration
\usepackage{type1cm}
\usepackage[explicit]{titlesec}

\titleformat{\chapter}[display]
{\normalfont\Large\rmfamily}
{\sffamily\flushright\fontsize{60}{0}\textbf{\textcolor{blue!40}{{\Huge\chaptername}~\thechapter\vskip0pt\rule{\textwidth}{2pt}}}}{0pt}
{\flushleft\fontsize{30}{0}{#1}\vskip60pt}
\titlespacing*{\chapter}
{0pt}{-40pt}{0pt}

%\usetikzlibrary{shadows}
\usetikzlibrary{shadows.blur}
\usetikzlibrary{shapes.symbols}

% Tcolorbox
\usepackage[most]{tcolorbox}

% Clean SI Units
\usepackage{siunitx}

% Enumerate thingies
\usepackage{enumitem}

% Cancel things out in equations
\usepackage[makeroom]{cancel}

\usepackage{multicol}

% Graphics and figures
\usepackage{graphicx}
\usepackage{wrapfig}
\usepackage{float}

\usepackage{cancel}

% Caption figures and tables
\usepackage{caption,subcaption}

% Generate symbols
\usepackage{textcomp} % Include this line to avoid output errors
\usepackage{gensymb}

% Make multiple rows in a table
\usepackage{multirow}

% Booktabs tables
\usepackage{booktabs}

%\usepackage[utopia,sfscaled]{mathdesign}
% Useful frames
\usepackage{mdframed}

% Comment-out large sections
\usepackage{comment}

% No auto-indent
\setlength{\parindent}{0pt}

% Asymptote - 3D vector graphics
\usepackage{asymptote}

% Tikz Package Stuff
\usepackage{pgf,tikz,pgfplots}
\usepackage{tikz-3dplot}
\usepackage{tabularx}
\usepackage{array}
\usepackage{colortbl}
\tcbuselibrary{skins}
\usepackage{tkz-euclide}

\newcolumntype{Y}{>{\raggedleft\arraybackslash}X}

\tcbset{tab1/.style={fonttitle=\bfseries\large,fontupper=\normalsize\sffamily,
		colback=yellow!10!white,colframe=red!75!black,colbacktitle=Salmon!40!white, halign=center,
		coltitle=black,center title,freelance,frame code={
			\foreach \n in {north east,north west,south east,south west}
			{\path [fill=red!75!black] (interior.\n) circle (3mm); };},}}

\tcbset{tab2/.style={enhanced,fonttitle=\bfseries,fontupper=\normalsize\sffamily, halign=center, box align=center,
		colback=yellow!10!white,colframe=red!50!black,colbacktitle=Salmon!40!white,
		coltitle=black,center title}}


% Use various tikz libraries
\usetikzlibrary{decorations.pathmorphing, decorations.markings, decorations.pathreplacing, patterns} % Decorate paths!
\usetikzlibrary{calc, patterns, shapes.geometric, positioning, through, intersections}
\usetikzlibrary{scopes}
\usetikzlibrary{angles, quotes}
\usetikzlibrary{svg.path}
\usetikzlibrary{arrows, arrows.meta}
\usetikzlibrary{fadings}
% pgfplots package settings
\pgfplotsset{compat=1.15}
% \pgfplotsset{width=10cm,compat=1.9} % Taken from latest overleaf.
% plot arc easily
\def\centerarc[#1](#2)(#3:#4:#5)% Syntax: [draw options] (center) (initial angle:final angle:radius)
{ \draw[#1] ($(#2)+({#5*cos(#3)},{#5*sin(#3)})$) arc (#3:#4:#5); }

% Awesome circled numbers
\newcommand*\circled[4]{\tikz[baseline=(char.base)]{\node[shape=circle, fill=#2, draw=#3, text=#4, inner sep=2pt] (char) {#1};}}

% Control size of text
\usepackage{relsize}

% Extend conditional commands
\usepackage{xifthen}
\usepackage{xcolor}
\definecolor{termcolor}{cmyk}{.21,.97,.0,.0}
\definecolor{darkred}{cmyk}{.27,1,1,.32}
\definecolor{darkblue}{cmyk}{1,.98,.10,.11}
\definecolor{darkgreen}{cmyk}{.29,0,87,0}
\definecolor{darkmycolor}{cmyk}{99,59,22,3}
\definecolor{for_eyes}{RGB}{253,247,228}
%change color of math equation
%\everymath{\color{darkred}}
% Scale math by size
\newcommand*{\Scale}[2][4]{\scalebox{#1}{\ensuremath{#2}}}

% Big integrals
\usepackage{bigints}

% Number equations within sections
\numberwithin{equation}{section}

% Generate blind text
\usepackage{blindtext}

% Useful symbols
\usepackage{marvosym}

\newcounter{problem}[section]
\newcounter{example}[section]

% cancel 색상 변경
\newcommand\Ccancel[2][black]{\renewcommand\CancelColor{\color{#1}}\cancel{#2}}

%%%% 원문자
\newcommand*\ocircled[1]{\tikz[baseline=(char.base)]{
		\node[shape=circle,draw,inner sep=2pt] (char) {#1};}}
	
%%%%% 보기 스타일 %%%%%
\usepackage{tabu}
\newcommand{\questwo}[2]{
	\vskip 6pt
	\noindent\begin{tabu}{X[0.2] X[6] X[0.2] X[6]}
		(1)&$#1$ &(2) &$#2$
	\end{tabu}
}
\newcommand{\questhree}[3]{
	\vskip 3pt
	\noindent\begin{tabu}{X[0.2] X[6] X[0.2] X[6] X[0.2] X[6]}
		(1)&$#1$ &(2) &$#2$ &(3) & $#3$
	\end{tabu}
	\vskip 5pt
}
\newcommand{\quesfour}[4]{
	\vskip 6pt
	\noindent\begin{tabu}{X[0.2] X[6] X[0.2] X[6]}
		(1)&$#1$ &(2) &$#2$\\
		(3)&$#3$ &(4) &$#4$
	\end{tabu}
}
\newcommand{\quesfive}[5]{
	\vskip 6pt
	\noindent\begin{tabu}{X[0.2] X[3] X[0.2] X[3] X[0.2] X[3]}
		(1)&$#1$ &(2) &$#2$ &(3) &$#3$\\
		(4)&$#4$ &(5) &$#5$
	\end{tabu}
}

\newcommand{\oquesfive}[5]{
	\vskip 6pt
	\noindent\begin{tabu}{X[0.2] X[3] X[0.2] X[3] X[0.2] X[3] X[0.2] X[3] X[0.2] X[3]}
		\ding{172}&$#1$ &\ding{173} &$#2$ &\ding{174} &$#3$&	\ding{175}&$#4$ &\ding{176} &$#5$
	\end{tabu}
}
%%%% sample(보기) %%%%
\newenvironment{sample}{\vskip 10pt\noindent\begin{tikzpicture}[yshift=1.5pt]%
		\draw[rounded corners=1ex,overlay,draw=blue] (0pt,-4pt) rectangle (19pt,9pt);
		\node[rectangle,overlay,xshift=9.8pt,yshift=2pt,color=blue] {{\footnotesize\sffamily 보기}};\end{tikzpicture}\phantom{\footnotesize\sffamily보기...}}{\vskip 10pt}
%%%% THEOREMS %%%%
\newenvironment{theorem}[1][\hspace{-0.36em}]
{
	\begin{mdframed}[backgroundcolor=purple!5, align=center, userdefinedwidth=40em, linecolor=purple!30, linewidth=2pt, roundcorner=7pt, innertopmargin=10pt, shadow=true, shadowcolor=black!20, roundcorner=7pt, innerbottommargin=10pt, frametitle = {#1}]
	}
	{
	\end{mdframed}
}

%%%% LEMMAS %%%%
\newenvironment{lemma}[1][\hspace{-0.36em}]
{
	\begin{mdframed}[backgroundcolor=black!8, align=center, userdefinedwidth=40em, topline=false, bottomline = false, leftline = false, rightline = false, frametitle = {#1 보조정리}]
	}
	{
	\end{mdframed}
}

%%%% COROLLARY %%%%
\newenvironment{corollary}[1][\hspace{-0.36em}]
{
	\begin{mdframed}[backgroundcolor=black!8, align=center, userdefinedwidth=40em, topline=false, bottomline = false, leftline = false, rightline = false, frametitle = {#1 따름정리}]
	}
	{
	\end{mdframed}
}

%%%% DEFINITIONS %%%%
\newenvironment{definition}[1][\hspace{-0.36em}]
{
	\begin{mdframed}[backgroundcolor=cyan!14, align=center, userdefinedwidth=40em, linecolor=cyan!60, linewidth=2pt,roundcorner=7pt, innertopmargin=10pt, shadow=true, shadowcolor=black!20, roundcorner=7pt, innerbottommargin=10pt, frametitle = {정의 : #1}]
	}
	{
	\end{mdframed}
}

%%%% PROPOSITION %%%%
\newenvironment{proposition}
{
	\begin{mdframed}[backgroundcolor=black!4, align=center, userdefinedwidth=40em, topline=false, bottomline = false, leftline = false, rightline = false, frametitle = {Proposition}]
	}
	{
	\end{mdframed}
}
%%%% PROBLEM %%%%
\newenvironment{problem}{\refstepcounter{problem}
	\begin{mdframed}[linecolor=blue!35, linewidth=2pt, roundcorner=7pt, innertopmargin=10pt, shadow=true, shadowcolor=black!20, roundcorner=7pt, innerbottommargin=10pt, backgroundcolor=blue!5]
		\noindent 
		\noindent\begin{tikzpicture}[overlay,xshift=6pt,yshift=5pt]
			\draw[fill=violet!15,draw=violet!15] (0,0) circle (8pt);
			\draw[fill=violet!15,draw=violet!15] (9pt,0) circle (8pt);
			\node[rectangle,overlay,xshift=4pt] {\color{black}\sffamily\bfseries 문제};
		\end{tikzpicture}{\phantom{.........}\fontspec[Scale=1.1]{TeX Gyre Adventor}\color{darkblue} \theproblem}\hspace{5pt}}{
\end{mdframed}}
%%%% SOLUTION OF PROBLEM %%%%
\newenvironment{psolution}{\begin{description}\item[{\begin{tikzpicture}%
				\draw[rounded corners=1ex,overlay] (-3pt,-3pt) rectangle (20pt,9pt);\end{tikzpicture}\footnotesize\sffamily풀이}\hspace{6pt}]}{\end{description}}
			
%%%% EXAMPLE %%%%		
\newenvironment{example}{\refstepcounter{example}
	\begin{mdframed}[roundcorner=7pt,linecolor=termcolor,linewidth=2pt,innertopmargin=10pt, shadow=true, shadowcolor=black!20, innerbottommargin=10pt, backgroundcolor=termcolor!2]
		\noindent 
		\noindent\begin{tikzpicture}[overlay,xshift=6pt,yshift=5pt]
			\draw[fill=darkred!60,draw=darkred!60] (0,0) circle (7pt);
			\draw[fill=darkred!60,draw=darkred!60] (9pt,0) circle (7pt);
			\node[rectangle,overlay,xshift=4pt] {\color{white}\sffamily\bfseries 예제};
		\end{tikzpicture}{\phantom{.........}\fontspec[Scale=1.1]{TeX Gyre Adventor}\color{darkred} \theexample}\hspace{5pt}}{
\end{mdframed}}		
%%%%%%% SOLUTION OF EXAMPLE %%%%%%%%%
\newenvironment{solution}{\begin{description}\item[{\begin{tikzpicture}%
				\draw[rounded corners=1ex,overlay] (-3pt,-3pt) rectangle (20pt,9pt);\end{tikzpicture}\footnotesize\sffamily풀이}\hspace{6pt}]}{\end{description}}

% 보기 박스 정의 시작
\tcbuselibrary{breakable, skins}
\tcbset{enhanced}
\newtcolorbox{ChoiceBox}[1]{
		enhanced,
		before skip=2ex, after skip=2ex,
		boxrule=0.5pt, colframe=black, colback=white, arc=0.5ex,
		boxsep=0.5ex, top=1.5ex, bottom=1.5ex, left=0.5em, right=o0.5em,
		colbacktitle=white, coltitle=black,
		attach boxed title to top center={xshift=0cm, yshift=-1.5mm},
		boxed title style={size=minimal, enhanced, boxrule=0.25pt, colframe=white},
		breakable=false, title ={< #1 >}
}
% 보기 박스 정의 끝.
	
% Change end-of-proof symbol
\renewcommand\qedsymbol{$\blacksquare$}
%overline
\newcommand{\ovr}[1]{\overline{\textrm{#1}}}
% trigonometric function
\newcommand{\cosrm}[1]{\cos \textrm{#1}}
\newcommand{\sinrm}[1]{\sin \textrm{#1}}
\newcommand{\tanrm}[1]{\tan \textrm{#1}}
\newcommand{\cotrm}[1]{\cot \textrm{#1}}
\newcommand{\cscrm}[1]{\csc \textrm{#1}}
\newcommand{\secrm}[1]{\sec \textrm{#1}}
%%%% BLACKBOARD BOLD %%%%
\newcommand{\bbN}{\mathbb{N}} % Natural numbers
\newcommand{\bbZ}{\mathbb{Z}} % Zahlen
\newcommand{\bbQ}{\mathbb{Q}} % Rational numbers
\newcommand{\bbR}{\mathbb{R}} % Real numbers
\newcommand{\bbC}{\mathbb{C}} % Complex numbers
\DeclareSymbolFont{bbold}{U}{bbold}{m}{n} % Identity matrix
\DeclareSymbolFontAlphabet{\mathbbold}{bbold} % Identity matrix
\newcommand{\identitymatrix}{\mathbbold{1}} % Identity matrix

%%%% CODE LISTING %%%%
\usepackage{listings}
\definecolor{greencomments}{HTML}{00BA00}
\definecolor{graynumbers}{HTML}{4F4F4F}
\definecolor{purplestrings}{HTML}{AD00AA}
\definecolor{backgroundcolor}{HTML}{E8E8E8}


%%%% UNIT BASIS VECTORS %%%%
\newcommand{\ihat}{\bm{\hat{\imath}}} % Cartesian i hat (x-direction)
\newcommand{\jhat}{\bm{\hat{\jmath}}} % Cartesian j hat (y-direction)
\newcommand{\khat}{\bm{\hat{k}}} % Cartesian k hat (z-direction)
\newcommand{\rhat}{\bm{\hat{r}}} % Spherical r hat
\newcommand{\phihat}{\bm{\hat{\phi}}} % Spherical phi hat
\newcommand{\thetahat}{\bm{\hat{\theta}}} % Spherical theta hat
\newcommand{\nhat}{\bm{\hat{n}}} % Unit normal vector
\newcommand{\rhohat}{\bm{\hat{\rho}}} % Cylindrical rho hat
\newcommand{\zhat}{\bm{\hat{z}}} % Cylindrical z hat


%%%% COLORS: DEFINITIONS AND COMMANDS %%%%
% Miscellaneous
\definecolor{DARKBLUE}{HTML}{040080}
\definecolor{DARKBROWN}{HTML}{8B4513}
\definecolor{LIGHTBROWN}{HTML}{CD853F}
\definecolor{PINK}{HTML}{D147BD}
\definecolor{LIGHTPINK}{HTML}{DC75CD}
\definecolor{GREENSCREEN}{HTML}{00FF00}
\definecolor{ORANGE}{HTML}{FF862F}
\newcommand{\DARKBLUE}{\color{DARKBLUE}}
\newcommand{\DARKBROWN}{\color{DARKBROWN}}
\newcommand{\LIGHTBROWN}{\color{LIGHTBROWN}}
\newcommand{\PINK}{\color{PINK}}
\newcommand{\LIGHTPINK}{\color{LIGHTPINK}}
\newcommand{\GREENSCREEN}{\color{GREENSCREEN}}
\newcommand{\ORANGE}{\color{ORANGE}}
% Blue
\definecolor{BLUEE}{HTML}{1C758A}
\definecolor{BLUED}{HTML}{29ABCA}
\definecolor{BLUEC}{HTML}{58C4DD}
\definecolor{BLUEB}{HTML}{9CDCEB}
\definecolor{BLUEA}{HTML}{C7E9F1}
\definecolor{BLUE}{HTML}{0000FF}
\newcommand{\BLUEE}{\color{BLUEE}}
\newcommand{\BLUED}{\color{BLUED}}
\newcommand{\BLUEC}{\color{BLUEC}}
\newcommand{\BLUEB}{\color{BLUEB}}
\newcommand{\BLUEA}{\color{BLUEA}}
\newcommand{\BLUE}{\color{BLUE}}
% Teal
\definecolor{TEALE}{HTML}{49A88F}
\definecolor{TEALD}{HTML}{55C1A7}
\definecolor{TEALC}{HTML}{5CD0B3}
\definecolor{TEALB}{HTML}{76DDC0}
\definecolor{TEALA}{HTML}{ACEAD7}
\definecolor{TEAL}{HTML}{00FFFF}
\newcommand{\TEALE}{\color{TEALE}}
\newcommand{\TEALD}{\color{TEALD}}
\newcommand{\TEALC}{\color{TEALC}}
\newcommand{\TEALB}{\color{TEALB}}
\newcommand{\TEALA}{\color{TEALA}}
\newcommand{\TEAL}{\color{TEAL}}
% Green
\definecolor{GREENE}{HTML}{699C52}
\definecolor{GREEND}{HTML}{77B05D}
\definecolor{GREENC}{HTML}{83C167}
\definecolor{GREENB}{HTML}{A6CF8C}
\definecolor{GREENA}{HTML}{C9E2AE}
\definecolor{GREEN}{HTML}{00FF00}
\newcommand{\GREENE}{\color{GREENE}}
\newcommand{\GREEND}{\color{GREEND}}
\newcommand{\GREENC}{\color{GREENC}}
\newcommand{\GREENB}{\color{GREENB}}
\newcommand{\GREENA}{\color{GREENA}}
\newcommand{\GREEN}{\color{GREEN}}
% Yellow
\definecolor{YELLOWE}{HTML}{E8C11C}
\definecolor{YELLOWD}{HTML}{F4D345}
\definecolor{YELLOWC}{HTML}{FFFF00}
\definecolor{YELLOWB}{HTML}{FFEA94}
\definecolor{YELLOWA}{HTML}{FFF1B6}
\definecolor{YELLOW}{HTML}{FFFF00}
\newcommand{\YELLOWE}{\color{YELLOWE}}
\newcommand{\YELLOWD}{\color{YELLOWD}}
\newcommand{\YELLOWC}{\color{YELLOWC}}
\newcommand{\YELLOWB}{\color{YELLOWB}}
\newcommand{\YELLOWA}{\color{YELLOWA}}
\newcommand{\YELLOW}{\color{YELLOW}}
% Gold
\definecolor{GOLDE}{HTML}{C78D46}
\definecolor{GOLDD}{HTML}{E1A158}
\definecolor{GOLDC}{HTML}{F0AC5F}
\definecolor{GOLDB}{HTML}{F9B775}
\definecolor{GOLDA}{HTML}{F7C797}
\newcommand{\GOLDE}{\color{GOLDE}}
\newcommand{\GOLDD}{\color{GOLDD}}
\newcommand{\GOLDC}{\color{GOLDC}}
\newcommand{\GOLDB}{\color{GOLDB}}
\newcommand{\GOLDA}{\color{GOLDA}}
% Red
\definecolor{REDE}{HTML}{CF5044}
\definecolor{REDD}{HTML}{E65A4C}
\definecolor{REDC}{HTML}{FC6255}
\definecolor{REDB}{HTML}{FF8080}
\definecolor{REDA}{HTML}{F7A1A3}
\definecolor{RED}{HTML}{FF0000}
\newcommand{\REDE}{\color{REDE}}
\newcommand{\REDD}{\color{REDD}}
\newcommand{\REDC}{\color{REDC}}
\newcommand{\REDB}{\color{REDB}}
\newcommand{\REDA}{\color{REDA}}
\newcommand{\RED}{\color{RED}}
% Maroon
\definecolor{MAROONE}{HTML}{94424F}
\definecolor{MAROOND}{HTML}{A24D61}
\definecolor{MAROONC}{HTML}{C55F73}
\definecolor{MAROONB}{HTML}{EC92AB}
\definecolor{MAROONA}{HTML}{ECABC1}
\newcommand{\MAROONE}{\color{MAROONE}}
\newcommand{\MAROOND}{\color{MAROOND}}
\newcommand{\MAROONC}{\color{MAROONC}}
\newcommand{\MAROONB}{\color{MAROONB}}
\newcommand{\MAROONA}{\color{MAROONA}}
% Purple
\definecolor{PURPLEE}{HTML}{644172}
\definecolor{PURPLED}{HTML}{715582}
\definecolor{PURPLEC}{HTML}{9A72AC}
\definecolor{PURPLEB}{HTML}{B189C6}
\definecolor{PURPLEA}{HTML}{CAA3E8}
\definecolor{PURPLE}{HTML}{FF00FF}
\newcommand{\PURPLEE}{\color{PURPLEE}}
\newcommand{\PURPLED}{\color{PURPLED}}
\newcommand{\PURPLEC}{\color{PURPLEC}}
\newcommand{\PURPLEB}{\color{PURPLEB}}
\newcommand{\PURPLEA}{\color{PURPLEA}}
\newcommand{\PURPLE}{\color{PURPLE}}
% White and Black
\definecolor{WHITE}{HTML}{FFFFFF}
\newcommand{\WHITE}{\color{WHITE}}
\definecolor{BLACK}{HTML}{000000}
\newcommand{\BLACK}{\color{BLACK}}
% Different Grays
\definecolor{LIGHTGRAY}{HTML}{BBBBBB}
\definecolor{GRAY}{HTML}{888888}
\definecolor{DARKGRAY}{HTML}{444444}
\definecolor{DARKERGRAY}{HTML}{222222}
\definecolor{GRAYBROWN}{HTML}{736357}
\newcommand{\LIGHTGRAY}{\color{LIGHTGRAY}}
\newcommand{\GRAY}{\color{GRAY}}
\newcommand{\DARKGRAY}{\color{DARKGRAY}}
\newcommand{\DARKERGRAY}{\color{DARKERGRAY}}
\newcommand{\GRAYBROWN}{\color{GRAYBROWN}}


\usepackage{indentfirst}
\usepackage{amsmath}
\usepackage{setspace}
\usepackage{kotex}

\usepackage[margin=3.5cm,headsep=0.5cm]{geometry}
\newcommand{\Z}{\mathbb{Z}}
\newcommand{\vrm}[1]{\overrightarrow{\textrm{#1}}}
\usepackage{amsfonts}
\title{미적분에서의 존재성 증명의 한 전략}
\author{infgrp@hanmail.net}
\date{\today}

\begin{document}
\maketitle

\setstretch{1.5}


\section{논의의 기초}

    미적분학에서 존재성 증명에 활용되는 가장 기본적인 정리는 최대-최소 종리와 사잇값의 정리이다. 왜냐하면 이들 정리는 다른 정리에 의존하지 않고 단지 폐구간과 이 구간에서 연속인 함수라는 두 수학적 개념만을 활용하여 증명되는 수학적 사실이기 때문이다. 사실 이 두 정리의 증명은 고등학교 수준을 훨씬 뛰어 넘기 때문에 고등학교에서는 증명하지 않고 그 사실만을 이들을 기반으로 하는 정리의 증명에 활용한다. 
    
    이들 정리 다음으로 기본적인 정리는 Rolle의 정리이다. Rolle의 정리는 최대-최소 정리를 활용하여 증명된다. 따라서 Rolle의 정리는 최대-최소 정리 다음 수준의 기본적인 정리라 할 수 있다. 이제 Rolle의 정리와 그 증명을 살펴보자. 
\vspace{1em}
\begin{theorem}[Rolle의 정리]
함수 $f:[a,\: b]\to\mathbb{R}$가 구간 $[a,\: b]$에서 연속이고 개구간 $(a,\: b)$에서 미분가능하고
 $f(a)=f(b)$이면 $f^{\prime}(c)=0$을 만족시키는 실수 $c \in(a,\: b)$가 존재한다.
\begin{proof}
 주어진 함수가 폐구간 $[a,\: b]$에서 연속이므로 최대-최소정리에 의하여 이 함수는 이 구간에서 최댓값과 최솟값을 갖는다. 만약 이 구간의 양 끝점의 한 점에서 최댓값을 갖고 나머지 한 점에서 최솟값을 가지면 이 함수는 상수함수이다. 이 경우 미분계수가 정의되는 구간 $(a, \;b)$위의 모든 점에서 미분계수가 $0$이다. 이제 구간 내부의 한 점 $c$에서 최댓값을 갖는 경우를 생각하자. (최솟값의 경우는 함수 $-f$를 생각하면 되므로 이 경우는 증명을 생략하자.) 

이제 $f(c)$가 최댓값 이므로  $h>0$이면 
\[
\frac{f(c+h)-f(c)}{h} \le 0
\]
이다. 따라서 점 $c$에서의 우극한 $f(c^{+})$는 다음과 같다.
\[
f(c^{+}) = \lim\limits_{h \to 0^{+}} \frac{f(c+h)-f(c)}{h} \le 0
\]
마찬가지 방법으로 $f(c)$가 최댓값 이므로  $h<0$이면 
\[
\frac{f(c+h)-f(c)}{h} \ge 0
\]
이다. 따라서 점 $c$에서의 좌극한 $f(c^{-})$는 다음과 같다.
\[
f(c^{-}) = \lim\limits_{h \to 0^{+}} \frac{f(c+h)-f(c)}{h} \ge 0
\]
그런데 점 $c$에서 미분계수가 존재하므로 $f^{+}(c)= f^{-}(c)$이므로 $f^{\prime}(c)=0$이다.
\end{proof}
\end{theorem}
\vspace{1em}

이제 일반적으로 평균값의 정리(Mean Vaule Theorem)로 알려진 Lagrange의 평균값 정리에 대하여 알아보자. 정리의 증명은 주어진 함수와 주어진 구간을 지나는 직선을 이용하고 Rolle의 정리를 적용하여 쉽게 증명할 수 있다.

\begin{theorem}[Lagrange의 평균값 정리]
 함수 $f:[a,\: b]\to\mathrm{R}$가 구간 $[a,\: b]$에서 연속이고 개구간 $(a,\: b)$에서 미분가능하고 $f(a)=f(b)$이면 $f^{\prime}(c)=\frac{f(b)-f(a)}{b-a}$를 만족시키는 실수 $c \in(a,\: b)$가 존재한다.

\end{theorem}
  

  우리나라 교육과정에서 인수정리와 관련된 내용은 학생들이 인수정리의 본질적인 측면을 정확하게 알지 못하고 그 내용만 기계적으로 활용하도록 한다는 점이다. 이제 인수정리를 두 가지 관점에서 일반화 해보자.
  
\section{인수정리의 일반화 1}

  나머지 정리와 인수정리에서의 핵심이 되는 ‘나눈다’는 의미의 본질을 생각해 보자. 여러
수학의 분야에서 나누는 개념은 주요하게 활용되고 있다. 정수론에서의 잉여류, 대수학에서의 quotient group, quotient ring등의 개념과 위상수학에서의 quotient space등은 주어진 수학적 대상의 성질을 연구하거나 주어진 수학적 대상으로부터 새로운 수학적 대상을 얻는 하나의 방법으로 활용된다. 이러한 하나의 수학적 대상의 quotient가 가능하기 위에서는 주어진 수학적 대상에 잘 정의되는 동치관계가 존재하여야 한다. 이러한 동치관계에서 동치인 관계에 있는 것들은 나누기(quotient)한 집합에서는 모두 \textbf{‘동일시’}된다는 점에 주목할 필요가 있다. 즉 나눈다는 것의 본질은 특정한 관계에 있는 것들을 모두 동일시한다는 것이다. 예를 들어, 정수론에서 $5$로 나눈 잉여류를 얻는 과정을 살펴보자.

  임의의 두 정수 에 대하여 다음과 같이 관계 $\sim$를 정의 하자.
\[
a \sim b \Longleftrightarrow a-b=5k, \text{단 는 정수.}
\]
이 때 관계 $\sim$은 동치관계임을 쉽게 알 수 있고 정수를 $5$로 나눈 잉여류의 집합은 $\{ \bar{0},\: \bar{1}, \:\bar{2},\: \bar{3},\: \bar{4} \}$이다. 

  여기에서 우리의 논의에 중요한 사실은 $\bar{0}$는 $5$의 배수인 정수 전체이고 이들은 모두 위의 관계에서 모두 \textbf{동일시}되고 있다는 점이다. 이러한 사실은 정수론뿐만 아니라 앞에서 언급한 다양한 수학의 분야에서 논의하는 quotient에서도 똑같다.

  이제 나머지 정리를 동일시라는 관점에서 일반화 하면 다음과 같다.
 $x$에 대한 다항식 $f(x)$를 일차식 $x-\alpha$로 나누었을 때의 나머지는 다항식 $f(x)$에서 $x$와 $\alpha$를 동일시하여 얻을 수 있다. 즉,
\[
R=f(\alpha)
\]
  이것을 좀 더 일반적으로 생각하면 $X-Y$로 어떤 식을 나눈다는 것의 본질은  $X-Y=0$. 즉, $X$와 $Y$를 동일시한다는 것으로 파악할 수 있다. 이러한 측면에서 수학에서는 \textbf{동일시}는 \textbf{같다}와는 완전히 같은 개념은 아니므로 이를 구별하기 위해 등호 대신 기호 $\equiv$를 사용한다. 이러한 동일시의 개념으로, 나누는 것의 의미를 확장하면 몇 가지 좋은 점이 있는데 첫째 인수정리를 일차식이 아닌 일반적인 차수로 확장이 가능하다는 것이다. 즉, $f(x)$를 $P(x)$로 나누어 몫이 $Q(x)$이고 나머지가 $R(x)$라 가정 하면 다항식에서의 나머지 정리에 의해 다음이 성립한다.
\[
f(x) = P(x) Q(x) + R(x), \: \: 0 \leq \deg\left( R(x) \right) < \deg\left(Q(x) \right)
\]
이 때 $R(x)$는 $f(x)$에서 $P(x)$를 $0$과 동일시하면 얻을 수 있다. 다음의 예들은 이러한 경우의 대표적인 것이다.
\vspace{1em}
\begin{example}
 $x^4 + x^2 +1$을 $x^2 - x +1$로 나눈 나머지를 구하시오.
\end{example}

\textbf{풀이)} 주어진 식을 $x^2 - x +1$로 나눈다는 것은 $x^2 - x +1$을 $0$과 동일시한다는 것이고 이것은 $x^2$과 $x-1$을 동일시하는 것과 같은 의미이다. 따라서 $x^4 + x^2 +1$을 $x^2 - x +1$로 나눈 나머지는 $x^2 \equiv x -1$를 반복하여 사용하여
\begin{align*}
x^{4} + x^{2} + 1 & \equiv \left(x^2 \right)^2 + (x-1) +1 \pmod{x^{2} -x +1}\\
& \equiv (x-1)^2 + x \left(x^2 \right)^2 + (x-1) +1 \pmod{x^{2} -x +1} \\
& \equiv x^2 - x + 1 \equiv 0 \pmod{x^{2} -x +1}
\end{align*}
을 얻으므로 $x^4 + x^2 +1$은 $x^2 - x +1$로 나누어떨어진다.
\vspace{1em}
\begin{example}
 $x^{2021} -1$을 $x^2 + x +1$로 나눈 나머지를 구하시오.
\end{example}

\textbf{풀이)}  $x^2 + x +1$로 $x^{2021} -1$을 나누는 것은 $x^2 + x +1$을 $0$과 동일시하는 것이므로 $x^2 + x +1\equiv 0$이고 이 식의 양변에 $x-1$을 곱하면 $x^3 - 1\equiv 0$을 얻는다. 즉 $x^3$을 $1$과 동일시할 수 있음을 의미한다. 따라서 $x^3 \equiv 1$, $x^2 \equiv -x-1$의 관계를 이용하여
\begin{align*}
    x^{2021} -1 &\equiv \left(x^3 \right)^{673} \times x^{2} -1 \\
    &\equiv x^{2} - 1 \\
    &\equiv (-x-1) - 1 \\
    & \equiv -x -2  \pmod{x^{2} +x +1}
\end{align*}
임을 얻을 수 있고 따라서 구하는 나머지는 $-x-2$이다.

  한편 이러한 동일시의 관점으로 나누는 것을 일반화하여 생각하는 것의 또 다른 이점은 부정원이 여러 개인 다항식에서의 \textbf{나머지 정리} 및 \textbf{인수정리}에 적용이 가능하다는 점이다. 간단한 예로서 우리는 곱셈공식으로부터
\[
a^2 - b^2 = (a-b)(a+b)
\]
임을 쉽게 알 수 있다.

  그런데 동일시라는 관점에서 $a^2 -b^2$을 $a-b$로 나누는 것은 $a-b$를 $0$과 동일시하는 것이고 이것은 $a$와 $b$를 동일시하는 것이다. $a$와 $b$를 동일시하면 
  \[
  a^2 -b^2 \equiv a^2 -a^2 \equiv 0  \pmod{a-b}
  \]
  이고 따라서 $a^2 - b^2$은 $a-b$로 나누어떨어진다. 또한 $a$와 $-b$를 동일시하여도 마찬가지 이므로 $a+b$로도 나누어떨어짐을 알 수 있다.

  한편 $a^2 -b^2$을 $a-2b$로 나눈 나머지를 구하여 보면, $a$와 $2b$를 동일시함으로서
  \[
  a^2 - b^2 \equiv 4b^2 - b^2 \equiv 3b^2 \pmod{a-2b}
  \]
이므로 구하는 나머지는 $3b^2$이다.

  이제 몇 가지 예와 문제를 통하여 동일시라는 관점에서 주어진 식의 인수분해를 생각해 보자.
\vspace{1em}

\begin{example}
 $a(b^2 - c^2 ) +b(c^2 -a^2 ) + c(a^2 - b^2 )$을 인수분해 하시오.
\end{example} 

\textbf{풀이)} 주어진 식에서 $a$와 $b$를 동일시하면 주어진 식은 $0$이고 따라서 주어진 식은 $a-b$를 인수로 갖는다. 마찬가지 방법으로 주어진 식은 $b-c$, $c-a$를 인수로 가짐을 알 수 있다. 또한 주어진 식이 $3$차식이므로,
\[
a(b^2 - c^2) + b(c^2 - a^2 ) + c(a^2 - b^2 ) = K(a-b)(b-c)(c-a)
\]

임을 알 수 있고 위의 식은 $a,\:b,\: c$에 대한 항등식이므로 양변에 적당한 수를 대입하여 $K$를 결정하면 $K=-1$임을 알 수 있다. 즉
\[
a(b^2 - c^2) + b(c^2 - a^2 ) + c(a^2 - b^2 ) = - (a-b)(b-c)(c-a)
\]

\vspace{1em}

\begin{theorem}
 임의의 자연수 $n$에 대하여
\[
a^{n}(b-c) + b^{n} (c-a) + c^{n}(a-b)
\]
는 $(a-b)(b-c)(c-a)$로 나누어떨어짐을 보이시오.
\end{theorem}  

\textbf{증명)} 주어진 식에서 $a$와 $b$를 동일시하면 주어진 식은 $0$이 되므로 $a-b$로 주어진 식은 나누어떨어진다. 마찬가지 방법으로 $b-c$, $c-a$로 나누어떨어짐을 보일 수 있고 따라서 주어진 식은 $(a-b)(b-c)(c-a)$로 나누어떨어짐을 알 수 있다.

\vspace{1em}
\begin{problem}
 다음 식을 인수분해 하시오.
\[
a^{2}b^{2}(b-a) + b^{2}c^{2}(c-b) + c^{2}a^{2}(a-c)
\]
\end{problem}
\textbf{풀이)} 앞의 문제와 예와 같은 방법으로 주어진 식은 $(a-b)(b-c)(c-a)$로 나누어떨어짐을 알 수 있고 따라서
\[
a^{2}b^{2}(b-a) + b^{2}c^{2}(c-b) + c^{2}a^{2}(a-c)=(a-b)(b-c)(c-a)f(a, \:b,\: c)
\]
이고 양변의 차수를 비교하면 $f(a,\:b, \:c)$는 $a,\:b\;,c$에 대한 $2$차 대칭식이어야 하고 위의 식이 항등식임을 이용하면 
$$
f(a,\;b,\;c) = k_{1}(a+b+c)^2 + k_{2}(ab +bc +ca)
$$
임을 알 수 있다. 이제 양번에 적당한 수를 대입하여 $k_{1}$과 $k_{2}$를 구하면 
$k_{1}=0$, $k_{2}=1$임을 알 수 있다.
즉,
\[
a^{2}b^{2}(b-a) + b^{2}c^{2}(c-b) + c^{2}a^{2}(a-c)=(a-b)(b-c)(c-a)(ab+bc+ca)
\]
이다.
\vspace{1em}
\begin{definition}
  어떤 다항식이 구성하는 문자들의 위치를 바꿀 때마다 식의 부호가 변하는 식을 교대식이라고 한다. 즉,
  \[
  f(a,\: b,\: c) = - f(b, \:a, \: c) = f(b,\: c,\: a) = \cdots
  \]
\end{definition}
  위의 세 문제를 살펴보면 모두 $(a-b)(b-c)(c-a)$를 인수로 갖는다. 이제 이러한 성질을 갖는 다항식의 특징을 생각하는 것은 자연스러운 생각이다. 실제로 위의 세 문제의 다항식들은 모두 교대식임을 알 수 있다. 따라서 다음이 성립함을 알 수 있다.
  \vspace{1em}
  
\begin{theorem}
모든 교대식 $f(a,\:b,\:c)$는 다음과 같이 인수분해 됨을 보여라.
$$ f(a,\:b,\:c) = (a-b)(b-c)(c-a)g(a,\:b,\:c), \:\:(\text{단, }g(a,\:b,\:c)\text{는 대칭식})$$
\end{theorem}

\begin{problem}
 $c^{2}(a^2 + b^2 - c^2 ) - b^{2}(c^2 + a^2 - b^2 )$을 인수분해 하시오.
\end{problem}

\textbf{풀이)} $b^2$과 $c^2$을 동일시하면 주어진 식이 $0$이 되므로 $b^2 - c^2$을 인수로 갖고, 한편 $a^2$과 $b^2$을 동일시해도 주어진 식이 $0$이 되므로
\[
c^{2}(a^2 + b^2 - c^2 ) - b^{2}(c^2 + a^2 - b^2 ) = K (b^2 - c^2 )(a^2 -b^2 -c^2 )
\]
이다. 양 변이 $a,\:b,\: c$에 대한 항등식이고 차수비교를 하면 $K$는 상수이고 양변에 적당한 수들을 대입하면 $K=1$을 얻는다. 즉,
\[
c^{2}(a^2 + b^2 - c^2 ) - b^{2}(c^2 + a^2 - b^2 ) = K (b^2 - c^2 )(a^2 -b^2 -c^2 )
\]
이다.

\section{인수정리의 일반화 2}

  한편 인수정리의 일반화를 다항식의 근이라는 측면에서 접근해 보도록 하자. 다항식 $f(x)$가 $g(x)$로 나누어떨어진다고 가정하자. 즉,
  \[
  f(x) = g(x) Q(x)
  \]
라 하면 $g(x)$의 모든 근은 $f(x)$의 근이 되므로 다음과 같은 일반화된 인수정리를 생각할 수 있다. 
\vspace{1em}
\begin{theorem}
$x$에 대한 다항식 $f(x)$에 대하여,
\[
f(x) = g(x)Q(x) \Leftrightarrow \forall \alpha, \: g(\alpha)=0 \text{이면 } f(\alpha)=0
\]
이다.
\end{theorem}
 

  인수정리 2를 활용하면 다음 문제는 쉽게 해결할 수 있다.
\vspace{1em}
\begin{problem}
 다항식 $ x^4 + x^3 + x^2 + x+1$은 다항식 $x^{44} + x^{33} + x^{22} + x^{11} + 1$을 나눔을 보여라.
\end{problem}

\textbf{풀이)} $ x^4 + x^3 + x^2 + x+1$의 근은 $1$을 제외한 $1$의 $5$제곱근들이므로 이 $4$차 다항식의 임의의 근 $\alpha$에 대하여 $\alpha^{5} = 1$이다. 그런데
\[
\alpha^{44} + \alpha^{33} + \alpha^{22} + \alpha^{11} + 1 = \alpha^4 + \alpha^3 + \alpha^2 + \alpha + 1
\]
이므로 $ x^4 + x^3 + x^2 + x+1$의 모든 근은  $x^{44} + x^{33} + x^{22} + x^{11} + 1$의 근이된다. 따라서 문제가 증명 되었다.\vspace{1em}
 
 
  마지막으로 본 논의와 상관이 없지만 재미있는 방식의 인수분해를 하나 소개한다. 여기에 소개하는 방식의 인수분해는 $x^2+(a+b)x+ab$꼴의 인수분해는 비교적 잘 할 수 있으나 $acx^2+(ad+bc)x+bd$꼴의 인수분해를 어려워하는 학생들에게 제한적으로 활용할 수 있을 것이다. 먼저 $6x^2+x-15$의 인수분해를 다음과 같이 독특하게 해 보자.
  
  \begin{align*}
      6x^2+x-15 &\Rightarrow x^2+x-90 \\
      &\Rightarrow (x+1)(x-9) \\
      &\Rightarrow (6x+10)(6x-9) \\
      & \Rightarrow (3x+5)(2x-3)
  \end{align*}

 대수적으로 이것을 살펴보면 다음과 같다.
\begin{align*}
      acx^2 +(ad+bc)x + db &\Rightarrow x^2+(ad+bc)x +abcd \\
      &\Rightarrow (x+ad)(x+bc) \\
      &\Rightarrow (acx+ad)(acx+bc) \\
      & \Rightarrow (cx+d)(ax+b)
  \end{align*}
  이제 위에서 언급한 인수분해 이론을 적용하여 다음 몇 개의 식을 인수분해 하여 보도록 하자.

\section{일반화된 인수정리의 활용}
이제 일반화된 인수정리를 활용하여 여러 문제들을 해결해 보자.
\vspace{1em}
\begin{problem}
 다음의 다항식을 정수계수 다항식의 곱으로 인수분해 하시오.
 \[
 (x+y+z)^4 - (x+y)^4 - (y+z)^4 - (z+x)^4 +x^4 + y^4 + z^4
 \]
\end{problem}

\textbf{풀이)} 주어진 식을 $P_{1}(x,\:y,\:z)$라 하자. 
\[
P_{1}(0, \:y,\:z) = P_{1}(x,\:0, \:z) = P_{1}(x,\:y,\:0) =0
\]
이므로 일반화된 인수정리에 의하여 $P_{1}(x,\:y,\:z)$는 $x,\:y,\:z$로 나누어 떨어진다. 즉, 
\[
P_{1}(x,\:y,\:z) = x y z \: Q_{1}(x,\: y,\:z)
\]
이고 이때 $Q_{1}(x,\: y,\:z)$는 차수 비교에 의하여 $1$차식임을 알수 있다. 한편 $P_{1}(x,\:y,\:z)$는 대칭식이므로 $Q_{1}(x,\: y,\:z)$도 대칭식이다. 결론적으로 $Q_{1}(x,\: y,\:z)$는 $1$차의 대칭식이므로 
\[
Q_{1}(x,\: y,\:z) = k(x+y+z)
\]
임을 알 수 있다. 한편 $3k=P_{1}(1,\:1,\:1)=81-48+3=36$이므로 $k=12$이다. 따라서
\[
P_{1}(x,\:y,\:z) = 12 x y z (x+y+z)
\]
이다.
\vspace{1em}

\begin{problem}
 다음의 다항식을 정수계수 다항식의 곱으로 인수분해 하시오.
 \[
 x^2(y^3-z^3)+y^2(z^3-x^3)+z^2(x^3-y^3)
 \]
\end{problem}

\textbf{풀이)} 주어진 식을 $P_{2}(x,\:y,\:z)$라 하면 에 대한 교대식이므로
\[
P_{2}(x,\:y,\:z) = (x-y)(y-z)(z-x)Q_{2}(x,\:y,\:z)
\]
이고 $Q_{2}(x,\: y,\:z)$는 대칭식임을 알 수 있다. 양 변의 $x,\:y,\:z$에 대한 차수를 비교해 보면 $Q_{2}(x,\: y,\:z)$는 $2$차식임을 알 수 있다. 따라서 $Q_{2}(x,\: y,\:z)$는 $2$차의 대칭식이며 주어진 식이 $x,\:y,\:z$의 차수가 $4$차 이상인 경우가 없으므로 $Q_{2}(x,\: y,\:z)$는 $(x+y+z)^2$을 포함하지 않는다. 따라서
\[
P_{2}(x,\:y,\:z) =k(x-y)(y-z)(z-x)(xy+yz+zx)
\]
라 할 수 있고 $P_{2}(1,\:2,\:3) = 22k=22$에서 $k=1$이므로
\[
P_{2}(x,\:y,\:z) = (x+y)(y+z)(z+x) (xy+yz+zx)
\]
이다.
\vspace{1em}
\begin{problem}
  다음의 다항식을 정수계수 다항식의 곱으로 인수분해 하시오.
  \[
  (yz+zx+xy)^3 - y^3z^3 -z^3x^3-x^3y^3
  \]
\end{problem}

\textbf{풀이)} 주어진 다항식을 $P_{3}(x,\:y,\:z)$라 하면 
\[
P_{3}(0,\:y,\:z) = P_{2}(x,\:0,\:z) = P_{2}(x,\:y,\:0)=0
\]
이므로 $P_{3}(x,\:y,\:z)$는 $x, \:y,\:z$를 인수로 갖는다.

한편 
\[
P_{3}(x,\:-x,\:z) = P_{2}(x,\:y,\:-y) = P_{2}(-z,\:y,\:z)=0
\]
이므로 $P_{3}(x,\:y,\:z)$는 $(x+y), \:(y+z),\:(z+x)$를 인수로 갖는다. 또한 $P_{3}(x,\:y,\:z)$는 $x, \:y,\:z$에 대한 $6$차식이므로 
\[
P_{3}(x,\:y,\:z) = k x y z (x+y)(y+z)(z+x)
\]
이다. 그런데 $8k = P_{3}(1,\:1,\:1) =24$에서 $k=3$이고 따라서
\[
(xy+yz+zx)^3 - x^3y^3-y^3z^3-z^3x^3 =3xyz(x+y)(y+z)(x+x)
\]
이다.

\vspace{1em}
\begin{problem}
  다음의 다항식을 정수계수 다항식의 곱으로 인수분해하시오.
  \[
  x^3y^3+y^3z^3+z^3x^3 - x^4yz-xy^4z-xyz^4
  \]
\end{problem}

\textbf{풀이)} 주어진 다항식을 $P_{4}(x,\:y,\:z)$라 하고 $x^2$과 $yz$를 동일시 하면 $P_{4}(x,\:y,\:z)=0$이고,
$y^2$과 $zx$, $z^2$과 $xy$을 각각 동일시해도 마찬가지이다. 따라서 $P_{4}(x,\:y,\:z)$는 $x^2-yz$, $y^2-zx$, $z^2-xy$를 인수로 갖는다. 한편 $P_{4}(x,\:y,\:z)$는 $x, \:y,\:z$에 관한 $6$차식이므로
\[
P_{4}(x,\:y,\:z) = k (x^2-yz)(y^2-zx)(z^2-xy)
\]
라 놓을 수 있다. $x, \:y,\:z$에 적당한 값을 대입하여 $k=-1$임을 알 수 있고 따라서
\[
P_{4}(x,\:y,\:z) = - (x^2-yz)(y^2-zx)(z^2-xy) =(yz-x^2)(zx-y^2)(xy-z^2)
\]
이 성립한다.

\vspace{3em}

\section{ 연 습 문 제 }

\begin{problem}
다음 식이 성립함을 보이시오.
\[
\frac{b-c}{a} + \frac{c-a}{b} +\frac{a-b}{c} = \frac{(a-b)(b-c)(c-a)}{abc}
\]
\end{problem} 
\vspace{2em}
\begin{problem}
다항식 $x^{10}+x^5+1$은 $x^2+x+1$로 나누어 떨어짐을 보이시오.
\end{problem}

\end{document}