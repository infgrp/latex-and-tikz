% 반드시 XeLatex으로 컴파일을 해야 합니다.
% 같이 제공하는 preamblex.tex은 trig.tex보다 상위 디렉토리에 두어야 컴파일이 됩니다. 두 파일을 같은 디렉토리에 두려면 몇 줄 아래에 있는 % AMS and mathtools
\usepackage{amsmath,amsthm,amssymb,marvosym,mathrsfs,amsfonts,amscd,mathtools}
% Hyperlinks and URLs
\usepackage{url}
\usepackage{hyperref}
\hypersetup{
	colorlinks,
	citecolor=BLACK,
	filecolor=BLACK,
	linkcolor=BLACK,
	urlcolor=BLACK
}

% Colors
\usepackage[usenames,dvipsnames]{xcolor}
\usepackage{tikz}
\usepackage{tkz-euclide}

% shadowing mdframed
\usepackage[framemethod=tikz]{mdframed}
\usetikzlibrary{shadows}
% Bold math
\usepackage{bm}

%Use Korean Letter when enumerate
\usepackage{dhucs-enumerate}

\usepackage{anyfontsize}

% Bra Ket (Dirac) Notation
\usepackage{braket}

% Slashed characters (e.g. in Dirac equation)
\usepackage{slashed}
\usepackage{pifont} % 원문자 사용시 필요한 패키지

% chapter decoration
\usepackage{type1cm}
\usepackage[explicit]{titlesec}

\titleformat{\chapter}[display]
{\normalfont\Large\rmfamily}
{\sffamily\flushright\fontsize{60}{0}\textbf{\textcolor{blue!40}{{\Huge\chaptername}~\thechapter\vskip0pt\rule{\textwidth}{2pt}}}}{0pt}
{\flushleft\fontsize{30}{0}{#1}\vskip60pt}
\titlespacing*{\chapter}
{0pt}{-40pt}{0pt}

%\usetikzlibrary{shadows}
\usetikzlibrary{shadows.blur}
\usetikzlibrary{shapes.symbols}

% Tcolorbox
\usepackage[most]{tcolorbox}

% Clean SI Units
\usepackage{siunitx}

% Enumerate thingies
\usepackage{enumitem}

% Cancel things out in equations
\usepackage[makeroom]{cancel}

\usepackage{multicol}

% Graphics and figures
\usepackage{graphicx}
\usepackage{wrapfig}
\usepackage{float}

\usepackage{cancel}

% Caption figures and tables
\usepackage{caption,subcaption}

% Generate symbols
\usepackage{textcomp} % Include this line to avoid output errors
\usepackage{gensymb}

% Make multiple rows in a table
\usepackage{multirow}

% Booktabs tables
\usepackage{booktabs}

%\usepackage[utopia,sfscaled]{mathdesign}
% Useful frames
\usepackage{mdframed}

% Comment-out large sections
\usepackage{comment}

% No auto-indent
\setlength{\parindent}{0pt}

% Asymptote - 3D vector graphics
\usepackage{asymptote}

% Tikz Package Stuff
\usepackage{pgf,tikz,pgfplots}
\usepackage{tikz-3dplot}
\usepackage{tabularx}
\usepackage{array}
\usepackage{colortbl}
\tcbuselibrary{skins}
\usepackage{tkz-euclide}

\newcolumntype{Y}{>{\raggedleft\arraybackslash}X}

\tcbset{tab1/.style={fonttitle=\bfseries\large,fontupper=\normalsize\sffamily,
		colback=yellow!10!white,colframe=red!75!black,colbacktitle=Salmon!40!white, halign=center,
		coltitle=black,center title,freelance,frame code={
			\foreach \n in {north east,north west,south east,south west}
			{\path [fill=red!75!black] (interior.\n) circle (3mm); };},}}

\tcbset{tab2/.style={enhanced,fonttitle=\bfseries,fontupper=\normalsize\sffamily, halign=center, box align=center,
		colback=yellow!10!white,colframe=red!50!black,colbacktitle=Salmon!40!white,
		coltitle=black,center title}}


% Use various tikz libraries
\usetikzlibrary{decorations.pathmorphing, decorations.markings, decorations.pathreplacing, patterns} % Decorate paths!
\usetikzlibrary{calc, patterns, shapes.geometric, positioning, through, intersections}
\usetikzlibrary{scopes}
\usetikzlibrary{angles, quotes}
\usetikzlibrary{svg.path}
\usetikzlibrary{arrows, arrows.meta}
\usetikzlibrary{fadings}
% pgfplots package settings
\pgfplotsset{compat=1.15}
% \pgfplotsset{width=10cm,compat=1.9} % Taken from latest overleaf.
% plot arc easily
\def\centerarc[#1](#2)(#3:#4:#5)% Syntax: [draw options] (center) (initial angle:final angle:radius)
{ \draw[#1] ($(#2)+({#5*cos(#3)},{#5*sin(#3)})$) arc (#3:#4:#5); }

% Awesome circled numbers
\newcommand*\circled[4]{\tikz[baseline=(char.base)]{\node[shape=circle, fill=#2, draw=#3, text=#4, inner sep=2pt] (char) {#1};}}

% Control size of text
\usepackage{relsize}

% Extend conditional commands
\usepackage{xifthen}
\usepackage{xcolor}
\definecolor{termcolor}{cmyk}{.21,.97,.0,.0}
\definecolor{darkred}{cmyk}{.27,1,1,.32}
\definecolor{darkblue}{cmyk}{1,.98,.10,.11}
\definecolor{darkgreen}{cmyk}{.29,0,87,0}
\definecolor{darkmycolor}{cmyk}{99,59,22,3}
\definecolor{for_eyes}{RGB}{253,247,228}
%change color of math equation
%\everymath{\color{darkred}}
% Scale math by size
\newcommand*{\Scale}[2][4]{\scalebox{#1}{\ensuremath{#2}}}

% Big integrals
\usepackage{bigints}

% Number equations within sections
\numberwithin{equation}{section}

% Generate blind text
\usepackage{blindtext}

% Useful symbols
\usepackage{marvosym}

\newcounter{problem}[section]
\newcounter{example}[section]

% cancel 색상 변경
\newcommand\Ccancel[2][black]{\renewcommand\CancelColor{\color{#1}}\cancel{#2}}

%%%% 원문자
\newcommand*\ocircled[1]{\tikz[baseline=(char.base)]{
		\node[shape=circle,draw,inner sep=2pt] (char) {#1};}}
	
%%%%% 보기 스타일 %%%%%
\usepackage{tabu}
\newcommand{\questwo}[2]{
	\vskip 6pt
	\noindent\begin{tabu}{X[0.2] X[6] X[0.2] X[6]}
		(1)&$#1$ &(2) &$#2$
	\end{tabu}
}
\newcommand{\questhree}[3]{
	\vskip 3pt
	\noindent\begin{tabu}{X[0.2] X[6] X[0.2] X[6] X[0.2] X[6]}
		(1)&$#1$ &(2) &$#2$ &(3) & $#3$
	\end{tabu}
	\vskip 5pt
}
\newcommand{\quesfour}[4]{
	\vskip 6pt
	\noindent\begin{tabu}{X[0.2] X[6] X[0.2] X[6]}
		(1)&$#1$ &(2) &$#2$\\
		(3)&$#3$ &(4) &$#4$
	\end{tabu}
}
\newcommand{\quesfive}[5]{
	\vskip 6pt
	\noindent\begin{tabu}{X[0.2] X[3] X[0.2] X[3] X[0.2] X[3]}
		(1)&$#1$ &(2) &$#2$ &(3) &$#3$\\
		(4)&$#4$ &(5) &$#5$
	\end{tabu}
}

\newcommand{\oquesfive}[5]{
	\vskip 6pt
	\noindent\begin{tabu}{X[0.2] X[3] X[0.2] X[3] X[0.2] X[3] X[0.2] X[3] X[0.2] X[3]}
		\ding{172}&$#1$ &\ding{173} &$#2$ &\ding{174} &$#3$&	\ding{175}&$#4$ &\ding{176} &$#5$
	\end{tabu}
}
%%%% sample(보기) %%%%
\newenvironment{sample}{\vskip 10pt\noindent\begin{tikzpicture}[yshift=1.5pt]%
		\draw[rounded corners=1ex,overlay,draw=blue] (0pt,-4pt) rectangle (19pt,9pt);
		\node[rectangle,overlay,xshift=9.8pt,yshift=2pt,color=blue] {{\footnotesize\sffamily 보기}};\end{tikzpicture}\phantom{\footnotesize\sffamily보기...}}{\vskip 10pt}
%%%% THEOREMS %%%%
\newenvironment{theorem}[1][\hspace{-0.36em}]
{
	\begin{mdframed}[backgroundcolor=purple!5, align=center, userdefinedwidth=40em, linecolor=purple!30, linewidth=2pt, roundcorner=7pt, innertopmargin=10pt, shadow=true, shadowcolor=black!20, roundcorner=7pt, innerbottommargin=10pt, frametitle = {#1}]
	}
	{
	\end{mdframed}
}

%%%% LEMMAS %%%%
\newenvironment{lemma}[1][\hspace{-0.36em}]
{
	\begin{mdframed}[backgroundcolor=black!8, align=center, userdefinedwidth=40em, topline=false, bottomline = false, leftline = false, rightline = false, frametitle = {#1 보조정리}]
	}
	{
	\end{mdframed}
}

%%%% COROLLARY %%%%
\newenvironment{corollary}[1][\hspace{-0.36em}]
{
	\begin{mdframed}[backgroundcolor=black!8, align=center, userdefinedwidth=40em, topline=false, bottomline = false, leftline = false, rightline = false, frametitle = {#1 따름정리}]
	}
	{
	\end{mdframed}
}

%%%% DEFINITIONS %%%%
\newenvironment{definition}[1][\hspace{-0.36em}]
{
	\begin{mdframed}[backgroundcolor=cyan!14, align=center, userdefinedwidth=40em, linecolor=cyan!60, linewidth=2pt,roundcorner=7pt, innertopmargin=10pt, shadow=true, shadowcolor=black!20, roundcorner=7pt, innerbottommargin=10pt, frametitle = {정의 : #1}]
	}
	{
	\end{mdframed}
}

%%%% PROPOSITION %%%%
\newenvironment{proposition}
{
	\begin{mdframed}[backgroundcolor=black!4, align=center, userdefinedwidth=40em, topline=false, bottomline = false, leftline = false, rightline = false, frametitle = {Proposition}]
	}
	{
	\end{mdframed}
}
%%%% PROBLEM %%%%
\newenvironment{problem}{\refstepcounter{problem}
	\begin{mdframed}[linecolor=blue!35, linewidth=2pt, roundcorner=7pt, innertopmargin=10pt, shadow=true, shadowcolor=black!20, roundcorner=7pt, innerbottommargin=10pt, backgroundcolor=blue!5]
		\noindent 
		\noindent\begin{tikzpicture}[overlay,xshift=6pt,yshift=5pt]
			\draw[fill=violet!15,draw=violet!15] (0,0) circle (8pt);
			\draw[fill=violet!15,draw=violet!15] (9pt,0) circle (8pt);
			\node[rectangle,overlay,xshift=4pt] {\color{black}\sffamily\bfseries 문제};
		\end{tikzpicture}{\phantom{.........}\fontspec[Scale=1.1]{TeX Gyre Adventor}\color{darkblue} \theproblem}\hspace{5pt}}{
\end{mdframed}}
%%%% SOLUTION OF PROBLEM %%%%
\newenvironment{psolution}{\begin{description}\item[{\begin{tikzpicture}%
				\draw[rounded corners=1ex,overlay] (-3pt,-3pt) rectangle (20pt,9pt);\end{tikzpicture}\footnotesize\sffamily풀이}\hspace{6pt}]}{\end{description}}
			
%%%% EXAMPLE %%%%		
\newenvironment{example}{\refstepcounter{example}
	\begin{mdframed}[roundcorner=7pt,linecolor=termcolor,linewidth=2pt,innertopmargin=10pt, shadow=true, shadowcolor=black!20, innerbottommargin=10pt, backgroundcolor=termcolor!2]
		\noindent 
		\noindent\begin{tikzpicture}[overlay,xshift=6pt,yshift=5pt]
			\draw[fill=darkred!60,draw=darkred!60] (0,0) circle (7pt);
			\draw[fill=darkred!60,draw=darkred!60] (9pt,0) circle (7pt);
			\node[rectangle,overlay,xshift=4pt] {\color{white}\sffamily\bfseries 예제};
		\end{tikzpicture}{\phantom{.........}\fontspec[Scale=1.1]{TeX Gyre Adventor}\color{darkred} \theexample}\hspace{5pt}}{
\end{mdframed}}		
%%%%%%% SOLUTION OF EXAMPLE %%%%%%%%%
\newenvironment{solution}{\begin{description}\item[{\begin{tikzpicture}%
				\draw[rounded corners=1ex,overlay] (-3pt,-3pt) rectangle (20pt,9pt);\end{tikzpicture}\footnotesize\sffamily풀이}\hspace{6pt}]}{\end{description}}

% 보기 박스 정의 시작
\tcbuselibrary{breakable, skins}
\tcbset{enhanced}
\newtcolorbox{ChoiceBox}[1]{
		enhanced,
		before skip=2ex, after skip=2ex,
		boxrule=0.5pt, colframe=black, colback=white, arc=0.5ex,
		boxsep=0.5ex, top=1.5ex, bottom=1.5ex, left=0.5em, right=o0.5em,
		colbacktitle=white, coltitle=black,
		attach boxed title to top center={xshift=0cm, yshift=-1.5mm},
		boxed title style={size=minimal, enhanced, boxrule=0.25pt, colframe=white},
		breakable=false, title ={< #1 >}
}
% 보기 박스 정의 끝.
	
% Change end-of-proof symbol
\renewcommand\qedsymbol{$\blacksquare$}
%overline
\newcommand{\ovr}[1]{\overline{\textrm{#1}}}
% trigonometric function
\newcommand{\cosrm}[1]{\cos \textrm{#1}}
\newcommand{\sinrm}[1]{\sin \textrm{#1}}
\newcommand{\tanrm}[1]{\tan \textrm{#1}}
\newcommand{\cotrm}[1]{\cot \textrm{#1}}
\newcommand{\cscrm}[1]{\csc \textrm{#1}}
\newcommand{\secrm}[1]{\sec \textrm{#1}}
%%%% BLACKBOARD BOLD %%%%
\newcommand{\bbN}{\mathbb{N}} % Natural numbers
\newcommand{\bbZ}{\mathbb{Z}} % Zahlen
\newcommand{\bbQ}{\mathbb{Q}} % Rational numbers
\newcommand{\bbR}{\mathbb{R}} % Real numbers
\newcommand{\bbC}{\mathbb{C}} % Complex numbers
\DeclareSymbolFont{bbold}{U}{bbold}{m}{n} % Identity matrix
\DeclareSymbolFontAlphabet{\mathbbold}{bbold} % Identity matrix
\newcommand{\identitymatrix}{\mathbbold{1}} % Identity matrix

%%%% CODE LISTING %%%%
\usepackage{listings}
\definecolor{greencomments}{HTML}{00BA00}
\definecolor{graynumbers}{HTML}{4F4F4F}
\definecolor{purplestrings}{HTML}{AD00AA}
\definecolor{backgroundcolor}{HTML}{E8E8E8}


%%%% UNIT BASIS VECTORS %%%%
\newcommand{\ihat}{\bm{\hat{\imath}}} % Cartesian i hat (x-direction)
\newcommand{\jhat}{\bm{\hat{\jmath}}} % Cartesian j hat (y-direction)
\newcommand{\khat}{\bm{\hat{k}}} % Cartesian k hat (z-direction)
\newcommand{\rhat}{\bm{\hat{r}}} % Spherical r hat
\newcommand{\phihat}{\bm{\hat{\phi}}} % Spherical phi hat
\newcommand{\thetahat}{\bm{\hat{\theta}}} % Spherical theta hat
\newcommand{\nhat}{\bm{\hat{n}}} % Unit normal vector
\newcommand{\rhohat}{\bm{\hat{\rho}}} % Cylindrical rho hat
\newcommand{\zhat}{\bm{\hat{z}}} % Cylindrical z hat


%%%% COLORS: DEFINITIONS AND COMMANDS %%%%
% Miscellaneous
\definecolor{DARKBLUE}{HTML}{040080}
\definecolor{DARKBROWN}{HTML}{8B4513}
\definecolor{LIGHTBROWN}{HTML}{CD853F}
\definecolor{PINK}{HTML}{D147BD}
\definecolor{LIGHTPINK}{HTML}{DC75CD}
\definecolor{GREENSCREEN}{HTML}{00FF00}
\definecolor{ORANGE}{HTML}{FF862F}
\newcommand{\DARKBLUE}{\color{DARKBLUE}}
\newcommand{\DARKBROWN}{\color{DARKBROWN}}
\newcommand{\LIGHTBROWN}{\color{LIGHTBROWN}}
\newcommand{\PINK}{\color{PINK}}
\newcommand{\LIGHTPINK}{\color{LIGHTPINK}}
\newcommand{\GREENSCREEN}{\color{GREENSCREEN}}
\newcommand{\ORANGE}{\color{ORANGE}}
% Blue
\definecolor{BLUEE}{HTML}{1C758A}
\definecolor{BLUED}{HTML}{29ABCA}
\definecolor{BLUEC}{HTML}{58C4DD}
\definecolor{BLUEB}{HTML}{9CDCEB}
\definecolor{BLUEA}{HTML}{C7E9F1}
\definecolor{BLUE}{HTML}{0000FF}
\newcommand{\BLUEE}{\color{BLUEE}}
\newcommand{\BLUED}{\color{BLUED}}
\newcommand{\BLUEC}{\color{BLUEC}}
\newcommand{\BLUEB}{\color{BLUEB}}
\newcommand{\BLUEA}{\color{BLUEA}}
\newcommand{\BLUE}{\color{BLUE}}
% Teal
\definecolor{TEALE}{HTML}{49A88F}
\definecolor{TEALD}{HTML}{55C1A7}
\definecolor{TEALC}{HTML}{5CD0B3}
\definecolor{TEALB}{HTML}{76DDC0}
\definecolor{TEALA}{HTML}{ACEAD7}
\definecolor{TEAL}{HTML}{00FFFF}
\newcommand{\TEALE}{\color{TEALE}}
\newcommand{\TEALD}{\color{TEALD}}
\newcommand{\TEALC}{\color{TEALC}}
\newcommand{\TEALB}{\color{TEALB}}
\newcommand{\TEALA}{\color{TEALA}}
\newcommand{\TEAL}{\color{TEAL}}
% Green
\definecolor{GREENE}{HTML}{699C52}
\definecolor{GREEND}{HTML}{77B05D}
\definecolor{GREENC}{HTML}{83C167}
\definecolor{GREENB}{HTML}{A6CF8C}
\definecolor{GREENA}{HTML}{C9E2AE}
\definecolor{GREEN}{HTML}{00FF00}
\newcommand{\GREENE}{\color{GREENE}}
\newcommand{\GREEND}{\color{GREEND}}
\newcommand{\GREENC}{\color{GREENC}}
\newcommand{\GREENB}{\color{GREENB}}
\newcommand{\GREENA}{\color{GREENA}}
\newcommand{\GREEN}{\color{GREEN}}
% Yellow
\definecolor{YELLOWE}{HTML}{E8C11C}
\definecolor{YELLOWD}{HTML}{F4D345}
\definecolor{YELLOWC}{HTML}{FFFF00}
\definecolor{YELLOWB}{HTML}{FFEA94}
\definecolor{YELLOWA}{HTML}{FFF1B6}
\definecolor{YELLOW}{HTML}{FFFF00}
\newcommand{\YELLOWE}{\color{YELLOWE}}
\newcommand{\YELLOWD}{\color{YELLOWD}}
\newcommand{\YELLOWC}{\color{YELLOWC}}
\newcommand{\YELLOWB}{\color{YELLOWB}}
\newcommand{\YELLOWA}{\color{YELLOWA}}
\newcommand{\YELLOW}{\color{YELLOW}}
% Gold
\definecolor{GOLDE}{HTML}{C78D46}
\definecolor{GOLDD}{HTML}{E1A158}
\definecolor{GOLDC}{HTML}{F0AC5F}
\definecolor{GOLDB}{HTML}{F9B775}
\definecolor{GOLDA}{HTML}{F7C797}
\newcommand{\GOLDE}{\color{GOLDE}}
\newcommand{\GOLDD}{\color{GOLDD}}
\newcommand{\GOLDC}{\color{GOLDC}}
\newcommand{\GOLDB}{\color{GOLDB}}
\newcommand{\GOLDA}{\color{GOLDA}}
% Red
\definecolor{REDE}{HTML}{CF5044}
\definecolor{REDD}{HTML}{E65A4C}
\definecolor{REDC}{HTML}{FC6255}
\definecolor{REDB}{HTML}{FF8080}
\definecolor{REDA}{HTML}{F7A1A3}
\definecolor{RED}{HTML}{FF0000}
\newcommand{\REDE}{\color{REDE}}
\newcommand{\REDD}{\color{REDD}}
\newcommand{\REDC}{\color{REDC}}
\newcommand{\REDB}{\color{REDB}}
\newcommand{\REDA}{\color{REDA}}
\newcommand{\RED}{\color{RED}}
% Maroon
\definecolor{MAROONE}{HTML}{94424F}
\definecolor{MAROOND}{HTML}{A24D61}
\definecolor{MAROONC}{HTML}{C55F73}
\definecolor{MAROONB}{HTML}{EC92AB}
\definecolor{MAROONA}{HTML}{ECABC1}
\newcommand{\MAROONE}{\color{MAROONE}}
\newcommand{\MAROOND}{\color{MAROOND}}
\newcommand{\MAROONC}{\color{MAROONC}}
\newcommand{\MAROONB}{\color{MAROONB}}
\newcommand{\MAROONA}{\color{MAROONA}}
% Purple
\definecolor{PURPLEE}{HTML}{644172}
\definecolor{PURPLED}{HTML}{715582}
\definecolor{PURPLEC}{HTML}{9A72AC}
\definecolor{PURPLEB}{HTML}{B189C6}
\definecolor{PURPLEA}{HTML}{CAA3E8}
\definecolor{PURPLE}{HTML}{FF00FF}
\newcommand{\PURPLEE}{\color{PURPLEE}}
\newcommand{\PURPLED}{\color{PURPLED}}
\newcommand{\PURPLEC}{\color{PURPLEC}}
\newcommand{\PURPLEB}{\color{PURPLEB}}
\newcommand{\PURPLEA}{\color{PURPLEA}}
\newcommand{\PURPLE}{\color{PURPLE}}
% White and Black
\definecolor{WHITE}{HTML}{FFFFFF}
\newcommand{\WHITE}{\color{WHITE}}
\definecolor{BLACK}{HTML}{000000}
\newcommand{\BLACK}{\color{BLACK}}
% Different Grays
\definecolor{LIGHTGRAY}{HTML}{BBBBBB}
\definecolor{GRAY}{HTML}{888888}
\definecolor{DARKGRAY}{HTML}{444444}
\definecolor{DARKERGRAY}{HTML}{222222}
\definecolor{GRAYBROWN}{HTML}{736357}
\newcommand{\LIGHTGRAY}{\color{LIGHTGRAY}}
\newcommand{\GRAY}{\color{GRAY}}
\newcommand{\DARKGRAY}{\color{DARKGRAY}}
\newcommand{\DARKERGRAY}{\color{DARKERGRAY}}
\newcommand{\GRAYBROWN}{\color{GRAYBROWN}}

를 % AMS and mathtools
\usepackage{amsmath,amsthm,amssymb,marvosym,mathrsfs,amsfonts,amscd,mathtools}
% Hyperlinks and URLs
\usepackage{url}
\usepackage{hyperref}
\hypersetup{
	colorlinks,
	citecolor=BLACK,
	filecolor=BLACK,
	linkcolor=BLACK,
	urlcolor=BLACK
}

% Colors
\usepackage[usenames,dvipsnames]{xcolor}
\usepackage{tikz}
\usepackage{tkz-euclide}

% shadowing mdframed
\usepackage[framemethod=tikz]{mdframed}
\usetikzlibrary{shadows}
% Bold math
\usepackage{bm}

%Use Korean Letter when enumerate
\usepackage{dhucs-enumerate}

\usepackage{anyfontsize}

% Bra Ket (Dirac) Notation
\usepackage{braket}

% Slashed characters (e.g. in Dirac equation)
\usepackage{slashed}
\usepackage{pifont} % 원문자 사용시 필요한 패키지

% chapter decoration
\usepackage{type1cm}
\usepackage[explicit]{titlesec}

\titleformat{\chapter}[display]
{\normalfont\Large\rmfamily}
{\sffamily\flushright\fontsize{60}{0}\textbf{\textcolor{blue!40}{{\Huge\chaptername}~\thechapter\vskip0pt\rule{\textwidth}{2pt}}}}{0pt}
{\flushleft\fontsize{30}{0}{#1}\vskip60pt}
\titlespacing*{\chapter}
{0pt}{-40pt}{0pt}

%\usetikzlibrary{shadows}
\usetikzlibrary{shadows.blur}
\usetikzlibrary{shapes.symbols}

% Tcolorbox
\usepackage[most]{tcolorbox}

% Clean SI Units
\usepackage{siunitx}

% Enumerate thingies
\usepackage{enumitem}

% Cancel things out in equations
\usepackage[makeroom]{cancel}

\usepackage{multicol}

% Graphics and figures
\usepackage{graphicx}
\usepackage{wrapfig}
\usepackage{float}

\usepackage{cancel}

% Caption figures and tables
\usepackage{caption,subcaption}

% Generate symbols
\usepackage{textcomp} % Include this line to avoid output errors
\usepackage{gensymb}

% Make multiple rows in a table
\usepackage{multirow}

% Booktabs tables
\usepackage{booktabs}

%\usepackage[utopia,sfscaled]{mathdesign}
% Useful frames
\usepackage{mdframed}

% Comment-out large sections
\usepackage{comment}

% No auto-indent
\setlength{\parindent}{0pt}

% Asymptote - 3D vector graphics
\usepackage{asymptote}

% Tikz Package Stuff
\usepackage{pgf,tikz,pgfplots}
\usepackage{tikz-3dplot}
\usepackage{tabularx}
\usepackage{array}
\usepackage{colortbl}
\tcbuselibrary{skins}
\usepackage{tkz-euclide}

\newcolumntype{Y}{>{\raggedleft\arraybackslash}X}

\tcbset{tab1/.style={fonttitle=\bfseries\large,fontupper=\normalsize\sffamily,
		colback=yellow!10!white,colframe=red!75!black,colbacktitle=Salmon!40!white, halign=center,
		coltitle=black,center title,freelance,frame code={
			\foreach \n in {north east,north west,south east,south west}
			{\path [fill=red!75!black] (interior.\n) circle (3mm); };},}}

\tcbset{tab2/.style={enhanced,fonttitle=\bfseries,fontupper=\normalsize\sffamily, halign=center, box align=center,
		colback=yellow!10!white,colframe=red!50!black,colbacktitle=Salmon!40!white,
		coltitle=black,center title}}


% Use various tikz libraries
\usetikzlibrary{decorations.pathmorphing, decorations.markings, decorations.pathreplacing, patterns} % Decorate paths!
\usetikzlibrary{calc, patterns, shapes.geometric, positioning, through, intersections}
\usetikzlibrary{scopes}
\usetikzlibrary{angles, quotes}
\usetikzlibrary{svg.path}
\usetikzlibrary{arrows, arrows.meta}
\usetikzlibrary{fadings}
% pgfplots package settings
\pgfplotsset{compat=1.15}
% \pgfplotsset{width=10cm,compat=1.9} % Taken from latest overleaf.
% plot arc easily
\def\centerarc[#1](#2)(#3:#4:#5)% Syntax: [draw options] (center) (initial angle:final angle:radius)
{ \draw[#1] ($(#2)+({#5*cos(#3)},{#5*sin(#3)})$) arc (#3:#4:#5); }

% Awesome circled numbers
\newcommand*\circled[4]{\tikz[baseline=(char.base)]{\node[shape=circle, fill=#2, draw=#3, text=#4, inner sep=2pt] (char) {#1};}}

% Control size of text
\usepackage{relsize}

% Extend conditional commands
\usepackage{xifthen}
\usepackage{xcolor}
\definecolor{termcolor}{cmyk}{.21,.97,.0,.0}
\definecolor{darkred}{cmyk}{.27,1,1,.32}
\definecolor{darkblue}{cmyk}{1,.98,.10,.11}
\definecolor{darkgreen}{cmyk}{.29,0,87,0}
\definecolor{darkmycolor}{cmyk}{99,59,22,3}
\definecolor{for_eyes}{RGB}{253,247,228}
%change color of math equation
%\everymath{\color{darkred}}
% Scale math by size
\newcommand*{\Scale}[2][4]{\scalebox{#1}{\ensuremath{#2}}}

% Big integrals
\usepackage{bigints}

% Number equations within sections
\numberwithin{equation}{section}

% Generate blind text
\usepackage{blindtext}

% Useful symbols
\usepackage{marvosym}

\newcounter{problem}[section]
\newcounter{example}[section]

% cancel 색상 변경
\newcommand\Ccancel[2][black]{\renewcommand\CancelColor{\color{#1}}\cancel{#2}}

%%%% 원문자
\newcommand*\ocircled[1]{\tikz[baseline=(char.base)]{
		\node[shape=circle,draw,inner sep=2pt] (char) {#1};}}
	
%%%%% 보기 스타일 %%%%%
\usepackage{tabu}
\newcommand{\questwo}[2]{
	\vskip 6pt
	\noindent\begin{tabu}{X[0.2] X[6] X[0.2] X[6]}
		(1)&$#1$ &(2) &$#2$
	\end{tabu}
}
\newcommand{\questhree}[3]{
	\vskip 3pt
	\noindent\begin{tabu}{X[0.2] X[6] X[0.2] X[6] X[0.2] X[6]}
		(1)&$#1$ &(2) &$#2$ &(3) & $#3$
	\end{tabu}
	\vskip 5pt
}
\newcommand{\quesfour}[4]{
	\vskip 6pt
	\noindent\begin{tabu}{X[0.2] X[6] X[0.2] X[6]}
		(1)&$#1$ &(2) &$#2$\\
		(3)&$#3$ &(4) &$#4$
	\end{tabu}
}
\newcommand{\quesfive}[5]{
	\vskip 6pt
	\noindent\begin{tabu}{X[0.2] X[3] X[0.2] X[3] X[0.2] X[3]}
		(1)&$#1$ &(2) &$#2$ &(3) &$#3$\\
		(4)&$#4$ &(5) &$#5$
	\end{tabu}
}

\newcommand{\oquesfive}[5]{
	\vskip 6pt
	\noindent\begin{tabu}{X[0.2] X[3] X[0.2] X[3] X[0.2] X[3] X[0.2] X[3] X[0.2] X[3]}
		\ding{172}&$#1$ &\ding{173} &$#2$ &\ding{174} &$#3$&	\ding{175}&$#4$ &\ding{176} &$#5$
	\end{tabu}
}
%%%% sample(보기) %%%%
\newenvironment{sample}{\vskip 10pt\noindent\begin{tikzpicture}[yshift=1.5pt]%
		\draw[rounded corners=1ex,overlay,draw=blue] (0pt,-4pt) rectangle (19pt,9pt);
		\node[rectangle,overlay,xshift=9.8pt,yshift=2pt,color=blue] {{\footnotesize\sffamily 보기}};\end{tikzpicture}\phantom{\footnotesize\sffamily보기...}}{\vskip 10pt}
%%%% THEOREMS %%%%
\newenvironment{theorem}[1][\hspace{-0.36em}]
{
	\begin{mdframed}[backgroundcolor=purple!5, align=center, userdefinedwidth=40em, linecolor=purple!30, linewidth=2pt, roundcorner=7pt, innertopmargin=10pt, shadow=true, shadowcolor=black!20, roundcorner=7pt, innerbottommargin=10pt, frametitle = {#1}]
	}
	{
	\end{mdframed}
}

%%%% LEMMAS %%%%
\newenvironment{lemma}[1][\hspace{-0.36em}]
{
	\begin{mdframed}[backgroundcolor=black!8, align=center, userdefinedwidth=40em, topline=false, bottomline = false, leftline = false, rightline = false, frametitle = {#1 보조정리}]
	}
	{
	\end{mdframed}
}

%%%% COROLLARY %%%%
\newenvironment{corollary}[1][\hspace{-0.36em}]
{
	\begin{mdframed}[backgroundcolor=black!8, align=center, userdefinedwidth=40em, topline=false, bottomline = false, leftline = false, rightline = false, frametitle = {#1 따름정리}]
	}
	{
	\end{mdframed}
}

%%%% DEFINITIONS %%%%
\newenvironment{definition}[1][\hspace{-0.36em}]
{
	\begin{mdframed}[backgroundcolor=cyan!14, align=center, userdefinedwidth=40em, linecolor=cyan!60, linewidth=2pt,roundcorner=7pt, innertopmargin=10pt, shadow=true, shadowcolor=black!20, roundcorner=7pt, innerbottommargin=10pt, frametitle = {정의 : #1}]
	}
	{
	\end{mdframed}
}

%%%% PROPOSITION %%%%
\newenvironment{proposition}
{
	\begin{mdframed}[backgroundcolor=black!4, align=center, userdefinedwidth=40em, topline=false, bottomline = false, leftline = false, rightline = false, frametitle = {Proposition}]
	}
	{
	\end{mdframed}
}
%%%% PROBLEM %%%%
\newenvironment{problem}{\refstepcounter{problem}
	\begin{mdframed}[linecolor=blue!35, linewidth=2pt, roundcorner=7pt, innertopmargin=10pt, shadow=true, shadowcolor=black!20, roundcorner=7pt, innerbottommargin=10pt, backgroundcolor=blue!5]
		\noindent 
		\noindent\begin{tikzpicture}[overlay,xshift=6pt,yshift=5pt]
			\draw[fill=violet!15,draw=violet!15] (0,0) circle (8pt);
			\draw[fill=violet!15,draw=violet!15] (9pt,0) circle (8pt);
			\node[rectangle,overlay,xshift=4pt] {\color{black}\sffamily\bfseries 문제};
		\end{tikzpicture}{\phantom{.........}\fontspec[Scale=1.1]{TeX Gyre Adventor}\color{darkblue} \theproblem}\hspace{5pt}}{
\end{mdframed}}
%%%% SOLUTION OF PROBLEM %%%%
\newenvironment{psolution}{\begin{description}\item[{\begin{tikzpicture}%
				\draw[rounded corners=1ex,overlay] (-3pt,-3pt) rectangle (20pt,9pt);\end{tikzpicture}\footnotesize\sffamily풀이}\hspace{6pt}]}{\end{description}}
			
%%%% EXAMPLE %%%%		
\newenvironment{example}{\refstepcounter{example}
	\begin{mdframed}[roundcorner=7pt,linecolor=termcolor,linewidth=2pt,innertopmargin=10pt, shadow=true, shadowcolor=black!20, innerbottommargin=10pt, backgroundcolor=termcolor!2]
		\noindent 
		\noindent\begin{tikzpicture}[overlay,xshift=6pt,yshift=5pt]
			\draw[fill=darkred!60,draw=darkred!60] (0,0) circle (7pt);
			\draw[fill=darkred!60,draw=darkred!60] (9pt,0) circle (7pt);
			\node[rectangle,overlay,xshift=4pt] {\color{white}\sffamily\bfseries 예제};
		\end{tikzpicture}{\phantom{.........}\fontspec[Scale=1.1]{TeX Gyre Adventor}\color{darkred} \theexample}\hspace{5pt}}{
\end{mdframed}}		
%%%%%%% SOLUTION OF EXAMPLE %%%%%%%%%
\newenvironment{solution}{\begin{description}\item[{\begin{tikzpicture}%
				\draw[rounded corners=1ex,overlay] (-3pt,-3pt) rectangle (20pt,9pt);\end{tikzpicture}\footnotesize\sffamily풀이}\hspace{6pt}]}{\end{description}}

% 보기 박스 정의 시작
\tcbuselibrary{breakable, skins}
\tcbset{enhanced}
\newtcolorbox{ChoiceBox}[1]{
		enhanced,
		before skip=2ex, after skip=2ex,
		boxrule=0.5pt, colframe=black, colback=white, arc=0.5ex,
		boxsep=0.5ex, top=1.5ex, bottom=1.5ex, left=0.5em, right=o0.5em,
		colbacktitle=white, coltitle=black,
		attach boxed title to top center={xshift=0cm, yshift=-1.5mm},
		boxed title style={size=minimal, enhanced, boxrule=0.25pt, colframe=white},
		breakable=false, title ={< #1 >}
}
% 보기 박스 정의 끝.
	
% Change end-of-proof symbol
\renewcommand\qedsymbol{$\blacksquare$}
%overline
\newcommand{\ovr}[1]{\overline{\textrm{#1}}}
% trigonometric function
\newcommand{\cosrm}[1]{\cos \textrm{#1}}
\newcommand{\sinrm}[1]{\sin \textrm{#1}}
\newcommand{\tanrm}[1]{\tan \textrm{#1}}
\newcommand{\cotrm}[1]{\cot \textrm{#1}}
\newcommand{\cscrm}[1]{\csc \textrm{#1}}
\newcommand{\secrm}[1]{\sec \textrm{#1}}
%%%% BLACKBOARD BOLD %%%%
\newcommand{\bbN}{\mathbb{N}} % Natural numbers
\newcommand{\bbZ}{\mathbb{Z}} % Zahlen
\newcommand{\bbQ}{\mathbb{Q}} % Rational numbers
\newcommand{\bbR}{\mathbb{R}} % Real numbers
\newcommand{\bbC}{\mathbb{C}} % Complex numbers
\DeclareSymbolFont{bbold}{U}{bbold}{m}{n} % Identity matrix
\DeclareSymbolFontAlphabet{\mathbbold}{bbold} % Identity matrix
\newcommand{\identitymatrix}{\mathbbold{1}} % Identity matrix

%%%% CODE LISTING %%%%
\usepackage{listings}
\definecolor{greencomments}{HTML}{00BA00}
\definecolor{graynumbers}{HTML}{4F4F4F}
\definecolor{purplestrings}{HTML}{AD00AA}
\definecolor{backgroundcolor}{HTML}{E8E8E8}


%%%% UNIT BASIS VECTORS %%%%
\newcommand{\ihat}{\bm{\hat{\imath}}} % Cartesian i hat (x-direction)
\newcommand{\jhat}{\bm{\hat{\jmath}}} % Cartesian j hat (y-direction)
\newcommand{\khat}{\bm{\hat{k}}} % Cartesian k hat (z-direction)
\newcommand{\rhat}{\bm{\hat{r}}} % Spherical r hat
\newcommand{\phihat}{\bm{\hat{\phi}}} % Spherical phi hat
\newcommand{\thetahat}{\bm{\hat{\theta}}} % Spherical theta hat
\newcommand{\nhat}{\bm{\hat{n}}} % Unit normal vector
\newcommand{\rhohat}{\bm{\hat{\rho}}} % Cylindrical rho hat
\newcommand{\zhat}{\bm{\hat{z}}} % Cylindrical z hat


%%%% COLORS: DEFINITIONS AND COMMANDS %%%%
% Miscellaneous
\definecolor{DARKBLUE}{HTML}{040080}
\definecolor{DARKBROWN}{HTML}{8B4513}
\definecolor{LIGHTBROWN}{HTML}{CD853F}
\definecolor{PINK}{HTML}{D147BD}
\definecolor{LIGHTPINK}{HTML}{DC75CD}
\definecolor{GREENSCREEN}{HTML}{00FF00}
\definecolor{ORANGE}{HTML}{FF862F}
\newcommand{\DARKBLUE}{\color{DARKBLUE}}
\newcommand{\DARKBROWN}{\color{DARKBROWN}}
\newcommand{\LIGHTBROWN}{\color{LIGHTBROWN}}
\newcommand{\PINK}{\color{PINK}}
\newcommand{\LIGHTPINK}{\color{LIGHTPINK}}
\newcommand{\GREENSCREEN}{\color{GREENSCREEN}}
\newcommand{\ORANGE}{\color{ORANGE}}
% Blue
\definecolor{BLUEE}{HTML}{1C758A}
\definecolor{BLUED}{HTML}{29ABCA}
\definecolor{BLUEC}{HTML}{58C4DD}
\definecolor{BLUEB}{HTML}{9CDCEB}
\definecolor{BLUEA}{HTML}{C7E9F1}
\definecolor{BLUE}{HTML}{0000FF}
\newcommand{\BLUEE}{\color{BLUEE}}
\newcommand{\BLUED}{\color{BLUED}}
\newcommand{\BLUEC}{\color{BLUEC}}
\newcommand{\BLUEB}{\color{BLUEB}}
\newcommand{\BLUEA}{\color{BLUEA}}
\newcommand{\BLUE}{\color{BLUE}}
% Teal
\definecolor{TEALE}{HTML}{49A88F}
\definecolor{TEALD}{HTML}{55C1A7}
\definecolor{TEALC}{HTML}{5CD0B3}
\definecolor{TEALB}{HTML}{76DDC0}
\definecolor{TEALA}{HTML}{ACEAD7}
\definecolor{TEAL}{HTML}{00FFFF}
\newcommand{\TEALE}{\color{TEALE}}
\newcommand{\TEALD}{\color{TEALD}}
\newcommand{\TEALC}{\color{TEALC}}
\newcommand{\TEALB}{\color{TEALB}}
\newcommand{\TEALA}{\color{TEALA}}
\newcommand{\TEAL}{\color{TEAL}}
% Green
\definecolor{GREENE}{HTML}{699C52}
\definecolor{GREEND}{HTML}{77B05D}
\definecolor{GREENC}{HTML}{83C167}
\definecolor{GREENB}{HTML}{A6CF8C}
\definecolor{GREENA}{HTML}{C9E2AE}
\definecolor{GREEN}{HTML}{00FF00}
\newcommand{\GREENE}{\color{GREENE}}
\newcommand{\GREEND}{\color{GREEND}}
\newcommand{\GREENC}{\color{GREENC}}
\newcommand{\GREENB}{\color{GREENB}}
\newcommand{\GREENA}{\color{GREENA}}
\newcommand{\GREEN}{\color{GREEN}}
% Yellow
\definecolor{YELLOWE}{HTML}{E8C11C}
\definecolor{YELLOWD}{HTML}{F4D345}
\definecolor{YELLOWC}{HTML}{FFFF00}
\definecolor{YELLOWB}{HTML}{FFEA94}
\definecolor{YELLOWA}{HTML}{FFF1B6}
\definecolor{YELLOW}{HTML}{FFFF00}
\newcommand{\YELLOWE}{\color{YELLOWE}}
\newcommand{\YELLOWD}{\color{YELLOWD}}
\newcommand{\YELLOWC}{\color{YELLOWC}}
\newcommand{\YELLOWB}{\color{YELLOWB}}
\newcommand{\YELLOWA}{\color{YELLOWA}}
\newcommand{\YELLOW}{\color{YELLOW}}
% Gold
\definecolor{GOLDE}{HTML}{C78D46}
\definecolor{GOLDD}{HTML}{E1A158}
\definecolor{GOLDC}{HTML}{F0AC5F}
\definecolor{GOLDB}{HTML}{F9B775}
\definecolor{GOLDA}{HTML}{F7C797}
\newcommand{\GOLDE}{\color{GOLDE}}
\newcommand{\GOLDD}{\color{GOLDD}}
\newcommand{\GOLDC}{\color{GOLDC}}
\newcommand{\GOLDB}{\color{GOLDB}}
\newcommand{\GOLDA}{\color{GOLDA}}
% Red
\definecolor{REDE}{HTML}{CF5044}
\definecolor{REDD}{HTML}{E65A4C}
\definecolor{REDC}{HTML}{FC6255}
\definecolor{REDB}{HTML}{FF8080}
\definecolor{REDA}{HTML}{F7A1A3}
\definecolor{RED}{HTML}{FF0000}
\newcommand{\REDE}{\color{REDE}}
\newcommand{\REDD}{\color{REDD}}
\newcommand{\REDC}{\color{REDC}}
\newcommand{\REDB}{\color{REDB}}
\newcommand{\REDA}{\color{REDA}}
\newcommand{\RED}{\color{RED}}
% Maroon
\definecolor{MAROONE}{HTML}{94424F}
\definecolor{MAROOND}{HTML}{A24D61}
\definecolor{MAROONC}{HTML}{C55F73}
\definecolor{MAROONB}{HTML}{EC92AB}
\definecolor{MAROONA}{HTML}{ECABC1}
\newcommand{\MAROONE}{\color{MAROONE}}
\newcommand{\MAROOND}{\color{MAROOND}}
\newcommand{\MAROONC}{\color{MAROONC}}
\newcommand{\MAROONB}{\color{MAROONB}}
\newcommand{\MAROONA}{\color{MAROONA}}
% Purple
\definecolor{PURPLEE}{HTML}{644172}
\definecolor{PURPLED}{HTML}{715582}
\definecolor{PURPLEC}{HTML}{9A72AC}
\definecolor{PURPLEB}{HTML}{B189C6}
\definecolor{PURPLEA}{HTML}{CAA3E8}
\definecolor{PURPLE}{HTML}{FF00FF}
\newcommand{\PURPLEE}{\color{PURPLEE}}
\newcommand{\PURPLED}{\color{PURPLED}}
\newcommand{\PURPLEC}{\color{PURPLEC}}
\newcommand{\PURPLEB}{\color{PURPLEB}}
\newcommand{\PURPLEA}{\color{PURPLEA}}
\newcommand{\PURPLE}{\color{PURPLE}}
% White and Black
\definecolor{WHITE}{HTML}{FFFFFF}
\newcommand{\WHITE}{\color{WHITE}}
\definecolor{BLACK}{HTML}{000000}
\newcommand{\BLACK}{\color{BLACK}}
% Different Grays
\definecolor{LIGHTGRAY}{HTML}{BBBBBB}
\definecolor{GRAY}{HTML}{888888}
\definecolor{DARKGRAY}{HTML}{444444}
\definecolor{DARKERGRAY}{HTML}{222222}
\definecolor{GRAYBROWN}{HTML}{736357}
\newcommand{\LIGHTGRAY}{\color{LIGHTGRAY}}
\newcommand{\GRAY}{\color{GRAY}}
\newcommand{\DARKGRAY}{\color{DARKGRAY}}
\newcommand{\DARKERGRAY}{\color{DARKERGRAY}}
\newcommand{\GRAYBROWN}{\color{GRAYBROWN}}

로 수정하시기 바랍니다.

\documentclass[11pt, a4paper]{book}
%\documentclass[11pt,a4paper]{article} 
\usepackage[T1]{fontenc}
\usepackage{kotex}
\usepackage{secdot}
% Set page geometry
\usepackage[margin=2.5cm,headsep=0.5cm]{geometry}
\usepackage{setspace}
\usepackage{versions}
%\RequirePackage{latexalpha2wlua}
%\includeversion{aftercal}
\excludeversion{aftercal}
\includeversion{nogeo}
%\excludeversion{nogeo}
\includeversion{psol}
%\excludeversion{psol}
% AMS and mathtools
\usepackage{amsmath,amsthm,amssymb,marvosym,mathrsfs,amsfonts,amscd,mathtools}
% Hyperlinks and URLs
\usepackage{url}
\usepackage{hyperref}
\hypersetup{
	colorlinks,
	citecolor=BLACK,
	filecolor=BLACK,
	linkcolor=BLACK,
	urlcolor=BLACK
}

% Colors
\usepackage[usenames,dvipsnames]{xcolor}
\usepackage{tikz}
\usepackage{tkz-euclide}

% shadowing mdframed
\usepackage[framemethod=tikz]{mdframed}
\usetikzlibrary{shadows}
% Bold math
\usepackage{bm}

%Use Korean Letter when enumerate
\usepackage{dhucs-enumerate}

\usepackage{anyfontsize}

% Bra Ket (Dirac) Notation
\usepackage{braket}

% Slashed characters (e.g. in Dirac equation)
\usepackage{slashed}
\usepackage{pifont} % 원문자 사용시 필요한 패키지

% chapter decoration
\usepackage{type1cm}
\usepackage[explicit]{titlesec}

\titleformat{\chapter}[display]
{\normalfont\Large\rmfamily}
{\sffamily\flushright\fontsize{60}{0}\textbf{\textcolor{blue!40}{{\Huge\chaptername}~\thechapter\vskip0pt\rule{\textwidth}{2pt}}}}{0pt}
{\flushleft\fontsize{30}{0}{#1}\vskip60pt}
\titlespacing*{\chapter}
{0pt}{-40pt}{0pt}

%\usetikzlibrary{shadows}
\usetikzlibrary{shadows.blur}
\usetikzlibrary{shapes.symbols}

% Tcolorbox
\usepackage[most]{tcolorbox}

% Clean SI Units
\usepackage{siunitx}

% Enumerate thingies
\usepackage{enumitem}

% Cancel things out in equations
\usepackage[makeroom]{cancel}

\usepackage{multicol}

% Graphics and figures
\usepackage{graphicx}
\usepackage{wrapfig}
\usepackage{float}

\usepackage{cancel}

% Caption figures and tables
\usepackage{caption,subcaption}

% Generate symbols
\usepackage{textcomp} % Include this line to avoid output errors
\usepackage{gensymb}

% Make multiple rows in a table
\usepackage{multirow}

% Booktabs tables
\usepackage{booktabs}

%\usepackage[utopia,sfscaled]{mathdesign}
% Useful frames
\usepackage{mdframed}

% Comment-out large sections
\usepackage{comment}

% No auto-indent
\setlength{\parindent}{0pt}

% Asymptote - 3D vector graphics
\usepackage{asymptote}

% Tikz Package Stuff
\usepackage{pgf,tikz,pgfplots}
\usepackage{tikz-3dplot}
\usepackage{tabularx}
\usepackage{array}
\usepackage{colortbl}
\tcbuselibrary{skins}
\usepackage{tkz-euclide}

\newcolumntype{Y}{>{\raggedleft\arraybackslash}X}

\tcbset{tab1/.style={fonttitle=\bfseries\large,fontupper=\normalsize\sffamily,
		colback=yellow!10!white,colframe=red!75!black,colbacktitle=Salmon!40!white, halign=center,
		coltitle=black,center title,freelance,frame code={
			\foreach \n in {north east,north west,south east,south west}
			{\path [fill=red!75!black] (interior.\n) circle (3mm); };},}}

\tcbset{tab2/.style={enhanced,fonttitle=\bfseries,fontupper=\normalsize\sffamily, halign=center, box align=center,
		colback=yellow!10!white,colframe=red!50!black,colbacktitle=Salmon!40!white,
		coltitle=black,center title}}


% Use various tikz libraries
\usetikzlibrary{decorations.pathmorphing, decorations.markings, decorations.pathreplacing, patterns} % Decorate paths!
\usetikzlibrary{calc, patterns, shapes.geometric, positioning, through, intersections}
\usetikzlibrary{scopes}
\usetikzlibrary{angles, quotes}
\usetikzlibrary{svg.path}
\usetikzlibrary{arrows, arrows.meta}
\usetikzlibrary{fadings}
% pgfplots package settings
\pgfplotsset{compat=1.15}
% \pgfplotsset{width=10cm,compat=1.9} % Taken from latest overleaf.
% plot arc easily
\def\centerarc[#1](#2)(#3:#4:#5)% Syntax: [draw options] (center) (initial angle:final angle:radius)
{ \draw[#1] ($(#2)+({#5*cos(#3)},{#5*sin(#3)})$) arc (#3:#4:#5); }

% Awesome circled numbers
\newcommand*\circled[4]{\tikz[baseline=(char.base)]{\node[shape=circle, fill=#2, draw=#3, text=#4, inner sep=2pt] (char) {#1};}}

% Control size of text
\usepackage{relsize}

% Extend conditional commands
\usepackage{xifthen}
\usepackage{xcolor}
\definecolor{termcolor}{cmyk}{.21,.97,.0,.0}
\definecolor{darkred}{cmyk}{.27,1,1,.32}
\definecolor{darkblue}{cmyk}{1,.98,.10,.11}
\definecolor{darkgreen}{cmyk}{.29,0,87,0}
\definecolor{darkmycolor}{cmyk}{99,59,22,3}
\definecolor{for_eyes}{RGB}{253,247,228}
%change color of math equation
%\everymath{\color{darkred}}
% Scale math by size
\newcommand*{\Scale}[2][4]{\scalebox{#1}{\ensuremath{#2}}}

% Big integrals
\usepackage{bigints}

% Number equations within sections
\numberwithin{equation}{section}

% Generate blind text
\usepackage{blindtext}

% Useful symbols
\usepackage{marvosym}

\newcounter{problem}[section]
\newcounter{example}[section]

% cancel 색상 변경
\newcommand\Ccancel[2][black]{\renewcommand\CancelColor{\color{#1}}\cancel{#2}}

%%%% 원문자
\newcommand*\ocircled[1]{\tikz[baseline=(char.base)]{
		\node[shape=circle,draw,inner sep=2pt] (char) {#1};}}
	
%%%%% 보기 스타일 %%%%%
\usepackage{tabu}
\newcommand{\questwo}[2]{
	\vskip 6pt
	\noindent\begin{tabu}{X[0.2] X[6] X[0.2] X[6]}
		(1)&$#1$ &(2) &$#2$
	\end{tabu}
}
\newcommand{\questhree}[3]{
	\vskip 3pt
	\noindent\begin{tabu}{X[0.2] X[6] X[0.2] X[6] X[0.2] X[6]}
		(1)&$#1$ &(2) &$#2$ &(3) & $#3$
	\end{tabu}
	\vskip 5pt
}
\newcommand{\quesfour}[4]{
	\vskip 6pt
	\noindent\begin{tabu}{X[0.2] X[6] X[0.2] X[6]}
		(1)&$#1$ &(2) &$#2$\\
		(3)&$#3$ &(4) &$#4$
	\end{tabu}
}
\newcommand{\quesfive}[5]{
	\vskip 6pt
	\noindent\begin{tabu}{X[0.2] X[3] X[0.2] X[3] X[0.2] X[3]}
		(1)&$#1$ &(2) &$#2$ &(3) &$#3$\\
		(4)&$#4$ &(5) &$#5$
	\end{tabu}
}

\newcommand{\oquesfive}[5]{
	\vskip 6pt
	\noindent\begin{tabu}{X[0.2] X[3] X[0.2] X[3] X[0.2] X[3] X[0.2] X[3] X[0.2] X[3]}
		\ding{172}&$#1$ &\ding{173} &$#2$ &\ding{174} &$#3$&	\ding{175}&$#4$ &\ding{176} &$#5$
	\end{tabu}
}
%%%% sample(보기) %%%%
\newenvironment{sample}{\vskip 10pt\noindent\begin{tikzpicture}[yshift=1.5pt]%
		\draw[rounded corners=1ex,overlay,draw=blue] (0pt,-4pt) rectangle (19pt,9pt);
		\node[rectangle,overlay,xshift=9.8pt,yshift=2pt,color=blue] {{\footnotesize\sffamily 보기}};\end{tikzpicture}\phantom{\footnotesize\sffamily보기...}}{\vskip 10pt}
%%%% THEOREMS %%%%
\newenvironment{theorem}[1][\hspace{-0.36em}]
{
	\begin{mdframed}[backgroundcolor=purple!5, align=center, userdefinedwidth=40em, linecolor=purple!30, linewidth=2pt, roundcorner=7pt, innertopmargin=10pt, shadow=true, shadowcolor=black!20, roundcorner=7pt, innerbottommargin=10pt, frametitle = {#1}]
	}
	{
	\end{mdframed}
}

%%%% LEMMAS %%%%
\newenvironment{lemma}[1][\hspace{-0.36em}]
{
	\begin{mdframed}[backgroundcolor=black!8, align=center, userdefinedwidth=40em, topline=false, bottomline = false, leftline = false, rightline = false, frametitle = {#1 보조정리}]
	}
	{
	\end{mdframed}
}

%%%% COROLLARY %%%%
\newenvironment{corollary}[1][\hspace{-0.36em}]
{
	\begin{mdframed}[backgroundcolor=black!8, align=center, userdefinedwidth=40em, topline=false, bottomline = false, leftline = false, rightline = false, frametitle = {#1 따름정리}]
	}
	{
	\end{mdframed}
}

%%%% DEFINITIONS %%%%
\newenvironment{definition}[1][\hspace{-0.36em}]
{
	\begin{mdframed}[backgroundcolor=cyan!14, align=center, userdefinedwidth=40em, linecolor=cyan!60, linewidth=2pt,roundcorner=7pt, innertopmargin=10pt, shadow=true, shadowcolor=black!20, roundcorner=7pt, innerbottommargin=10pt, frametitle = {정의 : #1}]
	}
	{
	\end{mdframed}
}

%%%% PROPOSITION %%%%
\newenvironment{proposition}
{
	\begin{mdframed}[backgroundcolor=black!4, align=center, userdefinedwidth=40em, topline=false, bottomline = false, leftline = false, rightline = false, frametitle = {Proposition}]
	}
	{
	\end{mdframed}
}
%%%% PROBLEM %%%%
\newenvironment{problem}{\refstepcounter{problem}
	\begin{mdframed}[linecolor=blue!35, linewidth=2pt, roundcorner=7pt, innertopmargin=10pt, shadow=true, shadowcolor=black!20, roundcorner=7pt, innerbottommargin=10pt, backgroundcolor=blue!5]
		\noindent 
		\noindent\begin{tikzpicture}[overlay,xshift=6pt,yshift=5pt]
			\draw[fill=violet!15,draw=violet!15] (0,0) circle (8pt);
			\draw[fill=violet!15,draw=violet!15] (9pt,0) circle (8pt);
			\node[rectangle,overlay,xshift=4pt] {\color{black}\sffamily\bfseries 문제};
		\end{tikzpicture}{\phantom{.........}\fontspec[Scale=1.1]{TeX Gyre Adventor}\color{darkblue} \theproblem}\hspace{5pt}}{
\end{mdframed}}
%%%% SOLUTION OF PROBLEM %%%%
\newenvironment{psolution}{\begin{description}\item[{\begin{tikzpicture}%
				\draw[rounded corners=1ex,overlay] (-3pt,-3pt) rectangle (20pt,9pt);\end{tikzpicture}\footnotesize\sffamily풀이}\hspace{6pt}]}{\end{description}}
			
%%%% EXAMPLE %%%%		
\newenvironment{example}{\refstepcounter{example}
	\begin{mdframed}[roundcorner=7pt,linecolor=termcolor,linewidth=2pt,innertopmargin=10pt, shadow=true, shadowcolor=black!20, innerbottommargin=10pt, backgroundcolor=termcolor!2]
		\noindent 
		\noindent\begin{tikzpicture}[overlay,xshift=6pt,yshift=5pt]
			\draw[fill=darkred!60,draw=darkred!60] (0,0) circle (7pt);
			\draw[fill=darkred!60,draw=darkred!60] (9pt,0) circle (7pt);
			\node[rectangle,overlay,xshift=4pt] {\color{white}\sffamily\bfseries 예제};
		\end{tikzpicture}{\phantom{.........}\fontspec[Scale=1.1]{TeX Gyre Adventor}\color{darkred} \theexample}\hspace{5pt}}{
\end{mdframed}}		
%%%%%%% SOLUTION OF EXAMPLE %%%%%%%%%
\newenvironment{solution}{\begin{description}\item[{\begin{tikzpicture}%
				\draw[rounded corners=1ex,overlay] (-3pt,-3pt) rectangle (20pt,9pt);\end{tikzpicture}\footnotesize\sffamily풀이}\hspace{6pt}]}{\end{description}}

% 보기 박스 정의 시작
\tcbuselibrary{breakable, skins}
\tcbset{enhanced}
\newtcolorbox{ChoiceBox}[1]{
		enhanced,
		before skip=2ex, after skip=2ex,
		boxrule=0.5pt, colframe=black, colback=white, arc=0.5ex,
		boxsep=0.5ex, top=1.5ex, bottom=1.5ex, left=0.5em, right=o0.5em,
		colbacktitle=white, coltitle=black,
		attach boxed title to top center={xshift=0cm, yshift=-1.5mm},
		boxed title style={size=minimal, enhanced, boxrule=0.25pt, colframe=white},
		breakable=false, title ={< #1 >}
}
% 보기 박스 정의 끝.
	
% Change end-of-proof symbol
\renewcommand\qedsymbol{$\blacksquare$}
%overline
\newcommand{\ovr}[1]{\overline{\textrm{#1}}}
% trigonometric function
\newcommand{\cosrm}[1]{\cos \textrm{#1}}
\newcommand{\sinrm}[1]{\sin \textrm{#1}}
\newcommand{\tanrm}[1]{\tan \textrm{#1}}
\newcommand{\cotrm}[1]{\cot \textrm{#1}}
\newcommand{\cscrm}[1]{\csc \textrm{#1}}
\newcommand{\secrm}[1]{\sec \textrm{#1}}
%%%% BLACKBOARD BOLD %%%%
\newcommand{\bbN}{\mathbb{N}} % Natural numbers
\newcommand{\bbZ}{\mathbb{Z}} % Zahlen
\newcommand{\bbQ}{\mathbb{Q}} % Rational numbers
\newcommand{\bbR}{\mathbb{R}} % Real numbers
\newcommand{\bbC}{\mathbb{C}} % Complex numbers
\DeclareSymbolFont{bbold}{U}{bbold}{m}{n} % Identity matrix
\DeclareSymbolFontAlphabet{\mathbbold}{bbold} % Identity matrix
\newcommand{\identitymatrix}{\mathbbold{1}} % Identity matrix

%%%% CODE LISTING %%%%
\usepackage{listings}
\definecolor{greencomments}{HTML}{00BA00}
\definecolor{graynumbers}{HTML}{4F4F4F}
\definecolor{purplestrings}{HTML}{AD00AA}
\definecolor{backgroundcolor}{HTML}{E8E8E8}


%%%% UNIT BASIS VECTORS %%%%
\newcommand{\ihat}{\bm{\hat{\imath}}} % Cartesian i hat (x-direction)
\newcommand{\jhat}{\bm{\hat{\jmath}}} % Cartesian j hat (y-direction)
\newcommand{\khat}{\bm{\hat{k}}} % Cartesian k hat (z-direction)
\newcommand{\rhat}{\bm{\hat{r}}} % Spherical r hat
\newcommand{\phihat}{\bm{\hat{\phi}}} % Spherical phi hat
\newcommand{\thetahat}{\bm{\hat{\theta}}} % Spherical theta hat
\newcommand{\nhat}{\bm{\hat{n}}} % Unit normal vector
\newcommand{\rhohat}{\bm{\hat{\rho}}} % Cylindrical rho hat
\newcommand{\zhat}{\bm{\hat{z}}} % Cylindrical z hat


%%%% COLORS: DEFINITIONS AND COMMANDS %%%%
% Miscellaneous
\definecolor{DARKBLUE}{HTML}{040080}
\definecolor{DARKBROWN}{HTML}{8B4513}
\definecolor{LIGHTBROWN}{HTML}{CD853F}
\definecolor{PINK}{HTML}{D147BD}
\definecolor{LIGHTPINK}{HTML}{DC75CD}
\definecolor{GREENSCREEN}{HTML}{00FF00}
\definecolor{ORANGE}{HTML}{FF862F}
\newcommand{\DARKBLUE}{\color{DARKBLUE}}
\newcommand{\DARKBROWN}{\color{DARKBROWN}}
\newcommand{\LIGHTBROWN}{\color{LIGHTBROWN}}
\newcommand{\PINK}{\color{PINK}}
\newcommand{\LIGHTPINK}{\color{LIGHTPINK}}
\newcommand{\GREENSCREEN}{\color{GREENSCREEN}}
\newcommand{\ORANGE}{\color{ORANGE}}
% Blue
\definecolor{BLUEE}{HTML}{1C758A}
\definecolor{BLUED}{HTML}{29ABCA}
\definecolor{BLUEC}{HTML}{58C4DD}
\definecolor{BLUEB}{HTML}{9CDCEB}
\definecolor{BLUEA}{HTML}{C7E9F1}
\definecolor{BLUE}{HTML}{0000FF}
\newcommand{\BLUEE}{\color{BLUEE}}
\newcommand{\BLUED}{\color{BLUED}}
\newcommand{\BLUEC}{\color{BLUEC}}
\newcommand{\BLUEB}{\color{BLUEB}}
\newcommand{\BLUEA}{\color{BLUEA}}
\newcommand{\BLUE}{\color{BLUE}}
% Teal
\definecolor{TEALE}{HTML}{49A88F}
\definecolor{TEALD}{HTML}{55C1A7}
\definecolor{TEALC}{HTML}{5CD0B3}
\definecolor{TEALB}{HTML}{76DDC0}
\definecolor{TEALA}{HTML}{ACEAD7}
\definecolor{TEAL}{HTML}{00FFFF}
\newcommand{\TEALE}{\color{TEALE}}
\newcommand{\TEALD}{\color{TEALD}}
\newcommand{\TEALC}{\color{TEALC}}
\newcommand{\TEALB}{\color{TEALB}}
\newcommand{\TEALA}{\color{TEALA}}
\newcommand{\TEAL}{\color{TEAL}}
% Green
\definecolor{GREENE}{HTML}{699C52}
\definecolor{GREEND}{HTML}{77B05D}
\definecolor{GREENC}{HTML}{83C167}
\definecolor{GREENB}{HTML}{A6CF8C}
\definecolor{GREENA}{HTML}{C9E2AE}
\definecolor{GREEN}{HTML}{00FF00}
\newcommand{\GREENE}{\color{GREENE}}
\newcommand{\GREEND}{\color{GREEND}}
\newcommand{\GREENC}{\color{GREENC}}
\newcommand{\GREENB}{\color{GREENB}}
\newcommand{\GREENA}{\color{GREENA}}
\newcommand{\GREEN}{\color{GREEN}}
% Yellow
\definecolor{YELLOWE}{HTML}{E8C11C}
\definecolor{YELLOWD}{HTML}{F4D345}
\definecolor{YELLOWC}{HTML}{FFFF00}
\definecolor{YELLOWB}{HTML}{FFEA94}
\definecolor{YELLOWA}{HTML}{FFF1B6}
\definecolor{YELLOW}{HTML}{FFFF00}
\newcommand{\YELLOWE}{\color{YELLOWE}}
\newcommand{\YELLOWD}{\color{YELLOWD}}
\newcommand{\YELLOWC}{\color{YELLOWC}}
\newcommand{\YELLOWB}{\color{YELLOWB}}
\newcommand{\YELLOWA}{\color{YELLOWA}}
\newcommand{\YELLOW}{\color{YELLOW}}
% Gold
\definecolor{GOLDE}{HTML}{C78D46}
\definecolor{GOLDD}{HTML}{E1A158}
\definecolor{GOLDC}{HTML}{F0AC5F}
\definecolor{GOLDB}{HTML}{F9B775}
\definecolor{GOLDA}{HTML}{F7C797}
\newcommand{\GOLDE}{\color{GOLDE}}
\newcommand{\GOLDD}{\color{GOLDD}}
\newcommand{\GOLDC}{\color{GOLDC}}
\newcommand{\GOLDB}{\color{GOLDB}}
\newcommand{\GOLDA}{\color{GOLDA}}
% Red
\definecolor{REDE}{HTML}{CF5044}
\definecolor{REDD}{HTML}{E65A4C}
\definecolor{REDC}{HTML}{FC6255}
\definecolor{REDB}{HTML}{FF8080}
\definecolor{REDA}{HTML}{F7A1A3}
\definecolor{RED}{HTML}{FF0000}
\newcommand{\REDE}{\color{REDE}}
\newcommand{\REDD}{\color{REDD}}
\newcommand{\REDC}{\color{REDC}}
\newcommand{\REDB}{\color{REDB}}
\newcommand{\REDA}{\color{REDA}}
\newcommand{\RED}{\color{RED}}
% Maroon
\definecolor{MAROONE}{HTML}{94424F}
\definecolor{MAROOND}{HTML}{A24D61}
\definecolor{MAROONC}{HTML}{C55F73}
\definecolor{MAROONB}{HTML}{EC92AB}
\definecolor{MAROONA}{HTML}{ECABC1}
\newcommand{\MAROONE}{\color{MAROONE}}
\newcommand{\MAROOND}{\color{MAROOND}}
\newcommand{\MAROONC}{\color{MAROONC}}
\newcommand{\MAROONB}{\color{MAROONB}}
\newcommand{\MAROONA}{\color{MAROONA}}
% Purple
\definecolor{PURPLEE}{HTML}{644172}
\definecolor{PURPLED}{HTML}{715582}
\definecolor{PURPLEC}{HTML}{9A72AC}
\definecolor{PURPLEB}{HTML}{B189C6}
\definecolor{PURPLEA}{HTML}{CAA3E8}
\definecolor{PURPLE}{HTML}{FF00FF}
\newcommand{\PURPLEE}{\color{PURPLEE}}
\newcommand{\PURPLED}{\color{PURPLED}}
\newcommand{\PURPLEC}{\color{PURPLEC}}
\newcommand{\PURPLEB}{\color{PURPLEB}}
\newcommand{\PURPLEA}{\color{PURPLEA}}
\newcommand{\PURPLE}{\color{PURPLE}}
% White and Black
\definecolor{WHITE}{HTML}{FFFFFF}
\newcommand{\WHITE}{\color{WHITE}}
\definecolor{BLACK}{HTML}{000000}
\newcommand{\BLACK}{\color{BLACK}}
% Different Grays
\definecolor{LIGHTGRAY}{HTML}{BBBBBB}
\definecolor{GRAY}{HTML}{888888}
\definecolor{DARKGRAY}{HTML}{444444}
\definecolor{DARKERGRAY}{HTML}{222222}
\definecolor{GRAYBROWN}{HTML}{736357}
\newcommand{\LIGHTGRAY}{\color{LIGHTGRAY}}
\newcommand{\GRAY}{\color{GRAY}}
\newcommand{\DARKGRAY}{\color{DARKGRAY}}
\newcommand{\DARKERGRAY}{\color{DARKERGRAY}}
\newcommand{\GRAYBROWN}{\color{GRAYBROWN}}

 % 작업하는 파일의 상위 디렉토리에 둬야 하는 파일.
\usetikzlibrary{intersections,decorations.text}
\definecolor{c1}{RGB}{62, 97, 127}
\definecolor{c2}{RGB}{104, 182, 182}
\definecolor{c3}{RGB}{107, 190, 190}
\definecolor{c4}{RGB}{100, 172, 174}
\title{수열}
\author{유익승}
\date{\today}
\newcommand{\plogo}{\fbox{\color{c4}$\mathcal{YHHS}$}}
\usepackage{fancyhdr,lastpage}
\pagestyle{fancy}
\fancyhf{}
\lhead{\color{c4}Trigonometry}
\rhead{\plogo}
\lfoot{\color{c1}\texttt{양현고등학교}}
\rfoot{\color{c1}\pageref{LastPage}페이지 중 \thepage 페이지}

\usetikzlibrary{positioning,calc,arrows.meta,bending}
\newcommand{\tikzmark}[3][]{\tikz[overlay,remember picture,baseline] \node 
	[anchor=base,#1](#2) {#3};}

\begin{document}
	\thispagestyle{empty}
	 \pagenumbering{gobble} 
\begin{tikzpicture}[remember picture,overlay]
	%%%%%%%%%%%%%%%%%%%% Background %%%%%%%%%%%%%%%%%%%%%%%%
	\fill[Dandelion] (current page.south west) rectangle (current page.north east);
	
	
	
	
	%%%%%%%%%%%%%%%%%%%% Background Polygon %%%%%%%%%%%%%%%%%%%%
	
	\foreach \i in {2.5,...,22}
	{
		\node[rounded corners,Dandelion!60,draw,regular polygon,regular polygon sides=6, minimum size=\i cm,ultra thick] at ($(current page.west)+(2.5,-5)$) {} ;
	}
	
	\foreach \i in {0.5,...,22}
	{
		\node[rounded corners,Dandelion!60,draw,regular polygon,regular polygon sides=6, minimum size=\i cm,ultra thick] at ($(current page.north west)+(2.5,0)$) {} ;
	}
	
	\foreach \i in {0.5,...,22}
	{
		\node[rounded corners,Dandelion!90,draw,regular polygon,regular polygon sides=6, minimum size=\i cm,ultra thick] at ($(current page.north east)+(0,-9.5)$) {} ;
	}
	
	
	\foreach \i in {21,...,6}
	{
		\node[Dandelion!85,rounded corners,draw,regular polygon,regular polygon sides=6, minimum size=\i cm,ultra thick] at ($(current page.south east)+(-0.2,-0.45)$) {} ;
	}
	
	
	%%%%%%%%%%%%%%%%%%%% Title of the Report %%%%%%%%%%%%%%%%%%%% 
	\node[left,black,minimum width=0.625*\paperwidth,minimum height=3cm, rounded corners] at ($(current page.north east)+(0,-9.5)$)
	{
		{\fontsize{25}{30} \selectfont \bfseries 수열(Sequence)}
	};
	
	%%%%%%%%%%%%%%%%%%%% Subtitle %%%%%%%%%%%%%%%%%%%% 
	\node[left,black,minimum width=0.625*\paperwidth,minimum height=2cm, rounded corners] at ($(current page.north east)+(0,-11)$)
	{
		{\huge \textit{}}
	};
	
	%%%%%%%%%%%%%%%%%%%% Author Name %%%%%%%%%%%%%%%%%%%% 
	\node[left,black,minimum width=0.625*\paperwidth,minimum height=2cm, rounded corners] at ($(current page.north east)+(0,-13)$)
	{
		{\Large \textsc{유  익  승}}
	};
	
	%%%%%%%%%%%%%%%%%%%% Year %%%%%%%%%%%%%%%%%%%% 
	\node[rounded corners,fill=Dandelion!70,text =black,regular polygon,regular polygon sides=6, minimum size=2.5 cm,inner sep=0,ultra thick] at ($(current page.west)+(2.5,-5)$) {\LARGE \bfseries 2022};
\end{tikzpicture}
\pagestyle{empty}
\setcounter{page}{1}
\setstretch{1.4}
\tikzset{every shadow/.style={opacity=1}}	
%	\maketitle
	\chapter{\Huge 지수}
	 \pagenumbering{arabic} 
	\pagestyle{fancy}
임의의 실수 $a$에 대하여 $a$를 $n$번 반복하여 곱한 수 즉
\begin{equation*}
	 \underbrace{a \times a \times \cdots \times a}_{\text{$n$개의 곱}}
\end{equation*}
을 $a^n$으로 나타내고 

\begin{figure}[H]
	\begin{center}
\begin{equation*}
	\tikzmark[blue]{base}{$a^{\,\,\tikzmark[red]{exponent}{n}}$}
\end{equation*}
\begin{tikzpicture}[overlay, remember picture,node distance =.2cm]
	\node[blue] (basedescr) [below right=of base]{밑};
	\draw[,-stealth,thick] (basedescr.west) to [in=315,out=180] (base.south);
	\node[red] (exponentdescr) [above right=of exponent] {지수};
	\draw[-stealth,thick] (exponentdescr.west) to [in=65,out=180] (exponent.north);
\end{tikzpicture}
\end{center}
\end{figure}
$a$를 {\color{blue}밑(base)}, $n$을 {\color{red}지수(exponent)}라고 한다.
지수의 정의로부터 임의의 실수 $a,\;b$와 자연수 $m,\;n$에 대하여 다음이 성립함을 쉽게 알 수 있다.
\begin{align*}
&(1)\; a^m a^n = a^{m+n} \\
&(2)\; \left(a^{m} \right)^{n} = a^{mn} \\
&(3)\; (ab)^{n} = a^n b^n \\
&(4)\; \left(\frac{a}{b}\right)^n = \frac{a^n}{b^n}
&(5)\; a^m \div a^n = \begin{cases}
	a^{m-n}, & (m>n) \\
	1, & (m=n)\\
	\frac{1}{a^{m-n}}, & (m<n)
\end{cases}\qquad (\text{단, }a \neq 0)
\end{align*}
이 성립함을 금방 알 수 있다. 정의로부터 지수 $n$은 자연수이어야 함을 알 수 있다. 그런데 실제로 지수는 자연수를 넘어 정수, 유리수, 실수 그리고 심지어는 복소수 범위까지 확장이 가능하다. 우리는 유리수 범위까지는 매우 정교하게 확장하고 실수 범위로의 확장은 정당화 수준에서 확장하고자 한다. 지수를 정수 범위로 확장하기 전에 우리는 다음 사실에 주목할 필요가 있다.
\begin{equation*}
	\color{blue}a^n \div \color{red} a^m =\frac{\color{blue}a^{n}}{\color{red}a^{m}} = \frac{\color{blue}\overbrace{a \times a \times \cdots \times a}^{\text{$n$개의 곱}}}{\color{red}\underbrace{a \times a \times \cdots \times a}_{\text{$m$개의 곱}}} \color{black} = a^{\color{blue}n\color{black}-\color{red}m} \color{black}
\end{equation*}
$a^{0}=1$이므로 이 식에서 $n$에 $0$을 대입하면 $a^{-m}=1 \div a^m = \frac{1}{a^m}$이다. 이를 이용하면 음의 정수 지수를 어떻게 처리하면 되는 지를 알 수 있다. 이를 이용하여 다음 절에서 우리는 지수를 자연수 범위에서 정수 범위로 확장할 것이다. 우리는 궁극적으로 지수를 실수범위로까지 확장할 것이다. 그런데 지수를 정수 범위에서 유리수 범위로의 확장은 거듭제곱근을 이용하여 확장할 수 있다.
  
 	\section{거듭제곱근}
제곱하여 실수 $a$가 되는 수, 즉 $x^2 =a$를 $a$의 제곱근이라 하고, 세제곱하여 $a$가 되는 수, 즉  $x^2 =a$인 $x$를 $a$의 세제곱근이라고 한다. 일반적으로 $n$이 $2$ 이상인 자연수일 때, $n$제곱하여 실수 $a$가 되는 수
$$x^n =a$$
를 만족시키는 수$x$를 $a$의 {\color{red}$n$제곱근} 이라고 한다. 일반적으로, 세제곱근, 네제곱근, $\cdots$을 통틀어 $a$의 {\color{red}거듭제곱근}이라고 한다.

예를 들어, 허수단위 $i$에 대하여 $i^4 =1$이므로 $i$는 $1$의 네제곱근이다.

 일반적으로 복소수 범위에서 실수 $a$의 $n$제곱근은 $n$개가 있음이 알려져 있다. 그러나 여기서는 $a$의 거듭제곱근 중에서 실수인 것만을 생각하기로 하자. 실수 $a$의 $n$제곱근 중에서 실수인 것은 함수 $y = x^n$의 그래프와 직선 $y=a$의 교점의 $x$좌표와 같다. 함수 $y = x^n$의 그래프를 이용하면 $a$의 $n$제곱근 중에서 실수인 것의 위치와 갯수를 알 수 있다.
 


\begin{enumerate}[label=\arabic*)]
	\item $n$이 홀수일 때,
	
	함수 $y=x^n$의 그래프는 다음 그림과 같이 원점에 대하여 대칭이다.
	\begin{figure}[H]%
		\begin{center}
			\begin{tikzpicture}[scale=0.75]
				\draw[-stealth, thick] (-2, 0) -- (2.5, 0) node[below]{$x$};
				\draw[-stealth, thick] (0, -3) -- (0, 3) node[left]{$y$};
				\draw[scale=0.8, domain=-2.3:2.3, smooth, variable=\x, blue, thick] plot ({\x}, {0.3*\x*\x*\x}) node[right]{$y=x^n$};
				\draw[thick, red] (-1.9, 2) -- (1.9, 2) node[right, yshift=5pt]{$y=a$};
				\node[red, yshift=-5pt, xshift=18pt] at (1.9, 2) {$a>0$};
				\draw[thick, cyan] (-1.9, -2) -- (1.9, -2) node[right, yshift=5pt]{$y=a$};
				\node[cyan, yshift=-5pt, xshift=18pt] at (1.9, -2) {$a<0$};
				\draw[thick, dashed] (1.63, 2) -- (1.63, 0) node[below, red]{$\sqrt[n]{a}$};
				\draw[thick, dashed] (-1.63, -2) -- (-1.63, 0) node [above, blue]{$\sqrt[n]{a}$};
				\node[below left, xshift=3pt, yshift=1pt] at (0,0) {$\textrm{O}$};		 	
			\end{tikzpicture}
		\end{center}
	\end{figure}
	 따라서 실수 $a$에 관계 없이 $a$의 $n$제곱근 중에서 실수인 것은 오직 하나 존재하며, 이것을 기호로 $\sqrt[n]{a}$로 나타낸다. 이 때  $\sqrt[n]{a}$와 $a$의 부호는 같다. 
	
	\item $n$ 이 짝수일 때,
	
		함수 $y=x^n$의 그래프는 다음 그림과 같이 $y$축에 대하여 대칭이다.
		\begin{figure}[H]%
			\begin{center}
				\begin{tikzpicture}[scale=0.75]
					\draw[-stealth, thick] (-4, 0) -- (4, 0) node[below]{$x$};
					\draw[-stealth, thick] (0, -1.5) -- (0, 5) node[left]{$y$};
					\draw[scale=0.8, domain=-3:3, smooth, variable=\x, blue, thick] plot ({\x}, {0.6*\x*\x}) node[right]{$y=x^n$};
					\draw[thick, red] (-2.5, 2) -- (2.5, 2) node[right, yshift=5pt]{$y=a$};
					\node[red, yshift=-5pt, xshift=18pt] at (2.5, 2) {$a>0$};
					\draw[thick, cyan] (-1.9, -1) -- (1.9, -1) node[right, yshift=5pt]{$y=a$};
					\node[cyan, yshift=-5pt, xshift=18pt] at (1.9, -1) {$a<0$};
					\draw[thick, dashed] (1.63, 2) -- (1.63, 0) node[below, red] {$\sqrt[n]{a}$};
					\draw[thick, dashed] (-1.63, 2) -- (-1.63, 0) node[below, red] {$-\sqrt[n]{a}$};
					\node[below left, xshift=3pt, yshift=1pt] at (0,0) {$\textrm{O}$};		 	
				\end{tikzpicture}
			\end{center}
		\end{figure}
따라서 $n$ 이 짝수일 때는 $a>0$이면 그림과 같이 $a$의 $n$제곱근 중에서 실수인 것은 $\sqrt[n]{a}$와 $-\sqrt[n]{a}$이다. $a=0$이면 모든 자연수 $n$에 대하여 $0^n =0$이므로 $0$의 $n$제곱은은 $0$하나 뿐이다. 마지막으로 $a<0$이면 그림에서 알 수 있듯이 $a$의 $n$제곱근은 존재하지 않는다.
\end{enumerate}

  이상을 정리하면 다음과 같다.
  

  \begin{tcolorbox}[tab2,tabularx={X||Y|Y|Y|},title=$n$이 짝수일 때,boxrule=0.5pt]
  		  & $a>0$    & $a=0$     & $a<0$       \\\hline\hline
  	$n$이 짝수 & $\sqrt[n]{a}$, 또는 $-\sqrt[n]{a}$ & $0$ &  없다  \\ \hline
   $n$이 홀수&  $\sqrt[n]{a}$ & 0   &   $\sqrt[n]{a}$ \\\hline\hline
  \end{tcolorbox}

참고로 $x^n$은 $n$이 짝수일 때는 우함수, $n$이 홀수일 때는 기함수이다. 다음 식을 보면 임의의 함수 $f(x)$는 우함수와 기함수의 합으로 나타낼 수 있음을 알 수 있다.
\begin{equation*}
	f(x) = \left(\frac{f(x)+f(-x)}{2} \right) + \left( \frac{f(x)-f(-x)}{2}\right)
\end{equation*}

$a>0$고 $n$이 $2$ 이상의 정수일 때, $a$의 양의 $n$제곱근인 $\sqrt[n]{a}$는 오직 하나 존재하
고, $\left(\sqrt[n]{a}\right)^{n}=a$가 성립한다. 지수법칙을 이용하여 거듭제곱근에 대하여 성립하는 여러 가지 성질을 알아보자.
\begin{example}
	$a>0, b>0$이고 $n$이 $2$ 이상인 자연수일 때,
	\[
	\sqrt[n]{a}\sqrt[n]{b} = \sqrt[n]{ab}
	\]
	가 성립함을 보여라.
	\begin{solution}
		$\left(\sqrt[n]{a} \sqrt[n]{b}\right)^n = \left(\sqrt[n]{a}\right)^n \left(\sqrt[n]{b}\right)^n =ab$이다. 그런데 $\sqrt[n]{a}>0$, $\sqrt[n]{b}>0$이므로 $\sqrt[n]{a} \sqrt[n]{b}>0$이다. 따라서 $\sqrt[n]{a} \sqrt[n]{b}$는 양수 $ab$의 양의 $n$제곱근인 $\sqrt[n]{ab}$와 같다. 즉 $\sqrt[n]{a}\sqrt[n]{b} = \sqrt[n]{ab}$이다. 
	\end{solution}
\end{example}

\begin{problem}
	$a>0$이고 $m, n$이 $2$ 이상의 자연수일 때, 다음이 성립함을 보이시오.
	\questwo{\left(\sqrt[n]{a}\right)^{m} =\sqrt[n]{a^m}} {\sqrt[m]{\sqrt[n]{a}}=\sqrt[mn]{a}}
	
\processifversion{psol}{\begin{psolution}
\begin{enumerate}[label=(\arabic*)]
\item $\left(\sqrt[n]{a}\right)^m$을 $n$제곱 하면
\begin{equation*}
	\left(\left(\sqrt[n]{a}\right)^m \right)^n = \sqrt[n]{a}^{mn} =\left(\left(\sqrt[n]{a}\right)^n \right)^m = a^m
\end{equation*}
이므로 $ \left(\sqrt[n]{a}\right)^m$는 $a^m$의 $n$제곱근이다. 따라서 $\left(\sqrt[n]{a}\right)^{m} =\sqrt[n]{a^m}$이다. 
\item $\sqrt[m]{\sqrt[n]{a}}$을 $mn$제곱하면 
\begin{equation*}
	\left( \sqrt[m]{\sqrt[n]{a}}\right)^{nm} = \left( \left( \sqrt[m]{\sqrt[n]{a}}\right)^{m}\right)^{n} = \left(\sqrt[n]{a}\right)^{n} = a
\end{equation*}
이므로 $\sqrt[m]{\sqrt[n]{a}}$는 $a$의 $mn$제곱근이다. 따라서 $\sqrt[m]{\sqrt[n]{a}}=\sqrt[mn]{a}$이다.
\end{enumerate}
\end{psolution}}
\end{problem}

이상을 정리하면 다음과 같다.
\begin{theorem}[거듭제곱근의 성질]
$a>0$, $b>0$이고 $m, n$이 $2$ 이상의 정수일 때,
	\quesfour{\sqrt[n]{a}\sqrt[n]{b} = \sqrt[n]{ab}}{\frac{\sqrt[n]{a}}{\sqrt[n]{b}} = \sqrt[n]{\frac{a}{b}}}{\left(\sqrt[n]{a}\right)^m = \sqrt[n]{a^m}}{\sqrt[m]{\sqrt[n]{a}} = \sqrt[mn]{a}}
\end{theorem}

거듭제곱근의 성질은 매우 중요하다. 다음 절에서 우리는 거듭제곱근의 성질을 이용하여 지수를 유리수 범위까지 확장할 것이다.

\section{지수의 확장}
 지금까지 우리는 지수가 양의 정수, 즉 자연수의 경우에만 생각해 왔으나 이제 지수가 $0$ 또는 음의 정수의 경우로 확장해 보자. $a \neq 0$이고 $m, n$이 자연수일 때 다음이 성립한다.
 \[
 a^m \div a^n = a^{m-n}
 \]
 여기에서 $m=n$이면 $a^{0} =a^{m-m} = a^m \div a^m =1$이다. 즉 $a^{0}=1$이다. 한편 위의 식에서 $m=0$을 대입하면,
 $a^0 \div a^n = a^{-n}$이므로 $a^{-n} =\frac{1}{a^n}$임을 알 수 있다. 이를 정리하면 다음과 같고 지수가 정수 범위로 확장되었음을 알 수 있다.
 \begin{theorem}[$0$ 또는 음의 정수의 지수]
 $a \neq 0$이고 $n$이 양의 정수일 때,
 \[
 a^{0} = 1, \quad a^{-n} = \frac{1}{a^n}
 \]
 \end{theorem}

지수가 $0$ 또는 음의 정수인 경우를 위와 같이 정의하면, 지수가 양의 정수일 때의 지수법칙 (5)를 다음과 같이 간단히 나타낼 수 있다. 
\begin{itemize}
	\item $m=n$일 때, $a^m \div a^n = 1=a^{0} = a^{m-n}$
	\item $m<n$일 때, $a^m \div a^n = 1=\frac{1}{a^{n-m}} = a^{-(n-m)} = a^{m-n}$
\end{itemize}
이다. 따라서 $m, n$의 대소와 관계 없이 임의의 양의 정수 $m,\;n$에 대하여 
\[
a^m \div a^n = a^{m-n}
\]
이다.

\begin{example}
	$a\neq 0$이고 $m,\;n$이 모두 음의 정수일 때, 다음의 지수법칙이 성립함을 보이시오.
	\[
	a^m a^n = a^{m+n}
	\]
	\begin{solution}
	 $m$과 $n$이 음의 정수이면 $-m=p$, $-n =q$라 하면 $p,\;q$는 양의 정수이고 따라서
	\[
	a^m a^n = a^{-p} a^{-q} =\frac{1}{a^p}\frac{1}{a^q} = \frac{1}{a^p a^q} =\frac{1}{a^{p+q}}=a^{-(p+q)} = a^{m+n}
	\]
	이 성립한다.
\end{solution}
\end{example}


예제의 방법을 적절히 변형하면 지수법칙 
\begin{equation*}
	a^m \div a^n = a^{m-n}
\end{equation*}
을 어렵지 않게 증명할 수 있다.
\begin{problem}
	$a\ne0$, $b\ne0$이고 $m$, $n$이 모두 음의 정수일 때, 다음 지수법칙이 성립함을 보여라.
	\questwo{(a^{m})^{n}=a^{mn}}{(ab)^{n}=a^{n}b^{n}}
	\processifversion{psol}{\begin{psolution}
			\begin{enumerate}[label=(\arabic*)]
				\item$m$, $n$이 모두 음의 정수라 하자. $-m=p$, $-n =q$라 하면 $p,\;q$는 양의 정수이고
				\begin{equation*}
					\left(a^{m}\right)^n = \left(a^{-p}\right)^n =\left(\frac{1}{a^{p}}\right)^{-q} = a^{pq} = a^{(-p)(-q)} = a^{mn}
				\end{equation*}
				이다.
				\item $n=-q$라 하면
				\begin{equation*}
					\left(ab\right)^{n} = \left(ab\right)^{-q} = \left(\frac{1}{(ab)^q}\right)= \frac{1}{a^q b^q} = \frac{1}{a^q} \frac{1}{b^q} =a^{-q} b^{-q}  =a^n b^n
				\end{equation*}
				이다.
			\end{enumerate}
	\end{psolution}}
\end{problem}

일반적으로 지수가 정수일 때, 다음과 같은 지수법칙이 성립함을 알 수 있다.
\begin{theorem}[지수가 정수일 때의 지수법칙]
	$a\ne0$, $b\ne0$이고 $m$, $n$이 정수일 때
	\quesfour{a^{m}a^{n}= a^{m+n}}{a^{m}div a^{n}= a^{m-n}}{(a^{m})^{n}= a^{mn}}{(ab)^{n}=a^{n}b^{n}}
\end{theorem}

\begin{sample}
	\begin{enumerate}[label = (\arabic*)]
		\item $5^{-3}\div  5^{-5}=5^{-3-(-5)}=5^{2}=25$
		\item $2^{-2}\times 5^{-2}=(2\times 5)^{-2}=10^{-2}=\frac{1}{100}$
	\end{enumerate}
\end{sample}
\begin{problem}
	$a\ne0$, $b\ne0$일 때, 다음 식을 간단히 하여라.
	\quesfour{a^{7}a^{-4}}{a^{4} \div a^{-3}}{(a^{-3})^{-2}}{(a^{2}b^{-4})^{-3}}
	
\end{problem}


우리는 지금까지 지수를 자연수에서 정수로 확장하는 방법과 확장후 지수법칙이 여전히 성립함을 증명하였다. 이제 우리는 지수를 유리수 범위로 확대하는데 앞 절에서 탐구한
거듭제곱근을 이용하여 확장할 것이다.

$a>0$이고 지수 $p$, $q$가 유리수일 때에도 지수법칙 $(a^{p})^{q}=a^{pq}$이 성립한다면 정수 $m$, $n$($n>0$)에 대하여
\[
\left(a^{\frac{m}{n}}\right)^{n}= a^{\frac{m}{n}\times n}= a^{m}
\]
이므로 거듭제곱근의 정의로부터 $a^{\frac{m}{n}}$은 $a^{m}$의 $n$제곱근 중의 하나이다.
여기서 $a^{\frac{m}{n}}$을 양수 $a^{m}$의 양의 $n$제곱근으로 정하면
\[
a^{\frac{m}{n}}=\sqrt[n]{a^{m}}
\]
이다. 따라서 지수가 유리수인 경우에는 다음과 같이 정의한다.
\begin{theorem}[유리수인 지수]
	$a>0$이고 $m$, $n$($n \geq 2$)이 정수일 때
	\[
	a^{\frac{m}{n}}=\sqrt[n]{a^{m}}, \text{ 특히 } a^{\frac{1}{n}}=\sqrt[n]{a}
	\]
\end{theorem}

\begin{sample}
	\questwo{8^{\frac{1}{3}}=\sqrt[3]{8}=2}{4^{\frac{3}{2}}=\sqrt{4^{3}}=\sqrt{64}=8}
\end{sample}

\begin{problem}
	다음 수를 근호를 사용하여 나타내고, 간단히 하여라.
\quesfour{64^{\frac{1}{2}}}{27^{\frac{4}{3}}}{27^{-\frac{1}{3}}}{\left(\frac{1}{16}\right)^{-0.75}}	
\end{problem}

거듭제곱근의 성질과 지수의 정의를 이용하여 앞에서 배운 지수법칙이 지수가 유리수인 경우에도 성립함을 알아보자.
\begin{example}
$a>0$이고 $r$, $s$가 유리수일 때, 다음 지수법칙이 성립함을 보여라.
\begin{equation*}
		a^{r} a^{s}= a^{r+s}
\end{equation*}	
\begin{solution}
	 정수 $m$, $n$, $p$, $q$($n \geq2$, $q \geq 2$)에 대하여 $r=\frac{m}{n}$, $s=\frac{p}{q}$로 놓으면
	\begin{align*}
	a^{r}a^{s} &= a^{\frac{m}{n}}a^{\frac{p}{q}}\\
					&= a^{\frac{mq}{nq}}a^{\frac{np}{nq}} \\
			   		&=\sqrt[nq]{mq}\times\sqrt[nq]{np}\\
			   		&=\sqrt[nq]{a^{mq+np}}\\
			        &= a^{\frac{mq+np}{nq}}\\
			        &= a^{\frac{m}{n}+\frac{p}{q}}\\
			        &= a^{r+s}
\end{align*}
	따라서 $a^{r}a^{s}= a^{r+s}$이 성립한다.
\end{solution}
	
\end{example}

\begin{problem}
	 $a>0$, $b>0$이고 $r$, $s$가 유리수일 때, 다음 지수법칙이 성립함을 보여라.
\questhree{a^{r} \div a^{s}= a^{r-s}}{(a^{r})^{s}= a^{rs}}{(ab)^{r}= a^{r}b^{r}}
\end{problem}

이상을 정리하면 일반적으로 지수가 유리수일 때에도 다음과 같은 지수법칙이 성립함을 알 수 있다.

\begin{theorem}[지수가 유리수일 때의 지수법칙]
	$a>0$, $b>0$이고 $r$, $s$가 유리수일 때
\quesfour{ a^{r}a^{s}= a^{r+s}}{ a^{r} \div a^{s}= a^{r-s}}{(a^{r})^{s}= a^{rs}}{(ab)^{r}= a^{r}b^{r}}		
\end{theorem}
 \begin{sample}
\quesfour{3^{\frac{1}{2}}\times 3^{-\frac{3}{4}}= 3^{\frac{1}{2}-\frac{3}{4}}= 3^{-\frac{1}{4}}}{5^{\frac{5}{2}} \div 5^{2}= 5^{\frac{5}{2}- 2}= 5^{\frac{1}{2}}}{(2^{\frac{2}{3}})^{6}= 2^{\frac{2}{3}\times 6}= 2^{4}}{(2^{3}\times 3)^{\frac{2}{3}}=(2^{3})^{\frac{2}{3}}\times 3^{\frac{2}{3}}= 2^{2}\times 3^{\frac{2}{3}}}
 \end{sample}

\begin{problem}
	다음 식을 간단히 하여라.  
\quesfour{3^{\frac{2}{3}}\times 3^{\frac{3}{2}}}{4^{\frac{3}{2}}\times 4^{\frac{1}{2}}}{(2^{\frac{3}{4}})^{-\frac{8}{3}} }{(16\times 25)^{\frac{3}{4}} }	
\end{problem}
 \vskip 10pt
 
 \begin{problem}
 	 $(2^{0.2})^{2}$은 실수 $a$의 양의 $n$제곱근과 같다. 자연수 $a$, $n$의 순서쌍 $(a,\:n)$을 두 개만 말하여라.
 	 \processifversion{psol}{
 \begin{psolution}
 \begin{align*}
 	(2^{0.2})^{2} &= 2^{0.4} \\
 							&=2^{\frac{4}{10}} = \sqrt[10]{2^4} \\
 							&=2^{\frac{2}{5}} = \sqrt[5]{2^2}
 \end{align*}
이므로 주어진 수는 $16$의 $10$제곱근 또는 $4$의 $5$제곱근이다. 따라서 $(16, 10)$ 또는 $(4, 5)$이다.
 \end{psolution}	 
  }
 \end{problem}

이제 지수를 실수로 확장하는 데 고등학교 수준을 조금 넘어서는 수준에서 이론을 전개하기 위하여 {\color{red}같음}을 새롭게 정의해 보자. 

만약 두 수 $a,\;b$가 같으면 $a-b=0$이므로 임의의 양의 실수 $\epsilon$에 대하여 $\vert a-b \vert < \epsilon$이다.  이제 임의의 양의 실수 $\epsilon$에 대하여 $\vert a-b \vert < \epsilon$
이면 $a=b$임을 보이자. 이를 위해 대우를 이용한 증명을 할 것이다. 이를 위해 $a \neq b$라 가정하자. 그러면 일반성을 잃지 않고 $a>b$라 가정할 수 있다. 이제 $a-b>0$이다. 이제 $\epsilon =\frac{a-b}{2}$라 하면 
\[
a-b > \frac{a-b}{2} =\epsilon
\]
이고 이것은 가정에 모순이다. 따라서 대우를 이용한 증명이 완성되었다. 

이상을 정리하면 다음을 정의할 수 있다.

\begin{definition}[$a=b$]\vspace{-1.2em}
	\begin{equation*} 
			a-b  = 0 \Leftrightarrow  \text{ 모든 양의 실수 } \epsilon \text { 에 대하여 } \vert a-b \vert < \epsilon
	\end{equation*}
\end{definition}

이 정의로부터 우리는 두 실수의 차를 얼마든지 작게 할 수 있다면 그 두 수는 같다는 것을 알 수 있다.

이제 다음을 생각하자.
\begin{align*}
	1.4 < &\sqrt{2}<1.5 \\
	1.41 <&\sqrt{2} <1.42 \\
	1.414 <  & \sqrt{2} <1.415 \\
	&\phantom{x}\vdots
\end{align*}
이제,
\begin{align*}
	2^{1.4} < &2^{\sqrt{2}}< 2^{1.5} \\
	2^{1.41} <& 2^{\sqrt{2}}< 2^{1.42} \\
	2^{1.414 }<  & 2^{\sqrt{2}} < 2^{1.415} \\
	&\phantom{x}\vdots
\end{align*}
이다. 이 과정을 계속 반복하면 $2^{\sqrt{2}}$의 좌, 우에 있는  두 수의 차를 얼마든지 작게 할 수 있으므로 두 수는 위의 정의에 따라 같아진다고 할 수 있다. 따라서 그 사이에 있는 $2^{\sqrt{2}}$은 같아지는 그 수와 같다. 이로부터 우리는 정당화 수준에서 지수가 실수일 때 도 정의되며 더 나아가 어떤 밑에 대하여 어떤 실수 지수는 유일하게 어떤 실수값이 된다는 사실을 알 수 있다.

이로부터 우리는 지수를 실수 범위까지 확장이 가능함을 알 수 있고 다음의 지수법칙이 성립함이 알려져 있다.

\begin{theorem}[지수가 실수일 때의 지수법칙]\vspace {-0.5em}
	$a>0$, $b>0$이고 $x$, $y$가 실수일 때
\quesfour{a^{x}a^{y}= a^{x+y} }{ a^{x}div a^{y}= a^{x-y}}{ (a^{x})^{y}= a^{xy} }{ (ab)^{x}= a^{x}b^{x} }	
\end{theorem}
\begin{sample}\vspace{-1em}
	\begin{enumerate}[label=(\arabic*)]
		\item $3^{\sqrt{2}}\times 3^{3\sqrt{2}} = 3^{\sqrt{2}+2\sqrt{2}} =3^{3\sqrt{2}}=\left(3^{3}\right)^{\sqrt{2}}= 26^{\sqrt{2}}$
		\item $\left(2^{\sqrt{3}}\right)^{\sqrt{3}} = 2^{\sqrt{3}\times \sqrt{3}} = 2^{3} = 8$
	\end{enumerate}
\end{sample}

\begin{problem}
	다음 식을 간단히 하여라.
\quesfour{2^{2\sqrt{2}}\times 2^{-\sqrt{2}}}{5^{\sqrt{27}}\div  5^{\sqrt{3}}}{(5^{\sqrt{2}})^{-\sqrt{2}}}{ 9^{\sqrt{3}}\times\left(\frac{1}{3}\right)^{\sqrt{3}}}

\end{problem}

\begin{problem}
	\begin{enumerate}[label=\arabic*]
		\item  $(\text{유리수})^{\text{(유리수)}}=\text{(유리수)}$를 만족시키는 예로는 $2^{3}=8$을 들 수 있다. 다음을 만족하는 예가 있는지 찾아보아라.
		\begin{enumerate}[label=(\arabic*)]
			\item $\text{(유리수)}^{\text{(유리수)}}=\text{(무리수)}$
			\item $\text{(무리수)}^{\text{(유리수)}}=\text{(유리수)}$
			\item $\text{(무리수)}^{\text{(유리수)}}=\text{(무리수)}$
		\end{enumerate}
	\item $(\sqrt{3})^{\sqrt{2}}$은 무리수이다. 이를 이용하여 $\text{(무리수)}^{\text{(무리수)}}=\text{(유리수)}$가 되는 경우를 찾아보아라.
	\end{enumerate}

\end{problem}

	\chapter{\Huge 로그}

	리히터 규모 $7.0$의 지진, $\textrm{pH}$ $4.3$의 산성 용액, 밝기가 $1$등급인 별, $80$ 데시벨($\textrm{dB}$)의 소음 등과 같은 표현을 접할 수 있는데, 이는 모두 ‘로그’라고 하는 수학적 도구와 관련이 있다. 로그는 물리적 양을 간편하게 표현하는 장점이 있기 때문에 $17$세기 초 처음 등장했을 때, 유럽 전체에서 열광적인 환영을 받았다. 특히, 많은 계산을 해야 하는 천문학에서는 로그의 탄생만을 기다렸다고 해도 과언이 아니다. 이 단원에서는 일상생활에서 흔히 접할 수 있고, 유용하게 활용할 수 있는 로그에 대하여 공부한다.
	
%	
%	\begin{figure}[H]
%		\begin{center}
%			\begin{equation*}
%				\tikzmark[blue]{base}{$\text{log}_{\;a}{\;\;\tikzmark[red]{exponent}{b}}$}
%			\end{equation*}
%			\begin{tikzpicture}[overlay, remember picture,node distance =.2cm]
%				\node[blue] (basedescr) [below right=of base]{밑};
%				\draw[,-stealth,thick] (basedescr.west) to [in=270,out=180] (base.south);
%				\node[red] (exponentdescr) [above right=of exponent] {진수};
%				\draw[-stealth,thick] (exponentdescr.west) to [in=90,out=180] (exponent.north);
%			\end{tikzpicture}
%		\end{center}
%	\end{figure}

$2$를 $x$번 거듭제곱하여 $8$이 되게 하는 $x$의 값은 $3$이다. 즉,
\[
2^{x}=8 \text{ 이면 } x=3
\]
이다.

이제, $a^{x}=b$가 되는 $x$를 $a$, $b$를 이용하여 나타내는 방법을 생각해 보자.
$a>0$, $a\ne 1$일 때, 양수 $b$에 대하여
\[
a^{x}=b
\]
를 만족하는 실수 $x$는 반드시 하나 존재한다.

이 실수 $x$를 기호로
\[
x=\log_{a} b
\]
와 같이 나타내고, $a$를 {\color{red}밑}으로 하는 $b$의 {\color{red}로그(log)}라고 한다. 
이때, $b$를 $\log_{a}b$의 {\color{red}진수}라고 한다. 다시 말하면 $\log_{a} b$는 $b$를 $a$를 밑으로 하는 지수로 표현하였을 때의 지수라고 정의할 수 있다. 따라서 로그의 정의에 따라 다음이 성립함을 이해할 수 있다.
\[
 a^{\log_{a} b} = b
\]

로그는 {\color{blue} (밑)$>0$, (밑)$\ne 1$, (진수)$>0$}일 때에만 정의된다.  아래의 로그의 정의에서 $\Leftrightarrow$의 좌변을 {\color{red}지수표현}, 우변을 {\color{red}로그표현}이라고 하자.
\begin{definition}[로그의 정의]\vspace{-0.5em}
	$a>0$, $a \ne 1$, $b>0$일 때 
\[
	a^{x}=b  \Leftrightarrow x=\log_{a}b
\]	
\end{definition}

\begin{sample}
	\questwo{2^{4}=16 \Leftrightarrow 4 =\log_{2}16 }{ \log_{3}\frac{1}{9}= -2 \Leftrightarrow 3^{-2}=\frac{1}{9}}
\end{sample}

\begin{problem}
	다음 등식을 로그를 사용하여 나타내어라.
	\questhree{5^{3}=125 }{10^{-2}=0.01}{3^{\frac{1}{2}}=\sqrt{3}}
\end{problem}

\begin{problem}
	다음 등식을 지수를 사용하여 나타내어라.
\questhree{ \log_{3}81=4}{\log_{\frac{1}{2}}8 = -3}{\log_{8}2 =\frac{1}{3}}	
\end{problem}
\begin{example}
	다음 로그의 값을 구하여라.
	\questwo{ \log_{9}27}{\log_{\frac{1}{2}}8}
	\begin{solution}
		\begin{enumerate}[label=(\arabic*)]
			\item $\log_{9}27=x$로 놓고 지수 표현으로 변형하면  $9^{x}=27$이고 따라서 $(3^{2})^{x}=3^{3}$이다. 즉, $3^{2x}=3^{3}$에서  밑이 같으므로   $2x=3$ 이고 $x=\frac{3}{2}$이다.
			\item 로그의 정의를 이용하여 간단히 계산할 수도 있다. 진수 $8$을 밑 $\frac{1}{2}$을 이용하여 지수표현하면 지수가 $-3$이므로 구하는 값은 $-3$이다.
		\end{enumerate}
	\end{solution}
\end{example}

\begin{problem}
	 다음 로그의 값을 구하여라.
\questhree{ \log_{2}32}{\log_{5}\sqrt{5}}{\log_{10}\frac{1}{1000}}
\end{problem}

\section{로그의 성질}

$a>0$, $a\ne 1$일 때, $a^{0}=1$, $a^{1}=a$이므로 로그의 정의에 의하여 다음이 성립한다.
\[
\log_{a}1=0, \quad \log_{a}a = 1
\]

또, $M>0$, $N>0$일 때, $\log_{a}M=x$, $\log_{a}N=y$로 놓으면 $M=a^{x}$, $N=a^{y}$이므로
\[
MN=a^{x}a^{y}=a^{x+y}
\]
이다. 따라서 로그의 정의에 의하여
\[
\log_{a}MN=x+y=\log_{a}M+\log_{a}N
\]
이 성립한다.

\begin{problem}
	$a>0$, $a \ne 1$, $M>0$, $N>0$일 때, 다음이 성립함을 보여라.
\questwo{ \log_{a}M =\log_{a}M -\log_{a}N}
{\log_{a}M^{k}=k\log_{a}M\; (k\text{는 실수})}
\processifversion{psol}{
\begin{psolution}
	\begin{enumerate}[label=(\arabic*)]
		\item $M>0$, $N>0$일 때, $\log_{a}M=x$, $\log_{a}N=y$로 놓으면 $M=a^{x}$, $N=a^{y}$이므로
		\[
		\frac{M}{N}=\frac{a^{x}}{a^{y}}=a^{x-y}
		\]
		이다. 따라서 로그의 정의에 의하여
		\[
		\log_{a}\frac{M}{N}=x-y=\log_{a}M-\log_{a}N
		\]
		이 성립한다.
		\item $\log_{a}M =m$이라 하자. 즉, 정의에 의해 $M=a^{m}$이다. 그런데 
		\[
		M^{k} = \left(a^{m}\right)^{k}  = a^{mk}
		\]
		이므로 $\log_{a}M^{k} = mk = k \log_{a}M$이다.
	\end{enumerate}
\end{psolution}
}
\end{problem}

일반적으로 로그에 대하여 다음과 같은 성질이 성립한다.

\begin{theorem}[로그의 성질]
	$a>0$, $a\ne 1$, $M>0$, $N>0$이고, $k$가 임의의 실수일 때,
	\quesfour
	{\log_{a}1=0, \; \log_{a}a=1 }
	{\log_{a}MN =\log_{a}M +\log_{a}N}
	{\log_{a}\frac{M}{N}=\log_{a}M -\log_{a}N}
	{\log_{a}M^{k}=k\log_{a}M}
\end{theorem}

\begin{sample}
	\begin{enumerate}[label=(\arabic*)]
		\item $\log_{\frac{1}{2}}1 =0$, $\log_{\frac{1}{2}}\frac{1}{2}= 1$
		\item $\log_{2}10 =\log_{2}(2\times 5)=\log_{2}2 +\log_{2}5 = 1 +\log_{2}5$
		\item $\log_{10}5 =\log_{10}\frac{10}{2}=\log_{10}10 -\log_{10}2 = 1 -\log_{10}2$
		\item $\log_{10}\frac{1}{100}=\log_{10}10^{-2}= -2\log_{10}10=-2$
	\end{enumerate}
\end{sample}
\begin{problem}
	 다음을 간단히 하여라.
	\quesfour
	{ \log_{3}1 +\log_{\sqrt{3}}\sqrt{3}}
	{\log_{6}4+\log_{6}9}
	{\log_{5}15 -\log_{5}3}
	{\log_{7}\frac{1}{49}}
\end{problem}
\begin{problem}
	 $a>0$, $a \ne1$, $M>0$일 때, 다음 등식이 성립함을 설명하여라.
	 \begin{enumerate}[label=(\arabic*)]
	 	\item $\log_{a}\frac{1}{M}=-\log_{a}M$
	 	\item $\log_{a}\sqrt[n]{M}=\frac{1}{n}\log_{a}M$ ($n$은 $2$이상의 자연수)
	 \end{enumerate}
 \processifversion{psol}{
\begin{psolution}
	\begin{enumerate}[label=(\arabic*)]
		\item 로그의 성질 (1)과 (3)에 의해
	\[
	\log_{a}\frac{1}{M} = \log_{a}1 - \log_{a}M = 0 - \log_{a}M = -\log_{a}M
	\]
	이다.
	\item $\sqrt[n]{M} =M^\frac{2}{n}$이므로 로그의 성질 (4)에 의해
	\[
	\log_{a}\sqrt[n]{M}=\log_{a}M^{\frac{1}{n}}=\frac{1}{n}\log_{a}M
	\]
	이다.
	\end{enumerate}
\end{psolution} 
}
\end{problem}
\begin{example}
	다음 식을 간단히 하여라.
	\questwo{\log_{2}6 - 2\log_{2}\sqrt{3}}{\log_{10}\sqrt{5}-\dfrac{1}{2}\log_{10}\dfrac{1}{2}}
\begin{solution}
\begin{enumerate}[label=(\arabic*)]
	\item $2\log_{2}\sqrt{3}=\log_{2}(\sqrt{3})^{2}=\log_{2}3$이므로 주어진 식을 변형하면
	\begin{align*}
		\log_{2}6 - 2\log_{2}\sqrt{3}& =\log_{2}6 -\log_{2}3\\
		                                         	&=\log_{2}\frac{6}{3}\\
		                                     		&=\log_{2}2=1
	\end{align*}
	\item $\sqrt{5}= 5^{\frac{1}{2}}$, $\frac{1}{2}= 2^{-1}$이므로 주어진 식을 변형하면
	\begin{align*}
		\log_{10}\sqrt{5}-\frac{1}{2}\log_{10}\frac{1}{2}&=\log_{10}5^{\frac{1}{2}}-\frac{1}{2}\log_{10}2^{-1}\\
		&=\frac{1}{2}\log_{10}5 +\frac{1}{2}\log_{10}2\\
		&=\frac{1}{2}(\log_{10}5 +\log_{10}2)\\
	    &=\frac{1}{2}\log_{10}10 =\frac{1}{2}
	\end{align*}
\end{enumerate}
\end{solution}
\end{example}

\begin{problem}
	 다음 식을 간단히 하여라.
\questwo{\log_{2}\frac{4}{3}+ 3\log_{2}\sqrt[3]{12}}{\log_{2}\sqrt{8}-\log_{2}\frac{2}{3}-\log_{2}\sqrt{18}}
\end{problem}

$\log_{a}b$를 $1$이 아닌 양수 $c$를 밑으로 하는 로그로 바꾸는 방법을 알아보자. $\log_{a}b$에서 $b$는 $a$를 밑으로 하는 지수 표현이 되었다고 볼 수 있으므로 $a$를 $c$를 밑으로 하는 지수 표현을 하면 $b$도 자연스럽게 $c$를 밑으로 하는 지수 표현이 된다. 따라서 

$\log_{a}b=x$, $\log_{c}a=y$라고 하면 로그의 정의에 의하여 $a^{x}=b$, $c^{y}=a$이므로 지수의 성질에 의하여
\[
b=a^{x}=(c^{y})^{x}=c^{xy}
\]
이고, 다시 로그의 정의에 의하여 $xy=\log_{c}b$, 즉
\[
\log_{a}b\times\log_{c}a=\log_{c}b
\]

그런데 $a\ne1$일 때, $\log_{c}a\ne0$이므로 위의 식의 양변을 $\log_{c}a$로 나누면
\[
\log_{a}b =\frac{1}{\log_{b}a} \quad (a \ne 1 \text{ 이므로 } \log_{c}a \ne 0)
\]
이다. 이상을 정리하면 다음과 같다.

\begin{theorem}[밑의 변환 공식]
	$a>0$, $a\ne1$, $b>0$, $c>0$, $c\ne1$일 때 
	\[
	\log_{a}b =\frac{\log_{c}b}{\log_{c}a}
	\]
\end{theorem}


\begin{sample}
	\questwo{\log_{4}8=\frac{\log_{2}8}{\log_{2}4}=\frac{\log_{2}2^{3}}{\log_{2}2^{2}}=\frac{3}{2} }{\log_{3}10 =\frac{\log_{10}10}{\log_{10}3}=\frac{1}{\log_{10}3}}
\end{sample}
\begin{problem}
	다음 값을 구하여라.
\quesfour{\log_{9}\frac{1}{27}}{\log_{6}4 +\log_{\sqrt{6}} 3}{\log_{2}3 \cdot \log_{3}2}{\log_{3}5 \cdot \log_{5}7 \cdot \log_{7}9}
\end{problem}

\begin{example}
	$\log_{10}2=a$, $\log_{10}3=b$일 때, $\log_{5}12$를 $a$, $b$로 나타내어라.
	\begin{solution}
		 밑의 변환 공식에 의하여   $\log_{5}12 =\dfrac{\log_{10}12}{\log_{10}5}$이고
		$\log_{10}12$를 $a$, $b$로 나타내면 
		\begin{align*}
			\log_{10}12 &=\log_{10}(2^{2}\times 3)=\log_{10}2^{2}+\log_{10}3\\
			 &=2\log_{10}2 +\log_{10}3 = 2a+b
		\end{align*}
		이제 $\log_{10}5$를 $a$, $b$로 나타내면     
		\begin{align*}
		\log_{10}5 =\log_{10}\frac{10}{2}&=\log_{10}10 -\log_{10}2\\
		&=1-\log_{10}2=1-a
		\end{align*}
		따라서 $\log_{5}12$를 $a$, $b$로 나타내면  $\log_{5}12 =\frac{2a+b}{1-a}$
	\end{solution}
\end{example}

\begin{problem}
	$\log_{10}2=a$, $\log_{10}3=b$일 때, 다음을 $a$, $b$로 나타내어라.
\questwo{\log_{6}24}{\log_{5}18}	
\end{problem}
이제 로그의 성질 마지막으로 지수와 로그에 관한 문제해결과 지수함수의 미분에서 유용한 로그의 성질에 대하여 알아 본다.
$a>0, \; a \ne 1$, $b>0, \; c \ne 1$, $c>0, \; c \ne 1$라 하자.

$x = a^{\log_{b}c}$라 하고 이 식의 양변에 $b$를 밑으로 하는 로그를 취하면
\begin{align*}
	\log_{b}x &= \log_{b}a^{\log_{b}c} \\
					&= \log_{b}c \times \log_{b}a \\
					&= \log_{b}a \cdot \log_{b}c \\
					&=\log_{b}c^{\log_{b}a}
\end{align*}
이다. 따라서 $\log_{b}a^{\log_{b}c} = \log_{b}c^{\log_{b}a}$이므로 $a^{\log_{b}c} = c^{\log_{b}a}$이다.
이상을 정리하면 다음과 같다.
\begin{theorem}
	$a>0, \; a \ne 1$, $b>0, \; c \ne 1$, $c>0, \; c \ne 1$일 때,
	\[
	a^{\log_{b}c} = c^{\log_{b}a}
	\]
\end{theorem}
  이 성질을 이용하면 우리는 $\text{(자연수)}^{\text{(무리수)}}=\text{(자연수)}$가 되는 예를 찾을 수 있다. 즉, 
  \[
    2^{\log_{2}3} = 3^{\log_{2}2} =3
  \]
  이다. 이제 $\log_{2}3$이 무리수임을 보이면 된다. 
  
  $\log_{2}3$이 유리수라 가정하자. 이 수는 $1$보다 크므로 서로소인 두 자연수 $m, \;n$에 대하여 $\log_{2}3=\frac{n}{m}$이라 할 수 있다. 따라서
  $2^{\frac{n}{m}} = 3$이고 $2^n = 3^m$이다. 그런데 이 식의 좌변은 짝수이고 우변은 홀수 이므로 모순이다. 따라서 $\log_{2}3$은 무리수이다.
  
  위의 두 결과들을 종합하면 $\text{(자연수)}^{\text{(무리수)}}=\text{(자연수)}$이 가능함을 알 수 있다.

\section{상용로그}

  $10$을 밑으로 하는 로그를 상용로그라 하고, 양수 $N$의 상용로그 $\log_{10}N$은 보통 밑 $10$을 생략하여 $\log N$과 같이 나타낸다.  예를 들어, $2$의 상용로그의 값은 $\log 2$이고, 이때 $\log 2=\log_{10}2$이다. 일반적으로 $10^{n}$($n$은 실수)인 꼴의 수에 대한 상용로그의 값은 로그의 성질을 이용하여 쉽게 구할 수 있다.
  
  \begin{sample}\vspace{-1em}
  	\questwo{\log 1000=\log_{10}10 =3}{\log\sqrt[3]{100}=\log_{10}10^{\frac{2}{3}}=\frac{2}{3}}
  \end{sample}

\begin{problem}
	다음 표를 완성하여라.
	
	 \begin{tcolorbox}[tab2,tabularx={X||Y|Y|Y|Y|Y|Y|Y|},boxrule=0.5pt]
	$x$	       & $0.001$    & $0.01$     & $0.1$  & $1$ & $10$ & $100$ & $1000$     \\\hline\hline
	$\log x$ &                   &                 &             &        &          &             &        \\ \hline
	
	\end{tcolorbox}

\processifversion{psol}{\begin{psolution}
		\mbox{} \\ \vspace{-1.5em}
		 \begin{tcolorbox}[tab2, halign=center,tabularx={X||Y|Y|Y|Y|Y|Y|Y|},boxrule=0.5pt]
			$x$	       & $0.001$    & $0.01$     & $0.1$  & $1$ & $10$ & $100$ & $1000$     \\\hline\hline
			$\log x$ &             -3      &         -2        &      -1       &     0   &     1     &       2      &     3   \\ \hline
			
		\end{tcolorbox}
	\end{psolution}
}
\end{problem}

일반적으로 교과서의 상용로그표는 $1.00$에서 $9.99$ 까지의 수에 대하여 $0.01$의 간격으로 상용로그의 값을 반올림하여 소수 넷째 자리까지 나타낸 것이다. 일반적으로 공학에서 사용하는 로그표는 더 많은 소숫점 아래의 숫자까지 계산할 수 있다. 한편 상용로그표의 값은 근삿값이지만 보통 $\log 8.15 \fallingdotseq 0.9112$보다는 $\log 8.15=0.9112$로 나타낸다.
일반적인 상용로그표는 $1.00$에서 $9.99$까지의 수에 대하여 $0.01$의 간격으로 상용로그의 값을 반올림하여 소수 넷째 자리까지 나타낸 것이다.

예를 들어, $\log 8.15$의 값은  상용로그표에서 $8.1$의 행과 $5$의 열이 만나는 곳에 있는 수인 $0.9112$이다. 즉, $\log 8.15=0.9112$이다.

\begin{sample}
	\questwo{\log 8.21=0.9143 }{ \log x=0.9090 \text{ 이면 } x=8.11}
\end{sample}

\begin{problem}
	상용로그표를 이용하여 다음 $x$의 값을 구하여라.
\questwo{ \log 8.08=x}{\log x=0.9175}
\end{problem}
$N ≥10$ 또는 $0<N<1$일 때, $\log N$의 값을 구하는 방법을 알아보자.

일반적으로 양수 $N$은
\[
N=a\times 10^{n}\quad (1≤a<10, \;n \text{은 정수})        
\]
의 꼴로 나타낼 수 있다. 이 식의 양변에 상용로그를 취하면
\[
\log N=\log(a\times 10^{n})=\log a+\log 10^{n}=n+\log a
\]
이므로 $\log N$의 값은 상용로그표에서 $\log a$를 찾고 이 값에 $n$을 더하면 된다. 이때, 정수 $n$을 $\log N$의 {\color{red}지표}, $\log a$를 $\log N$의 {\color{red}가수}라고 한다. 여기서 $1 \le a < 10$이므로 $0 \le \log a <1$이다.

예를 들어, 상용로그표에서 $\log 5.15=0.7118$이므로
\[
\log 515=\log(5.15\times 10^{2})=\log 10^{2}+\log 5.15=2+0.7118
\]
%
%$0^{\circ}$부터 $90^{\circ}$까지의 삼각함수의 값을 알면 모든 각에 대한 삼각함수의 값을 알 수 있는 것처럼 $1$ 이상 $10$ 미만의 상용로그의 값을 알면 모든 양수의 상용로그의 값을 알 수 있다.
이다. 따라서 $\log 515$의 지표는 $2$, 가수는 $0.7118$이다. 또,
\begin{align*}
	\log 0.00515&=\log(5.15\times 10^{-3})=\log 10^{-3}+\log 5.15\\
	                     &=-3+0.7118
\end{align*}
이므로 $\log 0.00515$의 지표는 $-3$, 가수는 $0.7118$이다.

한편, $\log 0.00515=-3+0.7118=-2.2882$로 나타내면 지표와 가수를 찾아내기가 불편하므로 $-2.2882=-3+0.7118=\overline{3}.7118$과 같이 지표가 음의 정수 $-n$인 경우에는 $\overline{n}$으로 나타내기도 한다.

\begin{definition}[지표와 가수]
	양수 $N$에 대하여
	\[
	\log N=n+\alpha
	\]
	($n$은 정수, $0≤\alpha < 1$)이면 $\log N$의 지표는 $n$, 가수는 $\alpha$이다.
\end{definition}
\begin{sample}
	상용로그표에서 $\log 2.83=0.4518$이므로
	\begin{enumerate}[label=(\arabic*)]
		\item $28.3$의 상용로그의 값을 구하면
		\begin{align*}
			\log 28.3&=\log(10\times 2.83)=\log 10+\log 2.83\\
			      			&=1+0.4518=1.4518
		\end{align*}
		따라서 $\log 28.3$의 지표는 $1$, 가수는 $0.4518$이다.
		\item $0.0283$의 상용로그의 값을 구하면
		\begin{align*}
		\log 0.0283  &=\log(10^{-2}\times 2.83)=\log 10^{-2}+\log 2.83\\
		    		          &=-2+0.4518=\overline{2}.4518
		 \end{align*}
		따라서 $\log 0.0283$의 지표는 $-2$, 가수는 $0.4518$이다.
	\end{enumerate}
\end{sample}

\begin{problem}
	$\log 1.09=0.0374$임을 이용하여 다음 상용로그의 지표와 가수를 구하여라.
	\quesfour{\log 109000}{\log 10.9}{\log 0.000109}{\log\frac{1}{1.09}}
\end{problem}

상용로그의 지표와 가수의 성질에 대하여 알아 보기  위해 숫자의 배열은 같고 소수점의 위치만 다른 수들의 상용로그의 지표와 가수가 갖는 규칙성에 대하여 생각해 보자.

예를 들어, 상용로그표에서 $\log 4.16=0.6191$이므로
\begin{align*}
    \log 41.6=&\log(4.16\times 10) =\log 4.16+\log 10 =1+0.6191\\
	\log 416=&\log(4.16\times 10^{2})=\log 4.16+\log 10^{2} =2+0.6191\\
	\log 4160=&\log(4.16\times 10^{3})=\log 4.16+\log 10^{3}=3+0.6191\\
& \qquad\qquad\qquad \quad\vdots 
\end{align*}
이고,
\begin{align*}
	\log 0.416=&\log(4.16\times 10^{-1})=\log 4.16+\log 10^{-1}=-1+0.6191\\
	\log 0.0416=&\log(4.16\times 10^{-2})=\log 4.16+\log 10^{-2}=-2+0.6191\\
	\log 0.00416=&\log(4.16\times 10^{-3})=\log 4.16+\log 10^{-3}=-3+0.6191\\
	& \qquad\qquad\qquad \qquad\vdots 
\end{align*}
이다. 따라서 다음과 같은 성질을 알 수 있다.
주어진 양수 $A$가 정수부분이 존재할 때,  $A$의 상용로그의 지표가 $k$이면 $A$의 정수부분은 $k+1$자리의 자연수이다. 또 주어진 양수 $A$의 정수부분이 존재하지 않으면 $A$의 상용로그의 지표는 음의 정수이고 이 수의 절댓값에서 처음으로 $0$이 아닌 숫자가 나타난다.

일반적으로 상용로그의 지표와 가수에 대하여 다음과 같은 성질이 성립한다.
\begin{theorem}[지표와 가수의 성질]
\begin{enumerate}[label=(\arabic*)]
	\item 정수 부분이 $n$자리인 수의 상용로그의 지표는 $n-1$이다.
	\item 소수 $n$째 자리에서 처음으로 $0$이 아닌 숫자가 나타나는 소수의 상용로그의 지표는 $\overline{n}$, 즉 $-n$이다.
	\item 숫자의 배열이 같고 소수점의 위치만 다른 수들의 상용로그의 가수는 모두 같다.
\end{enumerate}

\end{theorem}
\begin{example}
	$\log 2=0.3010$임을 이용하여 다음 물음에 답하여라.
\begin{enumerate}[label=(\arabic*)]
	\item $2^{30}$은 몇 자리의 자연수인지 구하여라.
	\item $\frac{1}{2^{20}}$은 소수 몇째 자리에서 처음으로 $0$이 아닌 숫자가 나타나는지 구하여라.
\end{enumerate}
\begin{solution}
	\begin{enumerate}[label=(\arabic*)]
		\item $2^{30}$의 상용로그의 값을 구하면
		 \begin{align*}
		 	\log 2^{30}&=30\log 2\\
		 		 	&=30\times 0.3010=9.030
		 \end{align*}
		따라서 $\log 2^{30}$의 지표가 $9$이므로 $2^{30}$은 $10$자리의 자연수이다.
		\item $\frac{1}{2^{20}}$의 상용로그의 값을 구하면
		\begin{align*}
			\log\dfrac{1}{2^{20}}&=\log 2^{-20}= -20\log 2\\
						                       &=(-20)\times 0.3010\\
			                     			    &= -6.02 =\overline{7}.98
		\end{align*}
	따라서 $\log\frac{1}{2^{20}}$의 지표가 $-7$이므로 $\frac{1}{2^{20}}$은 소수 $7$째 자리에서 처음으로 $0$이 아닌 숫자가 나타난다.
	\end{enumerate}
\end{solution}
\end{example}

\begin{problem}
$\log 2=0.3010$, $\log 3=0.4771$임을 이용하여 다음 물음에 답하여라.
\begin{enumerate}[label=(\arabic*)]
	\item $6^{10}$은 몇 자리의 자연수인지 구하여라.
	\item $\left(\dfrac{2}{3}\right)^{100}$은 소수 몇째 자리에서 처음으로 $0$이 아닌 숫자가 나타나는지 구하여라.
\end{enumerate}
\end{problem}

지금까지 결과를 종합해 보면 $0^{\circ}$부터 $90^{\circ}$까지의 삼각함수의 값을 알면 모든 각에 대한 삼각함수의 값을 알 수 있는 것처럼 $1$ 이상 $10$ 미만의 상용로그의 값을 알면 모든 양수의 상용로그의 값을 알 수 있다.
\begin{problem}
	$\log a$의 지표는 $\log 2002$의 지표와 같고, $\log a$의 가수는 $\log 202$의 가수와 같을 때, 양수 $a$의 값을 구하는 방법을 설명하여라.
	\processifversion{psol}{
\begin{psolution}
	$\log 2002$의 지표는 $2002$가 $4$자리 수이므로 $3$임을 알 수 있고 $\log 202$의 가수는 $\log 2.02$이다. 따라서 
	\[
	\log a = 3 + \log 2.02 = \log 1000 + \log 2.02 = \log 2020
	\]
	이다. 따라서 $a=2020$이다.
\end{psolution}	
}
\end{problem}
상용로그를 이용하면 복잡한 계산을 간편하게 할 수 있으며, 이를 활용하여 여러 가지 실생활 문제를 해결할 수 있다.  

원금 $A$원을 연이율 $r\%$의 복리로 예금할 때, $n$년 후의 원리합계는 $A\left(1 +\dfrac{r}{100}\right)^{n}$원임을 이용하여 다음의 예제를 해결해 보자.
\begin{example}
	$100$만 원을 연이율 $6\%$, $1$년마다의 복리로 계산하여 $10$년간 예금하였을 때, 원리합계는 약 얼마가 되는지 상용로그표를 이용하여 구하여라.
	\begin{solution}
		$10$년 후의 원리합계를 $x$원이라고 하면
	\begin{align*}
		x&=10^{6}(1+0.06)^{10}\\
		  &=10^{6}\times 1.06^{10} 
	\end{align*}
이고 	$\log 1.06=0.0253$임을 이용하여 $\log 1.06^{10}$의 값을 구하면
	\begin{align*}
		\log 1.06^{10}&=10\log 1.06\\
		                    	&=10\times 0.0253=0.253
	\end{align*}
이다. 상용로그표에서 $0.253$에 가장 가까운 값을 찾으면
	\[
	\log 1.79=0.2529\qquad    \therefore  1.06^{10} \fallingdotseq 1.79
	\]
이므로 구하는 원리합계는 
	\begin{align*}
		x &\fallingdotseq 10^{6}\times 1.79 \\
			&=1790000
	\end{align*}
		
	\end{solution}
\end{example}

\begin{problem}
	어느 자동차는 신차의 가격이 $2000$만 원이고, 중고차의 가격은 매년 $10\%$씩 하락한다고 한다. 이 차를 새로 사고 $10$년 지난 후의 가격은 얼마인지 구하여라.(단, $\log 3=0.4771$, $\log 3.48=0.542$로 계산한다.)
	\processifversion{psol}{
\begin{psolution}
	신차 가격이 $2000$만원이고 중고차 가격가격은 매년 $10\%$씩 하락하므로 $10$년 후의 차의 가격은 $2000(0.9)^{10}$만원 이다. 따라서
	\begin{align*}
		\log 2000(0.9)^{10} & = \log 2000 + 10 \log\frac{9}{10}\\
		& = \log2000 + 10(2\log 3 - 1) \\
		& = \log2000 + 10(2 \times 0.4771 - 1)\\
		& = \log2000 - 0.458 = \log200 + 0.542\\
		& = \log200 + \log3.48 = \log696		
	\end{align*}
이다. 따라서 구하는 금액은 $696$만원이다.
\end{psolution}	
}
\end{problem}

\begin{problem}
	상용로그표를 이용하여 다음 물음에 답하여라.
	\begin{enumerate}[label=(\arabic*)]
		\item $\log 2^{500}$의 지표 $n$과 가수 $a$를 구하여라.
		\item $\log k <\alpha <\log(k+1)$을 만족하는 자연수 $k$의 값을 구하여라.
		\item $2^{500}$은 몇 자리의 자연수인지 말하여라.
		\item $2^{500}$의 가장 큰 자리의 숫자를 말하고, 그 이유를 설명하여라.
	\end{enumerate}
\processifversion{psol}{
\begin{psolution}
	\begin{enumerate}[label=(\arabic*)]
		\item $\log 2^{500}=500 \log 2 = 500 \times 0.3010 =150.5$이므로 지표는 $150$이고 가수 $\alpha$는 $0.5$이다. 
		\item $\log3 = 0.4771$이고 $\log 4 = 2\log2=0.6020$이므로 구하는 자연수 $k$는 $3$이다.
		\item $2^{500}$의 지표가 $150$이므로 $151$자리 수이다.
		\item $2^{500}$의 가수가 $0.5$이므로 (2)에 의해 $3$임을 알 수 있다.
	\end{enumerate}
\end{psolution}
}
\end{problem}
\newpage
\begin{tcolorbox}
	\textbf{로그와  감각}
\end{tcolorbox}

지진의 규모, 용액의 산성도, 별의 밝기, 소리의 크기 등을 나타낼 때 모두 로그가 사용된다. 그런데 흥미롭게도 이와 같이 로그가 사용되는 경우는 대부분 인간의 감각과 관련되어 있다.

일반적으로 인간이 감각으로 구별할 수 있는 한계는 두 자극의 물리적인 양의 차에 의해서가 아니라 비율의 차에 의해서 결정된다는 사실을 $19$세기의 생리학자인 베버가 발견하였다.

예를 들어, 어떤 사람이 $30\textrm{g}$의 무게와 $31\textrm{g}$의 무게를 손바닥에 놓고 겨우 구별할 수 있다고 할 때, 이 사람은 $60\textrm{g}$과 $61\textrm{g}$의 차이는 구별할 수 없고, $60\textrm{g}$과 $62\textrm{g}$의 차이는 겨우 구별할 수 있다는 것이다.

따라서 자극을 받고 있는 감각기에 약한 자극을 주면 자극의 변화가 작아도 그 변화를 쉽게 느낄 수 있으나, 처음에 강한 자극을 주면 약한 자극은 느낄 수 없으며, 더 강한 자극을 주어야만 변화를 느낄 수 있다고 한다.

이러한 사실을 근거로 물리학자이며 철학자인 페히너는


\begin{tcolorbox}
	“감각의 세기 $S$는 그 감각이 일어나게 한 자극의 물리적인 양 $I$의 로그에 비례한다. 즉,
	\begin{equation*}
		S= k\log I \qquad (k\text{는 상수})
	\end{equation*}
	이다.”
\end{tcolorbox}

라는 가설을 발표하였다. 이를 {\color{blue}베버-페히너의 법칙(Weber-Fechner's law)}이라고 한다.

\begin{problem}
	리히터 규모가 $M$인 지진의 진원지에서의 에너지의 크기를 $E$라고 하면 관계식
	\[
		\log E=11.8+1.5M
	\]
	이 성립한다고 한다.
	
	$2008$년 중국에서 일어난 대지진의 리히터 규모는 $8.0$이었다. 대지진 후 여러 차례의 여진이 추가로 발생했는데, 그 중에는 규모가 $6.0$인 것도 있었다고 한다. 규모 $8.0$과 $6.0$인 지진의 진원지에서의 에너지의 크기를 각각 $E_{1}$, $E_{2}$라고 할 때, 다음 물음에 답하여라.
	\begin{enumerate}[label=(\arabic*)]
		\item $E_{1}$, $E_{2}$의 값을 구하여, $\dfrac{E_{1}}{E_{2}}$의 값을 구하여라.
		\item $E_{1}$, $E_{2}$의 값을 구하지 않고, $\dfrac{E_{1}}{E_{2}}$의 값을 구하는 방법을 설명하여라.
	\end{enumerate}
\processifversion{psol}{
\begin{psolution}
	\begin{enumerate}[label=(\arabic*)]
		\item $\log E_{1} = 11.8 + 1.5\times 8 = 23.8$이고 $\log E_{2} = 11.8 + 1.5 \times 6 = 20.8$이다. 따라서 $E_{1} = 10^{23.8}$이고 $E_{2} = 10^{20.8}$이다.  따라서 
		\begin{equation*}
				\frac{E_{1}}{E_{2}} = \frac{10^{23.8}}{10^{20.8}} = 10^{3}
		\end{equation*}
	이다.
	\item $\log E_1  -\log E_2 = \log \frac{E_{1}}{E_{2}} = 3$이므로 $\frac{E_1}{E_{2}}=10^3$이다.
	
	\end{enumerate}
\end{psolution}
}
\end{problem}

\input{exponential.tex}

\input{logarithm.tex}

\end{document}